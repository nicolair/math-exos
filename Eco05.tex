\begin{tiny}(Eco05)\end{tiny} Propriétés des tangentes : définition bifocale.\newline
Soit $F$ et $F'$ deux points distincts et $t\rightarrow f(t)$ une courbe paramétrée $\mathcal{C}^1$ dont le support est une conique $\mathcal C$ de foyers $F$ et $F'$. En calculant les dérivées de
\begin{align*}
 t\rightarrow \Vert \overrightarrow{Ff(t)}\Vert & &
 t\rightarrow \Vert \overrightarrow{F'f(t)}
\end{align*}
montrer que la tangente en $f(t)$ à $\mathcal C$ est une bissectrice (à préciser selon les cas) des droites $(Ff(t))$, $(F'f(t))$.
