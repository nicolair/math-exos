\begin{tiny}(Cen12)\end{tiny} On classe les parisiens suivant leur nombre de cheveux. On note $P_i$ l'ensemble de ceux ayant exactement $i$ cheveux. Les $P_i$ forment une partition de l'ensemble des parisiens.\newline
Notons $n$ le nombre total de parisiens et $p$ le nombre de cheveux sur le plus chevelu d'entre eux. Notons $m$ le plus petit des $\sharp P_i$ et $M$ le plus grand. Par le plus simple des encadrements appliqué au dénombrement attaché à la partition, il vient
\begin{displaymath}
 (p+1)m\leq n \leq (p+1)M
\end{displaymath}
On en déduit 
\begin{displaymath}
 m \leq \frac{2500000}{500001}>4
\end{displaymath}
Il existe donc bien dans la partition une classe particulière contenant au moins cinq individus.
