\begin{tiny}(Cdt24)\end{tiny}
\begin{enumerate}
 \item Si $T$ est de rang 1, toutes ses colonnes sont engendrées par une seule colonne non nulle. Notons la $C$. Il existe alors $l_1, \cdots, l_n$ non tous nuls tels que
 \begin{multline*}
  \left( \forall j \in \llbracket 1,n \rrbracket, \; C_j(T) = l_iC\right) \\ 
  \Rightarrow
  T = C \,L\text{ avec } 
  L =
  \begin{pmatrix}
   l_1 & \cdots & l_n
  \end{pmatrix}.
 \end{multline*}

 \item Notons $\delta$ le déterminant cherché et $X_1, \cdots, X_n$ les colonnes de $I_n$. Ce sont aussi les vecteurs de la base canonique des matrices colonnes. Alors:
 \[
  C_j(I_n + CL) = X_j + l_j\,C
 \]
Dans le développement du déterminant obtenu par multilinéarité, il ne subsiste plus que les termes contenant au plus une fois la colonne $C$
\begin{multline*}
 \delta = \underset{=1}{\underbrace{\det(X_1,\cdots,X_n)}} \\
 + \sum_{j=1}^{n}\underset{ = l_j c_j}{\underbrace{\det(X_1,\cdots,X_{j-1},l_jC,X_{j+1},\cdots,X_n)}}\\
 = 1 + L\,C.
\end{multline*}

 \item Dans le cas où $A$ est inversible, on se ramène à la question précédente en factorisant pas $A$
\begin{multline*}
 \det(A + CL) = \det(A)\det(I_n + (A^{-1}C)\,L)\\
 = \det(A) + \det(A)LA^{-1}C\\
 = \det(A) + L\, \trans{\,\mathop{\mathrm{Com }}}(A) C
\end{multline*}

à cause de l'expression de la matrice inverse avec la transposée de la comatrice.\newline
Lorsque $A$ n'est pas inversible, on obtient la formule en l'approchant par des matrices inversibles par exemple de la forme $A + \lambda I_n$ avec $\lambda$ non nul mais assez petit pour n'être pas une valeur propre.
\end{enumerate}
