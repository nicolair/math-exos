\begin{tiny}(Cao04)\end{tiny} Posons 
\begin{multline*}
 \overrightarrow{a} 
 = \frac{1}{\Vert \overrightarrow{v} \wedge \overrightarrow{x}\Vert}\left( \overrightarrow{v} \wedge \overrightarrow{x}\right) \wedge \overrightarrow{v}, \\
 \overrightarrow{b} 
 = \frac{1}{\Vert \overrightarrow{v} \wedge \overrightarrow{x}\Vert}\left( \overrightarrow{v} \wedge \overrightarrow{x}\right).
\end{multline*}

On vérifie que 
$(\overrightarrow{a} , \overrightarrow{b} , \overrightarrow{v})$ est une base orthonormée directe. La fonction proposée n'est que la transcription vectorielle de la rotation d'angle $\theta$ autour de $\overrightarrow{v}$ définie par sa matrice dans une base orthonormée directe dont $\overrightarrow{v}$ est le troisième vecteur. 
