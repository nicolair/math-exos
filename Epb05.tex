\begin{tiny}(Epb05)\end{tiny} \label{fPoinc} Fonctions caractéristiques et formule de Poincaré.\newline
Soit $\Omega$ un ensemble quelconque fixé. On désigne par $\mathcal{F}$ l'ensemble des fonctions définies dans $\Omega$ et à valeurs dans $\{0,1\}$. C'est une partie de l'espace noté $\mathcal{F}_\R$ des fonctions définies dans $\Omega$ et à valeurs dans $\Z$. La somme et le produit (fonctionnels) sont bien définis dans $\mathcal{F}_\R$ et lui confèrent une structure d'anneau. Le neutre multiplicatif est la fonction (notée $U$) constante de valeur $1$.\newline
On associe à chaque partie $A$ de $\Omega$ une fonction notée $\varphi_A$ définie dans $\Omega$ qui prend les valeurs $0$ ou $1$:
\begin{align*}
\Omega &\rightarrow \{0,1\}\\
\omega &\mapsto
\left\lbrace 
\begin{aligned}
 1 &\text{ si } \omega\in A\\
 0 &\text{ si } \omega \notin A 
\end{aligned}
\right. 
\end{align*}
Cette fonction est appelée la fonction \emph{caractéristique} de la partie $A$.  
\begin{enumerate}
 \item Montrer, pour toutes parties $A$ et $B$ de $\Omega$, les égalités (entre fonctions) suivantes
\begin{displaymath}
 \varphi_{\overline{A}} = U - \varphi_A,\hspace{0.5cm}
\varphi_{A\cap B} = \varphi_A \varphi_B
\end{displaymath}
 \item  Comment s'exprime $\varphi_{A\cup B}$ ?\newline
Dans la suite de l'exercice, $A_1, \cdots, A_n$ est une famille de parties de $\Omega$. Pour toute partie $I$ de $\{1,\cdots,n\}$, on définit $A_I$ par:
\begin{displaymath}
 A_I = \cap_{i\in I}A_i
\end{displaymath}
Exprimer $\varphi_{A_1\cup A_2\cup \cdots \cup A_n}$ avec des $\varphi_{A_I}$ (on regroupera les parties $I$ de $\{1,\cdots,n\}$ qui ont le même nombre d'éléments).
\item Dans la suite de l'exercice, l'ensemble $\Omega$ est \emph{fini}. On considère une fonction $C$ définie dans $\mathcal{F}_\R$ et à valeurs dans $\R$ par 
\begin{displaymath}
 C(f) = \sum_{\omega \in \Omega} f(\omega)
\end{displaymath}
Que valent $C(\lambda f)$ et $C(f+g)$?\newline
Montrer les formules de Poincaré
\begin{align*}
 \sharp(A_1\cup \cdots \cup A_n) &=
\sum_{k=1}^{n}(-1)^{k+1}\sum_{I\text{ tq }\sharp I = k}\sharp(A_I) \\
 \p(A_1\cup \cdots \cup A_n) &=
\sum_{k=1}^{n}(-1)^{k+1}\sum_{I\text{ tq }\sharp I = k} \p(A_I)
\end{align*}
pour une probabilité quelconque sur l'univers $\Omega$.
\end{enumerate}
