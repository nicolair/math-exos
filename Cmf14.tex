\begin{tiny}(Cmf14)\end{tiny} Comme $\mathcal{U}$ est formé de 3 vecteurs et que l'on a besoin des matrices de passage, le plus économique est de montrer que la famille est génératrice en exprimant les $e$ en fonction des $u$. On tire des relations
\[
\left\lbrace
  \begin{aligned}
    e_1 &= \frac{1}{3}\left(u_1 + u_2\right)\\
    e_2 &= \frac{1}{2}\left(e_1 -u_1 + u_3\right) = \frac{1}{6}\left(-2u_1 + u_2 + 3u_3\right)\\
    e_3 &= \frac{1}{2}\left(e_1 -u_1 - u_3\right) =\frac{1}{6}\left(-2u_1 + u_2 - 3u_3\right)
  \end{aligned}
\right.
\]
On en déduit
\begin{multline*}
  P = P_{\mathcal{E}\mathcal{U}}=
  \begin{pmatrix}
    1 & 2 & 0 \\ -1 & 1 & 1 \\ -1 & 1 & -1
  \end{pmatrix},\\
P^{-1} = P_{\mathcal{U}\mathcal{E}}= \frac{1}{6}
  \begin{pmatrix}
    2 & -2 & -2 \\ 2 & 1 & 1 \\ 0 & 3 & -3
  \end{pmatrix}.
\end{multline*}
De plus,
\begin{multline*}
  A\begin{pmatrix}
     1 \\ -1 \\-1
   \end{pmatrix}
= \begin{pmatrix}
     -1 \\ 1 \\1
   \end{pmatrix} = -\begin{pmatrix}
     1 \\ -1 \\-1
   \end{pmatrix},\\
   A\begin{pmatrix}
     2 \\ 1 \\1
   \end{pmatrix}
= \begin{pmatrix}
     4 \\ 2 \\ 2
   \end{pmatrix} = 2\begin{pmatrix}
     2 \\ 1 \\ 1
   \end{pmatrix},\\  
   A\begin{pmatrix}
     0 \\ 1 \\ -1
   \end{pmatrix}
= \begin{pmatrix}
     0 \\ 0 \\ 0
   \end{pmatrix}
\end{multline*}
On en déduit 
\[
  \MatB{U}{f}=
  \begin{pmatrix}
    -1 & 0 & 0 \\ 0 & 2 & 0 \\ 0 & 0 & 0
  \end{pmatrix} = D.
\]
La formule de changement de base donne
\[
  A = P D P^{-1}\Rightarrow A^n = P D^n P^{-1}.
\]

