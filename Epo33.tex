\begin{tiny}(Epo33)\end{tiny} Division suivant les puissances croissantes.\newline
La \emph{valuation} d'un polynôme (notée $\val(P)$ ) est le plus grand des entiers $k$ tels que $X^k$ divise $P$. On convient que le polynôme nul est de valuation $-\infty$ conventionellement inférieure à toutes les autres.
\begin{enumerate}
 \item Montrer que $\val(PQ)=\val(P)+\val(Q)$ et que $\val(P+Q)\geq \val(P)+\val(Q)$.
 \item Soit $A\in \K[X]$ non nul de valuation nulle. Montrer que pour tout $B\in \K[X]$ de valuation nulle et tout $n\in \N$, il existe un unique couple $(Q_n,R_n)$ dans $\K[X]$ tels que
\begin{displaymath}
 B = Q_n A + R_n \text{ avec  } \val(R_n)>n
\end{displaymath}
\item Exemple Diviser $B=1+X+X^2$ par $A=1-X^3$ suivant les puissances croissantes avec $n=4$.
\end{enumerate}
