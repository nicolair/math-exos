\begin{tiny}(Epb02)\end{tiny} Mon ami Paul travaille dans un cercle de jeux. Il m'en a expliqué un. Trois boîtes sont alignées sur la table, une seule contient un jeton, les deux autres sont vides. Paul sait dans quelle boîte se trouve le jeton. Le client a deux possibilités
\begin{itemize}
 \item[S1:] il choisit une boîte en disant \og C'est mon dernier choix\fg.
 \item[S2:] il désigne une boîte, Paul élimine une boîte vide parmi les deux non désignées et le client choisit une des deux qui restent.
\end{itemize}
Le client gagne s'il choisit la boîte contenant le jeton. Quelle est la meilleure stratégie?  