On note 
\begin{displaymath}
 f(u) = (\sqrt{\cos u }, \sqrt{\sin u })
\end{displaymath}
pour $u$ entre $0$ et $\frac{\pi}{2}$. Le cercle osculateur en $f(u)$ passe par l'origine si et seulement si le rayon de courbure en $f(u)$ est égal à $\Vert \overrightarrow{Of(u)}\Vert$.\newline
On utilise la formule du déterminant pour calculer ( avec Maple) le rayon de courbure :
\begin{verbatim}
 restart;
#appel de la bibliothèque d'algèbre linéaire
with(LinearAlgebra):

f:= [sqrt(cos(u)),sqrt(sin(u))];

# tracé de la courbe.
# Le op pour enlever les crochets de liste
#plot([op(f),u=0..1.5]);

#calcul des dérivées,
# transformation en type vecteur  
ff:=diff(f,u): v:=<ff[1],ff[2]>:
fff:=diff(ff,u): w:=<fff[1],fff[2]>:

#carré de la distance origine-point
l:=cos(u)+sin(u):

#carré de la norme de la vitesse
# simplification :
nv:= v.v :
nv := simplify(nv);

#déterminant de la matrice 
#famille (vitesse, accélération)
#simplification
dd:=Determinant(<v|w>):
dd:=simplify(dd);
\end{verbatim} 

On obtient en particulier :
\begin{displaymath}
 \det(\overrightarrow{f'}(u),\overrightarrow{f''}(u))
=\dfrac{3}{8\sqrt{\sin u \cos u}}
\end{displaymath}

En posant $s$ pour $\sin u$ et $c$ pour $\cos u$, la condition cherchée s'écrit :
\begin{multline*}
 \dfrac{1}{8}\left(\dfrac{s^2}{c}+\dfrac{c^2}{s} \right)^3=
 \left( \dfrac{3}{8}\right)^2\dfrac{s+c}{sc} \\
\Leftrightarrow
9(cs)^2(c+s)=8(c^3+s^3)^3 
\end{multline*}
Or $c^3+s^3 = (c+s)(1-cs)$, la condition s'écrit donc
$9(cs)^2 = 8 (1+2cs)(1-sc)^3$ Comme $(s+c)^2 = 1+2sc$ la condition s'écrit finalement
\begin{displaymath}
 9(cs)^2=8(1+2sc)(1-cs)
\end{displaymath}
On obtient une équation en $\sin 2u$ qui admet une seule solution. Je ne sais pas si cette solution admet une expression particulière.
