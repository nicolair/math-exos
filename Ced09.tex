\begin{tiny}(Ced09)\end{tiny}
\begin{enumerate}
  \item Il s'agit d'un simple calcul de dérivée d'une fonction composée. 
  \item Calcul trigonométrique:
\[
\sqrt{\frac{1 + \cos(2t)}{1 - \cos(2t)}}
= \sqrt{\frac{1 + (2\cos^2 t -1)}{1 - (1-2\sin^2t)}}
= \cotan t
\]
car $\sin t$ et $\cos t$ sont positifs dans l'intervalle.

  \item D'après les questions a. et b. 
 \begin{center}
 \renewcommand{\arraystretch}{1.2}
\begin{tabular}{|l|c|c|} \hline
équation  & (2)              & (3)\\ \hline
solutions & $\lambda \cotan$ & $\lambda \tan$ \\ \hline
 \end{tabular}
 \end{center}

  \item Dérivons $uv = 1$
\[
\begin{aligned}
  uv                   &= 1 &\times& -16 = -8 +-8 \\
  u'v + uv'            &= 0 &\times& 2\sin(4t) \\
  u''v + uv'' + 2 u'v' &= 0 &\times& 1 - \cos(4t) \\ \hline
  0 + 0 + 2(1-4 \cos t) u'v' &= 16 &  &
\end{aligned}
\]
car $u$ et $v$ sont solutions. On en déduit
\[
  u'v'\sin^2(2t) = -4 .
\]
Utilisons $v = \frac{1}{u}$ pour former une équation dont $u$ est solution.
\begin{multline*}
\left.
\begin{aligned}
  v' = -\frac{u'}{u^2} \\ u'v'\sin^2(2t) = -4  
\end{aligned}
\right\rbrace 
\Rightarrow \left( \frac{u'}{u} \right)^2 = \frac{4}{\sin^2(2t)}\\
\Rightarrow 
u \text{ solution  de $(2)$ ou $(3)$}.
\end{multline*}

Remarque. Un argument de continuité du type théorème de la valeur intermédiaire est nécessaire pour assurer que l'égalité à $\frac{2}{\sin(2t)}$ ou $-\frac{2}{\sin(2t)}$ est la même dans tout l'intervalle.
\end{enumerate}
