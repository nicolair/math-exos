\begin{tiny}(Edi09)\end{tiny} Preuve du théorème de prolongement linéaire.\newline
Soit $(a_1,\cdots,a_p)$ une base d'un $\K$-espace vectoriel $E$ et $(\alpha_1,\cdots,\alpha_p)$ la famille des formes coordonnées dans cette base. Soit $(y_1,\cdots,y_p)$ une famille de vecteurs dans un $\K$-espace vectoriel $F$.
\begin{enumerate}
 \item Soit $f\in \mathcal{L}(E,F)$ telle que $f(a_i)=y_i$ pour $i$ de $1$ à $p$. Pour tout $x\in E$, exprimer $f(x)$ à l'aide des $y_i$ et des $\alpha_i(x)$. Que peut-on en déduire quant au théorème de prolongement linéaire.
 \item Achever la démonstration du théorème.
\end{enumerate}

