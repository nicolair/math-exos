\begin{tiny}(Ccu16)\end{tiny} On peut déduire du tableau, la \og vitesse de peinture\fg~ de chacun. Notons les $V_1$, $V_2$, $V_3$ et $S$ la surface de l'appartement (en $m^2$).
\begin{center}
\renewcommand{\arraystretch}{1.5}
\begin{tabular}{|c|c|c|} \hline
$V_1$           & $V_2$           & $V_3$\\ \hline
$\frac{S}{3}$   & $\frac{S}{2}$   & $\frac{S}{5}$ \\ \hline
\end{tabular}
\end{center}
Au bout d'un temps $t$ de travail (en heure), $P_i$ a peint $V_i t$ mètres carrés. Ensemble, les 3 peintres ont couvert
\[
 S_t = (V_1 + V_2 + V_3)t = (\frac{1}{3} + \frac{1}{2}+ \frac{1}{5})St
\]
Ils ont terminé lorsque $S_t = S$, le temps cherché (en heures) est donc
\[
 \frac{1}{\frac{1}{3} + \frac{1}{2}+ \frac{1}{5}} = \frac{30}{31} = \frac{1}{1+\frac{1}{30}}
 \simeq 1 - \frac{1}{30} \simeq 58 \text{mn}.
\]
car $1/30$ ème d'heure égale 2 minutes.