\begin{tiny}(gp8)\end{tiny}
\begin{figure}[ht]
 \centering
\input{Egp8_1.pdf_t}
\caption{Exercice \arabic{enumi} : définitions.}
\label{fig:Egp8_1}
\end{figure}
\begin{figure}[ht]
 \centering
\input{Egp8_2.pdf_t}
\caption{Exercice \arabic{enumi}: symétrique de $\mathcal D_A$.}
\label{fig:Egp8_2}
\end{figure}
Soit $A,B,C$ trois points non align{\'e}s dans un plan euclidien. Le cercle inscrit dans ce triangle est de centre un point $O$ et de rayon $r$. Les vecteurs $\overrightarrow a$, $\overrightarrow b$, $\overrightarrow c$ sont de norme $1$ et définis comme dans la figure \ref{fig:Egp8_1}. On définit aussi des fonctions $\alpha$, $\beta$, $\gamma$ dans le plan et à valeurs réelles :
\begin{align*}
 \alpha(M) =& (\overrightarrow{OM}/\overrightarrow a) -r \\
 \beta(M) =& (\overrightarrow{OM}/\overrightarrow b) -r \\
 \gamma(M) =& (\overrightarrow{OM}/\overrightarrow c) -r 
\end{align*}

\begin{enumerate}
\item Comment écrit-on qu'un point $M$ est dans une des droites $(AB)$, $(BC)$, $(CA)$?
 \item On note $X$ et $Y$ les fonctions coordonnées dans un  repère orthogonal mais pas forcément orthonormé. On considère une droite $\mathcal D$ dont l'équation exprimée avec ces coordonnées est
\begin{displaymath}
 uX + vY = 0
\end{displaymath}
Montrer que l'image de $\mathcal D$ par une symétrie orthogonale par rapport à un des axes a pour équation:
\begin{displaymath}
 uX - vY = 0
\end{displaymath}
\item On considère une droite $\mathcal D_A$ passant par $A$ et on note $\mathcal D'_A$ la droite sym{\'e}trique de $\mathcal D_A$ par rapport {\`a} la bissectrice de $(AB)$ et $(AC)$. On suppose qu'il existe des réels $\lambda$ et $\mu$ tels que
\begin{displaymath}
 M\in\mathcal D_A \Leftrightarrow \lambda \beta(M) + \mu \gamma(M) = 0
\end{displaymath}
Montrer que 
\begin{displaymath}
 M\in\mathcal D'_A \Leftrightarrow \mu \beta(M) + \lambda \gamma(M) = 0
\end{displaymath}
\item On d{\'e}finit comme dans la question précédentes, (avec $A$ puis $B$) des droites $\mathcal D'_B$ et $\mathcal D'_{C}$ à partir de droites $\mathcal D_B$ et $\mathcal D._{C}$. Montrer que les trois droites $\mathcal D'_A$, $\mathcal D'_B$, $\mathcal D'_C$ sont parall{\`e}les ou concourantes si et seulement si $\mathcal D_A$, $\mathcal D_B$, $\mathcal D_C$ sont parallèles ou concourantes.
\end{enumerate}

 