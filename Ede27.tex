\begin{tiny}(Ede27)\end{tiny} On définit une nouvelle relation locale (en $a$) entre deux fonctions (continues et à valeurs strictement positives). On dit que $f$ et $g$ sont \emph{de même ordre} en $a$ si et seulement si $o(f)=o(g)$. C'est à dire qu'une fonction est négligeable devant l'une si et seulement si elle est négligeable devant l'autre.
\begin{enumerate}
  \item Montrer que deux fonctions qui se dominent mutuellement sont de même ordre.
  \item Soit $h$ une fonction définie au voisinage de $a$ à valeurs positives et non majorée au voisinage de $a$. Montrer qu'il existe des fonctions $\varphi$ et $\psi$ à valeurs positives, définies au voisinage de $a$ et telles que
\begin{displaymath}
  h = \varphi \psi, \hspace{0.5cm} \varphi \xrightarrow{a} +\infty, \hspace{0.5cm} \psi \text{ ne tend pas vers $0$ en $a$}
\end{displaymath}
\item Montrer que deux fonctions de même ordre se dominent mutuellement.
\end{enumerate}
