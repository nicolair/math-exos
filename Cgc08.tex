\begin{tiny}(Cgc08)\end{tiny} Par hypothèse $b-a\in \mathcal{T}$.
\begin{enumerate}
  \item On suppose (faire un dessin)
\begin{displaymath}
  a < x_{min} < b \hspace{0.3cm}\text{avec} \hspace{0.3cm} f(x_{min}) < f(a)=f(b)
\end{displaymath}
Soit $T\in ]0,\alpha]$. Alors $x_{min}-T$ et $x_{min}+T$ sont dans $[a,b]$ par définition de $\alpha$. Définissons $g$:
\begin{displaymath}
  g:
\left( 
\begin{aligned}
  [a, b-T] &\rightarrow \R \\ x &\mapsto f(x+T) - f(x)
\end{aligned}
\right) 
\end{displaymath}
Comme $x_{min}$ réalise la plus petite valeur de $f$:
\begin{displaymath}
  g(x_{min}-T) \leq 0 \; \text{ et } \; g(x_{min}) \geq 0
\end{displaymath}
On peut alors appliquer le TVI pour justifier que $T\in \mathcal{T}$.
  \item Si $f(a)=f(b)$ n'est pas la plus grande valeur de $f$, on peut raisonner avec la valeur maximale comme en a. Si $f(a)=f(b)$ est à la fois la plus grande et la plus petite valeur de $f$ c'est qu'elle est constante.
  \item à compléter
  \item à compléter
  \item à compléter
\end{enumerate}
