\begin{tiny}(Ccu22)\end{tiny} Une première idée est de remarquer que $\binom{2n}{n}$ est le plus grand des $\binom{2n}{k}$. On en déduit que
\begin{displaymath}
  4^2 = (1+1)^{2n} \leq (2n+1)\binom{2n}{n}
\end{displaymath}
mais l'inégalité obtenue est plus faible que celles qui sont demandées.\newline
Une deuxième idée est de tirer parti de l'expression multiplicative d'un coefficient du binôme pour en déduire
\begin{displaymath}
  Q_n = \frac{1}{4^n}\binom{2n}{n} = \frac{\text{Pdt des impairs de $1$ à $2n$}}{\text{Pdt des pairs de $1$ à $2n$}}
\end{displaymath}
Notons 
\begin{displaymath}
  u_n = \frac{1}{2\sqrt{n}},\hspace{0.5cm} v_n =\frac{1}{n^{\frac{1}{3}}}
\end{displaymath}
et comparons les quotients de termes consécutifs.
\begin{multline*}
  \frac{u_{n+1}}{u_n} = \left( \frac{n}{n+1}\right)^{\frac{1}{2}} ,\hspace{0.5cm}
  \frac{v_{n+1}}{v_n} = \left( \frac{n}{n+1}\right)^{\frac{1}{3}}\\
  \frac{Q_{n+1}}{Q_n} = \frac{2n+1}{2(n+1)}
\end{multline*}
Après calculs, on trouve que
\begin{displaymath}
 \left( \frac{Q_{n+1}}{Q_n}\right)^{2} - \left(\frac{u_{n+1}}{u_n} \right)^2
 = \frac{1}{4(n+1)^2}>0
\end{displaymath}
et que
\begin{displaymath}
 \left(\frac{u_{n+1}}{u_n} \right)^3 - \left( \frac{Q_{n+1}}{Q_n}\right)^{3}
 = -\frac{4n^2+2n-1}{8(n+1)^3}>0
\end{displaymath}
Comme
\begin{displaymath}
  Q_1 = u_1 = \frac{1}{2} < v_1 = 1
\end{displaymath}
On obtient bien l'encadrement demandé.