\begin{tiny}(Ecg08)\end{tiny} Soit $\Gamma$ la courbe de niveau $0$ de la fonction $x-\sin(y)$ où $x$ et $y$ sont les fonctions coordonnées dans un repère fixé (origine $O$) d'un plan $E$. On définit une fonction $f$ dans $E\setminus \Gamma$ par:
\begin{displaymath}
 f= \frac{\sin x -y}{x- \sin y}
\end{displaymath}
 \begin{enumerate}
  \item Montrer que $f$ n'admet pas de limite en $O$. Préciser une courbe paramétrée $\gamma$ telle que $\gamma(t)$ converge vers $O$ pour $t$ en $0$ et que $|f(\gamma(t))|$ diverge vers l'infini.
  \item La fonction admet-elle une limite en un point $A\neq O$ de $\Gamma$?
 \end{enumerate}
