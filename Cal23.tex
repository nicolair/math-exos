\begin{tiny}(Cal23)\end{tiny}
\begin{itemize}
  \item Montrons que $f(e)=e$. Par surjectivité, pour tout $x\in \Omega$, il existe $y\in \Omega$ tel que $f(y)=x$. Alors: 
\begin{displaymath}
\begin{aligned}
&x*f(e) = f(y*e) = f(y) = x \\
&f(e) * x = f(e*y) = f(y) =x  
\end{aligned}
\end{displaymath}
  \item Réciproquement, par injectivité de $f$
\begin{displaymath}
  f(a)=e \Rightarrow f(a)=f(e) \Rightarrow a = e
\end{displaymath}
  \item Si $x\in I$ alors $f(x)\in I$. En effet:
\begin{displaymath}
\begin{aligned}
  &x*x^{-1} = e \Rightarrow f(x) * f(x^{-1})=f(e)=e\\
  &x^{-1}*x = e \Rightarrow f(x^{-1}) * f(x)=f(e)=e
\end{aligned}
\end{displaymath}
ce qui entraine $f(x)$ inversible d'inverse $f(x^{-1})$.
\item Supposons $f(x)$ inversible. Il admet un inverse et, comme $f$ est surjective, il existe $y$ tel que
\begin{multline*}
  f(x)^{-1} = f(y)
\Rightarrow f(x*y) = f(x)*f(y) = e\\
\Rightarrow x*y = e
\end{multline*}
De même de l'autre côté, $x$ est donc inversible d'inverse $y$.
\end{itemize}
