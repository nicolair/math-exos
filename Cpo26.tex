\begin{tiny}(Cpo26)\end{tiny} Si $P$ n'est pas nul, en examinant les termes de plus haut degré, on trouve une condition nécessaire sur le degré. On cherche ensuite l'ensemble des solutions à l'aide de coefficients indéterminés et d'un système. On trouve
\begin{itemize}
 \item Degré $2$, solutions  $\lambda(X^2-1)$ avec $\lambda\in\C$.
 \item Degré $1$, solutions  $\lambda(X+2)$ avec $\lambda\in\C$.
 \item Degré $3$, une seule solution : $\frac{4}{9}X^3$.
\end{itemize}
