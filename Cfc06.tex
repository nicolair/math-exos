\begin{tiny}(Cfc06)\end{tiny} La fonction en escalier $f$ est bornée et atteint ses bornes qui sont donc dans $I$. L'intégrale est la valeur moyenne de $f$. Elle est comprise entre les bornes et donc dans $I$ car $I$ est un intervalle.\newline
Soit $\mathcal{S}=(x_0,\cdots,x_n)$ une subdivision de $[0,1]$ adaptée à $f$. Les nombres $x_{i+1}-x_i$ sont positifs et de somme $1$. On peut donc écrire une inégalité de convexité pour $g$:
\begin{multline*}
 g\left( \int_{[0,1]}f\right) =g\left(\sum_{i=0}^{n-1}(x_{i+1}-x_i)v_i \right)\\
\leq  \sum_{i=0}^{n-1}(x_{i+1}-x_i)g(v_i)
= \int_{[0,1]}g\circ f
\end{multline*}
 où $g\circ f$ est encore en escalier.\newline
L'extension au cas où $f$ est continu repose sur le théorème d'approximation d'une fonction continue par une fonction en escalier et sur le théorème de Heine.\newline
Soit $\varepsilon>0$ arbitraire. D'après le théorème de Heine (appliqué à $g$), il existe un $\alpha>0$ tel que 
\begin{displaymath}
 \forall (u,v)\in I^2,\; |u-v|\leq\alpha \Rightarrow
|g(u)-g(v)|<\frac{\varepsilon}{2}
\end{displaymath}
D'après le théorème d'approximation d'une fonction continue par une fonction en escalier (appliqué à $f$), il existe $\varphi$ en escalier tel que
\begin{displaymath}
 \forall x\in [0,1],\;|f(x)-\varphi(x)|<\alpha
\end{displaymath}
Comme les deux intégrales sont dans $I$ (intervalle), on peut en déduire
\begin{multline*}
 \left|\int_{[0,1]}f - \int_{[0,1]}\varphi \right|\leq \alpha \\
\Rightarrow g\left( \int_{[0,1]}f\right) \leq  g\left( \int_{[0,1]}\varphi\right) + \frac{\varepsilon}{2}
\end{multline*}
On applique alors la propriété dans le cas des fonctions en escalier qui donne
\begin{displaymath}
 g\left( \int_{[0,1]}f\right) \leq \int_{[0,1]}g\circ \varphi + \frac{\varepsilon}{2}
\end{displaymath}
Puis, pour tout $x$ dans $[0,1]$
\begin{displaymath}
 |f(x)-\varphi(x)|<\alpha \Rightarrow
|g(f(x))-g(\varphi(x))|<\frac{\varepsilon}{2}
\end{displaymath}
ce qui entraine
\begin{displaymath}
 \int_{[0,1]}g\circ \varphi \leq \int_{[0,1]}g\circ f + \frac{\varepsilon}{2}
\end{displaymath}
On termine par un raisonnement à la Cauchy car $\varepsilon$ est arbitraire.