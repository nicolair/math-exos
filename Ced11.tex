\begin{tiny}(Eed11)\end{tiny}
On écrit l'équation différentielle dont $f$ et $g$ sont solutions sous la forme
\begin{displaymath}
  y'' +ay' +by = 0
\end{displaymath}
\begin{enumerate}
  \item On dérive la fonction $W$ (wronskien)
\begin{displaymath}
  W' = fg'' - f''g = -aW'
\end{displaymath}
\item Comme l'équation est à coefficients constants, la fonction $g'$ est encore solution de la même équation différentielle. On en tire que le wronskien de $g$ et $g'$ est soit identiquement nul soit jamais nul.\newline
Si le wronskien n'est jamais nul, en raisonnant comme dans la démonstration de cours sur l'ensemble des solutions de l'équation homogène, on montre que $f$ est combinaison linéaire à coefficients constants de $g$ et $g'$. On en tire l'expression demandée de la primitive.\newline
Si le wronskien est identiquement nul, comme il est le numérateur de la dérivée du quotient, la fonction $g(x)$ est de la forme $e^{\lambda x}$ et $x$ lui même se met sous la forme $\frac{1}{\lambda}\ln(g(x))$
\end{enumerate}
