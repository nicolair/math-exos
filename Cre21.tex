\begin{tiny}(Cre21)\end{tiny} Séparons les indices pairs et impairs de la formule du binôme:
\begin{multline*}
 (\sqrt{3}+1)^{2n+1}= \underset{= A}{\underbrace{\sum_{k \text{ pair}}\binom{2n+1}{k}\sqrt{3}^{2n+1-k}}}\\
 + \underset{= B}{\underbrace{\sum_{k \text{ impair}}\binom{2n+1}{k}\sqrt{3}^{2n+1-k}}}
\end{multline*}

Lorsque $k$ est impair, $2n+1-k$ est pair donc $B$ est entier. De plus, en considérant le développement de $(\sqrt{3}+1)^{2n+1}$, on obtient que 
\begin{align*}
 A &= \frac{1}{2}\left((\sqrt{3}+1)^{2n+1} + (\sqrt{3}-1)^{2n+1} \right) \\
 B &= \frac{1}{2}\left((\sqrt{3}+1)^{2n+1} - (\sqrt{3}-1)^{2n+1} \right)
\end{align*}
On en déduit
\begin{displaymath}
 (\sqrt{3}+1)^{2n+1}= 2B + (\sqrt{3}-1)^{2n+1}
\end{displaymath}
avec 
\begin{displaymath}
 0 < (\sqrt{3}-1)^{2n+1} < 1
\end{displaymath}
car $1< \sqrt{3} < 2$.\newline
Si $(a-b\sqrt{d})^{2n+1} = \left\lbrace (a+b\sqrt{d})^{2n+1} \right\rbrace $ alors on doit avoir
\[
 0 < a-b\sqrt{d} <1
\]
La condition est suffisante car on peut raisonner exactement comme avec $\sqrt{3}+1$. On remarque que la condition se traduit par 
\[
 a = \lceil b\sqrt{d} \rceil.
\]
qui permet des calculs fournissant d'autres exemples:
\[
 5+2\sqrt{5}, \;8+3\sqrt{7}, \; \cdots
\]
