\begin{tiny}(Cva01)\end{tiny} L'ensemble des valeurs prises par la variable est $I = \llbracket 2, 12\rrbracket$. Pour $k\in I$, la probabilité d'obtenir $k$ est égale au nombre de solutions dans $\llbracket 1,6\rrbracket^2$ de l'équation $x+y = k$ multiplié par $\frac{1}{36}$ (probabilité d'un couple quelconque).\newline
Pour $k\leq 7$, ces solutions sont 
\begin{displaymath}
  (1,k-1), (2,k-2),\cdots , (k-1,1) \Rightarrow p(X=k) = \frac{k-1}{36}
\end{displaymath}
Pour $k\geq 8$, ces solutions sont 
\begin{displaymath}
  (k-6,6), (k-5,5),\cdots , (6,k-6) \Rightarrow p(X=k) = \frac{13-k}{36}
\end{displaymath}
