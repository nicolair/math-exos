\begin{tiny}(Cgd33)\end{tiny} La fonction $f$ est définie dans $\R \setminus\left\lbrace -1, 0, 1\right\rbrace$. 
\[
 \forall x \in \R \setminus\left\lbrace -1, 0, 1\right\rbrace,\;
f(x) = \frac{e^{x-\frac{1}{x}}-1}{x-\frac{1}{x}}.
\]
En $0$, on ne peut la prolonger par continuité car
\begin{displaymath}
 f(x) = \frac{x}{x^2 - 1}\left( e^{x} e^{-\frac{1}{x}} - 1 \right)
 \rightarrow
 \left\lbrace 
 \begin{aligned}
  0 &\text{ si } x > 0 \\
  +\infty &\text{ si } x < 0
 \end{aligned}
\right.  
\end{displaymath}
En revanche, en $a\in\left\lbrace -1,+1\right\rbrace$ on peut la prolonger par continuité en posant $f(a) = 1$ car $1$ est le nombre dérivé de la fonction exponentielle en $0$ et 
\[
 x - \frac{1}{x} \rightarrow 0.
\]
Reprendre cet exercice dans le cadre de la feuille sur les développements pour étudier la dérivabilité en $-$ ou $+1$.