\begin{tiny}(Cmo08)\end{tiny} Commençons par montrer l'implication suggérée
\begin{multline*}
  \mathstrut^t\!AAX = 0_{\mathcal M_{q,1}(\R)}
  \Rightarrow \mathstrut^t\!X\mathstrut^t\!AAX = 0_{\R}\\
  \Rightarrow \mathstrut^t\!(AX)\,AX = 0_{\R} \\
  \Rightarrow y_1^2+\cdots + y_p^2 = 0 
  \text{ avec } AX =
\begin{pmatrix}
  y_1\\\vdots \\ y_p
\end{pmatrix}
\end{multline*}

On en déduit $\ker(\mathstrut^t\!AA) \subset \ker(A)$.\newline
L'inclusion réciproque étant évidente, les noyaux sont égaux. On conclut en utilisant la version matricielle du théorème du rang: le rang d'une matrice est égale à son nombre de colonne mois la dimension de son noyau. 
