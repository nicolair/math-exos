\begin{tiny}(Cee17)\end{tiny}
\begin{enumerate}
 \item Il s'agit d'un calcul classique à savoir faire très rapidement. En linéarisant, on trouve que l'intégrale est nulle pour $i\neq j$ et égale à $\frac{\pi}{2}$ si $i=j$. On en déduit que la famille est orthogonale pour le produit scalaire habituel définie par l'intégrale du produit. Comme elle est constituée de fonctions non nulle, cette famille est libre.
 \item \'Ecrivons chaque $\sin$ avec une exponentielle.
\begin{displaymath}
 \sin(kt)=\frac{1}{2i}\left((e^{it})^k - (e^{it})^{-k}\right) 
\end{displaymath}
En factorisant par $e^{ikt}$, on peut écrire la somme de $\sin$ sous la forme
\begin{displaymath}
 e^{ikt}P(e^{it})
\end{displaymath}
où $P$ est un polynôme à coefficients complexes (et s'exprimant simplement en fonction des $\lambda_k$) de degré $2p$. Lorsque ce polynôme admet strictement plus de $2p$ racines, tous ses coefficients sont nuls donc les $\lambda_k$ sont tous nuls.\newline
On en déduit que la famille des restrictions à un intervalle non réduit à un point est libre. En effet un tel intervalle contient une infinité de points.
\end{enumerate}
 