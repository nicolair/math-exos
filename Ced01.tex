\begin{tiny}(Ced01)\end{tiny} Solutions des équations différentielles proposées. Le paramètre $\lambda$ est réel sauf pour $(6)$ et $(7)$ où il est complexe.
\begin{multline*}
 (1)\hspace{0.5cm} -\frac{3}{8}\sin t -\frac{3}{8}\cos t + \frac{1}{40}\sin (3t)+\frac{3}{40}\cos(3t)\\ + \lambda e^{t}
\end{multline*}

\begin{align*}
&(2)&  &e^{t-t^2} + \lambda e^{-t^2}\\
&(3)&  &\frac{t}{2\sin t} - \frac{\cos t}{2} + \frac{\lambda}{\sin t}\\
&(4)& &x+\frac{\lambda}{\sqrt{t^2+1}} \\
&(5)& &\frac{t}{2}(\sin t - \cos t)+\frac{1}{2}\cos t + \lambda e^{-t}\\
&(6)& &\frac{1}{2}\sh(t)+\frac{i}{2}\ch(t) + \lambda e^{it} \\
&(7)& &\frac{t}{2}e^{it}+\frac{i}{4}e^{-it}+ \lambda e^{it} \\
&(8)& &te^{-t} -\frac{1}{2}\cos t + \frac{1}{2}\sin t + \lambda e^{-t}\\
&(9)& &\frac{1}{2} + \frac{1}{10}\cos (2t) + \frac{1}{5}\sin(2t) + \lambda e^{-t}
\end{align*}
Remarques sur les méthodes.\newline
Pour l'équation $(1)$, il faut commencer par linéariser le second membre. Deux méthodes sont possibles.\newline
 Première méthode : utiliser $\sin^2 t = \frac{1}{2}-\frac{1}{2}\cos(2t)$ puis $\sin a \cos b = \frac{1}{2}\left(\sin(a+b)+\sin(a-b) \right)$.\newline
Deuxième méthode : utiliser la formule du binôme
\begin{displaymath}
 \sin ^3 t = \frac{1}{(2i)^3}\left(e^{it}-e^{-it} \right)^3 
\end{displaymath}
Par les deux méthodes, on trouve
\begin{displaymath}
 \sin ^3 t = -\frac{1}{4}\sin(3t) + \frac{3}{4}\sin t
\end{displaymath}
  


