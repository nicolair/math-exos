\begin{tiny}(Ecg05)\end{tiny}
Norme $N_p$.\newline
Soit $p \geq 1$, on veut montrer que
\[N_p(x,y)=(|x|^p + |y|^p)^{\frac{1}{p}}\]
définit une norme sur $\R^2$. Le nombre $q$ est défini par :
\[\frac{1}{p}+\frac{1}{q}=1\]
\begin{enumerate}
\item Calculer une jolie expression pour la dérivée seconde dans $]0,1[$ de
\[f_p(x)=(1-x^p)^{\frac{1}{p}}\]
Tracer les graphes pour $p=1$, $p=1.5$, $p=2$, $p=4.5$.
\item Montrer que $\Omega_p$ est convexe avec
\[(x,y) \in \Omega_p \Leftrightarrow |x|^p+|y|^p \leq 1\]
\item Montrer que $N_p$ est une norme (utiliser l'exercice Ecg04)
\end{enumerate}