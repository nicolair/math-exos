\begin{tiny}(Cgs13)\end{tiny} Considérons successivement
\[
  k_1=\theta^{-1}(1), k_2=\theta^{-1}(2), \cdots , k_i=\theta^{-1}(i), \cdots
\]
Ils forment une énumération de $\llbracket 1,n \rrbracket$ car $\theta$ est bijective.
\begin{multline*}
  \sigma(k_1) \leq \theta(k_1)= 1 \Rightarrow \sigma(k_1) = \theta(k_1)= 1 \\
  \sigma(k_2) \leq \theta(k_2)= 2 \Rightarrow \sigma(k_2) = \theta(k_2)= 2
\end{multline*}

car $\sigma(k_2) \neq \sigma(k_1)$.\newline
On continue par récurrence.\newline
Si $\sigma$ et $\theta$ coincident sur $\left\lbrace k_1, \cdots, k_{i-1} \right\rbrace$ alors 
\[
  \sigma(k_i) \notin \left\lbrace \sigma(k_1), \cdots, \sigma(k_{i-1})\right\rbrace = \llbracket 1, i-1 \rrbracket 
\]
donc $\sigma(k_i)\leq \theta(k_i)=i$ entraine $\sigma(k_i) = i$.
