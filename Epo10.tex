\begin{tiny}(Epo10)\end{tiny} \'Etant donn{\'e} $p \in \N^*$, on pose
\begin{displaymath}
\binom{X}{0}=1
\text{ et }
\binom{X}{p}=\frac{1}{p!}X(X-1)\cdots (X-p+1) 
\end{displaymath}
On définit une application $\Delta$ de $\K[X]$ dans  $\K[X]$ par
\begin{displaymath}
\forall P \in \K[X],\; \Delta(P) = \widehat{P}(X+1) - P
\end{displaymath}
\begin{enumerate}
\item Montrer que la fonction polyn{\^o}miale associ{\'e}e {\`a} $\binom{X}{p}$ ne prend sur $\Z$ que des valeurs enti{\`e}res.

\item  Montrer que $\deg(\Delta(P))=\deg(P)-1$. Préciser le coefficient dominant de $\Delta(P)$. Calculer $\Delta (\binom{X}{p})$.

\item Exprimer $\Delta^n(P) = \underset{ n \text{ fois}}{\underbrace{\Delta \circ \cdots \circ \Delta}} (P)$ en vous inspirant de la formule du binôme.

\item Soit $P$ de degré au plus $n-1$. Montrer que
\begin{displaymath}
 \sum_{k=0}^n(-1)^k\binom{n}{k}\widetilde{P}(k) = 0 .
\end{displaymath}

\item Soit $P$ de degré $n$. Exprimer $P$ en fonction des $\Delta^k(P)$ et des $\binom{X}{p}$ en vous inspirant de la formule de Taylor.\newline 
D{\'e}terminer tous les polyn{\^o}mes {\`a} coefficients r{\'e}els tels que les fonctions associ{\'e}es ne prennent sur
$\Z$ que des valeurs enti{\`e}res.
\end{enumerate}
