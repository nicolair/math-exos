\begin{tiny}(Cmf06)\end{tiny}
\begin{enumerate}
 \item On cherche les solutions sous la forme
\[
 X =
\begin{pmatrix}
 x & y \\ z &t
\end{pmatrix} \; \Rightarrow \;
 X^2 =
\begin{pmatrix}
 x^2 + yz & xy + yt \\ zx + tz & zy + t^2
\end{pmatrix}.
\]
\[
 X ^2 = \begin{bmatrix}
 9 & 0 \\
 4 & 9
\end{bmatrix}
\Leftrightarrow
\left\lbrace 
\begin{aligned}
x^2 + yz &= 9 \\
y(x+t) &= 0\\
(x+t)z &= 4 \\
zy + t^2 &= 9
\end{aligned}
\right. 
\Leftrightarrow
\left\lbrace 
\begin{aligned}
x^2  &= 9 \\
y &= 0\\
(x+t)z &= 4 \\
t^2 &= 9
\end{aligned}
\right. 
\]
Comme $x+t \neq 0$, il y a seulement deux solutions
\[
 \begin{pmatrix}
  3 & 0 \\ 3 & \frac{2}{3}
 \end{pmatrix}, \hspace{0.5cm}
 - \begin{pmatrix}
  3 & 0 \\ 3 & \frac{2}{3}
 \end{pmatrix}.
\]
\[
 X ^2 = \begin{bmatrix}
 \frac{1}{4} & 0 \\
 1 & \frac{9}{4}
\end{bmatrix}
\Leftrightarrow
\left\lbrace 
\begin{aligned}
x^2 + yz &= \frac{1}{4} \\
y(x+t) &= 0\\
(x+t)z &= 1 \\
zy + t^2 &= \frac{9}{4}
\end{aligned}
\right. 
\Leftrightarrow
\left\lbrace 
\begin{aligned}
x^2  &= \frac{1}{4} \\
y &= 0\\
(x+t)z &= 1 \\
t^2 &= \frac{9}{4}
\end{aligned}
\right. 
\]
Cette fois il y a 4 solutions car $|x| \neq |t|$, 
\[
X_1 = \frac{1}{2}\begin{pmatrix}
  1 & 0 \\ 1 & 3
 \end{pmatrix}, 
X_2 = \frac{1}{2}\begin{pmatrix}
  1 & 0 \\ -2 & -3
 \end{pmatrix}, 
- X_1,  -X_2.
\]

 \item Inversons les expressions
\begin{multline*}
 \left\lbrace 
 \begin{aligned}
  v_1 &= e_1 \\ v_2 &= e_1 + e_2
 \end{aligned}
 \right. 
\Leftrightarrow
 \left\lbrace 
 \begin{aligned}
  e_1 &= v_1 \\ e_2 &= - v_1 + v_2
 \end{aligned}
 \right. \\
 \Leftrightarrow P_{\mathcal{E} \mathcal{V}} = \begin{pmatrix}
                                                1 & 1 \\ 0 & 1
                                               \end{pmatrix} 
\Leftrightarrow
P_{\mathcal{V} \mathcal{E} } = \begin{pmatrix}
                                                1 & -1 \\ 0 & 1
                                               \end{pmatrix}.
\end{multline*}

N'utilisons pas la formule de changement de base mais exprimons les images des vecteurs de $\mathcal{V}$ qui se lisent sur la matrice de $f$ dans $\mathcal{E}$:
\begin{multline*}
 f(e_1) = f(e_2)= e_1 + e_2 = v_2 \\
 \Rightarrow
 \left\lbrace 
 \begin{aligned}
  f(v_1) &= f(e_1) = v_2\\ f(v_2) &= f(e_1) + f(e_2) = 2v_2
 \end{aligned}
\right. \\
\Rightarrow
\MatB{V}{f}=
\begin{pmatrix}
 0 & 0 \\ 1 & 2
\end{pmatrix}.
\end{multline*}

 \item Suivons l'indication de l'énoncé en ajoutant $\frac{1}{4}I_2$:
\begin{multline*}
  X^2 + X = \begin{pmatrix} 1 & 1 \\ 1 & 1 \end{pmatrix}
  \Leftrightarrow
  X^2 + X + \frac{1}{4}I_2 = \begin{pmatrix} \frac{5}{4} & 1 \\ 1 & \frac{5}{4} \end{pmatrix}\\
  \Leftrightarrow
  (X - \frac{1}{2}I_2)^2 = \begin{pmatrix} \frac{5}{4} & 1 \\ 1 & \frac{5}{4} \end{pmatrix}
\end{multline*}

Avec la formule de changement de base, le résultat de la question b. s'écrit
\[
\MatB{V}{f}=
\begin{pmatrix} 0 & 0 \\ 1 & 2 \end{pmatrix}
 = P_{\mathcal{V}\mathcal{E}} \underset{= \MatB{E}{f}}{\underbrace{\begin{pmatrix} 1 & 1 \\ 1 & 1 \end{pmatrix}}}P_{\mathcal{E}\mathcal{V}} 
\]
Multiplions à gauche par $P_{\mathcal{V}\mathcal{E}}$ et à droite par $P_{\mathcal{E}\mathcal{V}}$. La relation devient
\[
  Y^2 = \begin{pmatrix} 0 & 0 \\ 1 & 2 \end{pmatrix} + \frac{1}{4} I_2 = \frac{1}{4}\begin{pmatrix} 1 & 0 \\ 4 & 9 \end{pmatrix}  
\]
avec $Y = P_{\mathcal{V}\mathcal{E}} X P_{\mathcal{E}\mathcal{V}} - \frac{1}{4}I_2$.
\end{enumerate}
On retombe sur la deuxième matrice de a. L'équation proposée ici admet donc 4 matrices solutions
\[
  \forall i \in \llbracket 1,4 \rrbracket, \hspace{0.2cm} P_{\mathcal{E}\mathcal{V}}\, X_i\, P_{\mathcal{V}\mathcal{E}} + \frac{1}{4} I_2
\]
Après calculs ces solutions sont
\[
  \frac{1}{2}\begin{pmatrix} 1 & 1 \\ 1 & 1 \end{pmatrix},\hspace{0.2cm}
  -\begin{pmatrix} 1 & 1 \\ 1 & 1 \end{pmatrix},\hspace{0.2cm}
  \begin{pmatrix} 0 & 1 \\ 1 & 0 \end{pmatrix},\hspace{0.2cm}
  -\frac{1}{2}\begin{pmatrix} 3 & 1 \\ 1 & 3 \end{pmatrix}.
\]
