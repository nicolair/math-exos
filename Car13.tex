\begin{tiny}(Car13)\end{tiny} Supposons qu'un tel polynôme $P$ existe. Il est de degré $3$ et unitaire donc $P(X+1)-P$ est de degré $2$ et de coefficient dominant $3$. De plus il est divisible par $X-2$ et $X-3$. On en déduit
\begin{displaymath}
 P(X+1)-P = 3(X-2)(X-3)=3X^2-15X+18
\end{displaymath}
On utilise alors des coefficients indéterminés. Si
\begin{displaymath}
 P = X^3 +aX^2+bX+c
\end{displaymath}
En identifiant les coefficients, l'expression de $P(X+1)-P$ conduit à:
\begin{displaymath}
 \left\lbrace 
\begin{aligned}
 3+2a=-15\\a+b+1=18
\end{aligned}
\right. 
\Leftrightarrow
 \left\lbrace 
\begin{aligned}
 a=-9\\a+b+1=26
\end{aligned}
\right. 
\end{displaymath}
Donc $P$ est de la forme
\begin{displaymath}
 P=X^3-9X^2+26X+c
\end{displaymath}
Le $c$ est déterminé par $P(1)=0$. On trouve $c=-18$.
Le seul polynôme possible est donc
\begin{displaymath}
 P=X^3-9X^2+26X+-18
\end{displaymath}
Il est effectivement divible par $X-1$ et prend la valeur $24$ en $2$, $3$, $4$.