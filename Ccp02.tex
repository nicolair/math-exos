\begin{tiny}(Ccp02)\end{tiny} Utilisons l'indication en supposant démontrée l'implication suggérée.
\begin{displaymath}
 |z|>1 \Rightarrow |z|<|p|+|q|   
\end{displaymath}
On en déduit
\begin{multline*}
|z|>1 \Rightarrow 1<|z|<|p|+|q|\\
 \Rightarrow \max(1,|p|+|q|)=|p|+|q|
\end{multline*}
Dans ce cas on a bien $|z|< \max(1,|p|+|q|)$.\newline
Dans l'autre cas, $|z|\leq 1$. On a encore 
\begin{displaymath}
 |z|\leq \max(1,|p|+|q|)
\end{displaymath}
Ce qui prouve l'inégalité générale demandée.\newline
Il reste à montrer l'inégalité indiquée. Lorsque $z\neq 0$, on peut diviser par $z$ la relation qu'il vérifie:
\begin{displaymath}
 |z|>1 \Rightarrow z^2 = -p+\frac{q}{z}
\Rightarrow |z|^2 \leq |p|+|q|
\end{displaymath}
en utilisant l'inégalité triangulaire et le fait que $|z|>1$.
On termine par :
\begin{displaymath}
 |z|>1\Rightarrow |z|<|z|^2\leq |p|+|q|
\end{displaymath}

