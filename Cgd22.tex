\begin{tiny}(Cgd22)\end{tiny} Notons $\varphi$ la fonction auxiliaire proposée par l'énoncé. Comme $\varphi(a)=0$, on choisit $K$ pour que $\varphi(b)=0$ ce qui permet d'utiliser le théorème de Rolle ($\varphi$ est $\mathcal{C}^1$) entre $a$ et $b$. Il existe $c\in \left] a,b \right[$ tel que $\varphi'(c)=0$ avec 
\begin{multline*}
 \varphi'(x) = -(a-x)f''(x) + 2K(a-x) \\
 =(a-x)\left( -f''(x) + 2K\right) 
\end{multline*}
On en déduit $K=\frac{f''(c)}{2}$.