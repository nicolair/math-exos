\begin{tiny}(Cdi01)\end{tiny} 
On vérifie facilement que $\Phi$ est linéaire et que son image est $A+B$. Son noyau est formé par les couples $(a,b)\in A\times B$ tels que 
\begin{multline*}
  a+b = 0_E \Rightarrow a = -b \in A\cap B\hspace{0.5cm} \\ \text{ d'où }\hspace{0.5cm} \ker \Phi = \left\lbrace (x,-x), x\in A\cap B \right\rbrace .
\end{multline*}
L'application de $A\cap B$ dans $\ker \Phi$ qui à $x$ associe $(x,-x)$ est clairement un isomorphisme. On en déduit l'égalité des dimensions. Par le théorème du rang:
\begin{multline*}
\dim(A\times B) = \dim \ker \Phi + \rg \Phi \\ 
\Leftrightarrow \dim A + \dim B = \dim(A\cap B) + \dim(A+B).
\end{multline*}

