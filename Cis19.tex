\begin{tiny}(Cis19)\end{tiny}
\begin{enumerate}
 \item Vérification facile. L'intégrale se calcule.
 \item On décompose arbitrairement l'intégrale avec un $a>0$ par relation de Chasles et on la majore par des inégalités de la moyenne. On obtient
\begin{displaymath}
 |K_n(f)|\leq \sup_{[0,a]}|f| + M_fk_n(a)
\end{displaymath}
où $M_f= \sup_{[0,1]}|f|$.\newline
Pour tout $\varepsilon >0$, comme $f\rightarrow 0$ en $0$, on choisit un $a>0$ assez petit pour que $\sup_{[0,a]}|f|<\frac{\varepsilon}{2}$. Ce $a$ étant fixé, comme la suite $\left( k_n(a)\right) _{n\in \N}\rightarrow 0$, il existe un $N$ tel que $M_fk_n(a)<\frac{\varepsilon}{2}$ dès que $n\geq N$. On en déduit la convergence vers $0$ demandée.
\item Dans le cas général,
\begin{displaymath}
 \left( K_n(f)\right) _{n\in \N} \rightarrow f(0)
\end{displaymath}
En effet, par hypothèse et question b.,
\begin{displaymath}
 K_n(f) = f(0)\underset{\rightarrow 1}{\underbrace{K_n(1)}} 
+ \underset{\rightarrow 0}{\underbrace{K_n(f-f(0))}}
\end{displaymath}

\end{enumerate}
