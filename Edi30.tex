\begin{tiny}(Edi30)\end{tiny} Dans $E = \R^5$, on se donne les vecteurs suivants
\[
 \begin{aligned}
  u_1 &= (1,2,-1,-1,0)\\
  u_2 &= (1,1,1,0,-1)\\
  u_3 &= (-2,1,0,1,2)\\
  a_1 &= (1,1,-1,0,1)\\
  a_2 &= (0,1,1,-1,0)\\
  a_3 &= (1,0,1,1,-1)\\
  a_4 &= (2,2,1,0,0)
 \end{aligned}
\]
et $U= \Vect(u_1,u_2,u_3)$, $A= \Vect(a_1,a_2,a_3,a_4)$.\newline
Exprimer $U$ et $A$ comme des intersections d'hyperplans. Les formes linéaires définissant ces hyperplans seront données dans la base duale de la base canonique notée $(\varepsilon_1,\varepsilon_2,\varepsilon_3,\varepsilon_4,\varepsilon_5)$.