\begin{tiny}Cfu14\end{tiny}
\begin{multline*}
  \left.
  \begin{aligned}
    e^a &= \ch a + \sh a \\ e^b &= \ch b + \sh b
  \end{aligned}
\right\rbrace 
\Rightarrow
e^{a + b}
= (\ch a \ch b + \sh a \sh b) \\ + (\ch a \sh b + \sh a \ch b) 
\end{multline*}
En multipliant $a$ et $b$ par $-1$, la première parenthèse est conservée et la deuxième multipliée par -1. On en déduit
\[
\begin{aligned}
\ch (a+b) &= \ch a \ch b + \sh a \sh b\\
\sh (a+b) &= \ch a \sh b + \sh a \ch b
\end{aligned}
\]
\begin{multline*}
\left.
  \begin{aligned}
  e^a + e^b &= e^{\frac{a+b}{2}}\,2\ch \frac{a-b}{2} \\
  e^{-a} + e^{-b} &= e^{-\frac{a+b}{2}}\,2\ch \frac{a-b}{2} 
  \end{aligned}
\right\rbrace \\
  \Rightarrow
\left\lbrace
\begin{aligned}
  \ch a + \ch b &= 2\ch \frac{a-b}{2} \ch \frac{a+b}{2} \\ \sh a + \sh b &= 2\ch \frac{a-b}{2} \sh \frac{a+b}{2}
\end{aligned}
\right.
\end{multline*}
\begin{multline*}
\left.
  \begin{aligned}
  e^a - e^b &= e^{\frac{a+b}{2}} \,2\sh \frac{a-b}{2} \\
  e^{-a} - e^{-b} &= e^{-\frac{a+b}{2}} \,2\sh \frac{a-b}{2} 
  \end{aligned}
\right\rbrace \\
  \Rightarrow
  \ch a - \ch b = 2\sh \frac{a-b}{2} \sh \frac{a+b}{2} .
\end{multline*}
On utilise la formule du binôme et on sépare les parties paires et impaires
\begin{multline*}
  e^{3x} = (\ch x + \sh x)^3\\
  = {\ch}^3x + 3 {\ch}^2x \sh x  + 3 \ch x {\sh}^2 x + {\sh}^3 x \\
  \Rightarrow
  \left\lbrace
  \begin{aligned}
    \ch(3x) &= {\ch}^3x  + 3 \ch x {\sh}^2 x \\
    \sh(3x) &= 3 {\ch}^2 x \sh x  + {\sh}^3 x
  \end{aligned}
\right. .
\end{multline*}
