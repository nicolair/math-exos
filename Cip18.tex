\begin{tiny}(Cip18)\end{tiny} 
\begin{enumerate}
 \item On obtient $I(-x) = I(x)$ avec le changement de variable $u = \pi -t$ dans l'intégrale exprimant $I(-x)$.\newline
 On obtient 
 \[
 I(\frac{1}{x}) = I(x) -\pi\ln(x^2) = I(x) -2\pi \ln|x| 
 \]
en réduisant au même dénominateur et développant le $\ln$. Pour $I(x^2)$, on factorise d'abord
\begin{multline*}
 \left| x^2 - e^{it}\right| = \left| x - e^{i\frac{t}{2}}\right| \left| x + e^{i\frac{t}{2}}\right|\\
 \Rightarrow I(x^2)
 = \int_{0}^{\pi }\ln (x^{2} - 2x\cos\frac{t}{2} +1)dt \\ 
 + \int_{0}^{\pi }\ln (x^{2}+ 2x\cos\frac{t}{2} + 1)dt.
\end{multline*}
En posant $u=\frac{t}{2}$ dans la première intégrale et $u = \pi - \frac{t}{2}$ dans la seconde, on obtient
\[
 I(x^2) = 2I(x).
\]

 \item Voir \href{\baseurl temptex/fexcu.pdf}{exercice cu31}.

 \item
 Soit $|x|<1$. Posons $a = |x|$ et $x_0 = x$ et 
\[
 \forall n \in \N, \; x_{n+1} = x_n^2. 
\]
Alors:
\[
\forall n \in \N,\; x_n\in \left[ -a,a\right],\;
I(x_n) = 2^nI(x).
\]
Or, d'après b.
\[
\forall n \in \N,\; |I_n| \leq -2\pi \ln(1-a).
\]
La suite géométrique (de raison 2) $I(x_n)$ est bornée donc elle est nulle c'est à dire $I(x)=0$. Pour $|x|>1$,
\[
 I(x) = \underset{=0}{\underbrace{I(\frac{1}{x})}} + 2\pi\ln(x) = 2\pi\ln(x).
\]

\end{enumerate}

