\begin{tiny}(Cmm18)\end{tiny} Pour toute $A \in \mathcal{M}_p(\K)$, considérerons
\[
\varphi_A : \;
 \left\lbrace 
 \begin{aligned}
  \mathcal{M}_p(\K) &\rightarrow \K \\
  M &\mapsto \tr(AM)
 \end{aligned}
\right. .
\]
Il est clair que $\varphi_A$ est une forme linéaire. Il s'agit de montrer que toute forme linéaire sur les matrices est de la forme $\varphi_A$ pour une unique matrice $A$. Introduisons
\[ \Phi:\;
 \left\lbrace 
 \begin{aligned}
  \mathcal{M}_p(\K) &\rightarrow \mathcal{M}_p(\K)^{*} \\
  A &\mapsto \varphi_A
 \end{aligned}
\right. .
\]
Il s'agit de montrer que $\Phi$ est bijective. Comme elle est clairement linéaire entre deux espaces de même dimension $p^2$, il suffit de montrer qu'elle est injective.\newline
Tout $A \in \ker \Phi$ est nul car $0 = \tr(AE_{i j}) = a_{j i } = 0$ pour toutes les matrices élémentaires.
