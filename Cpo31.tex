\begin{tiny}(Cpo31)\end{tiny} Si un polynôme réel de degré $n$ est à valeurs réelles strictement positives, il s'exprime comme un produit de polynômes du second degré sans racine réelles
\begin{displaymath}
  X^2 -2 \Re z X + |z|^2 = (X-\Re z)^2 + (\Im z)^2
\end{displaymath}
Chacun de ces polynômes est donc une somme de deux carrés. Or le produit de deux polynômes sommes de 2 carrés est lui même une somme de deux carrés à cause de l'identité
\begin{displaymath}
  (A^2 + B^2)(U^2 + V^2) = (AU - BV)^2 + (AV + BU)^2
\end{displaymath}
On remarque l'analogie avec le carré du module du produit de deux nombres complexes.