\begin{tiny}(Cee20)\end{tiny} Le déterminant de $S$ se calcule facilement avec la méthode du pivot standard. On se ramène à une matrice triangulaire supérieur avec des $1$ sur la diagonale sauf pour le terme $p,p$. On obtient
\begin{displaymath}
 \det S = c -a_1^2 - a_2^2 - \cdots - a_{p-1}^2
\end{displaymath}
Il reste à vérifier que $\beta(x,x)\geq 0$ et que $\beta(x,x)=0$ entraine $x=0$ pour tout vecteur $x$ de $E$.\newline
Considérons donc un vecteur $x$ quelconque de coordonnées $(x-1,\cdots,x_p)$. On calcule en utilisant d'abord l'expression matricielle puis en faisant apparaitre des carrés
\begin{multline*}
 \beta(x,x)\\
=
\begin{pmatrix}
 x_1 & \cdots & x_p
\end{pmatrix}
\begin{pmatrix}
 x_1+a_1 x_p \\ x_2+a_2 x_p \\ \vdots \\ x_{p-1}+a_{p-1} x_p \\
a_1x_1+\cdots+a_{p-1}x_{p-1} +cx_p  
\end{pmatrix}
\\
=x_1^2+\cdots + x_{p-1}^2 + cx_p^2 +
2\left( a_1x_1+\cdots+a_{p-1}x_{p-1} \right) x_p\\
=\left((x_1 + a_1x_p)^2-a_1^2x_p^2 \right) + \cdots  \\+ \left((x_{p-1} + a_{p-1}x_p)^2-a_{p-1}^2x_p^2 \right) + cx_p^2\\
= (x_1 + a_1x_p)^2 + \cdots +(x_{p-1} + a_{p-1}x_p)^2+ (\det S )x_p^2
\end{multline*}
On en déduit les propriétés demandées.  