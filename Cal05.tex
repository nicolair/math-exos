\begin{tiny}(Cal05)\end{tiny}
\begin{enumerate}
 \item L'application $x\rightarrow gx$ est injective car on peut multiplier à gauche par $g^{-1}$. On en déduit qu'elle est bijective car c'est une application d'un ensemble \emph{fini} dans lui même.
 \item D'après la première question $P$ est aussi le produit des $gx$ donc
\begin{displaymath}
 P = \prod_{x\in G}(gx)=g^m P\Rightarrow g^m = e
\end{displaymath}
en multipliant par $P^{-1}$.
\end{enumerate}