\begin{tiny}(Ccu29)\end{tiny} En calculant le module, le calcul de $S_n$ se ramène à une somme de $\sin$ associée à une progression géométrique. On trouve
\begin{displaymath}
 S_n = 2n
\end{displaymath}
Pour calculer $T_n$, on développe les termes de la somme suivant la formule du binôme puis on permute les sommations. Seuls deux termes subsistent. On obtient
\begin{displaymath}
 T_n = n(z^n +1)
\end{displaymath}
