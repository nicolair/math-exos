\begin{tiny}(Eva12)\end{tiny} Loi faible des grands nombres.\newline
On considère une suite $\left( X_n\right) _{n\in \N^*}$ de variables aléatoires définies sur le même espace probabilisé fini $(\Omega, \p)$. On suppose qu'elles sont deux à deux indépendantes avec la même espérance $e$ et la même variance $v$.\newline
On introduit les variables moyennes: pour $n\in \N^*$,
\begin{displaymath}
 \overline{X}_n = \frac{1}{n}(X_1+\cdots+X_n)
\end{displaymath}
Montrer que $E(\overline{X}_n)=e$ et que, pour tout $\varepsilon >0$,
\begin{displaymath}
 \left( \p(\left|\overline{X}_n-e\right|\geq \varepsilon) \right) _{n\in \N} \rightarrow 0
\end{displaymath}
  