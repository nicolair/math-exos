\begin{tiny}(Cev20)\end{tiny}
\begin{enumerate}
 \item L'intersection entre le sous-espace des suites qui convergent vers $0$ et celui des suites constantes est la suite nulle. Toute suite convergente est la somme de la suite constante égale à sa limite et d'une suite qui converge vers $0$.
 \item Si $f\in F \cap G$, il existe $a$ et $b$ tels que $f(x) = ax + b$ donc $a=f(0)=0$ et $b=f'(0)=0$ car $f\in G$. L'intersection se réduit au vecteur nul. La deuxième propriété est assurée par une décomposition idiote
\begin{multline*}
 f(x) = f(0) + f'(0)x + r(x) \\ 
 \text{ avec } r(x) = f(x) -f(0) - f'(0)x.
\end{multline*}

Il est immédiat que $\left( x\rightarrow f(0) + xf'(0)\right)\in F$ et $r\in G$. 
\end{enumerate}
