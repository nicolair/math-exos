\begin{tiny}(Eed16)\end{tiny}
\begin{enumerate}
  \item Montrer que la dérivée d'une fonction paire est impaire et que la dérivée d'une fonction impaire est paire.
  \item Soit $f$ une fonction de $\R$ dans $\C$. Montrer qu'elle se décompose de manière unique en
\[
  f = a + b + ic + id
\]
où $a$, $b$, $c$, $d$ sont à valeurs réelles avec $a$, $c$ paires et $b$, $d$ impaires. On conviendra de les appeler respectivement les parties PR, IR, PI, II de $f$.
 \item Préciser les parties PR, IR, PI, II de $t\rightarrow e^{(1+i)t}$ et de $t \rightarrow \frac{1}{1+i}e^{(1+i)t}$.
 \item En déduire une primitive de $t \rightarrow \sin t \ch t$.
\end{enumerate}

