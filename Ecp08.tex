\begin{tiny}(Ecp08)\end{tiny} \emph{Orthocentre}\newline
Dans cet exercice, on pourra consid{\'e}rer
\begin{displaymath}
 (d-a)(\overline{b}-\overline{c})+(d-b)(\overline{c}-\overline{a})+(d-c)(\overline{a}-\overline{b})
\end{displaymath}
Montrer que, si deux des trois complexes 
\begin{displaymath}
\frac{d-a}{b-c}, \;\frac{d-b}{c-a},\; \frac{d-c}{a-b} 
\end{displaymath}
sont imaginaires purs, le troisi{\`e}me l'est aussi. En d{\'e}duire que les trois hauteurs d'un triangle
se coupent.