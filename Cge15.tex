Dans chaque produit vectoriel le même vecteur figure deux fois donc un terme s'annule. Il reste un déterminant dont chaque vecteur est une somme de trois. En développant grace à la multilinéarité, on devrait donc obtenir 27 termes. En fait il n'en reste que 6 car les mêmes vecteurs se retrouvent souvent. Parmi ces 6 termes, par antisymétrie, tous sont égaux à 
\begin{displaymath}
 \det(\overrightarrow a \wedge \overrightarrow c ,\overrightarrow b \wedge \overrightarrow a,
      \overrightarrow c \wedge \overrightarrow b )
\end{displaymath}
sauf le dernier qui est l'opposé de ce déterminant. En utilisant le produit mixte et la formule du double produit vectoriel, on obtient finalement
\begin{displaymath}
 -4\det(\overrightarrow a,\overrightarrow b,\overrightarrow c)
\end{displaymath}
