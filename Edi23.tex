\begin{tiny}(Edi23)\end{tiny} (base usuelle de $\mathcal{L}(E,F)$: démonstration)\newline
Soit $(a_1,\cdots,a_p)$ une base d'un $\K$-espace vectoriel $E$ et $(\alpha_1, \cdots, \alpha_p)$ la famille des formes coordonnées dans cette base. Soit $(b_1,\cdots,b_q)$ une base d'un $\K$-espace vectoriel $F$ et $(\beta_1, \cdots, \beta_q)$ la famille des formes coordonnées dans cette base.\newline
Pour tout couple $(i,j)\in \{1, \cdots, p\}\times \{1, \cdots, q\}$, on définit $f_{i,j}$ par le théorème du prolongement linéaire avec
\begin{displaymath}
 \forall k\in \{1, \cdots, p\},\; 
f_{i,j}(a_k)=\delta_{ik}b_j
\end{displaymath}
Soit $\lambda_{i,j}$ avec $(i,j)\in \{1, \cdots, p\}\times \{1, \cdots, q\}$ une famille d'éléments de $\K$ et 
\begin{displaymath}
 l=\sum_{(i,j)\in \{1, \cdots, p\}\times \{1, \cdots, q\}}\lambda_{i,j}f_{i,j}
\end{displaymath}
Exprimer $\lambda_{i,j}$ en fonction des $a_i$, $\alpha_i$, $b_j$, $\beta_j$. Démontrer que les $f_{i,j}$ forment une base de $\mathcal{L}(E,F)$.