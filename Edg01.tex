\begin{tiny}(Edg01)\end{tiny}
Soit $U$ un ouvert de $\R^{2}$ (resp $\R^{3}$), une
application num{\'e}rique $f$ d{\'e}finie dans $U$ est dite \emph{harmonique} si et seulement si $\Delta f=0$ o{\`u}
\[
\Delta f=\dfrac{\partial ^{2}f}{\partial \,x^{2}}+\dfrac{\partial ^{2}f}{%
\partial \,y^{2}}\quad \text{(resp)\quad }\dfrac{\partial ^{2}f}{\partial
\,x^{2}}+\dfrac{\partial ^{2}f}{\partial \,y^{2}}+\dfrac{\partial ^{2}f}{%
\partial \,z^{2}}
\]
est le \emph{laplacien de }$f$.

\begin{enumerate}
\item  Pour $(x,y)\in \R^{2}$ et $z=x+iy$, montrer que la fonction $f$ définie par :
\begin{displaymath}
 \quad f((x,y))=\ln \left|e^{ze^{-z}}\right|
\end{displaymath}
est harmonique.
\item  Montrer que si $f$ est harmonique et de classe $\mathcal{C}^{3}$
alors $\dfrac{\partial f}{\partial \,x}$ et $y\dfrac{\partial f}{\partial \,x}%
-x\dfrac{\partial f}{\partial \,y}$ sont harmoniques.

\item  V{\'e}rifier que
\[
f((x,y,z))=\arctan \dfrac{y}{x}+\arctan \dfrac{z}{y}+\arctan \dfrac{x}{z}
\]
est harmonique sur $\R^{*3}$.
\end{enumerate}