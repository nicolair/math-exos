\begin{tiny}(Car06)\end{tiny} Les solutions sont les polynômes de la forme
\[
 -2^9 + (X+2)^3 A = 2^9 +(X-2)^3B
\]
avec 
\[
 2^{10} = (X+2)^3 A - (X-2)^3 B.
\]
Pour que la condition sur le degré soit vérifiée, il faut trouver des solutions de degré au plus $2$ de cette l'équation de Bezout. On pourrait le faire en utilisant l'algorithme d'Euclide étendu. Pour changer, proposons une solution à base de formule du binôme. Comme
\[
 \left( (X+2) - (X-2)\right)^5 = 4 ^5 = 2^{10}, 
\]
on peut former $A$ avec les termes en $(X+2)^k$ pour $k\in \llbracket 3,5\rrbracket$ et $B$ avec ceux pour $k\in \llbracket 0,2\rrbracket$. On en déduit
\begin{multline*}
 A = (X+2)^2 - 5(X+2)(X-2 + 10(X-2)^2 \\
 = 6X^2 - 38X + 24\\
 B = 10(X+2)^2 - 5(X+2)(X-2) + (X-2)^2\\
 = 6X^2 + 38X +24.
\end{multline*}

