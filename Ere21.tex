\begin{tiny}(Ere21)\end{tiny} On rappelle que la partie fractionnaire d'un réel $x$ est $x -\lfloor x \rfloor$. Elle est souvent notée $\{x\}$.\newline
Soit $n$ entier naturel, montrer que $(\sqrt{3}-1)^{2n+1}$ est la partie fractionnaire de $(\sqrt{3}+1)^{2n+1}$.\newline
Soit $d$ un nombre naturel qui n'est pas le carré d'un nombre naturel, soit $a$ et $b$ des naturels non nuls. Sous quelle condition $(a-b\sqrt{d})^{2n+1}$ est-il la partie fractionnaire de $(a+b\sqrt{d})^{2n+1}$ pour $n\in \N^*$?