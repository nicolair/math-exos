\begin{tiny}(Ecu10)\end{tiny}
Soit $a$ et $b$ deux nombres r{\'e}els tels que $0<a<b$. On consid{\`e}re les équations d'inconnue r{\'e}elle $x$
\begin{eqnarray*}
\sqrt{x} = \sqrt{a-x}+\sqrt{b-x}  &(1)\\
3x-(a+b) = 2\sqrt{(a-x)(b-x)}  &(2)\\
5x^2 -2(a+b)x+(a-b)^2 = 0 &(3)
\end{eqnarray*}

\begin{enumerate}
\item Comparer les ensembles de solution de $(1)$, $(2)$, $(3)$.

\item  {\'E}tudier la fonction
\begin{displaymath}
x\rightarrow \sqrt{x}-\sqrt{a-x}-\sqrt{b-x}  
\end{displaymath}

\item  On fixe $b$. Discuter, suivant les valeurs de $a$, de l'existence et du nombre de solutions de $(1)$.
\end{enumerate}