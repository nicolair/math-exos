\begin{tiny}(Csn07)\end{tiny} Par définition:
\[
 d_1 = \underset{q \text{ nbs impairs}}{\underbrace{1 + \frac{1}{3} + \cdots + \frac{1}{2q-1}}} 
 - \left(\underset{p\text{ nbs pairs}}{\underbrace{ \frac{1}{2} + \frac{1}{4} + \cdots + \frac{1}{2p}}}\right)  
\]
\begin{multline*}
 d_n = \underset{nq\text{ premiers impairs}}{\underbrace{i_{nq}}} 
 - \underset{nq\text{ premiers pairs}}{\underbrace{p_{nq}}}\\
 = h_{2nq} - p_{nq} - p_{np}
 = h_{2nq} - \frac{h_{nq} + h_{np}}{2}
\end{multline*}

Introduisons une suite $\varepsilon_n$ qui tend vers $0$ pour traduire la propriété admise par l'énoncé:
\[
 \sum_{k=1}^{n}\frac{1}{k} = \ln n + \gamma + \varepsilon_n.
\]
Les $\gamma$ disparaissent de l'expression de $d_n$:
\begin{multline*}
 d_n = \ln(2nq) - \frac{\ln(nq) + \ln(np)}{2} + \varepsilon_{2nq} - \frac{\varepsilon_{np} + \varepsilon_{nq}}{2}\\
 = \ln 2 + \ln \sqrt{\frac{q}{p}} + \underset{\rightarrow 0}{\underbrace{\varepsilon_{2nq} - \frac{\varepsilon_{np} + \varepsilon_{nq}}{2}}}\\
 \rightarrow \ln 2 + \ln \sqrt{\frac{q}{p}}.
\end{multline*}
