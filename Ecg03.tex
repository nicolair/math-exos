\begin{tiny}(Ecg03)\end{tiny}
\textbf{Démonstration du théorème de d'Alembert}.\newline
Soit $P$ un polynôme à coefficients complexes de degré plus grand que 1.
\begin{enumerate}
\item En utilisant le théorème de Bolzano-Weirstrass, montrer qu'il existe un nombre complexe $z_0$ tel que
\[|P(z_0)|=\min \{|P(z)|, z\in \C\}\]

\item Les propriétés de la fonction exponentielle complexe montrent que, pour tout $Z$ complexe non nul et tout entier $k$, il existe un $z$ tel que $z^k=Z$.\newline
Montrer que si $|P(z_0)|>0$ alors il existe un $z$ tel que $|P(z)|<|P(z_0)|$. \newline On pourra s'approcher de $z_0$ sur une droite bien choisie.
\item En déduire le théorème de d'Alembert.
\end{enumerate} 