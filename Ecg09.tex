\begin{tiny}(Ecg09)\end{tiny} On désigne par $x$ et $y$ les fonctions coordonnées dans un repère \repere  et on définit une fonction $f$ dans le plan privé de la première bissectrice notée $\mathcal D$ par 
\begin{displaymath}
 f=\frac{\sin x - \sin y}{x-y}
\end{displaymath}
En utilisant le théorème des accroissements finis entre $x(m)$ et $y(m)$ pour un point $m$ quelconque et la fonction $\sin$, montrer que $f$ admet une limite en un point $m_0\in \mathcal D$.