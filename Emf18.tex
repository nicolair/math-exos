\begin{tiny}(Emf18)\end{tiny} Endomorphismes de trace nulle.
\begin{enumerate}
 \item Soit $E$ un $\K$-espace vectoriel et $f \in \mathcal{L}(E)$ tel que,
\[
  \forall x\in E, x\neq 0_E,\; \exists \lambda(x)\in \K \text{ tel que } f(x)=\lambda(x)x.
\]
Montrer qu'il existe $\mu\in \K$ tel que 
\begin{displaymath}
 \forall x\in E,\; f(x)=\mu x
\end{displaymath}
\item Soit $E$ de dimension finie $n$ et $f$ un endomorphisme \emph{de trace nulle}. Montrer qu'il existe une base de $E$ dans laquelle la matrice de $f$ présente un $0$ en position $1,1$. Comment peut-on interpréter la matrice extraite associée aux indices $i$ et $j$ dans $\llbracket 2,n\rrbracket$.
\item Soit $E$ de dimension finie $n$ et $f$ un endomorphisme \emph{de trace nulle}. Montrer qu'il existe une base de $E$ dans laquelle la matrice de $f$ ne présente que des $0$ sur la diagonale.
\end{enumerate}
 
