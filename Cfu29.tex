\begin{tiny}(Cfu29)\end{tiny} Pour $x$ et $y$ fixés, considérons la fonction $f$
\begin{displaymath}
  z \mapsto |x+z| + |y+z|
\end{displaymath}
Pour fixer les idées supposons que $x < y$. Les valeurs absolues changent de signe en $-x$ et $-y$ (avec $-y < -x$). On en déduit que $f$ est constante entre $-y$ et $-x$ de valeur 
\begin{displaymath}
  x + z - y - z = x-y
\end{displaymath}
Dans le cas général, on a toujours un intervalle sur lequel la fonction est constante et de valeur
\begin{displaymath}
  \max(x,y) - \min(x,y) = |x-y|
\end{displaymath}
On a donc prouvé 
\begin{displaymath}
\forall (x,y,z)\in \R^3, \; |x-y| \leq | x+ z| + |y+z|
\end{displaymath}
On termine avec la relation de cours conséquence de l'inégalité triangulaire
\begin{displaymath}
\left| |x| - |y| \right| \leq |x-y|
\end{displaymath}
