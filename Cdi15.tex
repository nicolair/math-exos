\begin{tiny}(Cdi15)\end{tiny} La linéarité est facile:
\begin{multline*}
 \forall(z,z')\in \C^2, \forall \lambda \in \R,\\
 f(z+z') = z+z' + a\,(\overline{z + z'}) 
 = z+a\,\overline{z} + z'+a\,\overline{z'} \\
 = f(z) + f(z') \\
 f(\lambda z) = \lambda z + a\, \overline{\lambda z} 
 = \lambda (z + a\, \overline{z})
 = \lambda f(z).
\end{multline*}
Si $\ker f \neq \left\lbrace 0_\C\right\rbrace$, il existe $z\neq 0$ tel que
\[
 z+a\overline{z}= 0 \Rightarrow a = -\frac{z}{\overline{z}}
 \Rightarrow |a| = 1.
\]
On en déduit que si $a$ n'est pas de module $1$, $f$ est un automorphisme du $\R$-espace vectoriel $\C$.\newline
Si $a = e^{i \alpha}$, son image et son noyau sont des droites:
\[
 \ker f = \Vect(e^{i\frac{\alpha + \pi}{2}}), \hspace{0.5cm}
 \Im f = \Vect(e^{i\frac{\alpha}{2}}).
\]
