\begin{tiny}Eig04\end{tiny}
Calculer $\int_D f dxdy$ dans les cas suivants {\`a} l'aide d'un changement de variable.
    \begin{enumerate}
        \item $D$ est limit{\'e} par deux paraboles donn{\'e}es par les in{\'e}quations
            \[y^2-2px\leq 0\quad,\quad    x^2-2py \leq 0\\\]
            $f=\exp\frac{x^3+y^3}{xy}$, utiliser le changement de
            variable $x=u^2v,y=uv^2$.
        \item Soit $a$ et $b$ tels que $0<a<b$, le domaine $D$ est
        d{\'e}fini par les {\'e}quations
        \[a\leq xy \leq b,\quad 0\leq x \leq y,\quad y^2-x^2\leq 1\]
        $f=(y^2-x^2)(x^2+y^2)$, utiliser le changement de variable
        $u=xy,v=y^2-x^2$.\newline
Les fonctions $u$ et $v$ sont définies dans le premier quart de plan.
        \item $D$ est la partie du plan $y>0$ limit{\'e} par les quatre paraboles
        \begin{eqnarray*}
            P_1:y^2=4x+4,\quad P_2:y^2=2x+1\\
            P_3:y^2=9-6x,\quad P_4:y^2=-4x+4\\
        \end{eqnarray*}
        $f=\frac{y}{\sqrt{x^2+y^2}}$, utiliser le changement de variable $x=u^2-v^2,y=2uv$.
    \end{enumerate}
