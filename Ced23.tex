\begin{tiny}(Ced23)\end{tiny} Si $\lambda = 0$, les solutions de l'équation différentielle sont les fonctions affines. Une telle fonction peut s'annuler en 0 ou en 1 mais pas aux deux à la fois (sauf si elle est nulle).\newline
Soit $\lambda \neq 0$ et $\delta$ une racine carrée de $-\lambda$. Les solutions de l'équation différentielles (sans condition en $0$ et $1$) sont les fonctions
\[
 t \mapsto A e^{\delta t} + B e^{-\delta t} \hspace{0.5cm} (A,B)\in \C^2
\]
La condition s'écrit donc
\begin{displaymath}
 A + B = A e^{\delta} + Be^{-\delta} = 0 \Rightarrow 
 B = -A 
\end{displaymath}
Pour une fonction non nulle, $(A,B)\neq(0,0)$ donc
\[
 e^{\delta} = e^{-\delta}\Leftrightarrow e^{2\delta} = 1
 \Leftrightarrow \delta \in i\pi \Z
\]
la condition cherchée est donc que $\lambda$ est de la forme
\[
 (n\pi)^2, \; n\in \N^*.
\]
