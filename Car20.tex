\begin{tiny}(Car20)\end{tiny} Notons $A$ le polynôme donné par l'énoncé et $B$ celui obtenu en substituant $-X$ à $X$ dans $A$.\newline
Le polynôme $A$ admet deux racines opposées si et seulement si $A$ et $B$ ont une racine en commun. On utilise donc l'algorithme d'Euclide qui conduit aux polynômes
\begin{multline*}
 X^4-3X^3-12X^2+48X-64,\\
X^4+3X^3-12X^2-48X-64,\;\\
-6X^3+96X,\; -64+4X^2,\;0
\end{multline*}
On en déduit que les racines opposées sont $4$ et $-4$. Le polynôme se factorise en
\begin{displaymath}
 (X^2-16)(X^2-3X+4)
\end{displaymath}
