\begin{tiny}(Ccu03)\end{tiny} La fonction
\begin{displaymath}
  f : x \mapsto \sqrt{3-x}-\sqrt{x+1}>\frac{1}{2}
\end{displaymath}
est définie dans $I=[-1,3]$, décroissante dans cet intervalle, avec $f(-1)=2$ et $f(3)=-2$. L'ensemble des nombres pour lesquels cette fonction prend une valeur $>\frac{1}{2}$ est donc $[-1,a[$ où $a$ est la solution de $f(x)=\frac{1}{2}$.\newline
On peut obtenir une expression de $a$ avec des racines carrées.\newline
Pour $x\in I$, notons $u=\sqrt{3-x}$ et $v=\sqrt{x+1}$, alors:
\begin{displaymath}
  u^2 + v^2 = 4
\end{displaymath}
Si $f(x)=\frac{1}{2}$ alors
\begin{multline*}
  u = v + \frac{1}{2} \Rightarrow
  (v + \frac{1}{2})^2+v^2 = 4 \\
  \Rightarrow 2v^2 + v -\frac{15}{4} = 0
  \text{ racines }
  \frac{-1 \pm \sqrt{31}}{4}
\end{multline*}

Comme $5<\sqrt{31}$,
\begin{displaymath}
  \frac{-1-\sqrt{31}}{4}< -\frac{6}{4}=-\frac{3}{2}<-1
\end{displaymath}
l'ensemble des solutions est:
\begin{displaymath}
  \left[ -1, \frac{-1+\sqrt{31}}{4}\right[ 
\end{displaymath}
