\begin{tiny}(Egs05)\end{tiny} \label{Egs05} Conjugaison.\newline
On dit que deux permutations $\sigma$ et $\sigma'$ dans $\mathfrak{S}_n$ sont \emph{conjuguées} si et seulement si il existe une permutation $\theta$ telle que $\sigma' = \theta\circ \sigma \circ \theta^{-1}$.
\begin{enumerate}
 \item Soit $\theta \in \mathfrak{S}_n$ et $\begin{pmatrix} a_{1} & a_{2} & \cdots  & a_{k}\end{pmatrix}$ un cycle. Montrer que $\theta \circ 
\begin{pmatrix}
 a_{1} & a_{2} & \cdots & a_{k}
\end{pmatrix}
\circ \theta ^{-1}
$ 
est un cycle {\`a} pr{\'e}ciser.
\item \`A toute permutation, on peut associer de manière unique la suite rangée par ordre croissant des longueurs des cycles disjoints de sa décomposition. Montrer que deux permutations sont conjuguées si et seulement si elles ont la même suite de longueurs.
\item Montrer qu'une permutation et sa réciproque sont conjuguées.\newline
Cas particulier. Préciser $\theta$ tel que $\theta\circ \sigma \circ \theta^{-1} = \sigma^{-1}$ avec
\begin{displaymath}
 \sigma =
\begin{pmatrix}
 1 & 2 & 3 & 4 & 5 & 6 & 7 & 8 \\8 & 1 & 7 & 3 & 2 & 6 & 4 & 5 
\end{pmatrix}
\end{displaymath}  
\end{enumerate}
