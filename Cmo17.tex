\begin{tiny}(Cmo17)\end{tiny} Diagonalisation de 
\[
  \begin{pmatrix}
  -2 & -1 & 0 \\ -1 & 0 & 1 \\ 0 & 1 & 2
\end{pmatrix},
\]
à compléter\newline

Diagonalisation de
\[
\begin{pmatrix}
  0 & 1 & 1 & 1 \\ 1 & 0 & -1 & -1\\ 1 & -1 & 0 & -1 \\ 1 & -1 & -1 & 0
\end{pmatrix}
\]
Calcul du polynôme caractéristique $P$
\begin{align*}
&\begin{pmatrix}
  -\lambda & 1 & 1 & 1 \\ 1 & -\lambda & -1 & -1\\ 1 & -1 & -\lambda & -1 \\ 1 & -1 & -1 & -\lambda
\end{pmatrix}& \\
&\begin{pmatrix}
  1 & -1 & -1 & -\lambda \\
  1 & -\lambda & -1 & -1\\
  1 & -1 & -\lambda & -1 \\
  -\lambda & 1 & 1 & 1
\end{pmatrix}&
\begin{aligned}
  \leftarrow L_4 \\  \\  \\ \leftarrow L_1
\end{aligned}
\\
&\begin{pmatrix}
  1 & -1        & -1        & -\lambda     \\
  0 & 1-\lambda & 0         & -1 + \lambda \\
  0 & 0         & 1-\lambda & -1+\lambda   \\
  0 & 1-\lambda & 1-\lambda & 1-\lambda^2
\end{pmatrix}&
\begin{aligned}
 \leftarrow  \\ \leftarrow L_2 - L_1\\ \leftarrow L_3 - L_1 \\ \leftarrow L_4 + \lambda L_1
\end{aligned}
\\
&\begin{pmatrix}
  1 & -1        & -1        & -\lambda     \\
  0 & 1-\lambda & 0         & -1 + \lambda \\
  0 & 0         & 1-\lambda & -1+\lambda   \\
  0 & 0         & 0         & 3-\lambda^2 -2 \lambda
\end{pmatrix}&
\begin{aligned}
 \leftarrow  \\ \leftarrow \\ \leftarrow  \\ \leftarrow L_4 - \lambda L_2 - L_3
\end{aligned}
\\
\end{align*}
\[
  P = (1-X)^2(-X^2- 2X +3) 
  = (1-X)^3(3+X)
\]
Les valeurs propres sont $1$ et $-3$.\newline
Recherche du noyau de $A - I_4$:
\begin{align*}
  &\left\lbrace
  \begin{aligned}
    -x_1 + x_2 + x_3 + x_4 &= 0\\
    x_1 - x_2 - x_3 - x_4 &= 0\\
    x_1 - x_2 - x_3 - x_4 &= 0\\
    x_1 - x_2 - x_3 - x_4 &= 0
  \end{aligned}
  \right. \\
  \Leftrightarrow
  &    x_1 - x_2 - x_3 - x_4 = 0
  \\
  \Leftrightarrow
  &
  \begin{pmatrix}
    x_1 \\ x_2 \\ x_3 \\ x_4 
  \end{pmatrix}
  =
  x_2
  \begin{pmatrix}
    1 \\ 1 \\ 0 \\ 0
  \end{pmatrix}
  + x_3
  \begin{pmatrix}
    1 \\ 0 \\ 1 \\ 0
  \end{pmatrix}
  + x_4
  \begin{pmatrix}
    1 \\ 0 \\ 0 \\ 1
  \end{pmatrix}
\end{align*}

Recherche du noyau de $A + 3I_4$:
\begin{align*}
  &\left\lbrace
  \begin{aligned}
    3x_1 + x_2 + x_3 + x_4 &= 0\\
    x_1 + 3x_2 - x_3 - x_4 &= 0\\
    x_1 - x_2 + 3x_3 - x_4 &= 0\\
    x_1 - x_2 - x_3 + 3x_4 &= 0
  \end{aligned}
  \right. \\
  \Leftrightarrow
  &\left\lbrace
  \begin{aligned}
    x_1 - x_2 - x_3 + 3x_4 &= 0\\
    x_1 + 3x_2 - x_3 - x_4 &= 0\\
    x_1 - x_2 + 3x_3 - x_4 &= 0\\
    3x_1 + x_2 + x_3 + x_4 &= 0
  \end{aligned}
  \right. 
  &\begin{aligned}
     L_1 \leftarrow L_4 \\
     L_4 \leftarrow L_1 
   \end{aligned}
\\
  \Leftrightarrow
  &\left\lbrace
  \begin{aligned}
    x_1 - x_2 - x_3 + 3x_4 &= 0\\
         4x_2       - 4x_4 &= 0\\
               4x_3 - 4x_4 &= 0\\
         4x_2 + 4x_3 - 8x_4 &= 0
  \end{aligned}
  \right. 
  &\begin{aligned}
     L_2 \leftarrow L_2 -L_1 \\
     L_3 \leftarrow L_3 -L_1 \\
     L_4 \leftarrow L_4 - 3L_1
   \end{aligned}
\\
  \Leftrightarrow
  &\left\lbrace
  \begin{aligned}
    x_1 - x_2 - x_3 + 3x_4 &= 0\\
          x_2        - x_4 &= 0\\
                x_3  - x_4 &= 0\\
          x_2  + x_3 - 2x_4 &= 0
  \end{aligned}
  \right. &\begin{aligned}
     L_2 \leftarrow \frac{1}{4}L_2 \\
     L_3 \leftarrow \frac{1}{4}L_3 \\
     L_4 \leftarrow \frac{1}{4}L_4 
   \end{aligned}
\\
  \Leftrightarrow
  &\left\lbrace
  \begin{aligned}
    x_1       + x_4 &= 0\\
       x_2    - x_4 &= 0\\
          x_3 - x_4 &= 0 \\
                 0  &= 0
  \end{aligned}
  \right.   &\begin{aligned}
     L_1 \leftarrow L_1 + L_2 + L_3 \\
     L_4 \leftarrow L_4 - L_2 -L_3 
   \end{aligned}\\
 \Leftrightarrow
  &\begin{pmatrix}
    x_1 \\ x_2 \\ x_3 \\ x_4 
  \end{pmatrix}
  = x_4
  \begin{pmatrix}
    -1 \\ 1 \\ 1 \\ 1
  \end{pmatrix}
\
\end{align*}
On en déduit
\[
  D=
  \begin{pmatrix}
    1 & 0 & 0 & 0\\ 0 & 1 & 0 & 0 \\ 0 & 0 & 1 & 0 \\ 0 & 0 & 0 & -3 
  \end{pmatrix}\hspace{0.5cm}
P = \begin{pmatrix}
       1 & 1 & 1 & -1\\ 1 & 0 & 0 & 1 \\ 0 & 1 & 0 & 1 \\ 0 & 0 & 1 & 1
    \end{pmatrix}.
\]

Diagonalisation de
\[
\begin{pmatrix}
 1 & 2 & 3 \\  1 & 2 & 3 \\  1 & 2 & 3 
\end{pmatrix}  
\]
Les deux dernières colonnes sont colinéaires à la première. Le rang est 1 donc $0$ est une valeur propre dont l'espace propre est le noyau qui est de dimension $2$ et d'équation $x_1 + 2x_2 + 3x_3 = 0$. Une base du noyau est 
\[
  (\begin{pmatrix}
    -2 \\ 1 \\ 0
  \end{pmatrix}, \;
  \begin{pmatrix}
    -3 \\ 0 \\ 1
  \end{pmatrix} )
\]
Comme le rang est $1$, les colonnes non nulles de l'image sont propres, en particulier,
\[
  \begin{pmatrix}
    1 \\ 1 \\ 1
  \end{pmatrix}
\text{ colonne propre de valeur propre } 6.
\]
On en déduit
\[
  D =
  \begin{pmatrix}
    0 & 0 & 0\\ 0 & 0 & 0\\ 0 & 0 & 6 
  \end{pmatrix},\hspace{0.5cm}
  P=
  \begin{pmatrix}
    -2 & -3 & 1 \\ 1 & 0 & 1 \\ 0 & 1 & 1
  \end{pmatrix}.
\]

