\begin{tiny}(Cgc19)\end{tiny} Une premiere manière de résoudre cet exercice est d'évoquer le théorème de cours: une fonction continue et injective sur un intervalle et strictement monotone. Puis de justifier soigneusement que la stricte monotonie interdit l'égalité des limites.\newline
L'outil semble disproportonné, proposons une preuve directe dans le cas où la limite commune est $l\in \R$ et la fonction non constante.\newline
Il existe alors $u\in I$ tel que $f(u)\neq l$, par exemple $f(u)>l$. Comme les lites aux extrémités sont $l$, il existe un $a$ assez proche de l'extrémité gauche et un $b$ assez proche de l'extrémité droite tels que 
\begin{displaymath}
  f(a) < \frac{f(u)+l}{2}\hspace{0.5cm} f(b) < \frac{f(u)+l}{2}
\end{displaymath}
En appliquant le théorème des valeurs intermédiaires dans $[a,u]$ et dans $[u,b]$, on prouve l'existence de deux antécédents distincts de $\frac{f(u) + l}{2}$ ce qui prouve que $f$ n'est pas injective.\newline
Adapter cette démonstration dans les autres cas.