\begin{tiny}(Ccp47)\end{tiny} Les conditions caractérisant la symétrie se traduisent par:
\begin{multline*}
  \exists (\lambda, \mu)\in \R^2 \text{ tq } 
  \left\lbrace 
  \begin{aligned}
    z + z' &= \lambda u \\
    z - z' &= i \mu u
  \end{aligned}
\right. 
\Rightarrow
2z = \lambda + i \mu \\
\Rightarrow \lambda + i \mu = \frac{2z}{u}.
\end{multline*}

Or:
\begin{multline*}
  2z' = \lambda - \mu i u = \overline{(\lambda + i \mu)}\, u
  \Rightarrow 2z' = \overline{\left( \frac{2z}{u}\right) }\, u\\
  \Rightarrow z' = \frac{u}{\overline{u}}\, \overline{z}.
\end{multline*}
