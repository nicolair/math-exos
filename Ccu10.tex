\begin{tiny}(Ccu10)\end{tiny}
\begin{enumerate}
  \item En élevant au carré, on obtient que les solutions de $(1)$ sont solution de $(2)$ et que les solutions de $(2)$ sont solution de $(3)$.
  \item La fonction est strictement croissante dans $[0,a]$ d'après les propriétés de la fonction racine carrée.
  \item On désigne par $\varphi$ la fonction de la question b.. L'équation $(1)$ revient à
\begin{displaymath}
  \varphi(x) = 0
\end{displaymath}
La fonction $\varphi$ prend ses valeurs dans 
\begin{displaymath}
  \left[ -(\sqrt{a} + \sqrt{b}), \sqrt{a} - \sqrt{b-a}\right] 
\end{displaymath}
L'équation $(1)$ admet une unique solution si et seulement si
\begin{displaymath}
  \sqrt{a} - \sqrt{b-a} \geq 0 \Leftrightarrow b \leq 2a
\end{displaymath}
Dans ce cas, cette solution est également solution de $(2)$ et de $(3)$. L'équation $(3)$ admet donc deux solutions réelles, l'une d'entre elle est celle de l'équation $(1)$.
\end{enumerate}
