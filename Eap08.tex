\begin{tiny}(\hyperdef{exo}{Eap08}{Eap08})\end{tiny} Fonction génératrice. 
L'objet de cet exercice est de donner une formule explicite pour le nombre $s_n$ de couples d'entiers naturels $(p,q)$ vérifiant
\begin{displaymath}
 p+2q = n
\end{displaymath}
avec $n\in \N$. On note $a_n$ le coefficient de $x^n$ dans un développement limité en $0$ à un ordre strictement plus grand que $n$ de 
\begin{displaymath}
 F(x) = \frac{1}{(1-x^2)(1-x)}
\end{displaymath}
\begin{enumerate}
 \item Montrer que $a_n=s_n$.
\item Trouver une formule explicite de $s_n$ en utilisant une décomposition en éléments simples et la formule de Taylor avec reste de Young.
\end{enumerate}
Voir sur ce thème un problème sur le \href{\baseurl devoirs\_nicolair/Apopovi.pdf}{théorème de Popoviciu}.