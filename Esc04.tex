\begin{tiny}(Esc04)\end{tiny} Pour tout $x\geq 0$, on définit des suites $(x_{n})_{n\in \N^{*}}$ et $(y_{n})_{n\in \N^{*}}$ par
\begin{displaymath}
x_{n}=\sqrt{1+\sqrt{2+\sqrt{3+\cdots+\sqrt{n +x}}}}
\end{displaymath}
De plus, $y_{n}$ est obtenu en remplaçant dans $x_{n}$ le $n$ de la dernière racine par $2n$.
\begin{enumerate}
  \item Montrer que $(x_{n})_{n\in \N^{*}}$ est croissante à partir d'un certain rang.
  \item Montrer que $(y_{n})_{n\in \N^{*}}$ est décroissante à partir d'un certain rang. En déduire la convergence de $(x_{n})_{n\in \N^{*}}$. On note $\varphi(x)$ sa limite.
  \item Montrer que la fonction $\varphi$ est croissante. Montrer qu'en fait elle est constante.
\end{enumerate}
