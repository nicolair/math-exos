\begin{tiny}(Cgc13)\end{tiny}
\begin{enumerate}
 \item De la croissance des fonctions partie entière et racine carrée, on déduit
\begin{displaymath}
  \lfloor \sqrt{\lfloor x \rfloor} \rfloor \leq \lfloor \sqrt{x} \rfloor
\end{displaymath}
Notons 
\begin{displaymath}
  m_x = \lfloor \sqrt{\lfloor x \rfloor} \rfloor \in \N
\end{displaymath}
On a donc $m_x \leq \lfloor \sqrt{x} \rfloor$ d'où $m_x\leq \sqrt{x}$. On veut montrer que $m_x = \lfloor \sqrt{x} \rfloor$ c'est à dire, par définition de la partie entière, que $\sqrt{x} < m_x +1$.
On raisonne par l'absurde:
\begin{multline*}
\sqrt{x}\geq m_x + 1 \Rightarrow x \geq (m_x+1)^2 \in \N \\
\Rightarrow \lfloor x \rfloor \geq (m_x+1)^2
\Rightarrow \sqrt{\lfloor x \rfloor} \geq m_x + 1 \in \N \\
\Rightarrow m_x = \lfloor \sqrt{\lfloor x \rfloor} \rfloor \geq m_x + 1
\end{multline*}
Absurde
 
 \item  Montrons d'abord que $(1)$ entraine $(2)$.\newline
On commence par reformuler $(1)$.\newline
Pour tout $x$ de $I$ notons
\begin{displaymath}
 m_x = \lfloor f(\lfloor x \rfloor)\rfloor
\end{displaymath}
Comme la fonction $f$ est croissante, on a toujours
\begin{displaymath}
 f(\lfloor x \rfloor) \leq f(x)
\end{displaymath}
La condition $(1)$ se traduit donc par :
\begin{displaymath}
 \forall x\in I : m_x\leq f(\lfloor x \rfloor) \leq f(x) < m_x +1
\end{displaymath}
Cette condition entraine l'implication $(2)$. En effet, lorsque $f(x)\in \Z$, $f(x)=m_x$ et l'encadrement précédent devient :
\begin{displaymath}
 f(x)\leq f(\lfloor x \rfloor) \leq f(x) < f(x) +1
\end{displaymath}
Cela entraine $f(x)=f(\lfloor x \rfloor)$ qui ne se produit (croissance stricte) que si $x=\lfloor x \rfloor$ c'est à dire si $x$ est entier.\newline
Remarquons que la continuité de $f$ n'est pas intervenue dans ce raisonnement.

Montrons que $(2)$ entraine $(1)$. Exprimons une conséquence de $(2)$ et de la continuité de $f$.
\begin{quote}
 L'image par $f$ d'un intervalle ne contenant qu'un seul entier est un intervalle contenant \emph{au plus} un entier.
\end{quote} 
En particulier, lorsque $a\in \Z$
\begin{align*}
 f(a)\in \Z &\Rightarrow f(a+1)\leq f(a)+1 \\
 f(a)\not \in \Z &\Rightarrow f(a+1) \leq \lfloor f(a)\rfloor +1
\end{align*}
Cela entraine dans les deux cas :
\begin{displaymath}
 \lfloor f(a)\rfloor \leq f(a+1) \leq \lfloor f(a)\rfloor +1
\end{displaymath}
Lorsque $a=\lfloor x \rfloor$, cette inégalité s'écrit (avec la notation $m_x$):
\begin{displaymath}
 m_x \leq f(\lfloor x \rfloor +1) \leq m_x +1
\end{displaymath}
Or $x < \lfloor x \rfloor +1$, donc $f(x)<f(\lfloor x \rfloor +1)$ et on a bien l'encadrement caractéristique de $(1)$ :
\begin{displaymath}
 \forall x\in I : m_x\leq f(\lfloor x \rfloor) \leq f(x) < m_x +1
\end{displaymath}
\end{enumerate}

