\begin{tiny}(Cli06)\end{tiny} Soit $A$ l'ensembles des réels dans $[a,b]$ en lesquels la fonction $f$ s'annule. Cet ensemble est une partie non vide de $\R$, minorée par $a$, elle admet donc une borne inférieure $c$ qui vérifie $a\leq c$ car $a$ est un minorant de $A$.\newline
D'autre part, il existe $u$ dans $A$ donc $c\leq u \leq b$. On en déduit que $c\in [a,b]$ donc $f$ est continue en $c$.\newline
D'après un résultat de cours, il existe une suite $\left( a_n\right)_{n\in \N}$ d'éléments de $A$ qui converge vers $c$.\newline
Par définition de $A$, la suite $\left( f(a_n)\right)_{n\in \N}$ est constante de valeur nulle. Elle converge donc vers $0$.\newline
Par continuité de $f$ en $c$, cette suite converge aussi vers $f(c)$. Par unicité de la limite, on obtient alors $f(c)=0$ donc $c\in A$ et $c=\min A$.

