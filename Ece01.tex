\begin{tiny}(Ece01)\end{tiny}
\textbf{Mouvements K{\'e}pl{\'e}riens}. On consid{\`e}re un mouvement
plan d{\'e}fini dans $\R$ et exprim{\'e} en coordonn{\'e}es polaires
dans le plan complexe
\[
f(t)=r(t)e^{i\theta (t)}
\]
On suppose que ce mouvement est {\`a} \emph{acc{\'e}l{\'e}ration} \emph{%
centrale} et {\`a} sym{\'e}trie circulaire c'est {\`a} dire qu'il existe une
fonction $m$ telle que
\[
f^{\prime \prime }(t)=m(r(t))e^{i\theta (t)}
\]
On se propose de montrer l'{\'e}quivalence entre les trois
propri{\'e}t{\'e}s suivantes :

\begin{itemize}
\item  il existe un r{\'e}el $K$ fix{\'e} tel que $m(r)=\frac{K}{r^{2}}$

\item  l'hodographe du mouvement est un cercle

\item  la trajectoire du mouvement est une conique de foyer
l'origine.
\end{itemize}

On rappelle que l'hodographe est l'arc param{\'e}tr{\'e} d{\'e}fini par la
d{\'e}riv{\'e}e de $f$.

\begin{enumerate}
\item  Montrer qu'il existe une constante $C$ telle que
\[
r^{2}(t)\theta ^{\prime }(t)=C
\]

\item  Pr{\'e}ciser la courbure de l'hodographe, montrer que $(1)$ est
{\'e}quivalent {\`a} (2).

\item  Montrer que (2) entra{\^\i}ne (3).

\item  Montrer que (3) entra{\^\i}ne (1)
\end{enumerate}