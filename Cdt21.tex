\begin{tiny}(Cdt21)\end{tiny}
\begin{enumerate}
 \item L'intersection est réduite à la matrice nulle et toute matrice $M$ se décompose en
\begin{displaymath}
 M = \underset{\in \mathcal{S}}{\underbrace{\frac{1}{2}(M+\trans M)}}+ \underset{\in \mathcal{I}}{\underbrace{\frac{1}{2}(M-\trans M)}}
\end{displaymath}

 \item Comme $\delta \circ \delta=\Id_{\mathcal{M}_n(\R)}$, on peut d'abord remarquer que $\det \delta =\pm 1$. La restriction de $\delta$ à $\mathcal S$ est $\Id_{\mathcal S}$.  La restriction de $\delta$ à $\mathcal I$ est $-\Id_{\mathcal I}$. En fait $\delta$ est la symétrie par rapport à $\mathcal{S}$ dans la direction $\mathcal{I}$. La matrice de $\delta$ dans une base de $\mathcal{M}_n(\R)$ dont les premiers vecteurs sont dans $\mathcal{S}$ et les suivants dans $\mathcal{I}$ est diagonale avec des $1$ au début et des $-1$ à la fin. On en déduit
\begin{displaymath}
\det \delta = (-1)^{\dim(\mathcal{I})}=(-1)^{\frac{n(n-1)}{2}} 
\end{displaymath}
\end{enumerate}