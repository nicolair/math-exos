\begin{tiny}(Edi28)\end{tiny} \label{exo: di28} Soit $E$ un $\K$-espace vectoriel de dimension finie.
\begin{enumerate}
 \item Soit $A$ un sous-espace de $E$ et $B$ un supplémentaire de $A$. Montrer que
\[
 \mathcal{A} = \left\lbrace f\in \mathcal{L}(E) \text{ tq } A \subset \ker f\right\rbrace
\]
est isomorphe à $\mathcal{L}(B,E)$. En déduire sa dimension.
 \item Soit $a\in \mathcal{L}(E)$ fixé. On définit la fonction $\delta$ par:
\begin{displaymath}
  \delta:
\left\lbrace 
\begin{aligned}
  \mathcal{L}(E) &\rightarrow \mathcal{L}(E) \\ f &\mapsto f \circ a
\end{aligned}
\right. 
\end{displaymath}
Montrer que $\delta$ est linéaire, préciser son noyau, son image et leurs dimensions.
\end{enumerate}
