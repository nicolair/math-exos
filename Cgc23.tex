\begin{tiny}(Cgc23)\end{tiny} L'unicité d'un éventuel point fixe est immédiate car $x\rightarrow f(x) - x$ est strictement décroissante. \newline
Pour l'existence, on commence par remarquer que $f\xrightarrow{-\infty}-\infty$ et $f\xrightarrow{+\infty}+\infty$ sont faux. \newline
Si on avait $f(x)-x\leq 0$ pour tous les $x$ on aurait $f(x)\leq x$ et donc  $f\xrightarrow{-\infty}-\infty$. Il existe donc un $x_1$ tel que $f(x_1)-x_1>0$. \newline
De même, si on avait $f(x)-x\geq 0$ pour tous les $x$ on aurait $f(x)\geq x$ et donc  $f\xrightarrow{+\infty}+\infty$. Il existe donc un $x_2$ tel que $f(x_2)-x_2<0$. \newline
On conclut par le théorème de la valeur intermédiaire.