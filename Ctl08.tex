\begin{tiny}(Ctl08)\end{tiny} Définissons une fonction $\varphi$ et dérivons la:
\begin{multline*}
  \varphi(x) = \frac{f(a)+f(x)}{2} - f(\frac{a+x}{2})\\
\varphi'(x) = \frac{1}{2}\left(f'(x)-f'(\frac{a+x}{2}) \right) 
\end{multline*}

Comme $x-\frac{a+x}{2}=\frac{x-a}{2}$, l'inégalité des accroissements finis appliquée à $\varphi'$ donne
\begin{displaymath}
  \frac{m_2}{4}(x-a)\leq \varphi'(x)\leq \frac{M_2}{4}(x-a) 
\end{displaymath}
avec $m_2$ et $M_2$ les valeurs minimale et maximale de la fonction continue $f''$ dans le segement.\newline
En intégrant entre $a$ et $b$, on obtient 
\begin{displaymath}
  \frac{m_2}{8}(b-a)^2\leq \varphi(b)-\varphi(a)\leq \frac{M_2}{8}(b-a)^2
\end{displaymath}
On termine en remarquant que $\varphi(a)=0$ et en appliquant le théorème de la valeur intermédiaire à la fonction continue $f''$:
\begin{displaymath}
  m_2\leq \frac{8\varphi(b)}{(b-a)^2}\leq M_2  
\end{displaymath}
entraine qu'il existe $c\in [a,b]$ tel que 
\begin{displaymath}
  \varphi(b) = \frac{(b-a)^2}{8}f''(c)
\end{displaymath}
