\begin{tiny}(Cml13)\end{tiny} Soit $m$ le plus petit entier $k$ tel que $f^k(x) = 0$. (Il en existe : au moins $n$). On considère une combinaison linéaire nulle. Alors $f^{m-1}(x)\neq 0_E$. 
\begin{displaymath}
 \lambda_0x + \cdots + \lambda_p f^p(x) = 0_E
\end{displaymath}
On compose par $f^{m-1}$. On en déduit $\lambda_0 =0$ et ainsi de suite en composant par $f^{m-2}, \cdots$.