\begin{tiny}(Eal07)\end{tiny} Théorème de Lagrange \newline
Soit $G$ un groupe (notation multiplicative) fini de cardinal $m$. On veut montrer que pour tout $g\in G$ :
\begin{displaymath}
g^m=e 
\end{displaymath}
Soit $H$ un sous-groupe de $G$. On définit une relation $R_H$ dans $G$ en posant
\begin{displaymath}
\forall (g,g^\prime)\in G^2 : g R_H g^\prime \Leftrightarrow g{g^\prime}^{-1}\in H
\end{displaymath}
\begin{enumerate}
\item Montrer que $Hg = Hg^\prime$ si et seulement si $g R_H g^\prime$.
\item Montrer que $R_H$ est une relation d'équivalence.
\item En déduire que $\card H$ divise $\card G$.
\item Soit $g\in G$, montrer qu'il existe un entier $m_g>0$ tel que 
\[\{k\in \Z \,\mathrm{ tq }\, g^k=e\}=\Z m_g\]
\item En considérant le sous-groupe engendré par $g$, montrer le théorème de Lagrange.
\end{enumerate}
