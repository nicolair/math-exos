\begin{tiny}(Evs02)\end{tiny} Soit $E$ un ensemble. On va étudier diverses équations d'inconnue $X$ dans $\mathcal{P}(E)$.
\begin{enumerate}
 \item Soit  $A$ et $B$ des parties de $E$. Discuter de l'existence de solutions de l'équation $E(A,B)$:
\begin{displaymath}
 X\cup A=B
\end{displaymath}
Lorsque $E(A,B)$ admet des solutions, caractériser par une double inclusion les parties $X$ de $E$ qui sont solutions.
 \item  En se ramenant à l'équation précédente, discuter  de l'existence de solutions pour les équations suivantes et caractériser les parties de $E$ qui sont solutions.
\begin{displaymath}
 (1)\; X\cap A=B, \hspace{0.4cm}(2)\;X\setminus A=B , \hspace{0.4cm}(3)\;A\setminus X=B
\end{displaymath}
\item Soit $I$ un ensemble et $\left( A_i\right) _{i\in I}$, $\left( B_i\right) _{i\in I}$ deux familles de parties de $E$. Résoudre les systèmes d'équations en précisant les conditions assurant l'existence de solutions.
\begin{align*}
 &(S_1):& \forall i\in I, \; A_i \cup X =B_i \\
 &(S_2):& \forall i\in I, \; A_i \cap X =B_i  
\end{align*}
\end{enumerate}
 