\begin{tiny}(Een05)\end{tiny} On définit une fonction $f:\N\times\N\rightarrow\N$ par le tableau suivant
\begin{center}
\renewcommand{\arraystretch}{1.2}
% use packages: array
\begin{tabular}{c|c|c|c|c|c}
4 &  &  &  &  &  \\ \hline
3 & 9 &  &  &  &  \\ \hline
2 & 5 & 8 & $\ddots$ &  &  \\ \hline
1 & 2 & 4 & 7 & 11 &  \\ \hline
0 & 0 & 1 & 3 & 6 & 10 \\ \hline
 & 0 & 1 & 2 & 3 & 4
\end{tabular}
\end{center}
qui se poursuit indéfiniment et où la case d'abscisse $i$ et d'ordonnée $j$ contient le nombre $f(i,j)$. Cette fonction est clairement bijective. Préciser explicitement $f(i,j)$.