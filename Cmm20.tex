\begin{tiny}(Cmm20)\end{tiny}
\begin{enumerate}
  \item Il est évident que $I_2$ et $A$ commutent avec $A$ donc $\Vect(I_2,A)\subset\ker \Phi$.
  \item Par le calcul:
\begin{displaymath}
M=
\begin{pmatrix}
x & y \\ z & t  
\end{pmatrix}
\Rightarrow 
\Phi(M) = (t-x)\Delta + yT_s + zT_i 
\end{displaymath}
On en déduit $\Im \Phi \subset \Vect(\Delta,T_i,T_s)$.

  \item Si $A\notin \Vect(I_2)$ alors $(A,I_2)$ libre et 
\begin{displaymath}
\Vect(I_2,A)\subset \ker \Phi \Rightarrow 2\leq \dim(\ker \Phi)
\Rightarrow \rg(\Phi) \leq 2
\end{displaymath}
On en déduit que la famille de trois vecteurs $(\Delta,T_i,T_s)$ est liée. On forme une relation linéaire en écrivant que $A\in \ker \Phi$ :
\begin{displaymath}
 (d-a)\Delta + bT_s + cT_i = 0_{\mathcal{M}_2(\K)} 
\end{displaymath}
avec
\begin{displaymath}
(d-a,b,c)=(0,0,0)\Leftrightarrow A \in \Vect(I_2)  
\end{displaymath}
 \item Si $A\in \Vect(I_2)$ tout le monde commute avec $A$ donc $\Phi$ est identiquement nulle.\newline
Si $A\notin \Vect(I_2)$, on a vu que $\rg(\Phi)\leq 2$ et 
\begin{displaymath}
(d-a,b,c)\neq(0,0,0)  
\end{displaymath}
et
\begin{align*}
d-a \neq 0 &\Rightarrow (T_s,T_i)\text{ libre } \\
b \neq 0 &\Rightarrow (\Delta,T_i)\text{ libre } \\
c \neq 0 &\Rightarrow (\Delta,T_s)\text{ libre } 
\end{align*}
Dans tous les cas le rang de la famille est supérieur ou égal à 2 donc le rang est 2.
Supposons $(I_2,A,B)$ liée.
\begin{itemize}
  \item Si $(I_2,A)$ liée, tout le monde commute avec $A$, en particulier $B$.
  \item Sinon $B\in \Vect(I_2,A)$ et il commute avec $A$.
\end{itemize}
Supposons que $B$ commute avec $A$.
\begin{itemize}
  \item Si $(I_2,A)$ liée, alors $(I_2,A,B)$ liée.
  \item Sinon,
\begin{multline*}
\rg(\Phi)=2 \Rightarrow \ker \Phi = \Vect(I_2,A) \\
\Rightarrow  B \in \Vect(I_2,A) \Rightarrow (I_2,A,B) \text{ liée }
\end{multline*}

\end{itemize}


\end{enumerate}
