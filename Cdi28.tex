\begin{tiny}(Cdi28)\end{tiny} 
\begin{enumerate}
 \item On vérifie que 
\[
 \left\lbrace 
 \begin{aligned}
  \mathcal{A} &\rightarrow \mathcal{L}(B,E)\\
  f &\mapsto f_{|B}
 \end{aligned}
\right. 
\]
est un isomorphisme. On en déduit 
\begin{multline*}
\dim \mathcal{A} = \dim(B) \dim(E)\\
= (\dim(E) - \dim(A))\dim(E). 
\end{multline*}


 \item La linéarité est immédiate car toutes les fonctions en jeu sont linéaires. D'autre part, il est clair que
\begin{multline*}
  \left\lbrace f\in \mathcal{L}(E) \text{ tq } \Im a \subset \ker f\right\rbrace = \ker \delta\\
  \Im(\delta) \subset \left\lbrace g\in \mathcal{L}(E) \text{ tq } \ker a \subset \ker g\right\rbrace.  
\end{multline*}

D'après la question a.
\begin{multline*}
 \dim(\ker \delta) = \dim(\ker a)\dim E, \\
 \dim \left\lbrace g\in \mathcal{L}(E) \text{ tq } \ker a \subset \ker g\right\rbrace = \rg a\,\dim E 
\end{multline*}

Or, d'après le théorème du rang 
\begin{multline*}
 \dim(\ker \delta) + \dim (\Im(\delta)) = \dim(E)^2\\
 \Rightarrow 
\dim (\Im(\delta)) =  \dim(E)^2 - \dim(\ker a)\dim E\\
= \rg a \dim E.
\end{multline*}

On en déduit 
\[
 \Im(\delta) = \left\lbrace g\in \mathcal{L}(E) \text{ tq } \ker a \subset \ker g\right\rbrace.
\]

\end{enumerate}

