\begin{tiny}(Eaz09)\end{tiny} Nombre de diviseurs.\newline
Soit $n$ un entier supérieur ou égal à $2$. On note $d(n)$ le nombre de diviseurs positifs de $n$. \footnote{On donnera deux solutions pour les questions b. et c.: une utilisant la paramétrisation de l'ensemble des diviseurs sous-jacente à la question a., l'autre le regroupement des diviseurs par paires $\left\lbrace d,d' \right\rbrace$ telles que $d d' = n$.} 
\begin{enumerate}
 \item Si la décomposition en facteurs premiers de $n$ est
\begin{displaymath}
 n=p_1^{m_1}p_2^{m_2}\cdots p_k^{m_k}
\end{displaymath}
Exprimer $d(n)$ à l'aide de $m_1,\cdots, m_k$. (valuations $p$-adiques de $n$) En déduire que $d$ est \emph{multiplicative} c'est à dire que
\begin{displaymath}
  a\wedge b = 1 \Rightarrow d(ab) = d(a)d(b)
\end{displaymath}
 \item Montrer que $n$ est un carré d'entier si et seulement si $d(n)$ est impair.
 \item Montrer que le produit de tous les diviseurs de $n$ est $\sqrt{n^{d(n)}}$.
 \item Montrer que le nombre de couples $(a,b)$ d'entiers tels que le ppcm de $a$ et de $b$ soit $n$ est égal au nombre de diviseurs de $n^2$.
\end{enumerate}
