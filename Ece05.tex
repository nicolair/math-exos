\begin{tiny}(Ece05)\end{tiny}
Fonctions à valeurs complexes. {\'E}quations intrinsèques. Dans cet exercice, $I$ est un intervalle et $R$ une fonction $\mathcal{C}^\infty(I)$ à valeurs strictement positives.
\begin{enumerate}
\item On considère la courbe paramétrée $\gamma$ à valeurs complexes et définie dans $I$ par :
\[\gamma(\alpha)=\int _{0}^{\alpha}R(\theta)e^{i\theta}\,d\theta\]
Interpréter géométriquement le paramètre $\alpha$ et le nombre $R(\alpha)$ relativement à la courbe.
\item Soit $u$ un nombre complexe de module 1 et $k$ un réel strictement positif. Soit $\gamma_1$ une courbe paramétrée dans $I$ telle que :
\begin{itemize}
\item $\alpha$ est l'angle orienté entre le vecteur d'affixe $u$ et $\overrightarrow{\gamma_1'}(\alpha)$
\item $kR(\alpha)$ est le rayon de courbure en $\gamma_1(\alpha)$
\end{itemize}
Montrer que le point $\gamma_1(\alpha)$ est l'image du point $\gamma(\alpha)$ par une similitude fixée à préciser.\newline
Ceci justifie la dénomination \emph{équation intrinsèque} donnée à une relation entre l'angle de la tangente avec une direction fixe et le rayon de courbure.
\item On suppose ici que la fonction $R$ est strictement croissante, on note $K$ la développée de $\gamma$. Préciser $K(\alpha)$, $\overrightarrow{K'}(\alpha)$ et son module. Vérifier que $\alpha$ est l'angle que fait $\overrightarrow{K'}(\alpha)$ avec une direction fixe. En déduire le rayon de courbure de la développée $K$. Que se passe-t-il lorsque le rayon de courbure est strictement décroissant ?
\end{enumerate}