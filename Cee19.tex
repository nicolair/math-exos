\begin{tiny}(Cee19)\end{tiny}
\begin{enumerate}
 \item Un sens est évident par spécialisation, l'autre est facile par linéarité en utilisant les identités de polarisation. 
 \item Il est évident qu'une similitude conserve l'orthogonalité d'après la deuxième propriété.\newline
Considérons un $f\in\mathcal{L}(E)$ qui conserve l'orthogonalité.\newline
Pour tous $x$ et $y$ dans $E$ avec $x\neq 0$, on a alors :
\begin{displaymath}
 (f(x)/f(y)) = \frac{\Vert f(x) \Vert ^2}{\Vert x \Vert ^2} (x/y)
\end{displaymath}
Il suffit pour cela d'orthogonaliser $(x,y)$ en $(x,y')$ et d'écrire que $f(x)$ et $f(y)$ sont orthogonaux.\newline
Il reste à montrer que tous les 
\begin{displaymath}
 \alpha_x = \frac{\Vert f(x) \Vert ^2}{\Vert x \Vert ^2}
\end{displaymath}
sont égaux entre eux pour tous les $x$ non nuls.\newline
En échangeant les rôles de $x$ et de $y$, on obtient $\alpha_x = \alpha_y$ pour $x$ et $y$ non nuls et non orthogonaux. Lorsque $x$ et $y$ sont orthogonaux, on peut toujours trouver un $z$ non nul et qui n'est orthogonal à aucun des deux. On en déduit l'égalité par transitivité.
\end{enumerate}
 