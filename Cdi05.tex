\begin{tiny}(Cdi05)\end{tiny} Il faut bien noter ici que $A$ et $B$ \emph{ne sont pas supplémentaires}. On considère deux bases:
\begin{displaymath}
  (a_1,\cdots,a_\alpha) \text{ base de } A, \hspace{0.5cm}
  (b_1,\cdots,b_\beta) \text{ base de } B
\end{displaymath}
avec $\alpha + \beta = n = \dim E$. D'après le théorème de la base incomplète, il existe $u_1,\cdots, u_\beta$ tels que
\begin{displaymath}
  (a_1,\cdots,a_\alpha, u_1,\cdots, u_\beta) \text{ base de } E
\end{displaymath}
On définit $f$ par prolongement linéaire:
\begin{displaymath}
\forall i\in \llbracket 1,\alpha\rrbracket,\; f(a_i)=0_E,\hspace{0.5cm}
\forall i\in \llbracket 1,\beta\rrbracket,\; f(u_i)= b_i
\end{displaymath}
On vérifie facilement que $\ker f = A$ et $\Im f = B$.