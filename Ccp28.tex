\begin{tiny}(Ccp28)\end{tiny} 
\begin{enumerate}
  \item Le point d'affixe $z$ est sur le cercle de diamètre $1, i$. En effet (voir exercice \ref{cp20} )
\begin{displaymath}
  \frac{z-i}{iz-i}\in \R \Leftrightarrow \frac{z-i}{z-1}\in i\R 
\end{displaymath}

  \item Le caractère isocèle et rectangle en $i$ se traduit par :
  \begin{multline*}
    z-i = \pm i (iz-i)
    \Leftrightarrow z -i = \pm(z-1)\\
    \Leftrightarrow z-i = -(z - 1)
    \Leftrightarrow z = \frac{1+i}{2}
  \end{multline*}

  \item Notons $x=\Re(z)$ et $y=\Im(z)$. \'Ecrivons le caractère équilatéral par une égalité de distance
\begin{multline*}
\left\lbrace 
\begin{aligned}
 \left|\frac{z-i}{iz-i}\right| &= 1 \\ \left|\frac{z-i}{iz-z}\right| &= 1 
\end{aligned}
\right. 
\Leftrightarrow
\left\lbrace 
\begin{aligned}
 \left|\frac{z-i}{z-1}\right| &= 1  
 \\ \left|\frac{z-i}{\sqrt{2}z}\right| &= 1 
\end{aligned}
\right. \\
\Leftrightarrow
\left\lbrace 
\begin{aligned}
 x &= y \\ 1-2y &= x^2 + y^2 
\end{aligned}
\right. \Leftrightarrow z = \frac{-1\pm \sqrt{3}}{2}(1+i)
\end{multline*}

\end{enumerate}
