\begin{tiny}(Een06)\end{tiny} Nombre de surjections.
Soit $E$ et $F$ deux ensembles finis, il est clair que le nombre d'applications surjectives de $E$ dans $F$ ne dépend que du nombre d'éléments dans $E$ et dans $F$. On le note
\begin{displaymath}
 s(\sharp E,\sharp F)
\end{displaymath}
Soit $n$ et $p$ deux entiers naturels fixés avec $1\leq p\leq n$. 
\begin{enumerate}
 \item En classant les surjections suivant l'ensemble des antécédents d'un élément fixé de l'espace d'arrivée, former une relation entre $s(n,p)$ et les $s(k,p-1)$ pour $k\in\llbracket0, n\rrbracket$.
 \item Former une autre relation en classant les surjections d'abord suivant l'image d'un élément fixé de l'espace de départ puis selon que cette image admette un seul antécedent ou plusieurs.
\end{enumerate}
