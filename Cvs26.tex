\begin{tiny}(Cvs26.tex)\end{tiny}
Définissons une relation binaire $\sim$ dans $\mathcal{F}$ en posant
\begin{displaymath}
  \forall (f,g) \in \mathcal{F}^2, \; f \sim  g \Leftrightarrow f(0) = g(0)
\end{displaymath}
On vérifie facilement que $\sim$ est une relation d'équivalence (réflexive, symétrique, transitive) pour laquelle $A_1$, $A_2$, $A_3$ sont les 3 classes d'équivalence.\newline
On peut conclure car les classes d'équivalence d'une relation d'équivalence forment une partition.
