\begin{tiny}(Cip21)\end{tiny} On utilise
\begin{displaymath}
 f(x)=\int_0^xf'\; \text{ et } F(x)= \int_0^x |f'|
\end{displaymath}
pour majorer:
\begin{multline*}
\int_0^a\left|f(x)f'(x)\right|\,dx
= \int_0^a\left|\int_0^xf'(t)\,dt\right|\left|f'(x)\right|\,dx\\
\leq \int_0^a \underset{=F(x)}{\underbrace{\left( \int_0^x\left|f'(t)\right|\,dt\right)}}  \underset{=F'(x)}{\underbrace{\left|f'(x)\right|}}\,dx \\
\leq \left[ \frac{1}{2}F^2\right]_0^a
\leq \frac{1}{2}\left( \int_0^a|f'(t)|\,dt\right)^2
\leq \frac{a}{2}\, \int_0^a(f'(t))^2\,dt
\end{multline*}

en utilisant l'inégalité de Cauchy-Schwarz.\newline
Dans un cs d'égalité, toutes les inégalités de la chaînes doivent être des égalité, en particulier la dernière (de Cauchy-Schwarz). On en déduit que $|f'|$ doit être constante. Comme $f'$ est supposée continue, elle doit être constante donc $f$ doit être polynomiale de degré $1$.
