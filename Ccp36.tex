\begin{tiny}(Ccp36)\end{tiny} En développant,
\begin{multline*}
  \frac{1}{2}\left( (b-a)^2+(c-b)^2+(a-c)^2\right)\\
  = a^2+b^2+c^2 -ab-bc-ac
\end{multline*} et
\begin{multline*}
  (aj^2+bj+c)(aj+bj^2+c)\\=
  a^2+b^2+c^2+ab(j+j^2)+ac(j+j^2)+bc(j+j^2)
\end{multline*}
On conclut avec
\begin{displaymath}
  1+j+j^2 = 0
\end{displaymath}
Ce calcul prouve directement l'équivalence entre les conditions 2 et 4 de la dernière question de l'exercice \ref{cp14}.