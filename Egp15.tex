\begin{figure}[ht]
 \centering
\input{Egp15_1.pdf_t}
\caption{Exercice \arabic{enumi}}
\label{fig:Egp15_1}
\end{figure}
\begin{tiny}(gp15)\end{tiny}
Dans un plan muni d'un repère, on se donne trois points $I$, $A$, $B$ respectivement de coordonnées $(1,1)$, $(1,0)$, $(0,1)$. Soit $\mathcal C$ le cercle de centre $I$ et de rayon $1$, soit $M$ un point de $\mathcal C$. La tangente en $M$ à $\mathcal C$ coupe l'axe $Ox$ en $P$ et l'axe $Oy$ en $Q$. (voir figure \ref{fig:Egp15_1})
\begin{enumerate}
 \item En utilisant la règle du dédoublement pour obtenir l'équation d'une tangente, donner des expressions simples de
\begin{displaymath}
(x(P)-1)(x(M)-1) \text{ et }(y(Q)-1)(y(M)-1) 
\end{displaymath}
en fonction de $x(M)$ et de $y(M)$.
 \item Former les équations des droites $(PB)$, $(QA)$, $(OM)$.
 \item Montrer que les droites $(PB)$, $(QA)$, $(OM)$ sont parallèles ou concourantes.
\end{enumerate}
