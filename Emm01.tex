\begin{tiny}(Emm01)\end{tiny}
Soit $n$ un entier pair, une matrice $M$ de $\mathcal{M}_{n}(\K)$ est dite \emph{en damier} si et seulement si il existe $\alpha $ et $\beta $ dans $\K$ tels que
\[
m_{ij}=\left\{
\begin{array}{ccc}
\alpha  & \text{si} & i+j\text{ pair} \\
\beta  & \text{si} & i+j\text{ impair}
\end{array}
\right.
\]
On note $\mathcal{D}$ l'ensemble des matrices en damier. {\'E}tudier les stabilit{\'e}s de $\mathcal{D}$. L'ensemble $\mathcal{D}$ est il une sous-alg{\`e}bre de $\mathcal{M}_{n}(\K)$ ?\newline Si $M$ est une matrice en damier, montrer qu'il existe $\lambda $ et $\mu$ dans $\K$ tels que $M^{3}=\lambda M^{2}+\mu M$.
