\begin{tiny}(Csn22)\end{tiny} On note 
\begin{displaymath}
 \forall n\in \N V_n\; V_n = \sum_{k=0}^n v_k
\end{displaymath}
Par hypothèse, il existe $V>0$ tel que 
\begin{displaymath}
 \forall n \in \N,\; |V_n| \leq V.
\end{displaymath}
Effectuons une transformation d'Abel en écrivant $v_k = V_k - V_{k-1}$ pour $k\geq 1$ et $v_0 = V_0$.
\begin{multline*}
 \sum_{k=0}^n u_kv_k
= u_0 V_0 + \sum_{k=1}^n u_k(V_k - V_{k-1})\\
= u_0 V_0 + \sum_{k=1}^n u_kV_k - \sum_{k=0}^{n-1} u_{k+1}V_{k}\\
= \sum_{k=1}^n (u_k-u_{k+1})V_k +u_{n+1}V_n
\end{multline*}

La suite $(u_{n+1}V_n)$ converge vers $0$ car $(u_n)$ converge vers $0$ et $(V_n)$ est bornée. La série
$\left( \sum (u_k-u_{k+1})V_k\right)$ est \emph{absolument} convergente car
\begin{multline*}
 \sum_{k=0}^n \left|(u_k-u_{k+1})V_k\right| 
= \sum_{k=0}^n (u_k-u_{k+1})\left|V_k\right| \\
\leq V \sum_{k=0}^n (u_k-u_{k+1})
\leq Vu_0
\end{multline*}

car la suite $(u_n)$ est décroissante.
