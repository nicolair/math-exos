\begin{tiny}(Ecp09)\end{tiny} \emph{Orthocentre et cercle circonscrit.} 
Soit $A$, $B$, $C$ trois points d'affixes $a$, $b$, $c$ distincts et de \emph{module 1}.
\begin{enumerate}
 \item Montrer que le point $H$ d'affixe $h=a+b+c$ est l'orthocentre du triangle $A,B,C$.
 \item Montrer que les six points d'affixes
\begin{displaymath}
\frac{b+c}{2},\frac{c+a}{2},\frac{a+b}{2},\frac{a+h}{2},\frac{b+h}{2},\frac{c+h}{2} 
\end{displaymath}
  sont sur un cercle dont l'affixe du centre est $\frac{h}{2}$.
  \item Déterminer les affixes des symétriques de $H$ par rapport à $(BC)$, $(CA)$, $(AB)$. (exercice \ref{Ecp40}) Préciser les modules. Que peut-on en déduire ?
\end{enumerate}

   