\begin{tiny}(Epo09)\end{tiny} Soit $u= \left( u_n\right) _{n\in \N^*}$ une suite de nombres complexes. On dira qu'une suite $\left( Q_n\right) _{n\in \N}$ de polynômes à coefficients complexes vérifie $E(u)$ si et seulement si:
\begin{displaymath}
 Q_0 = 1, \hspace{0.5cm}\forall n\in \N^*:
\left\lbrace 
\begin{aligned}
 \widetilde{Q_n}(0) &= u_n \\
 Q'_n &= \widehat{Q_{n-1}}(X+1) 
\end{aligned}
\right. 
\end{displaymath}
\begin{enumerate}
 \item Montrer qu'il existe une unique suite de polynômes vérifiant $E(u)$.
 \item Soit $\varepsilon$ la suite nulle: $\varepsilon_n=0$ pour tout $n\in \N^*$. Montrer que la suite $\left( P_n\right) _{n\in \N}$ avec 
\begin{displaymath}
 P_0=1,\hspace{0.5cm}\forall n\in \N^*:\;
P_n = \frac{1}{n!}X(X+n)^{n-1}
\end{displaymath}
vérifie $E(\varepsilon)$.
\item Soit $a\in \C$ et $\alpha = \left( \widetilde{P_n}(a)\right) _{n\in \N^*}$.\newline
Montrer que $\left( \widehat{P_n}(X+a)\right) _{n\in \N}$ vérifie $E(\alpha)$.
\item Soit $y\in \C$. Montrer qu'il existe une suite $u$ à déterminer telle que
\begin{displaymath}
 \left( \sum_{i=0}^{n}\widetilde{P_{n-i}}(y)P_i\right) _{n\in \N}
\end{displaymath}
vérifie $E(u)$. En déduire que:
\begin{displaymath}
 \widetilde{P_{n}}(x+y) = \sum_{i=0}^{n}\widetilde{P_{i}}(x)\widetilde{P_{n-i}}(y)
\end{displaymath}
pour tous $x, y$ complexes et $n$ naturel.
\end{enumerate}
