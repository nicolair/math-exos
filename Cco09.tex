Par symétries les distances entre les foyers et les asymptotes sont égales pour les deux foyers et asymptotes. Prenons l'asymptote d'équation
\begin{displaymath}
 \frac{x}{a}-\frac{y}{b}=1
\end{displaymath}
et le foyer de coordonnées $(c,0)$. La distance cherchée est
\begin{displaymath}
 \frac{\vert\frac{c}{a} \vert}{\sqrt{\frac{1}{a^2}+\frac{1}{b^2}}}
=\frac{cab}{a\sqrt{a^2+b^2}}=b
\end{displaymath}
car $c^2=a^2+b^2$.
