\begin{tiny}(Ccp17)\end{tiny} Les similitudes planes directes sont les fonctions complexes de la forme $z\mapsto az +b$ où $a$ et $b$ sont des complexes fixés. Les conditions se traduisent par un système
\begin{displaymath}
  \left\lbrace 
  \begin{aligned}
    b&= 1 \\ -ai + b &= \lambda
  \end{aligned}
\right. \Leftrightarrow
  \left\lbrace 
  \begin{aligned}
    b&= 1 \\ a &= i(\lambda-1)
  \end{aligned} \right.
\end{displaymath}
Soit $c$ l'affixe du centre d'une telle similitude, elle vérifie (en posant $\lambda = 1 +\tan \theta$)
\begin{multline*}
  i(\lambda -1) c +1= c \Leftrightarrow c = \frac{1}{1-i\tan \theta} = \frac{\cos \theta}{\cos \theta -i\sin \theta}\\
  = \cos \theta e^{i\theta} = \cos^2 \theta + i\cos \theta \sin \theta\\
  = \frac{1}{2} + \frac{1}{2}\cos(2\theta) +\frac{i}{2} \sin(2\theta)
  = \frac{1}{2}(1+e^{2i\theta})
\end{multline*}
L'ensemble des centres est donc le cercle de centre le point d'affixe $-1$ et de rayon $\frac{1}{2}$.