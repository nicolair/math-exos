\begin{tiny}(Cip30)\end{tiny} Définissons la fonction $G$ dans $[a,b]$ par:
\begin{displaymath}
 G(x)=\int_a^xg(t)\,dt
\end{displaymath}
C'est la primitive de $g$ nulle en $a$. On peut intégrer par parties:
\begin{displaymath}
 \int_a^bf(t)g(t)\,dt = \left[fG \right]_a^b-\int_a^bf'(t)G(t)\,dt 
\end{displaymath}
Comme $f'(t)\geq0$,
\begin{multline*}
 (f(b)-f(a))\min_{[a,b]}G\leq\int_a^bf'(t)G(t)\,dt \\
\leq (f(b)-f(a))\max_{[a,b]}G
\end{multline*}

D'après le théorème des valeurs intermédiaires appliqué à $G$, il existe $x\in[a,b]$ tel que
\begin{displaymath}
 \int_a^bf'(t)G(t)\,dt = (f(b)-f(a))G(x)
\end{displaymath}
Comme $G(a)=0$, on en déduit:
\begin{multline*}
\int_a^bf(t)g(t)\,dt = f(b)(G(b)-G(x)) +f(a)G(x) \\
= f(b)\int_x^bg(t)\,dt + f(a)\int_a^xg(t)\,dt
\end{multline*}
