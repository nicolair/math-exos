Soit $f$ une application qui n'est pas supposée linéaire mais qui conserve le produit scalaire. Pour tous vecteurs $x$ et $y$, développons 
\begin{displaymath}
  \Vert f(x+y)-f(x)-f(y)\Vert^2
\end{displaymath}
On obtient une somme de produits scalaires avec des $f$ de chaque côté que l'on peut enlever par conservation du produit scalaire. Cela conduit au développement de
\begin{displaymath}
  \Vert (x+y) -x -y \Vert^2 = 0
\end{displaymath}
On montre de même que 
\begin{displaymath}
  \Vert f(\lambda x) - \lambda x\Vert^2 = 0
\end{displaymath}
Une fois que l'on sait que $f$ est linéaire, on montre qu'elle est injective en considérant la norme d'un vecteur du noyau puis qu'elle est bijective car un espace euclidien est de dimension finie.