\begin{tiny}(Edi29)\end{tiny} Dans $E = \R^5$, la base canonique est notée $(e_1,e_2,e_3,e_4,e_5)$ et $(\varepsilon_1,\varepsilon_2,\varepsilon_3,\varepsilon_4,\varepsilon_5)$ la base duale des formes coordonnées dans la base canonique. Soit
\[
 \begin{aligned}
  \alpha_1 &= \varepsilon_1 + \varepsilon_2 - \varepsilon_3 + \varepsilon_4 + \varepsilon_5 \\
  \alpha_2 &= 2\varepsilon_1 + \varepsilon_2 + \varepsilon_3 - \varepsilon_5 \\
  \alpha_3 &= -\varepsilon_1 - 2\varepsilon_2 + \varepsilon_4 + 2\varepsilon_5
 \end{aligned}
\]
et $A = \ker \alpha_1 \cap \ker \alpha_2 \cap  \ker \alpha_3$. Soit
\[
 x = (x_1,x_2,x_3,x_4,x_5) \in E.
\]
Caractériser $x\in A$. Former une base de $A$. Les vecteurs seront exprimés dans la base canonique.\newline
Mêmes questions avec $A= \ker \alpha_1 \cap \ker \alpha_2$ et
\[
 \begin{aligned}
  \alpha_1 &= \varepsilon_1 -2 \varepsilon_2 + 2\varepsilon_3 + 3\varepsilon_5 \\
  \alpha_2 &= \varepsilon_1 + \varepsilon_2 + \varepsilon_3
 \end{aligned}
\]
