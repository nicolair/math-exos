\begin{tiny}(Cee14)\end{tiny} Développons la quantité que l'énoncé nous demande de considérer (vecteurs non nuls)
\begin{multline*}
 \left\Vert \Vert y\Vert^2x - (x/y)y \right \Vert^2\\
= \Vert y\Vert^4\Vert x\Vert^2 + (x/y)^2 \Vert y\Vert^2 -2\Vert y\Vert^2(x/y)^2\\
= \Vert y\Vert^2\left(\Vert y\Vert\Vert x\Vert -|(x/y)|\right)
\underset{> 0}{\underbrace{\left(\Vert y\Vert\Vert x\Vert +|(x/y)|\right)}} 
\end{multline*}

On en déduit l'inégalité de Cauchy-Schwarz. Dans le cas d'égalité, on retrouve bien que les vecteurs sont colinéaires.
