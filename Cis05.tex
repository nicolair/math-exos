\begin{tiny}(Cis05)\end{tiny} Notons
\begin{displaymath}
 I_2 = \int_{a}^{b}f^{2}, \;
 I_3 = \int_{a}^{b}f^{3}, \;
 I_4 = \int_{a}^{b}f^{4}
\end{displaymath}

et formons
\begin{displaymath}
 0=I_2-2I_3+I_4 = \int_{[a,b]}f^2(f^2-1)^2
\end{displaymath}
Comme la fonction à intégrer est continue et positive elle est identiquement nulle. Comme $f$ est continue, le théorème des valeurs intermédiaires montre que $f$ est constante de valeur $1$ ou $0$.