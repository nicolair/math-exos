\begin{tiny}(Ccu23)\end{tiny}
\begin{enumerate}
 \item Par définition, l'expression est du second degré en $t$ et positive pour tous les $t$ réels. On peut l'ordonner en $t$:
\begin{displaymath}
 \left( \sum_{k=1}^n y_k^2\right)t^2 + 2\left(\sum_{k=0}^n x_ky_k \right)t + \left( \sum_{k=1}^n x_k^2\right)
\end{displaymath}
Comme elle est toujours positive ou nulle, son discriminant est négatif ou nul ce qui donne l'inégalité demandée.

 \item On développe $(x_ky_l-x_ly_k)^2$ et on somme. Les termes peuvent se réécrire:
\begin{multline*}
 \sum_{(k,l)\in\llbracket 1,n \rrbracket^2}x_k^2y_l^2
 = \left( \sum_{k=1}^{n}x_k^2\right) \left( \sum_{l=1}^{n}y_l^2\right)\\
 = \left( \sum_{k=1}^{n}x_k^2\right) \left( \sum_{k=1}^{n}y_k^2\right)
\end{multline*}

On montre de même que
\begin{displaymath}
\sum_{(k,l)\in\llbracket 1,n \rrbracket^2}x_l^2y_k^2
= \left( \sum_{k=1}^{n}x_k^2\right) \left( \sum_{k=1}^{n}y_k^2\right)
\end{displaymath}
Pour les termes croisés:
\begin{multline*}
\sum_{(k,l)\in\llbracket 1,n \rrbracket^2}x_ky_kx_ly_l
= \left( \sum_{k=1}^{n}x_ky_k\right) \left( \sum_{l=1}^{n}x_ly_l\right)\\
= \left( \sum_{k=1}^{n}x_ky_k\right)^2
\end{multline*}

On en déduit la formule demandée.

 \item On déduit l'inégalite de Cauchy-Schwarz de la question précédente car la somme à gauche de l'égalité est positive.
\end{enumerate}
