\begin{tiny}(Csn10)\end{tiny} Remarquons que $u_n$ est la différence entre deux termes consécutifs d'une suite dont on connait l'expression. Calculons par télescopage la somme partielle
\[
  U_n =
  \sum_{k=1}^{n}u_k
  = \frac{(-1)^{n+1}}{\sqrt{n+1}} -\frac{(-1)}{1}.
\]
On en déduit que la série $\left( \sum u_n \right)$ converge avec 
\[
  \sum_{k=1}^{+\infty} u_k = 1.
\]
Notons que cette série semi convergente c'est à dire qu'elle n'est pas absolument convergente car
\[
  |u_n| \sim \frac{2}{\sqrt{n}}
\]
qui est le terme général d'une série de Riemann divergente. Cette équivalence montre aussi que $\frac{1}{n}$ est négligeable devant $u_n$ donc que $u_n \sim v_n$.\newline
Pour des séries qui ne sont pas de signe constant, cette équivalence ne signifie pas qu'elles sont de même nature.
\[
  \left.
  \begin{aligned}
  \left( \sum u_n \right) &\text{ Cvgte} \\    
  \left( \sum \frac{1}{n} \right) &\text{ Dvgte}
  \end{aligned}
\right\rbrace
\Rightarrow
\left( \sum (u_n + \frac{1}{n}) \right) \text{ Dvgte}.
\]
