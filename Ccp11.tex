\begin{tiny}(Ccp11)\end{tiny} \label{Ccp11} Notons $\mathcal{I}$ l'application inversion définie de $\C^*$ dans $\C^*$, $\mathcal{C}_a$ le cercle de centre le point d'affixe $a$ réel passant par l'origine et $\mathcal{T}$ l'image de ce cercle par l'inversion.\newline
L'application $\mathcal{I}$ est une \emph{involution}, c'est à dire que $\mathcal{I} \circ \mathcal{I} = \Id_{\C^*}$. On en déduit que
\begin{multline*}
  z\in \mathcal{T} \Leftrightarrow \mathcal{I}(z) \in \mathcal{C}_a
  \Leftrightarrow \left|\frac{1}{z} - a\right| = |a| \\
  \Leftrightarrow\left|1-az\right|^2 = \left|az\right|^2
  \Leftrightarrow 1-a\Re(z) = 0
\end{multline*}
L'image cherchée est donc formée par les complexes dont la partie réelle est $\frac{1}{a}$.
