% Fichier fr.tex, v1.1b 17/01/97  (ML - lavaud@univ-orleans.fr)
% Adaptation de: frenchy.tex (D. Flipo), francais.sty (J. Braams),
% french.sty (B. Gaulle), german.sty (H. Partl), esperant.sty (J. Knappen)...
% Cf: Documentation du style Babel de J. Braams;
% A guide to LaTeX, de H. Kopka et P. Daly,
% (Addison-Wesley), section "Adapting LaTeX styles"
%
%--- Empecher de charger ce fichier deux fois
  \let\this\relax
  \ifx\captionsfrench\undefined\relax
  \else\let\this\endinput
  \fi
  \this

%--- Afficher le numero de version
\def\Msg{\immediate\write16}
\Msg{ Style fr, v.1.1a, 24/11/95}

%--- Le caractere @ est une lettre localement
 \chardef\FRatCat=\the\catcode`\@
 \catcode`\@=11 % \makeatletter

%--- Definitions par defaut des commandes \FR and \US
%    pour selectionner les motifs de cesures a utiliser
%    Normalement, USenglish est la langue de numero 0
%    et le francais est la langue de numero 1. 

\def\US{\language=0     % La langue USenglish a le numero 0
        \lccode`\'=0  } % L'apostrophe n'est pas une lettre en anglais
\def\FR{\language=1     % La langue French a le numero 1
        \lccode`\'=`\'} % L'apostrophe est une lettre en francais

%--- Traitement de la ponctuation double (; : ! ?)
\catcode`\;=\active
\catcode`\:=\active
\catcode`\!=\active
\catcode`\?=\active
%
\gdef\punctenglish{\def;{\string;}\def:{\string:}\def!{\string!}\def?{\string?}}%
\punctenglish
%
% - Pour eviter des problemes avec la ponctuation double active
%    dans \special et \message
\global\let\@specialORI\special
\gdef\special#1{\bgroup\punctenglish
                \protect\@specialORI{#1}\egroup}
\global\let\@messageORI\message
\gdef\message#1{\bgroup\punctenglish
                \protect\@messageORI{#1}\egroup}
%
% - Pour l'environnement verbatim
\global\let\@dsORI\dospecials
\gdef\dospecials{\@dsORI\do\!\do\?\do\;\do\:}%
\gdef\@EF{\penalty\@M\hskip 0.17em plus 0.08 em minus 0.03 em}
%
% Point virgule
\gdef\@PV{\relax\ifhmode
     \ifdim\lastskip>\z@\unskip\fi  % skip space before (if any);
     \@EF               % add a thin space (no line break)
     \fi\string;}
%
% Point d'exclamation
\gdef\@PE{\relax\ifhmode
     \ifdim\lastskip>\z@\unskip\fi
     \@EF
     \fi\string!}
%
% Point d'interrogation
\gdef\@PI{\relax\ifhmode
     \ifdim\lastskip>\z@\unskip\fi
     \@EF
     \fi\string?}
%
% Deux points
\gdef\@DP{\relax\ifhmode
     \ifdim\lastskip>\z@\unskip\fi
     \penalty\@M\ \fi           % add an normal space (no line break)
     \string:}
%
% Attention: ne pas rajouter de blanc au debut des lignes de ce groupe
\gdef\punctfrench{% 
\def;{\protect\@PV{}}%
\def!{\protect\@PE{}}%
\def?{\protect\@PI{}}%
\def:{\protect\@DP{}}%
}
%
%--- Dates
%    Date en francais
\gdef\datefrench{\def\today{\number\day 
  \ifnum1=\day \up{\lowercase{er}}\fi% \lowercase for uppercase headings
  \space\ifcase\month\or
  janvier\or f\'evrier\or mars\or avril\or mai\or juin\or
  juillet\or ao\^ut\or septembre\or octobre\or novembre\or d\'ecembre\fi
  \space\number\year}}
%
%   Date en anglais US
\gdef\dateUSenglish{\def\today{\ifcase\month\or
  January\or February\or March\or April\or May\or June\or
  July\or August\or September\or October\or November\or December\fi
  \space\number\day, \number\year}}
%
%   Date en anglais
%\gdef\dateenglish{\def\today{\number\day \ifcase\day \or
%  st\or nd\or rd\or th\or th\or th\or th\or th\or th\or th\or %  1..10
%  th\or th\or th\or th\or th\or th\or th\or th\or th\or th\or % 11..20
%  st\or nd\or rd\or th\or th\or th\or th\or th\or th\or th\or % 21..30
%  st\fi
%  ~\ifcase\month\or
%  January\or February\or March\or April\or May\or June\or
%  July\or August\or September\or October\or November\or December\fi
%  \space \number\year}}
%
%--- Symboles speciaux du francais (guillemets, degre, numero)
%    Guillemets 
\newbox\charCWM
% - French guillemets for Plain TeX with CMR fonts (two different shapes)
%
%   1ere forme pour Plain (R. Seroul) : 
\gdef\oguill{\leavevmode\raise0.3ex%
         \hbox{{$\scriptscriptstyle\langle\!\langle\,$}\nobreak\ignorespaces}}
\gdef\fguill{\ifdim\lastskip>\z@\unskip\penalty\@M\fi
         \leavevmode\raise0.3ex%
         \hbox{{$\scriptscriptstyle\,\rangle\!\rangle$}}}
%
%   2e forme pour Plain
\gdef\leftguill{\leavevmode\raise0.25ex%
         \hbox{{$\scriptscriptstyle\ll\,$}\nobreak\ignorespaces}}
\gdef\rightguill{\ifdim\lastskip>\z@\unskip\penalty\@M\fi
         \leavevmode\raise0.25ex%
         \hbox{{$\scriptscriptstyle\,\gg$}}}
%  Utiliser la premiere forme par defaut
\global\let\oPlainGuill\oguill
\global\let\fPlainGuill\fguill
%\global\let\oPlainGuill\leftguill
%\global\let\fPlainGuill\rightguill
%
%   Guillemets avec fontes LaTeX lasy
\gdef\oLaTeXGuill{\leavevmode\hbox{\ly(\kern-0.20em(\kern+0.20em}%
                  \nobreak\ignorespaces}
\gdef\fLaTeXGuill{\ifdim\lastskip>\z@\unskip\penalty\@M\fi
                  \leavevmode\hbox{\kern+0.20em\ly)\kern-0.20em)}}
%
%   Guillemets avec fontes dc
\gdef\oECGuill{\leavevmode\hbox{\char19\kern 1.5pt\nobreak\ignorespaces}}
\gdef\fECGuill{\ifdim\lastskip>\z@\unskip\penalty\@M\fi
               \leavevmode\hbox{\kern 1.5pt \char20}}
%
% -  Choix des guillemets en fonction des fontes et du paquet de macros utilises
\gdef\beginguillemets{%
   \ifx\ly\undefined
       \oPlainGuill
   \else
       \setbox\charCWM=\hbox{\char23}% % width is 0pt in DC/EC fonts (<cwm>)
       \ifdim\wd\charCWM=0pt
          \oECGuill % Are EC/DC fonts available ?
       \else
           \ifx\DeclareFontShape\undefined
               \oLaTeXGuill
           \else
               \oPlainGuill
           \fi
       \fi
   \fi
  }
\gdef\endguillemets{%
   \ifx\ly\undefined
      \fPlainGuill
   \else
      \setbox\charCWM=\hbox{\char23}% % width is 0pt in DC/EC fonts (<cwm>)
      \ifdim\wd\charCWM=0pt
         \fECGuill % Are EC/DC fonts available ?
      \else
         \ifx\DeclareFontShape\undefined
            \fLaTeXGuill
         \else
            \fPlainGuill
         \fi
      \fi
   \fi
  }
%
% Alias pour les guillemets
\let\og\beginguillemets
\let\fg\endguillemets
% Pour compatibilite avec Babel 
\let\flqq\beginguillemets
\let\frqq\endguillemets
%
% - \degre et \No 
\gdef\degre{%
    \ifx\postscript@font\undefined      % CMR or DC/EC fonts 
        \setbox\charCWM=\hbox{\char23}% % width is 0pt in DC/EC fonts (<cwm>)
        \kern-.15em\lower.2ex\hbox{%
                \ifdim\wd\charCWM=0pt \char06 \else \char23 \fi\kern-.55em}
    \else \char23 \fi}                  % PS fonts are in use
\def\No{N\degre}
%
%--- Definitions pour Plain TeX ou LaTeX
% -  Plain TeX
\ifx\documentstyle\undefined
 \global\let\captionsfrench\relax
 \global\let\captionsenglish\relax
 \global\let\itemsfrench\relax
 \global\let\itemsenglish\relax
 \global\let\protect\relax
% Two possibilities to typeset  Mme, 1er..
 \gdef\up#1{\raise 1ex\hbox{\sevenrm #1}}%
 \gdef\fup#1{\leavevmode\raise+0.55ex\hbox{\sevenrm#1}\kern+.17em}%
% Aliases for french guillemets
 \global\let\og\beginguillemets
 \global\let\fg\endguillemets
% - LaTeX
\else
 \global\let\@saORI\@sanitize
 \gdef\@sanitize{\@saORI\@makeother\!\@makeother\?\@makeother\;\@makeother\:}%
 \global\let\@typeoutORI\typeout
 \gdef\typeout#1{\bgroup\punctenglish
                 \protect\@typeoutORI{#1}\egroup}%
%                                          \label
\global\let\@labORI\label%
\gdef\label#1{\bgroup\punctenglish%
\protect\@labORI{#1}\egroup}%
%                                          \newlabel
\global\let\@nlabORI\newlabel%
\gdef\newlabel#1#2{\bgroup\punctenglish%
\protect\@nlabORI{#1}{#2}\egroup}%
%                                          \ref
\global\let\@refORI\ref%
\gdef\ref#1{\bgroup\punctenglish%
\protect\@refORI{#1}\egroup}%
%                                          \pageref
\global\let\@prefORI\pageref%
\gdef\pageref#1{\bgroup\punctenglish%
\protect\@prefORI{#1}\egroup}%
%                                          \@citex
\global\let\@@citexORI\@citex%
\gdef\@citex[#1]#2{\bgroup\punctenglish%
\protect\@@citexORI[#1]{#2}\egroup}%
%                                          \nocite
\global\let\@nociteORI\nocite%
\gdef\nocite#1{\bgroup\punctenglish%
\protect\@nociteORI{#1}\egroup}%
%                                          % \bibitem
% \global\let\@bibitemORI\bibitem%
% \gdef\bibitem#1#2{\bgroup\punctenglish%
% \protect\@bibitemORI{#1}{#2}\egroup}%
%                                          % \@bibitem
% \global\let\@@bibitemORI\@bibitem%
% \gdef\@bibitem#1{\bgroup\punctenglish%
% \protect\@@bibitemORI{#1}\egroup}%
%                                          % \@lbibitem
% \global\let\@@lbibitemORI\@lbibitem%
% \gdef\@lbibitem[#1]#2{\bgroup\punctenglish%
% \protect\@@lbibitemORI[#1]{#2}\egroup}%
%                                          \bibcite
\global\let\@bibciteORI\bibcite%
\gdef\bibcite#1#2{\bgroup\punctenglish%
\protect\@bibciteORI{#1}{#2}\egroup}%
%                                          % \@testdef
% \global\let\@@testdefORI\@testdef
% \gdef\@testdef#1#2{\bgroup\punctenglish
% \protect\@@testdefORI{#1}{#2}\egroup}
%
% Pour eviter le message \endcsname...
\global\let\enddocumentORI\enddocument
\gdef\enddocument{\punctenglish\enddocumentORI}


 \gdef\@up#1{\raise 1ex\hbox{\small#1}}%
 \gdef\up#1{\protect\@up{#1}}
 \gdef\@fup#1{\leavevmode\raise+0.55ex\hbox{\small#1}\kern+.17em}%
 \gdef\fup#1{\protect\@fup{#1}}
% Define \ly in new LaTeX (LaTeX + nfss1)
 \ifx\ly\undefined
    \gdef\ly{\fontfamily{lasy}\fontseries{m}\fontshape{n}\selectfont}
 \fi
%
%   Definitions speciales pour LaTeX
 \gdef\captionsfrench{%
%   Definitions ci-dessous pour les classes: article, report et book
% Attention: ne pas rajouter de blanc au debut des lignes de ce groupe
\def\pagename{page}%
\def\refname{R\'ef\'erences}%
\def\abstractname{R\'esum\'e}%
\def\bibname{Bibliographie}%
\def\contentsname{Table des mati\`eres}%
\def\listfigurename{Liste des figures}%
\def\listtablename{Liste des tableaux}%
\def\indexname{Index}%
\def\seename{{\em voir}}% used normally in makeidx.sty
\def\figurename{{\sc Fig.}}%
\def\tablename{{\sc Tab.}}%
\def\partname{\protect\@Fpt partie}% "Premi\`ere partie" instead of "Part I"
\def\@Fpt{{\ifcase\value{part}\or
   Premi\`ere\or Deuxi\`eme\or Troisi\`eme\or Quatri\`eme\or Cinqui\`eme\or
   Sixi\`eme\or Septi\`eme\or Huiti\`eme\or Neuvi\`eme\or Dixi\`eme\or
   Onzi\`eme\or Douzi\`eme\or Treizi\`eme\or Quatorzi\`eme\or Quinzi\`eme\or
   Seizi\`eme\or Dix-septi\`eme\or Dix-huiti\`eme\or Dix-neuvi\`eme\or
   Vingti\`eme\fi}\space\def\thepart{}}%
%
%   Definitions ci-dessous seulement pour les classes report et book
\def\chaptername{Chapitre}%
\def\appendixname{Annexe}%
%
%   Definitions ci-dessous seulement pour le style lettre
\def\headtoname{\`a}%
\def\enclname{P.j. }% Pieces jointes
\def\PSname{P.-S. \string:{}}% Post-Scriptum
\def\Objectname{Objet \string:{}}% Object de la lettre
\def\ccname{Copie \`a }% copie conforme
\def\YourRefname{V/r\'ef. \string:{}}% Votre numero de reference
\def\OurRefname{N/r\'ef. \string:{}}% Notre numero de reference
\def\dateprefix{le\ }%
\def\telabbrev{T\'el. :}%
\def\faxabbrev{Fax :}%
\def\emailabbrev{Email :}%
}%
%
% Attention: ne pas rajouter de blanc au debut des lignes de ce groupe
 \gdef\captionsenglish{%
\def\pagename{page}%
\def\refname{References}%
\def\abstractname{Abstract}%
\def\bibname{Bibliography}%
\def\contentsname{Table of Contents}%
\def\listfigurename{List of Figures}%
\def\listtablename{List of Tables}%
\def\indexname{Index}%
\def\seename{{\em see}}% used normally in makeidx.sty
\def\figurename{Figure}%
\def\tablename{Table}%
\def\partname{Part}%
%
% The following is only for report and book styles
\def\chaptername{Chapter}%
\def\appendixname{Appendix}%
%
% The following is only for the letter style
\def\headtoname{To}%
\def\ccname{Cc:}%
\def\enclname{Encl:}%
\def\PSname{PS:}% Post-Scriptum
\def\Objectname{Object:}% Object of the letter
\def\YourRefname{Your Ref:}% Your reference number
\def\OurRefname{Our Ref:}% Our reference number
\def\dateprefix{}%
\def\telabbrev{Tel.:}%
\def\faxabbrev{Fax:}%
\def\emailabbrev{Email:}%
}
%   Listes
%    A la francaise
 \gdef\itemsfrench{%
   \def\labelitemi{--}%
   \def\labelitemii{--}%
   \def\labelitemiii{--}%
   \def\labelitemiv{--}%
 }
%   A l'anglaise
 \gdef\itemsenglish{%
   \def\labelitemi{$\m@th\bullet$}%
   \def\labelitemii{\bf --}%
   \def\labelitemiii{$\m@th\ast$}%
   \def\labelitemiv{$\m@th\cdot$}%
 }
\fi
%
%--- Definitions des commandes principales \french et \english
% -  Anglais 
\def\english{\US
\captionsenglish
\dateUSenglish
\punctenglish
\itemsenglish
\nonfrenchspacing
}
%
%  - Francais
\def\french{\FR
\captionsfrench
\datefrench
\punctfrench
% \itemsfrench
% \frenchspacing
}
%--- Le francais est la langue par defaut
\french
%\english

\catcode`\@=\FRatCat % restore catcode

\endinput
