\begin{tiny}(Efr06)\end{tiny} Valuations d'une fraction rationnelle.\\
Pour tout $a\in \C$ et $F\in \C(X)$, on définit $v_a(F)\in\Z$ (\emph{valuation} de $F$ en $a$) de la manière suivante :
\begin{itemize}
 \item $v_a(F)=0$ si $a$ n'est ni un pôle ni un zéro de $F$.
 \item $v_a(F)=m$ si $a$ est un zéro de $F$ de multiplicité $m$.
 \item $v_a(F)=-m$ si $a$ est un pôle de $F$ de multiplicité $m$.
\end{itemize}
\begin{enumerate}
 \item Montrer que $v_a(F)=m$ si et seulement si il existe des polynômes $P$ et $Q$ tels que $F=(X-a)^m\frac{P}{Q}$ avec $\widetilde{P}(a)\neq0$ et $\widetilde{Q}(a)\neq0$.
 \item Soit $F$ et $G$ deux fractions rationnelles. Montrer que $v_a(FG)=v_a(F)+v_a(G)$.\\ Montrer que $v_a(F+G)\geq \min(v_a(F),v_a(G))$ avec égalité lorsque $v_a(F)\neq v_a(G)$.
 \item Soit $F$, $G_1$, $G_2$ trois fractions rationnelles vérifiant
\begin{displaymath}
 F^2 + G_1F+G_2 = 0_{\C(X)}
\end{displaymath}
Montrer qu'un pôle de $F$ est soit un pôle de $G_1$ soit un pôle de $G_2$.
\end{enumerate}

 