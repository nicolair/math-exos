\begin{tiny}(Car11)\end{tiny} Les conditions imposées à $P$ entraînent que $1$ est une racine de $P -1$ et que $-1$ est une racine de $P+1$ avec des multiplicités au moins $4$. En traduisant la multiplicité sur les dérivées, on déduit que $1$ et $-1$ sont des racines de $P'$ de multiplicité au moins $3$. Un théorème de cours assure alors que 
\begin{displaymath}
 (X-1)^3(X+1)^3 = (X^2-1)^3
\end{displaymath}
divise $P'$. On en déduit que si $P$ est de degré $7$, il doit vérifier
\begin{displaymath}
 P' = \lambda (X^2-1)^3 \text{ avec } \lambda\in \R
\end{displaymath}
Il doit donc être de la forme
\begin{displaymath}
 P = \mu + \lambda X +\lambda X^3 +\frac{3}{5}\lambda X^5 + \frac{1}{7}\lambda X^7
\end{displaymath}
et vérifier de plus 
\begin{displaymath}
 \widetilde{P}(1) = -1 \hspace{0.5cm} \widetilde{P}(-1) = 1  
\end{displaymath}
Le système conduit à l'unique solution
\begin{displaymath}
 P = -\frac{35}{96}\left( X + X^3 +\frac{3}{5} X^5 + \frac{1}{7} X^7\right) 
\end{displaymath}
