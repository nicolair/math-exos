\begin{tiny}(Eva21)\end{tiny} Un ascenseur dessert un bâtiment de $e$ étages. Il démarre du rez de chausssée avec $N$ personnes en montant toujours et tout le monde en sort. Les distributions d'étages et de personnes sont équiprobables. Comment modéliser mathématiquement l'expérience? Soit $X$ la variable égale au numéro de l'étage du premier arrêt. Calculer $\p(X>k)$ et en déduire une expression de $E(X)$.  Considérons, pour $N$ fixé, $E(X)$ comme une suite en $e$. Montrer qu'elle diverge vers $+ \infty$ et préciser une suite équivalente.\newline
Reprendre des questions analogues avec la variable $Y$ égale au plus grand étage atteint par l'ascenseur.