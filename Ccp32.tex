\begin{tiny}(Ccp32)\end{tiny} Il est évident que si le triangle est équilatéral, le centre du cercle circonscrit coïncide avec l'isobarycentre.\newline
Réciproquement, à une similitude près, on peut supposer que les trois points sont sur le cercle unité et que l'affixe de l'un d'entre eux est égal à $1$.\newline
On considère donc trois points d'affixes 
\begin{displaymath}
 z_1=e^{i\alpha_1},\; z_2 = e^{i\alpha_2},\; z_3 = 1 
\end{displaymath}
et vérifiant
\begin{displaymath}
 e^{i\alpha_1} + e^{i\alpha_2} + 1 = 0 
\end{displaymath}
On en déduit
\begin{displaymath}
 e^{i\frac{\alpha_1 + \alpha_2}{2}} 2 \cos\left(\frac{\alpha_1 - \alpha_2}{2} \right)  + 1 = 0 
\end{displaymath}
Donc
\begin{multline*}
 e^{i\frac{\alpha_1 + \alpha_2}{2}}\in \R
\Rightarrow 
\frac{\alpha_1 + \alpha_2}{2} \equiv 0 \mod \pi \\
\Rightarrow  \alpha_1 + \alpha_2 \equiv 0 \mod 2\pi
\Rightarrow z_2 = \overline{z_1}
\end{multline*}
On en tire facilement avec la relation et la valeur de $j$ que
\begin{displaymath}
 \left( z_1=j, z_2 = j^2\right) \text{ ou } \left( z_1=j^2, z_2 = j\right) 
\end{displaymath}
ce qui montre que le triangle est équilatéral.