\begin{tiny}(Cdi37)\end{tiny} Pour majorer la valeur absolue, écrivons
\begin{multline*}
  f = (f+g) - g 
  \Rightarrow \Im(f) \subset \Im(f+g) + \Im(g)\\
  \Rightarrow \rg(f) \leq \dim\left( \Im(f+g) + \Im(g)\right) \\
  \leq \rg(f+g) + \rg(g)
\end{multline*}
On en déduit une première inégalité, on obtient l'autre en échangeant les rôles de $f$ et $g$.\newline
Supposons
\begin{displaymath}
\left\lbrace 
\begin{aligned}
    &E = \ker f + \ker g \\ &\Im(f) \cap \Im(g) = \left\lbrace 0_F\right\rbrace 
  \end{aligned}
  \right.  
\end{displaymath}
Montrons
\begin{displaymath}
  \Im(f) + \Im(g) \subset \Im(f+g)
\end{displaymath}
En effet, pour tout $y \in \Im(f) + \Im(g)$, il existe $x_f$ et $x_g$ tels que
\begin{displaymath}
  y = f(x_f) + g(x_g)
\end{displaymath}
Chacun se décompose dans la somme de noyau et seul \og l'autre\fg~ noyau contribue. Il existe $a\in \ker(g)$ et $b\in \ker(f)$ tels que
\begin{displaymath}
  y = f(a) + g(b) = (f+g)(a+b)
\end{displaymath}
car $f(b)=g(a)=0$.\newline
On en déduit 
\begin{displaymath}
  \Im(f) + \Im(g) = \Im(f+g)
\end{displaymath}
car l'autre inclusion est évidente puis
\begin{displaymath}
  \rg(f) + \rg(g) = \rg(f+g)
\end{displaymath}
car la somme est directe d'après la deuxième hypothèse.\newline
Supposons 
\begin{displaymath}
  \rg(f+g) = \rg(f) + \rg(g)
\end{displaymath}
Alors, d'après le théorème du rang
\begin{multline*}
\dim(E) - \dim\left( \ker(f+g)\right)\\
= 2\dim(E)- \dim(\ker(f)) - \dim(\ker(g)) \\
\Rightarrow
\dim(\ker(f)) + \dim(\ker(g)) \\
= \dim(E) + \dim\left( \ker(f+g)\right) \\
\Rightarrow
\dim\left( \ker(f) + \ker(g)\right)
=\dim(E) \\
+ \underset{\geq 0}{\underbrace{\dim\left( \ker(f+g)\right) - \dim\left( \ker(f) \cap \ker(g)\right) }}
\end{multline*}
On en déduit $\ker(f) + \ker(g) = E$ et $\ker(f+g) = \ker(f) \cap \ker(g)$. \newline
Il reste à montrer que l'intersection des images se réduit au vecteur nul. Or
\begin{displaymath}
  \Im(f+g) \subset \Im(f) + \Im(g)
\end{displaymath}
avec
\begin{multline*}
  \dim(\Im(f+g)) \leq \dim(\Im(f) + \Im(g)) \\
  \leq \dim(\Im(f)) + \dim(\Im(g))
\end{multline*}
On en déduit $\Im(f+g) = \Im(f) + \Im(g)$ et 
\begin{displaymath}
\dim(\Im(f) + \Im(g))
=\dim(\Im(f)) + \dim(\Im(g))
\end{displaymath}
ce qui entraine que les images sont en somme directe.
