\begin{tiny}(Cpb18)\end{tiny} On utilise l'événement contraire:
\[
  y = 1 - \p(A \cup B)
  = 1 -\p(A) - \p(B) + \underset{=x}{\underbrace{\p(A\cap B)}}
\]
On décompose $A$ et $B$:
\begin{multline*}
\left.
  \begin{aligned}
    A &= (A\cap B) \cup (A \cap \overline{B}) &\Rightarrow \p(A) = x + a\\
    B &= (A\cap B) \cup (B \cap \overline{A}) &\Rightarrow \p(B) = x + b
  \end{aligned}
\right\rbrace \\ \Rightarrow
y = 1 - x -a -b.
\end{multline*}
Avec les mêmes relations
\begin{multline*}
  \p(A \cap B) -\p(A) \p(B)
  = x -(x+a)(x+b) \\
  = -x^2 + \underset{=x+y}{\underbrace{(1-a-b)}} x -ab
  = xy - ab.
\end{multline*}
La première relation conduit à des inégalités
\[
  a + b + x + y = 1
  \Rightarrow
  \left\lbrace
  \begin{aligned}
    a + b &\leq 1 \\ x + y &\leq 1
  \end{aligned}
  \right.
\]
car $a$, $b$, $x$, $y$ positifs. On en déduit
\[
\p(A \cap B) -\p(A) \p(B)
\left\lbrace
\begin{aligned}
   &\leq& xy  \leq x(1-x)  &\leq& \frac{1}{4} \\
   &\geq& -ab \geq -a(1-4) &\geq& -\frac{1}{4}
\end{aligned}
\right. .
\]
Pour l'étude des cas d'égalité, on suppose que la probabilité d'un singleton est toujours non nulle.\newline
Cas d'égalité à $\frac{1}{4}$ si et seulement si $ab =0$ et $x = y = \frac{1}{2}$.
\[
\begin{aligned}
  a= 0 &\Leftrightarrow \p(A \cap \overline{B}) = 0 \Leftrightarrow A \subset B\\
  b= 0 &\Leftrightarrow \p(B \cap \overline{A}) = 0 \Leftrightarrow B \subset A
\end{aligned}
\]
Examinons l'autre condition dans un des deux cas
\begin{multline*}
  A \subset B \Rightarrow
  \left\lbrace
  \begin{aligned}
    A\cap B &= A \\ \overline{A} \cap \overline{B} &= \overline{B}
  \end{aligned}
\right. \\
\Rightarrow \p(A) = \p(B) = \frac{1}{2} \text{ et } A = B.
\end{multline*}
Le raisonnement est analogue si $B\subset A$.\newline
Cas d'égalité à $-\frac{1}{4} \Leftrightarrow xy =0$ et $a = b = \frac{1}{2}$.
\begin{multline*}
  x = 0
  \Rightarrow A \cap B = \emptyset 
  \Rightarrow 
  \left\lbrace
  \begin{aligned}
    A \subset \overline{B} \\
    B \subset \overline{A}
  \end{aligned}
  \right. \\
  \Rightarrow a = \p(A)=  b = \p(B)=\frac{1}{2}  \text{ et } B = \overline{A}.
\end{multline*}

