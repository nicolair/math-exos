\begin{tiny}(Eaz08)\end{tiny} Soit $G$ un groupe commutatif fini de cardinal $n=pq$ avec $p$ et $q$ premier entre eux. On admet que $g^n=e$ pour tout $g$ dans $G$ (théorème de Lagrange \href{http://back.maquisdoc.net/data/temptex/fexal.pdf}{al07}). On note $G_p$ (respectivement   $G_q$) l'ensemble des $g$ de $G$ dont l'ordre divise $p$ (respectivement $q$).
\begin{enumerate}
 \item Montrer que $G_p$ et $G_q$ sont des sous-groupes de $G$.
 \item Montrer que l'application
\begin{displaymath}
 \left\lbrace
\begin{aligned}
 G_p\times G_q 	&\rightarrow G\\
 (a,b) &\rightarrow ab
\end{aligned}
 \right. 
\end{displaymath}
est un isomorphisme de groupe. En déduire que $\sharp G_p=p$ et $\sharp G_q = q$. 
\end{enumerate}
 