\begin{tiny}(Ecg04)\end{tiny}
\textbf{Norme attachée à un convexe symétrique - jauge}.\newline
Soit $\Omega$ une partie d'un plan vectoriel réel convexe et symétrique par rapport à l'origine. On suppose de plus que pour tout $z$ non nul du plan, il existe un unique réel $\lambda>0$ tel que
\[\Omega \cap \R_+z=\left[ O,\lambda z \right] \]
La demi-droite d'origine $O$ et passant par $z$ est désignée par $\R_+z$. On pose
\begin{displaymath}
N_\Omega(z)=\frac{1}{\lambda}, \hspace{0.5cm} N_\Omega(O)=0 
\end{displaymath}
Montrer que $N_\Omega$ est une norme. Pour l'inégalité triangulaire, 
\begin{displaymath}
 N_\Omega(z_1 + z_2) \leq N_\Omega(z_1) + N_\Omega(z_2)
\end{displaymath}
on pourra utiliser la figure \ref{fig:Ecg04_1}, introduire le point d'intersection $u$ du segment $[u_1,u_2]$ avec la demi-droite $\R_+ Z$ et calculer le réel $\mu$ tel que $\mu u = Z$.
\begin{figure}[ht]
\centering
\input{Ecg04_1.pdf_t}
\caption{inégalité triangulaire}
\label{fig:Ecg04_1}
\end{figure}