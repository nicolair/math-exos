\begin{tiny}(Eli19)\end{tiny} Fonctions localement symétriques.\newline
Soit $f$ une fonction définie dans $\R$. Pour chaque $a\in \R$, on définit une partie $A_a \subset \left[ 0, + \infty\right[$ par:
\begin{multline*}
  \forall \alpha >0, \;\alpha \in A_a \\
  \Leftrightarrow \left( \forall h \in \R, \;\left( |h| \leq \alpha \Rightarrow f(a+h) = f(a-h)\right) \right).
\end{multline*}

On introduit les définitions et notations suivantes
\begin{itemize}
  \item $f$ est localement symétrique en $a\in \R$ si et seulement si $A_a \neq \emptyset$. On note $f$ ls en $a$.
  \item $f$ est localement symétrique dans $\R$ si et seulement si, $\forall a\in \R$, $f$ est ls en $a$. On note $f$ ls dans $\R$.
  \item $f$ est uniformément localement symétrique si et seulement si
  \begin{multline*}
    \exists \alpha >0 \text{ tq } \forall (a,h) \in \R^2, \\ 
    |h| \leq \alpha \Rightarrow f(a+h) = f(a-h).
  \end{multline*}

  On note $f$ uls.
\end{itemize}
Montrer les implications suivantes
\begin{enumerate}
  \item $f$ ls en $a$ et dérivable en $a$ entraine $f'(a) = 0$.
  \item $f$ uls entraine $f$ constante. On proposera deux démonstrations: une directe en utilisant une subdivision régulière et une autre par contraposition avec une construction dichotomique.
  \item On suppose $f$ continue et ls dans $\R$.
  \begin{enumerate}
    \item $A_a$ majoré entraine $A_a$ admet un plus grand élément.
    \item $\forall a\in \R, \;A_a$ non majoré.
    \item $f$ constante.
  \end{enumerate}

\end{enumerate}
