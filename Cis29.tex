\begin{tiny}(Cis29)\end{tiny} On suppose $f$ $k$-lipschitzienne avec $k>0$. Elle est continue et ne s'annule pas: elle garde donc un signe constant. Soit $m = \inf |f|$. Il existe $x_{min}$ tel que $m = f(x_{min} > 0$ car elle est définie sur un segment. Alors
\[
  \left| \frac{1}{f(y)} - \frac{1}{x}\right| = \frac{\left|f(x) - f(y)\right|}{\left|f(x) f(y)\right|}
  \leq \frac{k}{m^2} \left|x - y\right|.
\]
