\begin{tiny}(Emo07)\end{tiny} D{\'e}composition $LU$.\label{exo Emo07}\newline
Soit $A\in \mathcal{M}_{m}(\K)$. On suppose que l'algorithme du pivot partiel (I) conduit {\`a} une matrice triangulaire sup{\'e}rieure avec des termes non nuls sur la diagonale \emph{sans qu'il soit n{\'e}cessaire de permuter les lignes}. (Voir l'exercice \ref{exo Emo25} sur cette condition.)\newline
Montrer qu'il existe une matrice triangulaire inf{\'e}rieure $L$ et une matrice triangulaire sup{\'e}rieure $U$  telles que $A=LU$. Expliciter $L$ et $U$ pour
\begin{displaymath}
A=\begin{pmatrix}
5 & 2 & 1 \\
5 & -6 & 2 \\
-4 & 2 & 1    
  \end{pmatrix}.
\end{displaymath}

