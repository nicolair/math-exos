\begin{tiny}(Cmo20)\end{tiny} D'après le résultat cité, chaque hyperplan matriciel est associé à une matrice $A$ non nulle définissant son équation avec la trace. Pour montrer que l'hyperplan matriciel contient une matrice inversible, il suffit de montrer que : pour toute matrice non nulle $A$ il existe $M \in GL_p(\K)$ telle que $\tr(AM) = 0$.\newline
Soit $r = \rg(A)>0$. Il existe des matrices inversibles $P$ et $Q$, produit de matrices élémentaires telles que
\[
 A = Q\,J_r\,P.
\]
Comme la trace se conserve par permutation,
\[
 \tr(AM) = \tr\left( J_r(P\, M\, Q)\right) 
\]
Il suffit donc de trouver une matrice inversible $S$ telle que $\tr(J_r\,S) = 0$.\newline
Si $r\geq 2$, on peut trouver $S$ diagonale. Si $r = 1$, on prend une matrice diagonale par bloc avec
\[
 \begin{pmatrix}
  0 & 1 \\ 1 & 0
 \end{pmatrix}
\]
en haut à gauche et $I_{p-2}$ en bas à droite.
