\begin{tiny}(Ere17)\end{tiny} Soit $\left( x_n\right) _{n\in \N}$ une suite de nombres réels. On définit, pour tout $n\in \N$:
\begin{align*}
 &d_n = x_{n+1} - x_n &
 &a_n = d_n^+ &
 &b_n = d_n^- \\
 &A_n = \sum_{k=0}^n a_k &
 &B_n = \sum_{k=0}^n b_k &
 &D_n = \sum_{k=0}^n |d_k|
\end{align*}
Montrer que $\left( x_n\right) _{n\in \N}$ est la différence de deux suites croissantes. Montrer que la convergence de $\left( D_n\right) _{n\in \N}$ entraine celle de $\left( x_n\right) _{n\in \N}$.\newline
voir l'exercice sur \href{http://back.maquisdoc.net/v-1/index.php?act=chelt&id_elt=4860}{la définition de l'exponentielle complexe}.