\begin{tiny}(Ega10)\end{tiny} Soit $E$ un $\R$-espace vectoriel de dimension $2$. On rappelle que l'ensemble des fonctions de $E$ dans $\R$ est un $\R$-espace vectoriel noté $\mathcal{F}(E,\R)$. L'ensemble $E^*$ des formes linéaires est un sous-espace vectoriel de $\mathcal{F}(E,\R)$. Pour tout $\lambda\in \R$, on note $\overline{\lambda}$ la fonction constante de valeur $\lambda$ de $E$ dans $\R$. On note ici $\overline{\R}$ l'ensemble des fonctions constantes de $\mathcal{F}(E,\R)$. Il est évident que $\overline{\R}$ est un sous-espace vectoriel de $\mathcal{F}(E,\R)$. On note
\begin{displaymath}
  F = E^* + \overline{\R}
\end{displaymath}
Un élément de $F$ sera appelé \emph{fonction numérique affine}.
\begin{enumerate}
  \item Montrer que $E^*$ et $\overline{\R}$ sont supplémentaires dans $F$. En déduire une base de $F$. Toute fonction numérique affine se décompose donc de manière unique comme la somme d'une \emph{partie linéaire} et d'une \emph{partie constante}.
  \item Soit $\varphi \in F$, discuter de la nature de $D_\varphi = \varphi^{-1}(\{0\})$ (image réciproque du singleton).
  \item Soit $\varphi_1$, $\varphi_2$, $\varphi_3$ des fonctions numériques affines non constantes. Discuter suivant $\rg(\varphi_1,\varphi_2,\varphi_3)$ de la configuration géométrique de $D_{\varphi_1}$, $D_{\varphi_2}$, $D_{\varphi_3}$.
\end{enumerate}
