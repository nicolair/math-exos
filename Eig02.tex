\begin{tiny}Eig02\end{tiny}
Calculer les int{\'e}grales curvilignes
  $\int_{\Gamma}\omega$ dans les cas suivants.

\begin{enumerate}
  \item $\omega=x^2dx+y^2dy$, $\Gamma$ est la demi-ellipse
  parcourue dans le sens indirect et d{\'e}finie par
  \[\left \{ \begin{array}{c}
    x^2+4y^2-4=0 \\
    y \geq 0 \
  \end{array} \right.  \]
  \item $\omega=\frac{x-y}{x^2+y^2}dx+\frac{x+y}{x^2+y^2}dy$, $\Gamma$ est
  le contour du carr{\'e} $A,B,C,D$ avec $A=(1,1)$, $B=(-1,1)$,
  $C=(-1,-1)$, $D=(1,-1)$.
  \item $\omega=ydx+2xdy$, $\Gamma$ est le contour, parcouru dans
  le sens direct du domaine d{\'e}fini par
  \begin{displaymath}
\left \{ \begin{array}{c}
    x^2+y^2-2x \leq 0 \\
    x^2+y^2-2y \leq 0 \
  \end{array} \right. 
\end{displaymath}
  \item $\omega=ydx+xdy$, $\Gamma$ est l'arc de parabole $y=x^2$
  parcourue de $O(0,0)$ vers $A(2,4)$.
\end{enumerate}