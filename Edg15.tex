\begin{tiny}(Edg15)\end{tiny} Le but de l'exercice est de déteminer les fonctions $f\in \mathcal{C}^2(\R)$ vérifiant:
\begin{displaymath}
 \forall (x,y)\in \R^2,\; f(x+y)+f(x-y) = 2f(x)f(y)
\end{displaymath}
 \begin{enumerate}
  \item Déterminer les solutions constantes.
  \item Soit $f$ une solution non constante.
\begin{enumerate}
 \item Montrer ue $f(0)=1$ et que $f'(0)=0$.
 \item Montrer que $f$ est une fonction paire.
\end{enumerate}
\item  Soit $f$ une solution non constante. On considère la fonction $F$ définie dans $\R^2$ par :
\begin{displaymath}
 F((x,y))=f(x+y)+f(x-y)
\end{displaymath}
\begin{enumerate}
 \item Justifer que $F\in \mathcal{C^2}(\R^2)$.
 \item Calculer les dérivées partielles secondes de $F$. En déduire que $f$ vérifie une équation différentielle de la forme
\begin{displaymath}
 f''-\alpha f = 0
\end{displaymath}
Donner les solutions de cette équation suivant les valeurs de $\alpha$.
\end{enumerate}
\item Déterminer toutes les solutions de l'équation.
 \end{enumerate}
