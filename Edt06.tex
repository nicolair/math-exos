\begin{tiny}(Edt06)\end{tiny}
Soit $(P_{1},\cdots ,P_{n})$ des polyn{\^o}mes dans $\C_{n-1}[X]$ et
\begin{displaymath}
 A = \Mat_{(1,X,\cdots,X^{n-1})}(P_{1},\cdots ,P_{n})
\end{displaymath}
Soit $(x_{1},\cdots ,x_{n})\in \C^{n}$ et $V$ la matrice de VanderMonde associée 
\begin{displaymath}
V =
\begin{vmatrix}
1 & \cdots  & \cdots  & 1 \\
x_{1} &  &  & x_{n} \\
\vdots  &  &  & \vdots  \\
x_{1}^{n-1} & \cdots  & \cdots  & x_{n}^{n-1}
\end{vmatrix}
\end{displaymath}
Soit $Y\in \mathcal{M}_p(\C)$ la matrice dont le terme d'indice $i,j$ est $P_{i}(x_{j})$.\newline
Former une relation entre les matrices $A$, $Y$, $V$ puis entre leurs déterminants.\newline
Application. Calculer
\begin{displaymath}
 D_n =
\begin{vmatrix}
 1^0     & 2^0         & \cdots & (n)^0 \\
 2^1     & 3^1         & \cdots & (n+1)^1 \\
 \vdots  & \vdots      &        & \vdots \\
 n^{n-1} & (n+1)^{n-1} & \cdots & (2n-1)^{n-1}
\end{vmatrix}
\end{displaymath}
On pourra utiliser le résultat de l'exercice \href{\exosurl Wdt01.html}{dt01} pour exprimer le déterminant de la matrice $V$.