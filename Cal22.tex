\begin{tiny}(Cal22)\end{tiny} Pour $g$ quelconque, il existe $g'$ tel que $g*g'=e$. Montrons que $g'*g=e$.\newline
Il existe $g''\in G$ tel que $g*g''=e$. En multipliant à gauche par $g'$ puis à droite par $g''$:
\begin{multline*}
g*g' = e \Rightarrow g'*g*g'=g'
\Rightarrow g' * g * g' * g'' = g' * g'' \\
\Rightarrow g' * g = e
\end{multline*}
On en déduit que $e$ est un véritable élément neutre c'est à dire qu'il est neutre à gauche.
\begin{displaymath}
\forall g\in G,\;
e*g = (g*g')*g = g
\end{displaymath}
