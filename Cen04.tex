\begin{tiny}(Cen04)\end{tiny} 
\begin{enumerate}
  \item Notons $\mathcal{I}$ l'ensemble des couples $(A,B)$ vérifiant $A\subset B \subset E$.\newline
Classons les éléments de $\mathcal{I}$ selon le deuxième élément des couples. Pour chaque $B \subset E$, notons
\[
  \mathcal{I}_B =
  \left\lbrace (A,B) \text{ tq } A \subset B \right\rbrace.
\]
Les $\mathcal{I}_B$ forment une partition de $\mathcal{I}$ et chaque $\mathcal{I}_B$ est en bijection avec $\mathcal{P}(B)$. On en déduit
\begin{multline*}
\sharp\, \mathcal{I} =  \sum_{B \in \mathcal{P}(E)} \sharp \mathcal{I}_B
 = \sum_{B \in \mathcal{P}(E)} 2^{\sharp B}\\
 = \sum_{k \in \llbracket 0,n\rrbracket} \binom{n}{k} 2^k 
 = (1+2)^n = 3^n.
\end{multline*}
en regroupant les $B$ avec le même nombre d'éléments puis en utilisant la formule du binôme.\newline
On peut retrouver ce résultat en formant une bijection entre $\mathcal{S}$ et $\mathcal{F}(E,\llbracket 0,2 \rrbracket)$.

  \item Notons $\mathcal{U}= \left\lbrace (A,B) \text{ tq } A\cup B = E\right\rbrace$. Les applications
\begin{multline*}
\left\lbrace
  \begin{aligned}
    \mathcal{U} &\rightarrow \mathcal{I} \\
    (A,B)       &\mapsto (A\cap B, B)
  \end{aligned}
\right., \\
\left\lbrace
  \begin{aligned}
    \mathcal{I} &\rightarrow \mathcal{U} \\
    (A,B) &\mapsto (A \cup \overline{B},B)
  \end{aligned}
\right..
\end{multline*}
sont des bijections réciproques l'une de l'autre donc $\sharp\, \mathcal{U} = 3^n$.
\end{enumerate}

