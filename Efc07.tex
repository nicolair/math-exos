\begin{tiny}(\hyperdef{exo}{Efc07}{Efc07})\end{tiny} Soit $J$ un intervalle de $\R$ et $(f_i)_{i\in \mathcal I}$ une famille de fonctions convexes définies dans $J$ et à valeurs réelles. On ne suppose rien sur l'ensemble $\mathcal I$. Il peut être fini ou infini. On suppose en revanche que, pour chaque $x\in J$, l'ensemble $\left\lbrace f_i(x),i\in \mathcal I \right\rbrace $ est majoré. On pose
\begin{displaymath}
 \forall x\in J :\; f(x) = \sup \left\lbrace f_i(x),i\in \mathcal I \right\rbrace 
\end{displaymath}
Montrer que $f$ est convexe.