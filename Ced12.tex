\begin{tiny}(Ced12)\end{tiny} Changement de variable.
\begin{enumerate}
  \item Calculons les dérivées de $z = u\circ f$
\begin{align*}
  z &= u \circ f \\
  z' &= f' u'\circ f \\
  z'' &= f''u'\circ f + f'^2 u''\circ f
\end{align*}
On en déduit que $z$ est solution de $(E)$ si et seulement si 
\begin{displaymath}
  (1)\hspace{0.5cm} Lu''\circ f + Mu'\circ f + c u\circ f = 0
\end{displaymath}
avec
\begin{displaymath}
  L = A f'^2 \hspace{0.5cm}
  M = A f'' + Bf'
\end{displaymath}
L'équation $(1)$ est à coefficients constants si et seulement si $L$  et $M$ sont des constantes. $L$ doit être positive donc (à une constante multiplicative près)
\begin{displaymath}
  f' = \frac{1}{\sqrt{A}} \Rightarrow
  f'' = -\frac{1}{2}\frac{A'}{A^{\frac{3}{2}}}
\end{displaymath}
En remplaçant dans l'expression de $M$, on trouve que $M$ est constante si et seulement si 
\begin{displaymath}
  -A' + 2B = 0
\end{displaymath}
La fonction $f$ étant alors donnée comme primitive de $\frac{1}{\sqrt{A}}$.
\item Dans cet exemple 
\begin{displaymath}
A(x) = 1-x^2 \hspace{0.5cm} B(x)=-x = \frac{1}{2}A'(x)  
\end{displaymath}
donc la condition de la question précédente est vérifiée.\newline
Pour $f=\arcsin$, l'équation $(1)$ s'écrit alors
\begin{displaymath}
  u'' + 9u = 0
\end{displaymath}
dont les solutions sont les fonctions définies dans $]-\frac{\pi}{2}, \frac{\pi}{2}[$
\begin{displaymath}
  t\mapsto \lambda \cos3t + \mu \sin3t
\end{displaymath}
Les solutions de $(E)$ sont alors les fonctions définies dans $]-1,1[$
\begin{displaymath}
  x\mapsto \lambda \cos(3\arcsin x) + \mu \sin(3\arcsin x)
\end{displaymath}
En utilisant
\begin{align*}
\cos 3t &= \cos t(1-4\cos t)\\
\sin 3t &= 3\sin t - 4\sin^3 t
\end{align*}
on peut les écrire aussi sous la forme
\begin{displaymath}
  x\mapsto \lambda \sqrt{1-x^2}(1-4x) + \mu (3x-4x^3)
\end{displaymath}

\end{enumerate}
