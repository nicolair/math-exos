\begin{tiny}(Eml07)\end{tiny}
Multiplicateurs de Lagrange.\newline
Soit $E$ un $\K$ espace vectoriel fixé. Sauf précision, toutes les formes linéaires sont relatives à cet espace. 
\begin{enumerate}
  \item Soit $\alpha$ une forme linéaire non nulle et $a\in E$ tel que $\alpha(a)\neq 0$. Montrer que $\alpha$ est surjective et que les sous-espaces $\Vect(a)$ et $\ker(\alpha)$ sont supplémentaires.
  \item Soient $\alpha, \beta$ des formes lin{\'e}aires non nulles.
Montrer que 
\begin{displaymath}
 \ker\alpha\subset\ker \beta \Rightarrow \ker\alpha = \ker \beta
\end{displaymath}
\item Montrer qu'il existe une relation lin{\'e}aire entre deux formes non nulles si et seulement si elles ont le m{\^e}me hyperplan noyau.
 \item Soient $\alpha , \beta , \gamma $ des formes lin{\'e}aires non nulles dont les noyaux sont deux {\`a} deux distincts mais tels que $$\ker \alpha \cap \ker \beta \subset \ker \gamma$$
En consid{\'e}rant la restriction not{\'e}e $\alpha'$ de $\alpha$ {\`a} $\ker \beta$ et la restriction not{\'e}e $\gamma'$ de $\gamma$ {\`a} $\ker \beta$, montrer qu'il existe une relation lin{\'e}aire entre $\alpha$, $\beta$, $\gamma$.
\end{enumerate}