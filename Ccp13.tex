\begin{tiny}(Ccp13)\end{tiny} L'énoncé n'est pas très précis, un des points du cercle ne peut pas avoir d'image mais on ne sait pas lequel. Cela se traduit par : le point d'affixe $-\frac{d}{c}$ est sur le cercle. \newline
On va supposer que c'est l'origine. On en tire $d=0$. Comme on peut multiplier $a$, $b$, $c$, $d$ par un même nombre complexe non nul sans changer l'homographie, on peut supposer que $c=1$.\newline
On cherche donc des conditions sur $a$ et $b$ pour que
\begin{displaymath}
  z \mapsto a + \frac{b}{z}
\end{displaymath}
réponde à la question.\newline
Considérons les images des points d'affixe $2$ et$1+i$ (ils sont bien sur le cercle). Alors:
\begin{multline*}
\Re(a+\frac{b}{2})=\frac{1}{2},\\
\Re(a+\frac{b}{1+i})= \Re(a+\frac{1}{2}b(1-i)) \\ 
= \Re(a+\frac{b}{2}) -\Re\frac{bi}{2} =\frac{1}{2}
\Rightarrow b \in \R
\end{multline*}

Montrons maintenant que
\begin{displaymath}
 h:  z \mapsto \frac{1}{z}
\end{displaymath}
est une homographie qui satisfait à la condition.\newline
Pour tout point $M$ (d'affixe $m$) du cercle, il existe un réel $t$ tel que
\begin{displaymath}
  m = 1+e^{it}
\end{displaymath}
alors:
\begin{displaymath}
  h(m)=\frac{1}{1+e^{it}} = \frac{1}{2\cos\frac{t}{2}}e^{-i\frac{t}{2}}
  = \frac{1}{2}+i\frac{\sin\frac{t}{2}}{2\cos\frac{t}{2}}
\end{displaymath}
qui sont bien les affixes des points de la droite voulue.
