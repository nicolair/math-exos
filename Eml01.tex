\begin{tiny}(Eml01)\end{tiny}
\textbf{Projecteurs}. On appelle \emph{projecteur} de $E$ tout endomorphisme $p$ de $E$ tel que $p\circ p=p$. Dans tout l'exercice $p$ et $q$ sont deux projecteurs.
\begin{enumerate}
\item Montrer que $u \in \mathcal{L}(E)$ commute avec $p$ si et seulement si $\mathrm{Im}(p)$ et $\ker(p)$ sont stables par $u$.

\item Montrer que
\begin{displaymath}
 p\circ q+q\circ p=0_{\mathcal{L}(E)} \Rightarrow p\circ q=q\circ p=0_{\mathcal{L}(E)} 
\end{displaymath}

\item Exprimer une condition nécessaire et suffisante pour que $p+q$ soit un projecteur. Pr{\'e}ciser alors $\mathrm{Im}(p+q)$ et $\ker (p+q)$ {\`a} l'aide de $\mathrm{Im}(p)$, $\ker (p)$, $\mathrm{Im}(q)$, $\ker (q)$.

\item Montrer que $p\circ q=q\circ p$ entraîne que $p\circ q$ est un projecteur. Pr{\'e}ciser alors son noyau et son image.
\end{enumerate}
