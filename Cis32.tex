\begin{tiny}(Cis32)\end{tiny} Comme les fonctions sont bornées, il existe $M_f$ et $M_g$ tels que $|f(t)|\leq M_f$ et $|g(t)|\leq M_g$ pour tous les $t$ dans $I$. Considérons alors des réels $x$ et $y$ quelconques. Pour tout $\varepsilon >0$, il existe $t_x\in I$ tel que
\begin{displaymath}
 \varphi(x) \leq f(t_x) + xg(t_x) + \varepsilon
\end{displaymath}
 D'autre part, par définition de $\varphi(y)$ comme borne supérieure:
\begin{displaymath}
  f(t_x) + yg(t_x) \leq \varphi(y)
\end{displaymath}
On en déduit
\begin{displaymath}
 \varphi(x) - \varphi(y) \leq (x-y)g(t_x)+\varepsilon
\leq |x-y|M_g+\varepsilon
\end{displaymath}
Comme ceci est valable pour tous les $\varepsilon >0$, on a
\begin{displaymath}
 \varphi(x) - \varphi(y) \leq (x-y)M_g
\end{displaymath}
En échangeant les rôles de $x$ et $y$, on obtient
\begin{displaymath}
 \varphi(y) - \varphi(x) \leq (x-y)M_g
\end{displaymath}
Ce qui montre que $\varphi$ et $M_g$-lipschitzienne.