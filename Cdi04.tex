\begin{tiny}(Cdi04)\end{tiny} La linéarité de $f$ découle de celle de la dérivation et de la prise de valeur en $a$. Comme $\deg f(P) \leq \deg P$, la fonction est bien un endomorphisme. par un calcul immédiat:
\begin{displaymath}
  \forall k\in \llbracket 2,n\rrbracket,\;
  f((X-a)^k) = (k-2)(X-a)^{k}
\end{displaymath}
car $a$ est racine au moins double. De plus
\begin{displaymath}
  f(1) = 0\hspace{0.5cm} f(X-a) = 0
\end{displaymath}
On en déduit que $1, X-a$ et $(X-a)^2$ sont dans le noyau donc
\begin{displaymath}
  \Vect(1,X-a, (X-a)^2)\subset \ker f
\end{displaymath}
et les $(X-a)^k$ pour $k\geq 3$ sont dans l'image:
\begin{displaymath}
\Vect\left( (X-a)^3,\cdots,(X-a)^n\right)
\subset \Im f
\end{displaymath}
Comme les familles de vecteurs sont libres:
\begin{displaymath}
\dim \ker f \geq 3, \hspace{0.5cm}
\dim \Im f \geq n-2
\end{displaymath}
\`A cause du théorème du rang
\begin{multline*}
\dim \ker f + \dim \Im f = n +1 \\
\Rightarrow \dim \ker f \geq n+1-(n-2) = 3 \\
\Rightarrow \ker f = \Vect(1,X-a, (X-a)^2)
\end{multline*}

De même pour l'image.
\begin{displaymath}
\Vect\left( (X-a)^3,\cdots,(X-a)^n\right) = \Im f  
\end{displaymath}
