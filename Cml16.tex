\begin{tiny}(Cml16)\end{tiny} De $f\circ f= -\Id_E$, on tire que $f$ est bijective de bijection réciproque $-f$.\newline
Soit $x$ non nul et $\lambda$, $\mu$ réels tels que
\begin{displaymath}
  \lambda x + \mu f(x) = 0_E
\end{displaymath}
On compose par $f$:
\begin{displaymath}
  - \mu x + \lambda f(x) = 0_E
\end{displaymath}
Si $\lambda \neq 0$, on peut multiplier la deuxième relation par $-\frac{\mu}{\lambda}$ et ajouter à la première. On en tire
\begin{displaymath}
  \frac{\lambda^2 + \mu^2}{\lambda} x  = 0_E
\end{displaymath}
Ce qui est impossible dans $\R$ lorsque $x\neq 0_E$. On doit donc avoir $\lambda = 0$. Comme $f$ est bijective, $x\neq 0_E$ entraîne $f(x)\neq 0_E$ donc $\mu=0$. 