\begin{tiny}(Esf06)\end{tiny} Théorème d'approximation de Weierstrass.\newline
Soit $n\in \N^*$, $x\in[0,1]$ et $f\in \mathcal{C}([0,1],\R)$.\newline
On se donne aussi des variables aléatoires $X_1,\cdots, X_n$ mutuellement indépendantes et suivant la loi de Bernoulli de paramètre $x$. On note
\begin{displaymath}
  S_n = \sum_{k=1}^n X_k,\hspace{0.3cm} Z_n = \frac{1}{n} S_n, \hspace{0.3cm}
  B_n(f)(x) = E(f(Z_n))
\end{displaymath}
\begin{enumerate}
  \item Quelle est la loi de $S_n$ ? Préciser
\begin{displaymath}
E(S_n),\hspace{0.3cm} V(S_n),\hspace{0.3cm} E(Z_n),\hspace{0.3cm} V(Z_n)
\end{displaymath}

  \item En utilisant l'inégalité de Bienaymé-Chebychev, montrer que 
\begin{displaymath}
\forall \alpha >0:
\sum_{\left|\frac{k}{n}-x\right|\geq \alpha} \binom{k}{n}x^k (1-x)^{n-k} 
\leq \frac{1}{4n\alpha^2}
\end{displaymath}
  \item Montrer que 
\begin{multline*}
B_n(f)(x)-f(x) \\ 
= \sum_{k=0}^{n} \binom{k}{n}x^k (1-x)^{n-k}\left(f(\frac{k}{n})-f(x) \right)  
\end{multline*}
  \item En déduire, en utilisant le théorème de Heine que la suite de fonctions $\left( B_n(f)\right)_{n\in \N}$ converge \emph{uniformément} vers $f$ dans $[0,1]$.
\end{enumerate}
