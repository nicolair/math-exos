\begin{tiny}(Cre22)\end{tiny} Notons, pour tout entier $p$
\begin{displaymath}
 X_p = \left\lbrace x_n , n\geq p \right\rbrace 
\end{displaymath}
Cela permet de reformuler la condition de l'énoncé :
\begin{displaymath}
 \forall n \geq p : x_n\leq x_p \Leftrightarrow x_p = \max X_p
\end{displaymath}
On nous demande donc de prouver que, lorsque la suite $(x_n)_{n\in \N}$ converge vers $0$, l'ensemble (notons le $A$) des $p$ tels que $x_p = \max X_p$ est infini.\newline
En fait on va montrer que lorsque cet ensemble est fini, on peut extraire de $(x_n)_{n\in \N}$ une suite \emph{croissante} ce qui est évidemment incompatible (thm de passage à la limite dans une inégalité) avec une convergence vers 0.

Si $A$ est fini, il existe un $N$ tel que pour tout $n\geq N$ : $x_n<\max X_n$. Posons $i_0=N$
\begin{align*}
 x_{i_0}<\max X_{i_0} &\Rightarrow \exists i_1 > i_0 \text{ tel que } x_{i_0}<x_{i_1} \\
 x_{i_1}<\max X_{i_1} &\Rightarrow \exists i_2 > i_1 \text{ tel que } x_{i_1}<x_{i_2} \\
\cdots
\end{align*}

