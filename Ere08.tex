\begin{tiny}(Ere08)\end{tiny} \textbf{Suites de Farey.} \label{def:farey} Soit $n \in \N$, $n\geq 2$.\newline
Parmi tous les rationnels de $\left]0,1\right[$, consid{\'e}rons ceux qui peuvent s'{\'e}crire avec un d{\'e}nominateur inf{\'e}rieur ou {\'e}gal {\`a} $n$. La suite ordonn{\'e}e de ces nombres forme, par d{\'e}finition, la \emph{suite de Farey} d'ordre $n$.\index{suite de Farey} On la note $\mathcal{S}_{n}$. 
\begin{enumerate}
  \item Former $\mathcal{S}_{n}$ pour $n$ entre 3 et 5.
  \item Pour les $\mathcal{S}_{n}$ calculées: v{\'e}rifier que le num{\'e}rateur de la difference entre deux termes cons{\'e}cutifs de $\mathcal{S}_{n}$ vaut 1 apr{\`e}s
simplification et que le deuxi{\`e}me terme d'une s{\'e}quence de trois termes cons{\'e}cutifs est la fraction m{\'e}diane des deux autres (définie en \ref{def:mediane} \begin{tiny}(Ere07)                                                                                                                                                                         \end{tiny}).

\end{enumerate}
