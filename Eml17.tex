\begin{tiny}(Eml17)\end{tiny} Lemme des cinq.\newline
On se donne des espaces vectoriels et des applications linéaires suivant le diagramme
\begin{displaymath}
  \begin{array}{lllllllll}
    E_1 & \xrightarrow{\alpha}  & E_2  & \xrightarrow{\beta}  & E_3 & \xrightarrow{\gamma}   & E_4  & \xrightarrow{\delta}  & E_5 \\ 
\downarrow f_1&     &\downarrow  f_2  &             &\downarrow f_3 &               &\downarrow f_4 &                    &\downarrow f_5  \\
    E_1' & \xrightarrow{\alpha'} & E_2' & \xrightarrow{\beta'} & E_3' & \xrightarrow{\gamma'} & E_4' & \xrightarrow{\delta'} & E_5' 
  \end{array}
\end{displaymath}
On suppose de plus: 
\begin{displaymath}
\Im \alpha = \ker \beta\hspace{0.5cm}  \Im \beta = \ker \gamma \hspace{0.5cm} \Im \gamma = \ker \delta  
\end{displaymath}
\begin{displaymath}
\Im \alpha' = \ker \beta' \hspace{0.5cm}\Im \beta' = \ker \gamma' \hspace{0.5cm}\Im \gamma' = \ker \delta'  
\end{displaymath}
\begin{multline*}
f_2 \circ \alpha = \alpha' \circ f_1 \hspace{0.5cm}
f_3 \circ \beta = \beta' \circ f_2 \hspace{0.5cm}
f_4 \circ \gamma = \gamma' \circ f_3 \\
f_5 \circ \delta = \delta' \circ f_4
\end{multline*}
et enfin $f_1$, $f_2$ $f_4$, $f_5$ bijectives.\newline
Montrer que $f_3$ est bijective.