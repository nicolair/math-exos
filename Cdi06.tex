\begin{tiny}(Cdi06)\end{tiny} \label{exo: di06}
\begin{enumerate}
  \item La fonction est clairement linéaire et son noyau est
\begin{displaymath}
  \ker(\alpha_1) \cap \cdots \cap \ker(\alpha_n)
\end{displaymath}
Soit $x$ dans ce noyau. Alors $\alpha(x)=0$ pour toute forme linéaire $\alpha$ car $\alpha$ est une combinaison linéaire des $\alpha_i$. En considérant pour $\alpha$ les formes coordonnées dans une base, on prouve que toutes les coordonnées de $x$ dans une base sont nulles donc $x=0_E$. L'application $\Phi$ est injective donc bijective car c'est un endomorphisme en dimension finie.

  \item On utilise l'isomorphisme de la question précédente. On définit $(a_1,\cdots,a_n)$ comme les antécédents des vecteurs de la base canonique de $\K^n$. 
  
  \item On vérifie facilement que
\begin{multline*}
  \alpha \in \Vect(\alpha_1,\cdots, \alpha_p) \\
  \Rightarrow \ker(\alpha_1) \cap \cdots \cap \ker(\alpha_p) \subset \ker(\alpha).
\end{multline*}

Pour la réciproque, on peut supposer que la famille est libre. En effet, si elle ne l'est pas, on extrait une sous-famille libre qui engendre le même sous-espace et on raisonne avec celle-ci.\newline
On complète en une base de $E^*$
\begin{displaymath}
  (\alpha_1,\cdots, \alpha_n)
\end{displaymath}
et on utilise la base antéduale de la question b.
\begin{displaymath}
  (a_1,\cdots, a_n)
\end{displaymath}
Considérons la forme linéaire
\begin{displaymath}
  \varphi = \alpha - \alpha(a_1)\alpha_1 - \cdots - \alpha(a_p)\alpha(p)
\end{displaymath}
La forme $\varphi$ est nulle car $\varphi(a_i)=0$ pour tous les $i$ entre $1$ et $n$ (les raisons sont différentes selon $i\leq p$ ou $i>p$). 

  \item Supposons
\begin{displaymath}
  \dim(\ker \alpha_1\cap\cdots\cap\ker\alpha_p) = \dim E -p
\end{displaymath}
Le théorème du rang appliqué à $\Phi_p$ montre que $\Phi_p$ est surjective. Il existe alors des antécédents $a_1,\cdots,a_p$ aux vecteurs de la base canonique. Ils permettent de montrer que la famille est libre.\newline
Supposons la famille libre; on la complète en une base. La fonction $\Phi_n$ est alors surjective ce qui entraine que $\Phi_p$ est surjective. On obtient la dimension avec le théorème du rang appliqué à $\Phi_p$.
\end{enumerate}
