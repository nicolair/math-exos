\begin{tiny}(Cef02)\end{tiny} Comme les familles ont un nombre d'éléments égal à la dimension, il suffit de montrer qu'elles sont génératrices pour montrer qu'elles sont libres. On le fait en exprimant les vecteurs de base ce qui est commode pour exprimer les coordonnées dans la nouvelle base.\newline
Cas 1.
\begin{multline*}
  \left\lbrace
  \begin{aligned}
    a_1 &= -b_1 + b_3 &\times& x_1\\
    a_2 &= b_1  &\times& x_2\\
    a_3 &= b_2 &\times& x_3\\
    a_4 &= -b_2 - b_3 + b_4 &\times& x_4
  \end{aligned}
\right.
\Rightarrow \text{ coord dans } \mathcal{B}\\
= (-x_1 + x_2, x_3 - x_4, x_1 - x_4, x_4).
\end{multline*}
Cas 2.
\begin{multline*}
  \left\lbrace
  \begin{aligned}
    a_1 &= b_2 &\times& x_1\\
    a_2 &= b_3  &\times& x_2\\
    a_3 &= b_1 + b_2 &\times& x_3\\
    a_4 &= b_1 + b_2 + b_3 + b_4 &\times& x_4
  \end{aligned}
\right.
\Rightarrow \text{ coord dans } \mathcal{B}\\
= (x_3 + x_4, x_1 + x_3 + x_4, x_2 + x_4, x_4).
\end{multline*}
