\begin{tiny}(Ecp47)\end{tiny} Symétrie par rapport à une droite passant par l'origine.\newline
Soit $u\in C$ non nul, $U$ le point d'affixe $u$, $O$ le point d'affixe $0$. Pour tout point $M$ d'affixe $z$, on note $M'$ le symétrique de $M$ par rapport à la droite $(OU)$ et $z'$ son affixe.\newline
En écrivant que le point d'affixe $z+z'$ appartient à $(OU)$ et que $\overrightarrow{MM'}$ est orthogonal à $(OU)$, déterminer $z'$ en fonction de $z$ et $u$.
