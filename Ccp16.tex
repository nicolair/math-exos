\begin{tiny}(Ccp16)\end{tiny} \label{Ccp16}
Soit $\delta$ une racine carrée du discriminant et notons $m'= a +\frac{\delta}{2}$, $m= a -\frac{\delta}{2}$ les deux racines.
\begin{enumerate}
  \item La pente de la droite $(mm')$ est $\frac{\Im \delta}{\Re \delta}$ qui est bien indépendante du choix de la racine carrée.
De plus,
\begin{displaymath}
  \frac{\Im \delta}{\Re \delta} = 1 \Leftrightarrow 
\left\lbrace  
\begin{aligned}
  &\Re(\delta ^2) = 0 \\ &\Im(\delta)\Re(\delta)>0
  \Leftrightarrow \Im(\delta^2) > 0
\end{aligned}
\right. 
\end{displaymath}
Or $\delta^2$ est le discriminant $4a^2+8(1+i)$. En notant $x=\Re(a)$ et $y=\Im(a)$, les conditions sont
\begin{displaymath}
  x^2 - y^2 + 2 = 0 \hspace{0.5cm} xy + 1 > 0
\end{displaymath}

\item L'angle $mOm'$ est droit si et seulement si
\begin{multline*}
  \frac{m'}{m}\in i\R \Leftrightarrow (a-\frac{\delta}{2})(\overline{a}-\frac{\overline{\delta}}{2}) \in i\R
  \Leftrightarrow 4|a|^2 = |\delta|^2\\
  \Leftrightarrow 4|a|^2 = |4a^2+8(1+i)|\\
  \Leftrightarrow 0 = \Re(a^2(1-i)) + |1+i|^2\\
  \Leftrightarrow x^2 - y^2 +2xy +2 = 0
\end{multline*}

\end{enumerate}


