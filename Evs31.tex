\begin{tiny}(Evs31)\end{tiny} Pour s'entrainer à raisonner: des démonstrations en exercices en partant d'un jeu d'axiomes.\newline
On admet les propriétés suivantes.\newline
$\N$ est un ensemble non vide, muni d'une relation d'ordre (notée $\leq$) et non majoré.\\
Toute partie non vide de $\N$ admet un plus petit élément.\\
On note $0=\min \N$ et $\llbracket 0, n\rrbracket$ l'ensemble des entiers plus petits que $n$.\\
Comme $\N$ est non majoré, aucun naturel $n$ n'est un majorant de $\N$. Le complémentaire $\N\setminus \llbracket 0, n\rrbracket$ est donc non vide pour tout $n\in \N$.
On définit une application $S$ de $\N$ dans $\N^*=\N \setminus\{0\}$ avec
\begin{displaymath}
 S(n) = \min (\N\setminus \llbracket 0, n\rrbracket)
\end{displaymath}
L'application $S$  est surjective.\newline
On convient de noter $1=S(0)$.\newline
On admet que pour tout $n\neq 0$, il existe $m\in \N$ tel que $S(m)=n$.\\
La notation usuelle $a<b$ signifie $a\leq b$ et $a\neq b$.\\
Vous devez prouver les propositions numérotées en utilisant uniquement les propriétés admises et les propriétes avec un numéro plus petit.
\begin{enumerate}
\item $\N$ est totalement ordonné.

\item Pour tous naturels $a$ et $b$,
\begin{displaymath}
 b < a \Leftrightarrow \left( a\leq b \text{ faux } \right) 
\end{displaymath}

\item Pour tout $n$ naturel, $n < S(n)$ et 
\begin{displaymath}
 \llbracket 0 , S(n) \rrbracket = \llbracket 0 , n \rrbracket \cup \{S(n)\}
\end{displaymath}

\item  $S$ est strictement croissante et injective.

\item  Principe de récurrence. Soit $A$ une partie de $\N$ telle que
\begin{displaymath}
 \exists a \in A \text{ et } \forall n \in \N :\; n\in A \Rightarrow S(n) \in A
\end{displaymath}
alors $A= \llbracket a , +\infty\llbracket =\left\lbrace k\in \N \text{ tq } a\leq k\right\rbrace$ .

\item  Récurrence descendante. Soit $n\in \N$ et $A\subset \llbracket 0,n \rrbracket$ vérifiant $n\in A$ et, pour $x\in \llbracket 0,n-1 \rrbracket$, 
\begin{displaymath}
 S(x) \in A \Rightarrow x \in A
\end{displaymath}
alors $\llbracket 0,n \rrbracket \subset A$.

\item  Pour tous éléments $x$ et $n$ de $\N$:
\begin{align*}
 n \leq x < S(n) \Rightarrow x=n & & n < x \leq S(n) \Rightarrow x = S(n)
\end{align*}

\item  Toute partie de $\N$ non vide et majorée admet un plus grand élément.

\item Définitions. Un ensemble $\Omega$ est \emph{fini} si et seulement si toute application injective de $\Omega$ dans lui même est surjective. Un ensemble est \emph{infini} si et seulement si il n'est pas fini. Un ensemble est \emph{infini dénombrable} si et seulement si il est en bijection avec $\N$.
 
\item $\N$ n'est pas fini.

\item  S'il existe une bijection entre deux ensembles $A$ et $B$, alors $A$ est fini si et seulement si $B$ est fini.

\item Toute partie d'un ensemble fini est finie.

\item  Soit $n\in \N$ et $\varphi$ une application strictement croissante de $\llbracket 0, n\rrbracket$ dans $\N$. Alors $x\leq \varphi(x)$ pour tous les $x\in \llbracket 0, n\rrbracket$.

\item  Soit $n\in\N$ et $\varphi$ strictement croissante de $\llbracket 0,n \rrbracket$ dans $\llbracket 0,n \rrbracket$, alors $\varphi$ est l'identité de $\llbracket 0,n \rrbracket$.

\item  Soit $n\in\N$ et $\varphi$ une application de $\llbracket 0,n \rrbracket$ dans $\N$, alors $\varphi(\llbracket 0,n \rrbracket)$ est une partie majorée de $\N$.

\item  La partie $\llbracket 0, n\rrbracket$ de $\N$ est finie pour tout entier naturel $n$.

\item   Soit $p$ et $q$ deux entiers naturels tels qu'il existe une application injective de $\llbracket 0,p \rrbracket$ dans $\llbracket 0,q \rrbracket$, alors $p\leq q$.

\item  Soit $A$ une partie de $\N$ pour laquelle, pour tout $n\in \N$, il existe une application injective de $\llbracket 0,n \rrbracket$ dans $A$, alors $A$ est infinie.

\item  Soit $A$ un ensemble. S'il existe une application injective de $\N$ dans $A$, alors $A$ n'est pas fini.

\item  Soit $A$ un ensemble fini non vide. Il existe un unique entier naturel $n$ pour lequel il existe une bijection entre $\llbracket 0, n\rrbracket$ et $A$.

\item Toute partie finie non vide de $\N$ admet un plus grand élément. 

\item Toute partie d'un ensemble fini est finie et son cardinal est inférieur ou égal au cardinal de l'ensemble qui le contient. 

\end{enumerate}
