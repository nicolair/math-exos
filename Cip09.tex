\begin{tiny}(Cip09)\end{tiny}
Les int{\'e}grales suivantes se calculent avec des primitives.
\begin{align*}
&\int_{0}^{1}e^{x+e^{x}}dx= e^{e}-e,\;\\
&\int_{1}^{e^{\frac{\pi}{2}}}\cos(\ln x)dx = \frac{1}{2}\left(e^{\frac{\pi}{2}}-1 \right),\;\\
&\int_{0}^{\frac{\pi }{6}}\frac{\sin ^{2}x}{\cos x}dx = \frac{1}{2}\left(\ln 3 -1 \right),\;\\
&\int_0^{\frac{\pi }{3}}\cos ^{3}x\sin 2x\,dx = \frac{31}{80},\;\\
&\int_{\ln 2}^{2\ln 2}\frac{dx}{e^{x}-1}= \ln\frac{3}{2},\;
&\int_{0}^{2\pi }\cos5x\cos x\,dx = 0,\;
\end{align*}
Notons 
\begin{align*}
 I = \int_{0}^{\frac{\pi }{2}}\frac{\cos x}{\cos x+\sin x}dx 
& &
 J = \int_{0}^{\frac{\pi }{2}}\frac{\sin x}{\cos x+\sin x}dx
\end{align*}
Le changement de variable $y=\frac{\pi}{2}-\theta$ dans $I$ montre que $I=J$. Comme $I+J=\frac{\pi}{2}$, on déduit
\begin{displaymath}
 I= J = \frac{\pi}{4}
\end{displaymath}
Le calcul est le même pour
\begin{displaymath}
\int_{0}^{\frac{\pi }{2}}\frac{\cos ^{3}x}{\cos ^{3}x+\sin^{3}x}dx 
\end{displaymath}
