\begin{tiny}(Cpo17)\end{tiny} Si $p=0$ une des racines est nulle et n'a pas d'argument. On suppose donc $p\neq 0$. \newline
Les racines $z$ et $z'$ ont le même argument lorsque leurs quotients sont réels et strictement positifs. Formons le polynôme unitaire $P$ dont les racines sont $\frac{z}{z'}$ et $\frac{z'}{z}$.
\begin{align*}
 &\frac{z}{z'} \frac{z'}{z}=1 \\
 &\frac{z}{z'}+\frac{z'}{z}=\frac{z^2+z'^2}{zz'}
= \frac{s^2-2p}{p}=\frac{s^2}{p}-2
\end{align*}
On en déduit 
\begin{displaymath}
 P = X^2 -u X +1
\end{displaymath}
On doit donc caractériser la propriété pour $P$ d'avoir deux racines réelles strictement positives. Cela se produit si et seulement si le discriminant est positif ou nul et la somme des racines strictement positive. Cela revient à $u$ réel et $>2$. En revenant aux notations de l'énoncé, la condition cherchée est
\begin{displaymath}
 \frac{s^2}{p}\in\R\text{ et } \frac{s^2}{p}>4
\end{displaymath}
