\begin{tiny}(ed12)\end{tiny} Changement de variable.\\
On considère une équation différentielle du second ordre à coefficients non constants.
\begin{displaymath}
 (E)\hspace{0.5cm}A(x)y''(x)+B(x)y'(x)+cy(x) = 0
\end{displaymath}
Les fonctions $A$ et $B$ sont définies dans un intervalle $J$ et la fonction $A$ est à valeurs strictement positives, $c$ est un réel fixé. On considère une fonction $f\in \mathcal C^2(J)$ dont la dérivée première est à valeurs strictement positives.
\begin{enumerate}
\item La fonction $f$ est-elle bijective?
 \item Soit $z = u\circ f$. Former une équation différentielle $(1)$ telle que :
 \begin{displaymath}
   u \text{ solution de $(1)$ } \Leftrightarrow z \text{ solution de $(E)$ }
 \end{displaymath}
Sous quelles conditions cette équation est-elle à coefficients constants ?
 \item Appliquer l'idée de la question précédente pour résoudre l'équation
\begin{displaymath}
 (1-x^2)y''(x)-xy'(x)+9y(x) = 0
\end{displaymath}
dans l'intervalle $J=]-1,1[$.
\end{enumerate}
 