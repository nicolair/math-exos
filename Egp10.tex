\begin{tiny}(gp10)\end{tiny}
 \textbf{Relations dans un triangle}.

Soit $A,B,C$ trois points non align{\'e}s:
\begin{itemize}
\item  $S$ l'aire du triangle
\item  $p=\frac{1}{2}(a+b+c)$
\item  $R$ le rayon du cercle circonscrit
\item  $r$ le rayon du cercle inscrit
\item  $r_{A}$ (respectivement $r_{B},$ $r_{C}$) le rayon du cercle
exinscrit en $A$ (respectivement en $B,$ $C$)
\end{itemize}
D{\'e}montrer les formules suivantes :
\begin{multline*}
 R = \frac{a}{2\sin \widehat{A}}=\frac{b}{2\sin \widehat{B}} =\frac{c}{2\sin\widehat{C}} \\
=\frac{a+b+c}{2(\sin \widehat{A}+\sin \widehat{B}+\sin \widehat{C})}
\end{multline*}
(utilser le théorème de l'angle au centre)
\begin{displaymath}
R =\frac{abc}{4S} 
\end{displaymath}
(utiliser l'expression de l'aire avec le déterminant)
\begin{align*}
 r=\frac{2S}{a+b+c} & & r_{A} =\frac{2S}{-a+b+c}\\
r_{B} = \frac{2S}{a-b+c} & & r_{C} = \frac{2S}{a+b-c}
\end{align*}
(utiliser des sommes d'aires de triangles)
\begin{displaymath}
abc = 2Rr(a+b+c) 
\end{displaymath}
\begin{displaymath}
S = \sqrt{p(p-a)(p-b)(p-c)}\hspace{0.5cm}\text{formule de Héron} 
\end{displaymath}
(développer $\Vert \overrightarrow{BC}\Vert^2$ pour exprimer $\cos \widehat A$ en fonction de $a$, $b$, $c$ en déduire une expression du $\sin$ avec une racine qui se factorise bien et conduit à la formule demandée)
\begin{align*}
R+r &=&R(\cos \widehat{A}+\cos \widehat{B}+\cos \widehat{C}) \\
r &=&\frac{a+b+c}{2\left( \cot \frac{\widehat {A}}{2}+\cot \frac{%
\widehat {B}}{2}+\cot \frac{\widehat {C}}{2}\right) }
\end{align*}
