\begin{tiny}(Esn07)\end{tiny} Dans cet exercice, on utilise le résultat
\begin{displaymath}
  \left( \sum_{k=1}^n \frac{1}{k} -\ln n\right)_{n\in \N^*} \rightarrow \gamma 
\end{displaymath}
démontré dans l'exercice \ref{Esn03}. Pour $n$ entier  non nul, on note
\begin{itemize}
  \item $h_n$ la somme des inverses des $n$ premiers entiers.
  \item $i_n$ la somme des inverses des $n$ premiers impairs.
  \item $p_n$ la somme des inverses des $n$ premiers pairs.
\end{itemize}
On imagine que les entiers pairs et impairs sont stockés dans deux distributeurs d'où on les tire en ajoutant chaque fois l'inverse du nombre tiré. Chaque fois que l'on tire un nombre pair, il est multiplié par $-1$.\newline
On fixe des entiers $p$ et $q$ et on procède à des séquences de $p+q$ tirages: $q$ nombres impairs puis $p$ nombres pairs. On note $d_n$ la somme obtenue après $n$ séquences. Exprimer $d_n$ en fonction de $i_n$ et $p_n$ puis en fonction de plusieurs $h_i$ seulement. En déduire que $\left( d_n\right)_{n\in \N^*}$ converge et préciser sa limite.