\begin{tiny}(Epo40)\end{tiny} Le quotient de la division d'une somme par un polynôme fixé (ici $(X-1)^n$) est la somme des quotients. Pour $(X-1)^{n+2}$, le quotient est $(X-1)^2$.\newline
Pour $-1$ le quotient est nul car $n\geq 1$.\newline
Pour $(X+3)^{n+1}$, le quotient s'obtient à partir de la formule du binôme pour 
\[
  (X+3)^{n+1} = \left(4 + (X-1)\right)^{n+1}.
\]
La contribution est donc
\[
  (X-1) + 4(n+1).
\]
Le quotient demandé est donc
\[
  (X-1)^2 + (X-1) + 4(n+1) = X^2 + X +4(n+1).
\]
