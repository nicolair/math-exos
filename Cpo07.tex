\begin{tiny}(Cpo07)\end{tiny} Le principe est de développer le produit de deux sommes déjà connues pour obtenir une seule nouvelle somme et d'autres sommes déjà connues.\newline
Notons $S_1, S_2,\cdots, S_8$ les 8 premières sommes présentées par l'énoncé. Ensuite
\[
  S_9 = \sum_{i\neq j}a_i^2 a_j^2,\;
  S_{10} = \sum_{i\neq j}a_i^3 a_j,\;
  S_{11} = \sum_{i}{a_i}^4.
\]
Présentons successivement les valeurs avec (si besoin) les produits (entre parenthèses) qui permettent de les obtenir
\begin{multline*}
  S_1 = \sigma_1, \; S_2 = 2 \sigma_2,\; 
  (S_1^2): S_3 = \sigma_1^2 - 2 \sigma_2, \\
  S_4 = 6\sigma_3,\; (S_1S_2): S_5 = \sigma_1 \sigma_2 - 3\sigma_3,\;\\
  (S_1S_3): S_6 = \sigma_1^3 - 3\sigma_1 \sigma_2 + 3\sigma_3.
\end{multline*}

Pour le degré 4: $S_7 = 4!\, \sigma_4 = 24 \sigma_4$,
\begin{multline*}
  (S_1 S_4) : S_8 = 2\sigma_1 \sigma3 - 8 \sigma_4,\\
  (S_2^2): S_9 = -4\sigma_1\sigma_3 + 2\sigma_2^2 + 4 \sigma_4,\\
  (S_1 S_5): S_{10} = -\sigma_1\sigma_3 + \sigma_1^2 \sigma_2 - 2\sigma_2^2 + 4\sigma_4,\\
  (S_1 S_6): S_{11} = \sigma_1^4 - 4 \sigma_1^2 \sigma_2 + 4 \sigma_1 \sigma_3 - 4 \sigma_4 + 2\sigma_2^2.
\end{multline*}

