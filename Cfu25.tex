\begin{tiny}(Cfu25)\end{tiny} Examinons s'il existe des $x$ tels que $A(x)=0$.
\begin{multline*}
A(x)=0 \Rightarrow 3-4\cos^2 x = (1+3\sin x)^2\\
\Rightarrow -2+9\sin^2 x + 4\cos^2x+6\sin  x = 0\\
\Rightarrow 2+5\sin^2x + 6\sin x =0
\end{multline*}

Or l'équation $5z^2 +6z +2=0$ est sans racine réelle car de discriminant $-4$.\newline
La fonction $A$ garde donc un signe constant dans chacun de ses intervalles de définition $I_+=[\frac{\pi}{6},\frac{5\pi}{6}]$ et 
$I_-=[-\frac{5\pi}{6}, -\frac{\pi}{6}]$.\newline
Comme $A(\frac{\pi}{2})=\sqrt{3} -4<0$ et $A(-\frac{\pi}{2})=\sqrt{3} +4>0$, la fonction $A$ est négative dans $I_+$ et positive dans $I_-$.
