\begin{tiny}(Efu11)\end{tiny} Fonctions r{\'e}ciproques en trigonom{\'e}trie hyperbolique.\newline
La fonction $\ch $ d{\'e}finit une bijection de $\left[ 0,+\infty
\right[ $ dans $\left[ 1,+\infty \right[ $, sa bijection r{\'e}ciproque est not{\'e}e $%
\arg \ch $. La fonction $\sh $ est bijective de $\R$ dans $\R$, sa bijection r{\'e}ciproque est not{\'e}e $\arg \sh $.
La fonction $\tanh $ d{\'e}finit une bijection de $\R$ dans
$\left] -1,+1\right[ $, sa bijection r{\'e}ciproque est not{\'e}e
$\arg \tanh $.

\begin{enumerate}
\item  Soit $x\in \left[ 1,+\infty \right[ $, exprimer $\arg \ch x$
{\`a} l'aide de $\ln $ et de $\sqrt{x^{2}-1}$.

\item  Soit $x$ un nombre r{\'e}el, exprimer $\arg \sh x$ {\`a} l'aide
de $\ln $ et de $\sqrt{x^{2}+1}$.

\item  Soit $x\in \left] -1,+1\right[ $, exprimer $\arg \th x$ {\`a}
l'aide de $\ln $.
\end{enumerate}