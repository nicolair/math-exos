\begin{tiny}(Cva14)\end{tiny} On démontre l'inégalité demandée avec une formule de Taylor avec reste intégral.\newline
 L'interprétation géométrique est que le graphe d'une fonction convexe est au dessus de ses tangentes.\newline
Par positivité, on déduit que $E(X)\in I$. On note $e$ cette espérance et on applique l'inégalité avec $e$ dans le rôle de $y$ et  $x=X(\omega)$ pour $\omega\in \Omega$.
\begin{displaymath}
 g(x)\geq g(e)+g'(e)(x-m)
\end{displaymath}
On multiplie par $p(\{\omega\})$ et on somme sur les $\omega$ de $\Omega$. La somme des $(X(\omega)-e)p(\{\omega\})$ est nulle et on obtient la formule demandée.