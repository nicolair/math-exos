\begin{tiny}(Emo27)\end{tiny} Soit $\mathcal{A}=(a_1,\cdots,a_p)$ et $\mathcal{B}=(b_1,\cdots,b_p)$ deux bases d'un $\K$-espace vectoriel $E$. Soit $(\alpha_1, \cdots, \alpha_p)$ (resp $(\beta_1, \cdots, \beta_p)$) la base duale des formes coordonnées dans $\mathcal{A}$ (resp dans $\mathcal{B}$).
\begin{enumerate}
  \item Pour tout $(i,j)\in \llbracket 1,p \rrbracket^2$, exprimer les termes $(i,j)$ de $P_{\mathcal{A} \mathcal{B}}$ et de $P_{\mathcal{B} \mathcal{A}}$ avec ces formes coordonnées.
  \item Pour tout $\lambda \neq 0$, on note 
\[
  \mathcal{B}_\lambda = (b_1,\cdots,b_{p-1}, \lambda b_p).
\]
Comment $P_{\mathcal{A} \mathcal{B}_\lambda}$ est-elle obtenue à partir de $P_{\mathcal{A} \mathcal{B}}$?
  \item Comment $P_{\mathcal{B}_\lambda \mathcal{A} }$ est-elle obtenue à partir de $P_{\mathcal{B}\mathcal{A}}$?\newline
Donner trois argumentations: avec les formes coordonnées, avec une composition d'endomorphismes, avec des opérations élémentaires.
\end{enumerate}
