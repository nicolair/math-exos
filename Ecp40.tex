\begin{tiny}(Ecp40)\end{tiny} \label{Ecp40} Soit $a$, $b$, $c$ trois nombres complexes distincts de module $1$.\newline
Soit $u$ et $v$ complexes, on considère l'application $s$ de $\C$ dans $\C$ défnie par
\begin{displaymath}
  z\mapsto u \overline{z} +v
\end{displaymath}
\begin{enumerate}
  \item Déterminer $u$ et $v$ pour que $s(b)=b$ et $s(c)=c$. On dit que $b$ et $c$ sont des points fixes de $s$. 
  \item Calculer alors $s\circ s$ et montrer que si $m$ est l'affixe d'un point de la droite $(BC)$  (points $B$ et $C$ d'affixes $b$ et $c$) alors $s(m)=m$.\newline
  On admettra que $s$ est l'expression complexe de la symétrie par rapport à la droite $(BC)$.
\end{enumerate}
