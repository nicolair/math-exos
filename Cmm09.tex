\begin{tiny}(Cmm09)\end{tiny} le terme $i,i$ de la matrice $B\,\mathstrut^t\! B$ est $\sum_{j}b_{i,j}^2$.\newline
Comme $AX\,\mathstrut^t\!X$ est une matrice à $q$ colonnes, on peut la multiplier par $\mathstrut^t\!A$ qui a $q$ lignes. On exploite ensuite l'associativité du produit matriciel.
\begin{displaymath}
 0_{\mathcal M_{p,q}(\R)}=AX\,\mathstrut^t\!X\mathstrut^t\!A= (AX)\,\mathstrut^t\!(AX)
\end{displaymath}
 Prenns $AX$ dans le rôle de $B$. Dans $\R$, une somme de carrés n'est nulle que si tous les termes sont nuls. Le calcul du début montre alors que $AX=0_{\mathcal M_{p,r}(\R)}$.