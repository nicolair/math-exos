\begin{tiny}(Ccp08)\end{tiny} Notons $A$, $B$, $C$, $D$ les points respectivement d'affixe $a$, $b$, $c$, $d$ et $T$ l'expression proposée par l'énoncé. Comme on ne sait pas trop quoi faire, on développe. Sur les 12 termes, la moitié se simplifie ce qui montre que $T$ est imaginaire pur.
\begin{multline*}
 T = -a\overline{b} + \overline{a}b  + a\overline{c} - \overline{c} a -b\overline{c} + b\overline{c}\\
 = 2i\Im\left( \overline{a}b  + a\overline{c} + b\overline{c} \right) \in i\R.
\end{multline*}

Si une somme de 3 nombres complexes est imaginaire pure, lorsque 2 sont imaginaires purs, le 3ième l'est automatiquement. Chaque quotient correspond à une terme de $T$. Par exemple
\begin{multline*}
 \left( \frac{d-a}{b-c} = \frac{1}{|b-c|^2}(d-a)(\overline{b} - \overline{c}) \right) \\
 \Rightarrow
\left( \frac{d-a}{b-c} \in i\R \Leftrightarrow (d-a)(\overline{b} - \overline{c})\right) .
\end{multline*}

On en déduit que les 3 hauteurs se coupent car le caractère imaginaire pur d'un quotient correspond à l'appartenance à une hauteur. Par exemple:
$ \frac{d-a}{b-c} \in i\R$ si et seulement si $D$ est sur la hauteur du triangle $ABC$ issue de $A$.\newline
On a montré que si un point se trouve sur deux hauteurs de $ABC$, il se trouve obligatoirement sur le troisième.


