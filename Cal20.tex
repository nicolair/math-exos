\begin{tiny}(Cal20)\end{tiny} 
\begin{enumerate}
 \item Si $a$ est inversible on vérifie facilement que $\delta_{a^{-1}}$ est la bijection réciproque de $\delta_a$.
\item $\delta_a \circ \delta_b = \delta_{b*a}$.
\item Supposons $\delta_a$ bijective et notons $b$ l'antécédent de $e$. On a donc $b * a=e$. Mais a-t-on $a*b=e$ ?\newline
En fait
\begin{displaymath}
 \delta_a \circ \delta_b = \delta_{b*a} = \delta_e = \Id_E
\end{displaymath}
Comme $\delta_a$ est bijective, on peut composer à gauche par la bijection réciproque et obtenir :
\begin{displaymath}
 \delta_b = \left( \delta_a\right)^{-1} 
\end{displaymath}
De plus $a=e*a=\delta_a(e)$ donc
\begin{displaymath}
 a*b = \delta_b(a)=\delta_b \circ \delta_a (e) = e
\end{displaymath}
\end{enumerate}
