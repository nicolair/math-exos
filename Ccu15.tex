\begin{tiny}(Ccu15)\end{tiny} En regroupant deux par deux les termes consécutifs,
\begin{displaymath}
  a_n = \frac{1}{1\times 2} + \frac{1}{3\times 4} +\cdots + \frac{1}{(2n-1)\times (2n)}
\end{displaymath}
On en tire que la suite est croissante avec
\begin{displaymath}
  a_n - a_{n-1} = \frac{1}{(2n-1)\times (2n)}
\end{displaymath}
On reconnait une suite qui s'exprime comme une différence de termes consécutifs:
\begin{displaymath}
  \frac{1}{(k-1)k} = \frac{1}{k-1} - \frac{1}{k}
\end{displaymath}
On ajoute les termes en 
\begin{displaymath}
\frac{1}{2\times 3}, \frac{1}{4\times 5} \cdots  
\end{displaymath}
qui manquent pour pouvoir sommer en domino. On en déduit
\begin{multline*}
  a_n \leq \frac{1}{1\times 2} + \frac{1}{2\times 3}+ \frac{1}{3\times 4} +\cdots \\
  + \frac{1}{(2n-2)\times (2n-1)}+ \frac{1}{(2n-1)\times (2n)}\\
  = 1 - \frac{1}{2n} \leq 1
\end{multline*}

On en tire la convergence de la suite comme suite croissante majorée.
