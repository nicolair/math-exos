\begin{tiny}(Cpo36)\end{tiny} D'après les relations entre coefficients et racines, ils sont de la forme
\begin{displaymath}
 P = X^3-X^2+uX-1
\end{displaymath}
avec $u\in \C$. Considérons un polynôme $Q$ obtenu en inversant soit
\begin{displaymath}
 Q = 1-X + \bar{u}X^2 - X^3
\end{displaymath}
Pour tout $z$ complexe non nul,
\begin{displaymath}
 \widetilde{P}(\frac{1}{\bar{z}})
= \frac{1}{\bar{z}^3}
\left(1 -\bar{z} + u\bar{z}^2-\bar{z}^3\right)
= \frac{\overline{\widetilde{Q}(z)}}{\bar{z}^3}  
\end{displaymath}
De plus, $z$ est de module 1 si et seulement si
\begin{displaymath}
 z=\frac{1}{\bar{z}}
\end{displaymath}
On en déduit que les racines de $P$ sont de module $1$ si et seulement si $P$ et $Q$ ont les mêmes racines c'est à dire  égaux à un facteur complexe non nul près. On doit donc avoir $Q=-P$ ce qui se produit si $u=1$. Le seul polynôme unitaire vérifiant les conditions est donc
\begin{displaymath}
 X^3-X^2+X-1=(X^2+1)(X-1)
\end{displaymath}
dont les racines sont $1, i, -i$.