\begin{tiny}(Csn19)\end{tiny} Notons 
\begin{displaymath}
  U_n = \sum_{k=0}^nu_k,\hspace{0.5cm} V_n = \sum_{k=0}^nv_k
\end{displaymath}
Comme $\sigma$ est une bijection, il est clair que
\begin{displaymath}
\forall n \in \N, \; \sum_{k=0}^n|v_k| = \sum_{k=0}^n|u_{\sigma(k)}| \leq \sum_{k=0}^{+\infty}|u_{k}|   
\end{displaymath}
On en déduit l'absolue convergence de $\serie{v}$.\newline
Pour montrer que l'égalité des sommes, considérons :
\begin{displaymath}
\forall n\in \N, \;
\varphi_n = 
\min\left\lbrace k\text{ tq } \llbracket 0,n \rrbracket \subset \sigma\left(\llbracket 0,k \rrbracket \right) \right\rbrace 
\end{displaymath}
On peut conclure avec
\begin{displaymath}
\left|V_{\varphi_n}-U_n \right| \leq \sum_{k=n+1}^{+\infty}|u_{k}|  
\end{displaymath}
