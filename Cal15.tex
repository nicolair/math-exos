\begin{tiny}(Cal15)\end{tiny}
\begin{enumerate}
 \item L'inversion définit une bijection de $A$ dans $A'$. La multiplication à droite par un $z$ fixé est injective car on peut multiplier à gauche par $z^{-1}$. On en déduit que $zA'$ a le même nombre d'éléments que $A$.
 \item Pour tout $z$,
\begin{displaymath}
 \sharp A + \sharp zA' = 2\sharp A > \sharp G
\end{displaymath}
On en déduit que ces deux parties ne sont pas disjointes. Il existe $a$ et $b$ dans $A$ tel que $a=zb^{-1}$ donc $z=ab$. Tout élément de $G$ est produit de deux éléments de $A$.
\end{enumerate}
