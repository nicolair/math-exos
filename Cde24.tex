\begin{tiny}(Cde24)\end{tiny}
\begin{multline*}
 \sqrt{\ln (x^{2}+1)}-\sqrt{\ln(x^{2}-1)} \xrightarrow{+\infty} 0 , \;\\
 \left( x\sin \frac{1}{x}\right)^{x^{2}} \xrightarrow{+\infty} e^{-\frac{1}{6}} , \;
 \left( \tan \left( \frac{\pi }{4}+\frac{1}{x}\right) \right) ^{x} \xrightarrow{+\infty} e^{2} , \; \\
 \left( \frac{\ln (1+x)}{\ln x}\right) ^{x\ln x} \xrightarrow{+\infty} e \\
\frac{\left( (x+1)^{\frac1x}-x^{\frac 1x}\right) \left( x\ln x\right) ^2}{x^{x^{\frac 1x}}-x} \xrightarrow{+\infty} 1 
\end{multline*}
Pour la dernière fonction
\begin{multline*}
  (x+1)\cdots(x+n) = x^n (1+\frac{1}{x})(1+\frac{2}{x})\cdots(1+\frac{n}{x})=\\
  =x^n\left(1 + \frac{n(n+1)}{2x} +o(\frac{1}{x})\right) 
\end{multline*}
On en déduit (car $n$ est fixé)
\begin{multline*}
\left((x+1)\cdots(x+n)\right)^\frac{1}{n} \\
= x\left(1 + \frac{n(n+1)}{2x} +o(\frac{1}{x})\right)^{\frac{1}{n}}\\
= x\left(1 + \frac{(n+1)}{2x} +o(\frac{1}{x})\right)
\end{multline*}
La limite cherchée est donc  $\frac{n+1}{2}$.
\begin{multline*}
 x^2\left( e^{\frac{1}{x}} - e^{\frac{1}{x+1}}\right) \rightarrow \text{à compléter}, \; \\
\ln^a(1+x) - \ln^a(x) \rightarrow \text{à compléter } a>0.
\end{multline*}
