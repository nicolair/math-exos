\begin{tiny}(Eal17)\end{tiny} Action de groupe.\newline
Soit $G$ un groupe et $\Omega$ un ensemble. Les deux sont supposés finis. On désigne par $S(\Omega)$ le groupe des bijections (permutations) de $\Omega$ muni de la composition.\newline
Une \emph{action} de $G$ sur $\Omega$ est un morphisme de groupe (noté $A$) de $G$ dans $S(\Omega)$. On notera $A_g$ au lieu de $A(g)$, il s'agit d'une permutation des éléments de $\Omega$.\newline
Pour tout $\omega\in \Omega$, on note :
\begin{align*}
 &G_\omega = \left\lbrace g\in G\text{ tq } A_g(\omega)=\omega \right\rbrace\\
 &O_\omega = \left\lbrace A_g(\omega) \text{ , } \forall g\in G \right\rbrace
\end{align*}
On dit que $G_\omega$ est le \emph{stabilisateur} de $\omega$ et que $O_\omega$ est l'\emph{orbite} de $\omega$.
\begin{enumerate}
 \item Montrer que $G_\omega$ est un sous-groupe de $G$.
 \item Soit $\omega_1\in O_\omega$ et $g_0\in A$ tel que $A_{g_0}(\omega)=\omega_1$. Soit $U$ l'ensemble des $g\in G$ tels que $A_g(\omega)=\omega_1$. Montrer que  $U$ et $G_\omega$ ont le même nombre d'éléments. En déduire le nombre d'éléments de l'orbite de $\omega$.
\item Montrer que les orbites constituent une partition de $\Omega$.
\end{enumerate}
