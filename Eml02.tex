\begin{tiny}(Eml02)\end{tiny}
Soit $\alpha $ une forme lin{\'e}aire non nulle sur un $\R$ espace vectoriel $E$ (c'est {\`a} dire un {\'e}l{\'e}ment de $\mathcal{L}(E,\R)$), soit $u$ un {\'e}l{\'e}ment non nul de $E$. On d{\'e}finit une application $f$ de $E$ dans $E$ en posant $f(x)=\alpha (x)u$ pour tout {\'e}l{\'e}ment $x$ de $E$.
  \begin{enumerate}
    \item Montrer que $f$ est lin{\'e}aire, montrer qu'il existe un unique r{\'e}el $\lambda $ tel que $f\circ f=\lambda f$.
    \item Dans quel cas $\lambda $ est-il non nul? Montrer que $\frac{1}{\lambda }f$ est un projecteur dans ce cas. Pr{\'e}ciser le noyau et l'image.
  \end{enumerate}