\begin{tiny}(Eev19)\end{tiny} Soit $\K$ un corps \emph{infini} (par exemple $\Q$, $\R$ ou $\C$) et $E$ un $\K$-espace vectoriel. On considère une famille finie $A_1,\cdots, A_p$ de sous-espaces vectoriels vérifiant:
\begin{itemize}
  \item $A_1 \cup \cdots \cup A_p$ est un sous-espace vectoriel de $E$,
  \item il existe $i$ et $j$ distincts tels que $A_i \nsubseteq A_j$.
\end{itemize}
Montrer que 
\begin{displaymath}
  A_j \subset \bigcup_{k\neq j} A_k
\end{displaymath}
On pourra considérer des $a_i + \lambda x$ avec $x\in A_j$ et $a_i\in A_i$ tel que $a_i\notin A_j$ puis utiliser le \emph{principe des tiroirs}. 
\begin{quote}
  Si on veut ranger strictement plus de $q$ objets dans $q$ tiroirs, au moins un tiroir contient plusieurs objets.
\end{quote}
