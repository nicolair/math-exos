\begin{tiny}(Cip29)\end{tiny}
\begin{enumerate}
 \item Une première démonstration par majoration. La fonction est de signe constant au voisinage de $+\infty$. Supposons la positive (on peut utiliser $-F$). L'hypothèse sur les degrés permet de la majorer par une fonction de la forme
\begin{displaymath}
 t\mapsto \frac{A}{t^2}
\end{displaymath}
Les primitives de $F$ sont donc croissantes et majorées au voisinage de $+\infty$ ce qui assure la convergence.\newline
Présentons une démonstration plus algébrique. 
Dans la décomposition en éléments simple de $F$ considérons la somme des résidus:
\begin{displaymath}
 \sum _{z\in \mathcal{P}_r}\frac{\lambda(z)}{t-z} 
+ \sum _{z\in \mathcal{P}_c}\frac{\lambda(z)}{t-z}
\end{displaymath}
où $\mathcal{P}_r$ $\mathcal{P}_c$ désignent respectivement les ensembles de pôles réels et non réels de la fraction.
D'après le cours, une primitive de cette fonction est
\begin{multline*}
 \sum _{z\in \mathcal{P}_r}\lambda(z) \ln|t-z| \\
+ \sum _{z\in \mathcal{P}_c}\lambda(z)\left(  \ln|t-z|+i\arctan\frac{t-\Re(z)}{\Im(z)}\right) 
\end{multline*}
Chacune des parties en $\arctan$ converge en $+\infty$ vers $\frac{\pi}{2}$. Pour montrer la convergence des parties logarithmiques, il faut les considérer toutes à la fois. Mélangeons tous les pôles, réels ou non:
\begin{multline*}
 \sum_{k=0}^{p}\lambda_k\ln|t-z_k| \\
= \left( \sum_{k=0}^{p}\lambda_k\right) \ln|t|
+ \sum_{k=0}^{p}\lambda_k\ln|1-\frac{z_k}{t}|
\end{multline*}
La somme des résidus est nulle car la fraction est de degré $\leq 2$. On le montre en multipliant par $t$ et en faisant tendre $t$ vers $0$. On en déduit la convergence des primitives car les autres termes de la décomposition en éléments simples conduisent à des fractions de degré négatifs.
 \item
 \item 
\end{enumerate}
 