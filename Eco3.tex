\begin{figure}[htp]
 \centering
 \input{Eco3_1.pdf_t}
 \caption{Exercice \arabic{enumi} : tangente à une parabole}
 \label{fig:Eco3_1}
\end{figure}
\begin{tiny}(Eco3)\end{tiny} Propriétés des tangentes : parabole.\newline
Dans un plan muni d'un repère orthonormé, on se donne un réel $p>0$, un point $F$ et une parabole $\mathcal P$ paramétrée par :
\begin{displaymath}
 M(\theta) = F + \frac{p}{1+ \cos\theta}\overrightarrow{e_\theta}
\end{displaymath}
Déterminer des expressions simples de $\lambda(\theta)$ et $\varphi(\theta)$ pour que :
\begin{displaymath}
 \overrightarrow{M'}(\theta) = \lambda(\theta)\overrightarrow{e_{\varphi(\theta)}}
\end{displaymath}
En déduire que la tangente (figure \ref{fig:Eco3_1}) est la médiatrice de $FH(\theta)$. Comment se réfléchit sur $\mathcal P$ un rayon parallèle à l'axe focal ?