\begin{tiny}(Cen15)\end{tiny} Dans $\llbracket 1,q\rrbracket$, on fixe une partie $F$ à 3 éléments (par exemple $\left\lbrace 1,2,3\right\rbrace$). On classe les parties de $\llbracket 1,q\rrbracket$ à $p$ éléments selon leur intersection avec $F$ puis on regroupe les parties pour lesquelles cette intersection contient $0,1,2$ ou $3$ éléments.\newline
Pour la deuxième formule, on fait la même chose avec une partie $F$ à $r$ éléments. Le $\binom{r}{k}$ compte les parties de $F$ avec $k$ éléments et le $\binom{q-k}{p-r}$ compte les parties à $p-r$ éléments dans le complémentaire de $F$.