\begin{tiny}(Cga04)\end{tiny}
\begin{enumerate}
  \item D'après l'expression barycentrique de la condition d'alignement, les points $L$, $M$, $N$ sont alignés si et seulement si
\begin{displaymath}
  \begin{vmatrix}
    0      & 1       & -\alpha \\
    -\beta & 0       & 1       \\
    1      & -\gamma & 0
  \end{vmatrix}
=0.
\end{displaymath}
Ce déterminant est égal à $1 - \alpha \beta \gamma$.\newline
En projetant sur un vecteur unitaire de la droite $(BC)$, on passe de la définition du barycentre à une expression de $\alpha$ comme quotient de valeurs algébriques:
\begin{displaymath}
  \overrightarrow{LB} - \alpha \overrightarrow{LC}= \overrightarrow{0}
  \Rightarrow
  \overline{LB} - \alpha \overline{LC}= 0.
\end{displaymath}
On en déduit la condition demandée.

  \item On utilise cette fois la condition de concours de trois droites exprimée comme la nullité d'un déterminant formé à partir des équations des droites. On forme les équations des droites dans le repère $(A,\overrightarrow{AB},\overrightarrow{AC})$:
\begin{align*}
  (AL) &:& \alpha x + y = 0 \\
  (BM) &:& x + (1-\beta)y - 1 = 0 \\
  (CN) &:& (1-\gamma)x - \gamma y + \gamma = 0
\end{align*}
La condition demandée (théorème de Ceva) vient du calcul du déterminant
\begin{displaymath}
  \begin{vmatrix}
    \alpha   & 1       & 0 \\
    1        & 1-\beta & -1 \\
    1-\gamma & -\gamma & \gamma 
  \end{vmatrix}
  =
  \begin{vmatrix}
    \alpha   & 1       & 0 \\
    0        & -\beta & -1 \\
    1        & 0      & \gamma 
  \end{vmatrix}
\end{displaymath}
en enlevant la troisième colonne aux deux premières.
\begin{displaymath}
  = - 1 - \alpha \beta \gamma.
\end{displaymath}

  \item Considérons
\begin{displaymath}
  \sin (\widehat{NB})\, \overrightarrow{OA} - \sin (\widehat{NA})\, \overrightarrow{OB}.
\end{displaymath}
Sa projection sur la direction orthogonale à $\overrightarrow{ON}$ est
\begin{displaymath}
  \sin (\widehat{NB})\, \sin (\widehat{NA}) - \sin (\widehat{NA})\, \sin (\widehat{NB})=0.
\end{displaymath}
Il est donc de même direction que $\overrightarrow{ON}$.\newline
Remarquons que sa projection sur $\overrightarrow{ON}$ est
\begin{multline*}
  \sin (\widehat{NB})\, \cos (\widehat{NA}) - \sin (\widehat{NA})\, \cos (\widehat{NB})\\
  = \sin \left( \widehat{NB} - \widehat{NA}\right).
\end{multline*}
Elle est nulle si et seulement si $A$ et $B$ sont diamétralement opposés.\newline
On peut former des vecteurs analogues pour $M$ et $L$. Les points $M$, $N$, $L$ sont sur un même grand cercle si et seulement si les vecteurs sont dans un même plan. Cela se traduit par la nullité du déterminant formé par les coordonnées dans la base $\left( \overrightarrow{OA}, \,\overrightarrow{OB}, \,\overrightarrow{OC}\right)$
\begin{displaymath}
  \begin{vmatrix}
    0                          & \sin(\overrightarrow{MC})  & \sin(\overrightarrow{NB}) \\ 
    \sin(\overrightarrow{LC})  & 0                          & -\sin(\overrightarrow{NA})  \\
    -\sin(\overrightarrow{LB}) & -\sin(\overrightarrow{MA}) & 0 
  \end{vmatrix}
.
\end{displaymath}
Ce déterminant se traite comme dans la question a.

  \item Pour caractériser l'alignement des milieux, on utilise encore la condition avec les coordonnées barycentriques. Il faut faire attention à prendre la même masse pour obtenir les coordonnées barycentriques d'un milieu. Par exemple, les coordonnées du milieu de $AL$ sont
\begin{displaymath}
  A (1-\alpha,0,0) + L(0,1,-\alpha) \rightarrow (1-\alpha, 1, -\alpha)
\end{displaymath}
De même pour les autres points. On conclut avec un calcul de déterminant
\begin{displaymath}
\begin{vmatrix}
  1- \alpha & 1       & -\alpha \\
  -\beta    & 1-\beta & 1 \\
  1         & -\gamma & 1-\gamma
\end{vmatrix}
= \cdots = 2(1-\alpha \beta \gamma).
\end{displaymath}

Chaque condition d'alignement se traduit par la nullité d'un déterminant
\begin{align*}
  L, M, N \text{ alignés } &\Leftrightarrow D = 0 \\
  L', M', N' \text{ alignés } &\Leftrightarrow  D' = 0
\end{align*}
avec
\begin{displaymath}
  D = 
\begin{vmatrix}
  0      & 1       & - \alpha \\
  -\beta & 0       & 1 \\
  1      & -\gamma & 0
\end{vmatrix}
\end{displaymath}
\begin{displaymath}
  D' = 
\begin{vmatrix}
  0       & - \alpha & 1  \\
  1       & 0       & -\beta \\
  -\gamma & 1 & 0
\end{vmatrix}
\end{displaymath}
En développant, on trouve 
\begin{displaymath}
  D = D' = 1 - \alpha \beta \gamma
\end{displaymath}
ce qui assure l'équivalence des alignements.
\end{enumerate}
