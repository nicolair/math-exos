\begin{tiny}(Cga04)\end{tiny}
Dans cet exercice, on utilise les conditions d'alignement ou de concours trouvées dans l'exercice ga01. 
\begin{enumerate}
  \item D'après l'expression barycentrique de la condition d'alignement, les points $L$, $M$, $N$ sont alignés si et seulement si
\begin{displaymath}
  \begin{vmatrix}
    0      & 1       & -\alpha \\
    -\beta & 0       & 1       \\
    1      & -\gamma & 0
  \end{vmatrix}
=0.
\end{displaymath}
Ce déterminant est égal à $1 - \alpha \beta \gamma$.\newline
En projetant sur un vecteur unitaire de la droite $(BC)$, on passe de la définition du barycentre à une expression de $\alpha$ comme quotient de valeurs algébriques:
\begin{displaymath}
  \overrightarrow{LB} - \alpha \overrightarrow{LC}= \overrightarrow{0}
  \Rightarrow
  \overline{LB} - \alpha \overline{LC}= 0.
\end{displaymath}
On en déduit la condition demandée.
  \item On utilise cette fois la condition de concours de trois droites exprimée comme la nullité d'un déterminant formé à partir des équations des droites. On forme les équations des droites dans le repère $(A,\overrightarrow{AB},,\overrightarrow{AC})$:
\begin{align*}
  (AL) &:& \alpha x + y = 0 \\
  (BM) &:& x + (1-\beta)y - 1 = 0 \\
  (CN) &:& (1-\gamma)x - \gamma y + \gamma = 0
\end{align*}
La condition demandée (théorème de Ceva) vient du calcul du déterminant
\begin{displaymath}
  \begin{vmatrix}
    \alpha   & 1       & 0 \\
    1        & 1-\beta & -1 \\
    1-\gamma & -\gamma & \gamma 
  \end{vmatrix}
= - 1 - \alpha \beta \gamma.
\end{displaymath}

  \item Pour caractériser l'alignement des milieux, on utilise encore la condition avec les coordonnées barycentriques. Il faut faire attention à prendre la même masse pour obtenir les coordonnées barycentrique d'un milieu. Par exemple, les coordonnées du milieu de $AL$ sont
\begin{displaymath}
  A (1-\alpha,0,0) + L(0,1,-\alpha) \rightarrow (1-\alpha, 1, -\alpha)
\end{displaymath}
De même pour les autres points. On conclut avec un calcul de déterminant
\begin{displaymath}
\begin{vmatrix}
  1- \alpha & 1       & -\alpha \\
  -\beta    & 1-\beta & 1 \\
  1         & -\gamma & 1-\gamma
\end{vmatrix}
= \cdots = 2(1-\alpha \beta \gamma).
\end{displaymath}

Chaque condition d'alignement se traduit par la nullité d'un déterminant
\begin{align*}
  L, M, N \text{ alignés } &\Leftrightarrow D = 0 \\
  L', M', N' \text{ alignés } &\Leftrightarrow  D' = 0
\end{align*}
avec
\begin{displaymath}
  D = 
\begin{vmatrix}
  0      & 1       & - \alpha \\
  -\beta & 0       & 1 \\
  1      & -\gamma & 0
\end{vmatrix}
\end{displaymath}
\begin{displaymath}
  D' = 
\begin{vmatrix}
  0       & - \alpha & 1  \\
  1       & 0       & -\beta \\
  -\gamma & 1 & 0
\end{vmatrix}
\end{displaymath}
En développant, on trouve 
\begin{displaymath}
  D = D' = 1 - \alpha \beta \gamma
\end{displaymath}
ce qui assure l'équivalence des alignements.
\end{enumerate}
