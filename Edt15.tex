\begin{tiny}(Edt15)\end{tiny} Soit $\alpha_1$, $\alpha_2$, $\alpha_3$, $\alpha_1'$, $\alpha_2'$, $\alpha_3'$ six éléments non nuls d'un corps $\K$. On considère la matrice
\begin{displaymath}
 M=
\begin{pmatrix}
 \alpha_2'-\alpha_3' & \alpha_2 - \alpha_3 & \alpha_2\alpha_2'-\alpha_3\alpha_3' \\
 \alpha_3'-\alpha_1' & \alpha_3 - \alpha_1 & \alpha_3\alpha_3'-\alpha_1\alpha_1' \\
 \alpha_1'-\alpha_2' & \alpha_1 - \alpha_2 & \alpha_1\alpha_1'-\alpha_2\alpha_2'
\end{pmatrix}
\end{displaymath}
Soit $C_1$, $C_2$, $C_3$ les trois colonnes de cette matrice. Montrer que $\det M = 0$ en considérant
\begin{displaymath}
 C_3-\alpha_2C_1-\alpha_3'C_2
\end{displaymath}
La nullité de ce déterminant est utilisée dans l'exercice ga03 de la feuille sur les \href{\exosurl _fex_ga.pdf}{espaces affines}