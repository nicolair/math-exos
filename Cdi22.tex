\begin{tiny}(Cdi22)\end{tiny} Soit $x$ dans $\ker f$, alors $x= (f+g)(x)=g(x)$ donc $\ker f \subset \Im g$. D'autre part, l'inégalité avec les rangs s'écrit aussi $\rg g \leq \dim E - \rg f = \dim \ker f$ d'après le théormème du rang. De $\ker f \subset \Im g$ et $\rg g \leq \dim \ker f$, on tire l'égalité des espaces $\ker f =\Im g$. Ceci entraine que $f\circ g$ est l'endomorphisme nul. On peut donc composer la somme à gauche par $f$ :
\begin{displaymath}
 f=f\circ (f+g) = f\circ f + f\circ g = f\circ f
\end{displaymath}
ce qui assure que $f$ est une projection.
 