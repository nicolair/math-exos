\begin{tiny}Cvs29\end{tiny} Comme $f$ est injective, $f(\llbracket 0,1 \rrbracket)$ est une paire qui est un intervalle c'est donc $\llbracket -1, 0 \rrbracket$ ou $\llbracket 0,1 \rrbracket$.\newline
Supposons  $f(1)= 1$ et montrons par récurrence $f(n)=n$ pour tous les $n\in \N$. Si $f(n)=n$, il suffit de considérer l'image de $\llbracket n, n+1 \rrbracket$.\newline
Pour les nombres négatifs, on considère $\llbracket n-1, n\rrbracket$ en supposant $f(n) = n$.\newline
Si $f(1)=-1$, on se ramène au cas précédent en considérant $-f$ qui vérifie les mêmes propriétés.\newline
Si on ne suppose plus $f(0)=0$, il existe d'autres solutions par exemple les translations. On peut montrer alors que les seules solutions sont les fonctions
\[
  n \mapsto \varepsilon n + c
\]
avec $c\in \Z$ et $\varepsilon = \pm 1$.
