\begin{tiny}(Cal09)\end{tiny} Il s'agit en fait de montrer que la partie $A$ est stable par inversion. C'est à dire que pour tout $a\in A$ l'inverse de $A$ est encore dans $A$.\newline
Si $a=e$ c'est évident. Supposons $a\neq e$. Comme $A$ est stable, on peut définir une application $n\mapsto a^n$ de $\N^{*}$ dans $A$. Comme $A$ est finie, cette application n'est pas injective. Il existe donc $p<q$ dans $\N^*$ tels que
\[
 a^p = a^q \Rightarrow a^{q-p-1}*a = e \Rightarrow a^{q-p-1} = a^{-1}.
\]
Par définition $q-p-1\geq 0$ et l'inégalité est stricte sinon on aurait $a^{-1}=e =a$. Donc $a^{-1}$ s'exprime comme une puissance de $a$ avec un exposant naturel non nul. Par stabilité de $A$, ceci entraîne $a^{-1}\in A$.
