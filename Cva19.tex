\begin{tiny}(Cva19)\end{tiny}
\begin{enumerate}
  \item La variable $X_2$ suit une loi de Bernoulli de paramètre $\frac{1}{2}$. La loi de $X_3$ est donnée par le tableau
\begin{center}
\renewcommand{\arraystretch}{1.5}
\begin{tabular}{|l|c|c|c|} \hline
$k\in X_3(\Omega)$ & 0             & 1 & 2\\ \hline
$\p(X_3=k)$   & $\frac{1}{4}$ & $\frac{1}{2}$ & $\frac{1}{4}$ \\ \hline
\end{tabular}
\end{center}\smallskip
En effet, PPP et FFF conduisent à 0, les suites PFP et FPF à 2 et les 4 autres à 1. On en déduit 
\begin{displaymath}
E(X_2)=\frac{1}{2}, \hspace{0.5cm} E(X_3)=1.
\end{displaymath}


  \item Il est clair que $X_n(\Omega) = \llbracket 0,n-1 \rrbracket$ et que 
\begin{displaymath}
\p(X_n=0)=\p(X_n=n-1) = \frac{2}{2^{n}} =2^{1-n} 
\end{displaymath}
car les événements réalisant ces valeurs sont fixés par le premier lancer: PPP... ou FFF... pour 0, PFPFPF... ou FPFP... pour $n-1$.
  \item Pour une pièce équilibrée,
\begin{displaymath}
\p(X_{n+1}=k) = \frac{1}{2}\left( \p(X_{n}=k) + \p(X_{n}=k-1)\right)   
\end{displaymath}
car cela revient à fixer le résultat du lancer $n+1$. Si $X_n=k$ avec F au lancer $n$ le lancer $n+1$ doit donner F etc.

\item La fonction génératrice $G_n$ est polynomiale car la somme est finie (entre $0$ et $n-1$). La relation est une conséquence de c. En dérivant, on obtient
\begin{multline*}
G_{n+1}'(s) = \frac{1}{2}G_n(s) + \frac{1+s}{2}G_n'(s)\\
\Rightarrow E(X_{n+1}) = \frac{1}{2} \underset{=1}{\underbrace{G_n(1)}} + E(X_n)
\Rightarrow E(X_n) = \frac{n}{2}.
\end{multline*}

\end{enumerate}
