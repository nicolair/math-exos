\begin{tiny}(Cal01)\end{tiny} Soit $A$ un anneau intègre fini de cardinal $n$ et $a$ un élément non nul de $A$. On veut montrer que $a$ est inversible. Considérons l'application de $\N^*$ dans $A$ qui à un entier $k$ associe $a^k$. Comme $A$ est fini et $\N^*$ infini, cette application n'est pas injective. Il existe donc des entiers $i<j$ dans $\N^*$ tels que $a^i = a^j$.\newline
Remarquons que $a^i\neq 0_A$ car s'il l'était, en simplifiant par $a$ on obtiendrait $a^{i-1}$ nul et ainsi de suite jusqu'à une contradiction. On peut donc simplifier par $a^i$ et obtenir
$a^{j-i}=1_A$ avec $j-i>0$. On en déduit que $a$ est inversible d'inverse $a^{j-i-1}$. 
