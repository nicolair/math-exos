\begin{tiny}(Cpo03)\end{tiny} 
\begin{itemize}
  \item Divisibilité pour a. b. c.\newline
Les polynomes de degré 2 se factorisent facilement.
\begin{multline*}
  H = (X-e^a)(X-e^{-a}), \hspace{0.2cm} C = (X-e^{ia})(X-e^{-ia}), \\
  P = (X-1)(X-2).
\end{multline*}
En substituant respectivement les racines au $X$ de $H_n$, $C_n$, $P_n$, on trouve à chaque fois $0$ ce qui assure (cours) que les restes sont nuls.

  \item Quotient pour $H_n$.\newline
De la relation trigonométrique
\[
  2\sh(na) \ch(a) = \sh((n+1)a) + \sh((n-1)a),
\]
on déduit $H_n - \sh(na)X^{n-1} = H_{n-1}$.\newline
Puis le quotient
\[
  \sh(na)X^{n-1} + \cdots + \sh(2a)X + \sh(a).
\]

  \item Quotient pour $C_n$.\newline
Notons $Q_n$ le polynôme tel que $C_n = Q_n C$. En posant les divisions, on trouve
\[
  Q_2 = \sin(a), \; Q_3 = \sin(a) X + \sin(2a).
\]
On induit que 
\[
  Q_n = \sin(a) X^{n-2} + \sin(2a)X^{n-3} + \cdots + \sin((n-1)a).
\]
Pour le vérifier, on multiplie ce polynôme par
\[
  C = X^2 - 2\cos(a) X + 1.
\]
Présentons les coefficients en tableau
\begin{align*}
  &X^n : \sin(a) \\
  &X^{n-1} : \sin(2a) - 2\cos(a)\sin(a) = 0 \\
  &X^{n-2} : \sin(3a) - 2\cos(a)\sin(2a) + \sin(a) = 0\\
  & \hspace{0.5cm} \vdots                                         \\
  &X^{2} : \sin(n-1)a -                    ...  = 0\\
  &X  : - 2\cos a\sin(n-1)a + \sin(n-2)a = -\sin na\\
  &X^0 : \sin(n-1)a  
\end{align*}
On a bien montré que
\[
  Q_n C = \sin(a)X^n - \sin(na) X + \sin((n-1)a.
\]

  \item Quotient pour $P_n$.\newline
Cette fois, on réussit à factoriser directement avec des sommes en progression géométriques.
\begin{multline*}
  (X-1)^n - 1 =   \left((X -1) - 1 \right) \\ 
  \left( (X-1)^{n-1} +  \cdots + (X-1) + 1\right)
\end{multline*}
Donc
\begin{multline*}
  P_n = (X-2)( \underbrace{(X-2)^{2n-1} +1} + (X-1) + \\
  \cdots + (X-1)^{n-1}).
\end{multline*}
De même
\begin{multline*}
  (X-2)^{2n-1} +1 = 1 - (2-X)^{2n-1} \\
  = (1-2 + X)\left(1 + (2-X) + \cdots +(2-X)^{2n-2}\right).
\end{multline*}
Finalement, le quotient est
\begin{multline*}
  1 + (2-X) + \cdots +(2-X)^{2n-2} \\
  + (X-1)^{n-1} +  \cdots + (X-1) .
\end{multline*}

\end{itemize}


