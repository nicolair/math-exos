\begin{tiny}(Cpb13)\end{tiny} 
\begin{enumerate}
  \item On modélise par l'ensemble des parties à 5 éléments dans l'ensemble des 52 cartes.\newline
Pour compter les mains contenant 3 carreaux, on classe par les ensembles de 3 carreaux puis par les ensembles de 2 cartes dans les autres couleurs.\newline
Pour l'évènement \og au moins une paire\fg, on considère l'événement contraire soit : \og aucune paire\fg. On classe les mains par l'ensemble des 5 hauteurs puis on identifie aux fonctions de l'ensemble des 5 hauteurs dans l'ensemble des 4 couleurs. On présente les résultats dans un tableau
\begin{center}
\renewcommand{\arraystretch}{2.5}
\begin{tabular}{l|l|l}
événement   & 3 carreaux                                        & au moins une paire\\ \hline 
probabilité & $\dfrac{\binom{13}{3}\binom{39}{2}}{\binom{52}{5}}$ & $1 - \dfrac{\binom{13}{3}\, 4^5}{\binom{52}{5}}$
\end{tabular}
\end{center}
  \item On numérote les joueurs de 1 à 4. L'univers est formé par les quadruplets de parties  de 13 cartes formant partition. Pour compter, on classe par la main du joueur 1 puis 2, .. donc
\[
  \binom{52}{13}\, \binom{39}{13}\,\binom{26}{13} \text{ distributions de mains}
\]
Pour compter les distributions dans lesquelles chaque joueur a un as. On classe selon les distributions des cartes hors as puis par la distribution des as; soit 
\[
  \binom{48}{12}\, \binom{36}{12}\,\binom{24}{12}\, 4!
\]
La probabilité que chacun ait un as est donc
\[
  \frac{13\times 39 \times 38 \times 37}{52\times 51 \times 50 \times 49}
  \times
  \frac{13\times 26 \times 25}{39\times 38 \times 37}
  \times
  \frac{13\times 13}{26\times 25} \times 4!
  = \frac{13^4 \times 4!}{52\times 51 \times 50 \times 49}.
\]

\end{enumerate}

