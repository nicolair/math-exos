\begin{tiny}(Cde26)\end{tiny} Appliquons le théorème des accroissements finis à la fonction exponentielle entre $v(x)\ln(u(x))$ et $u(x)\ln(v(x))$.\newline
Il existe $w_x$ entre ces deux nombres tel que
\begin{multline*}
 u(x)^{v(x)} - v(x)^{u(x)} \\
 = e^{w_x}\left( v(x)\ln(u(x)) - u(x)\ln(v(x))\right)\\
 = e^{w_x}u(x)v(x)\left( \frac{\ln(u(x))}{u(x)} - \frac{\ln(v(x))}{v(x)}\right). 
\end{multline*}
Appliquons le théorème des accroissements finis entre $u(x)$ et $v(x)$ à la fonction
\[
 t\mapsto \frac{\ln t}{t}
\]
Il existe $z_x$ entre ces deux nombres tel que
\begin{multline*}
 u(x)^{v(x)} - v(x)^{u(x)} \\
 = e^{w_x}u(x)v(x)\frac{1-\ln(z_x)}{z_x^2}\left(u(x) - v(x)\right). 
\end{multline*}
On peut alors simplifier $u(x) - v(x)$ et 
\begin{multline*}
 \left. 
 \begin{aligned}
  u(x) \text{ et } v(x) &\rightarrow a\\
  w_x &\rightarrow a\ln a \\
  z_x &\rightarrow a
 \end{aligned}
\right\rbrace 
\Rightarrow
\frac{u^{v} - v^{u}}{u - v} \\
\rightarrow e^{a\ln a} (1 - \ln a) = a^a (1 - \ln a).
\end{multline*}
