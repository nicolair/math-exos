\begin{tiny}(Ccu12)\end{tiny} Les deux premières relations s'obtiennent par sommation en dominos à partir des relations
\begin{align*}
  &\frac{1}{k(k+1)(k+2)} = \frac{1}{2}\left( \frac{1}{k(k+1)} - \frac{1}{(k+1)(k+2)}\right) \\
  &\frac{k}{(k+1)!} = \frac{1}{k!} - \frac{1}{(k+1)!}
\end{align*}
La troisième relation se démontre par récurrence.\newline
Pour $n=1$, elle s'écrit $2! \geq 2!$.\newline
Montrons que l'inégalité à l'ordre $n$ entraîne celle à l'ordre suivant.
\begin{multline*}
  ((n+2)!)^{n+1} = ((n+1)!)^{n}(n+1)!(n+2)^{n+1} \\
  \leq \left( 2!4!\cdots (2n)!\right)(n+1)!(n+2)^{n+1} 
\end{multline*}

On doit donc montrer que 
\begin{displaymath}
  (n+1)!(n+2)^{n+1} \leq (2n+2)!
\end{displaymath}
Or:
\begin{displaymath}
  \frac{(2n+2)!}{(n+1)!(n+2)^{n+1}} = \frac{(n+3)(n+4)\cdots (2n+2)}{(n+2)^{n}}
\end{displaymath}
et le dénominateur contient $2n+2-n-3+1=n$ facteurs tous plus grands que $n+2$. Le quotient est donc plus grand que $1$.\newline
La dernière relation est encore une sommation en dominos obtenue à partir de la formule du triangle de Pascal. Pour $n<k$,
\begin{multline*}
\binom{k}{n} + \binom{k}{n+1} = \binom{k+1}{n+1} \\
\Rightarrow \binom{k}{n} = \binom{k+1}{n+1} - \binom{k}{n+1} \\
\Rightarrow
\sum_{k=n}^p \binom{k}{n} = 1 + \sum_{k=n+1}^p\left( \binom{k+1}{n+1} - \binom{k}{n+1}\right)\\
= \binom{p+1}{n+1}
\end{multline*}

On peut démontrer ce résultat de plusieurs autres manières.
