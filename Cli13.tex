\begin{tiny}(Cli13)\end{tiny}
Comme pour chaque $a\in I$ l'ensemble de nombres réels
\begin{displaymath}
  F_a = \left\lbrace f_n(a) , n\in \N\right\rbrace 
\end{displaymath}
est borné, il admet une borne inférieure $g(a)$ et une borne supérieure $h(a)$.\newline
Montrons que $h$ est semi-continue inférieurement.\newline
Pour tout $\varepsilon >0$, le réel $h(a)-\frac{\varepsilon}{2}$ n'est pas un majorant de $F_a$. Il existe donc $n\in \N$ tel que
\begin{displaymath}
  h(a)-\frac{\varepsilon}{2} \leq f_n(a)
\end{displaymath}
Comme $f_n$ est continue en $a$, il existe $\alpha >0$ tel que
\begin{multline*}
\forall x\in I, \left|x-a\right| \leq \alpha \Rightarrow f_n(a) - \frac{\varepsilon}{2} \leq f_n(a) \\
\Rightarrow h(a) - \varepsilon \leq f_n(a) - \frac{\varepsilon}{2} \leq f_n(a) \leq h(a)
\end{multline*}
De même que $g$ est semi-continue supérieurement.