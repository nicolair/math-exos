\begin{tiny}(Eco12)\end{tiny} Courbe orthoptique pour coniques à centre. Cercle de Monge.\\
On considère une conique $\mathcal C$ d'équation réduite
\begin{displaymath}
 \frac{x^2}{a^2}+\varepsilon \,\frac{y^2}{b^2} = 1
\end{displaymath}
 avec $\varepsilon\in \{-1,+1\}$ de manière à couvrir les cas ellipse et hyperbole.\\
Soit $M_0$, de coordonnées $(x_0,y_0)$ un point quelconque du plan et $\overrightarrow u$ de coordonnées $(u,v)$ un vecteur quelconque.
\begin{enumerate}
 \item En admettant qu'une droite est tangente à la conique si et seulement si elle la coupe en deux points confondus, former une condition caractérisant que la droite $M_0+\Vect(\overrightarrow u)$ est tangente à $\mathcal C$.\\
On écrira cette condition sous la former
\begin{displaymath}
Au^2 + Buv + Cv^2 =0
\end{displaymath}
où $A$, $B$, $C$ sont des expressions de $x_0$, $y_0$, $a$, $b$, $\varepsilon$ à déterminer.
\item Lorsque par $M_0$ passent deux tangentes, on note $p_1$ et $p_2$ les pentes de ces droites. Exprimer le produit $p_1p_2$ en fonction de $A$, $B$,$C$ puis de $x_0$, $y_0$, $a$, $b$, $\varepsilon$.
\item Déterminer l'équation de l'ensemble des points $M_0$ par lesquels passent deux tangentes orthogonales. Discuter de l'existence et de la nature de cet ensemble suivant la nature et les caractéristiques de la conique.
\end{enumerate}
 