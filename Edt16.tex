\begin{tiny}(Edt16)\end{tiny} On considère $n$ nombres réels
\begin{displaymath}
 0 < a_1 < a_2 < \cdots < a_n
\end{displaymath}
 et la fonction $P_n$, définie pour tout $x$ réel par
\begin{displaymath}
 P_n(x)=
\begin{vmatrix}
x      & a_2 & \cdots & a_n    \\
a_1    & x   &        & \vdots \\
\vdots &     & \ddots &        \\
a_1    & a_2 &        & x
\end{vmatrix}.
\end{displaymath}
Montrer que $P_n$ admet $n$ zéros réels distincts. On pourra considérer la fraction
\begin{displaymath}
 \frac{P_n(x)}{\prod_{i=1}^n(x-a_i)}.
\end{displaymath}
