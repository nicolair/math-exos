  'cp' => 2002, nombres complexes
  'cu' => 2003, calculs usuels
  'fu' => 2004, fonctions usuelles, trigo
  'ed' => 1616, equa diff
  'gp' => 2005, g�om�trie plane el�m
  'ge' => 2006, g�eo espace
  'cr' => 6430, courbes para
  'vs' => 1627, vocabulaire ensembles op�rations
  'en' => 2007, entiers nat d�nombre
  're' => 2008, corps des r�els, suites
  'li' => 2064, limites etc
  'gc' => 2072, prop glob fonc continues
  'gd' => 2070, prop glob d�riv�es
  'sr' => 4792, suites def r�curr
  'vc' => 4791, suites et fonc valeurs complexes
  'al' => 2075, groupes anneaux corps
  'po' => 1622, polynomes
  'fr' => 1623, fract rationnelles
  'ar' => 5546, arithm�tique dans Z et K[X]
  'ev' => 2076, espaces vectoriels (sans dimension)
  'di' => 2112, dimension des espaces vectoriels
  'ip' => 2190, int�grales et primitives
  'is' => 1624, int�gration sur un segment
  'de' => 2199, d�veloppements (locaux) limit�s et asymptotiques
  'fc' => 5778, fonctions convexes
  'tl' => 5768, formules de Taylor
  'cm' => 2231, calcul matriciel
  'am' => 6030, avec maple
  'gs' => 2260, groupe sym�trique
  'dt' => 2261, d�terminant
  'ee' => 2262, espaces euclidiens
  'ao' => 2263, automorphismes orthogonaux
  'ce' => 2265, (courbes euclidiennes) �tude m�trique des courbes
  'cg' => 2267, (continuit� g�om�trique) Fonction d'une variable g�om�trique : continuit�
  'dg' => 2268, (diff�rentiabilit� g�om�trique) Calcul diff�rentiel
  'ig' => 2269, (int�gration g�om�trique) FVG : calcul int�gral
  'tr' => 1614, (transverses) Cours MPSI B
  'el' => 1625, syst�mes d'�quations lin�aires
  'ap' => 2194, approximations (z�ros, integrale, nb r�el)
  'mm' => 2232, les matrices pour elles m�mes
  'mf' => 2233, matrices de familles et d'applications
  'mo' => 7805, rangs, op�rations, syst�mes
  'ga' => 5727, g�om�trie affine
  'co' => 4893, coniques
  'sc' => 2069, suites de r�els, convergence, comparaison

