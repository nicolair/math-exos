\begin{tiny}(Cen11)\end{tiny} On note $\mathcal I(E)$ l'ensemble des involutions sur un ensemble $E$. Si $E$ est un singleton, il existe une seule involution qui est aussi l'identité. Si $E$ est une paire, il existe deux involutions, l'identité et la permutation des deux éléments. On a donc
\begin{displaymath}
 x_1=1 \hspace{0.5cm}x_2=2
\end{displaymath}
Dans un ensemble $E$ à $n$ éléments, on fixe un $a$. On classe les involutions suivant l'image de $a$. Soit 
$\mathcal I_b$ l'ensemble des involutions $f$ telles que $f(a)=b$. Les $\mathcal{I}_b$, pour $b$ décrivant $E$, constituent une partition de $\mathcal I$.\newline
Pour $b\neq a$ et $f\in \mathcal{I}_b$, on a obligatoirement $f(b)=a$, l'ensemble $\mathcal{I}_b$ est donc en bijection avec $\mathcal{I}(E\setminus\{a,b\})$. Il existe $n-1$ parties de ce type.\newline
En revanche $\mathcal I_a$ est en bijection avec $\mathcal{I}(E\setminus\{a\})$. On en déduit finalement:
\begin{displaymath}
 x_n = x_{n-1} + (n-1)x_{n-2}
\end{displaymath}
 