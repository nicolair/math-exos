\begin{tiny}(Cfu17)\end{tiny} Notons $A(x)$, $B(x)$, $C(x)$ les trois expressions à exprimer et utilisons
\begin{displaymath}
 \ch x + 1 = \frac{1}{2}\left(e^x + e^{-x} +2\right)
= \frac{1}{2}\left(e^{\frac{x}{2}}+e^{\frac{x}{2}} \right)^2  
\end{displaymath}
et la formule analogue pour $\ch x -1$ :
\begin{displaymath}
  \ch x + 1 = 2 \left( \ch\frac{x}{2}\right)^2 \hspace{0.5cm}\ch x - 1 = 2 \left( \sh\frac{x}{2}\right)^2 
\end{displaymath}
On en déduit
\begin{displaymath}
  A(x) =\ch\frac{x}{2},\hspace{0.5cm}
  B(x) = \left| \sh(\frac{x}{2})\right|
\end{displaymath}
Comme $B(x)$ est définie pour $x$ en dehors de $[-1,1[$, on peut considérer
\begin{displaymath}
x =
\left\lbrace 
\begin{aligned}
\ch t   &\text{ si } x\geq 1 \\
 -\ch t &\text{ si } x< -1   
\end{aligned}
\right. 
\end{displaymath}

On en tire, pour $x\geq 0$,
\begin{displaymath}
  C(x) = 
\left\lbrace  
\begin{aligned}
  \th\frac{t}{2} &\text{ si } x\geq 1 \\
  \frac{\ch \frac{t}{2}}{\sh \frac{t}{2}} &\text{ si } x < -1 
\end{aligned}
\right. 
\end{displaymath}
