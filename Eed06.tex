\begin{tiny}(ed06)\end{tiny}
On considère les équations différentielles
\begin{align*}
(x+i)y'+y =& 0 & & (1)\\
(x+i)y'+y =& 1+2x\arctan x & & (2)
\end{align*}
dont l'inconnue $y$ est une fonction définie dans $\R$ et à valeurs complexes.
\begin{enumerate}
 \item Soit $x$ un nombre réel quelconque, préciser un argument puis l'expression exponentielle de $1+ix$.
\item Calculer une solution de $(1)$ par la formule du cours (exponentielle de l'opposée d'une primitive). Cette solution sera exprimée avec la primitive de
\begin{displaymath}
 x\rightarrow \dfrac{1}{x+z} \text{ avec }z\in \C-\R
\end{displaymath}
donnée dans le formulaire de cours. Vérifier que cette solution est une fraction rationelle simple (à préciser).
\item Vérifier que la fonction suivante est solution de $(1)$.
\begin{displaymath}
 x\rightarrow \frac{1}{x+i}
\end{displaymath}
\item Déterminer l'ensemble des solutions à valeurs complexes de $(2)$.\newline
Pour le calcul de primitive dans la méthode de variation de la constante, on pourra passer en notation intégrale et utiliser une intégration par parties.
\end{enumerate}


