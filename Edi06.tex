\begin{tiny}(Edi06)\end{tiny}
Soit $E$ un $\K$-espace vectoriel de dimension $n$.
\begin{enumerate}
\item Soit $(\alpha_1,\cdots,\alpha_n)$ une base de $E^*$ et $\Phi$ définie par :
\begin{displaymath}
 \Phi :\left\lbrace 
\begin{aligned}
 E \rightarrow& \K^n \\
 x \rightarrow& (\alpha_1(x),\cdots,\alpha_n(x))
\end{aligned}
\right. 
\end{displaymath}
Montrer que $\Phi$ est un isomorphisme.

\item Base antéduale.\newline
Soit $(\alpha_1,\cdots,\alpha_n)$ une base de $E^*$. Montrer qu'il existe une base $(a_1,\cdots,a_n)$ de $E$ dont la famille des formes coordonnées est $(\alpha_1,\cdots,\alpha_n)$. On dit que $(a_1,\cdots,a_n)$ est \emph{antéduale} de $(\alpha_1,\cdots,\alpha_n)$.

\item Multiplicateurs de Lagrange.\newline
Soit $(\alpha_1,\cdots,\alpha_p)$ une famille de vecteurs de $E^*$ et $\alpha\in E^*$. Montrer que
\begin{multline*}
  \ker(\alpha_1) \cap \cdots \cap \ker(\alpha_p) \subset \ker(\alpha) \\
  \Leftrightarrow
  \alpha \in \Vect(\alpha_1,\cdots, \alpha_p)
\end{multline*}

\item Soit $(\alpha_1,\cdots,\alpha_p)$ une famille de vecteurs de $E^*$. Montrer que  
\begin{multline*}
(\alpha_1,\cdots,\alpha_p) \text{ libre } \\
\Leftrightarrow
 \dim(\ker \alpha_1\cap\cdots\cap\ker\alpha_p) = \dim E -p
\end{multline*}
On considèrera une base $(\alpha_1,\cdots,\alpha_n)$ de $E^*$ et des applications
\begin{align*}
 \Phi_n :\left\lbrace 
\begin{aligned}
 E \rightarrow& \K^n \\
 x \rightarrow& (\alpha_1(x),\cdots,\alpha_n(x))
\end{aligned}
\right. \\
 \Phi_p :\left\lbrace 
\begin{aligned}
 E \rightarrow& \K^p \\
 x \rightarrow& (\alpha_1(x),\cdots,\alpha_p(x))
\end{aligned}
\right. 
\end{align*}

\end{enumerate}
