\begin{tiny}(Cal24)\end{tiny} Montrer que les opérations sont égales, c'est montrer que
\begin{displaymath}
  \forall (a,b)\in E^2, \; a+b = a*b
\end{displaymath}
Considérons la deuxième propriété dans le cas particulier
\begin{displaymath}
  (a + e) * (b +e) = (a*b) + (e*e)
\end{displaymath}
puis utilisons la première ($e$ est neutre pour les deux opérations)
\begin{displaymath}
  a + b = a*b + e = a*b 
\end{displaymath}
On considère ensuite
\begin{displaymath}
  (e + b) * (a + e) = (e*a)+(b*e)\Rightarrow b * a = a + b = a*b
\end{displaymath}
et enfin, pour tout $c\in E$,
\begin{displaymath}
  (a+b)*(e+c) = a + (b*c)\Rightarrow (a+b)*c = a+(b*c)
\end{displaymath}
comme $*=+$, on obtient bien l'associativité.
