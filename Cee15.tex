\begin{tiny}(Cee15)\end{tiny}
\begin{enumerate}
 \item On utilise la formule de la question b. On obtient
\renewcommand{\arraystretch}{1.2}
\begin{displaymath}
 P=
\begin{pmatrix}
 1 & 1 & -1 \\ 2 & 0 & 1 \\ 1 & 1 & 0
\end{pmatrix}, \;
P^{-1}=
\begin{pmatrix}
 \frac{1}{2} & \frac{1}{2} & -\frac{1}{2} \\
 -\frac{1}{2} & -\frac{1}{2} & \frac{3}{2} \\
 -1 & 0 & 1
\end{pmatrix}
\end{displaymath}
La matrice cherchée est 
\begin{displaymath}
 \begin{pmatrix}
 \frac{3}{2} & \frac{1}{2} & -2 \\
\frac{1}{2} & \frac{1}{2} & -1 \\
-2 & -1 & \frac{7}{2}
 \end{pmatrix}
\end{displaymath}
Remarque. Le code Maple suivant permet d'effectuer le calcul
\begin{verbatim}
with(LinearAlgebra):
P := Matrix([[1,1,-1],[2,0,1],[1,1,0]]);
Q :=P^(-1);
MatrixMatrixMultiply(Transpose(Q),Q);
\end{verbatim}

 \item D'après les formules (de cours) de changement de base pour la matrice d'un produit scalaire:
\begin{displaymath}
 \Mat_{\mathcal{B}}(/) = \trans P_{\mathcal{B}\mathcal{B}}\, \underset{=I_n}{\Mat_{\mathcal{B}}(/)} \,P_{\mathcal{B}\mathcal{B}}
= \trans P ^{-1} P ^{-1}
\end{displaymath}

 \item On peut exprimer assez facilement $a_1$, $a_2$, $a_3$ en fonction des $b_i$ puis vérifier la formule
\begin{displaymath}
 b_{k-1} - 2b_k +b_{k+1} = a_{k+1}
\end{displaymath}
pour $k$ entre $2$ et $n-1$. On en déduit la matrice de passage 
\begin{displaymath}
 P^{-1}=
\begin{pmatrix}
 1      & -2 & 1        & 0     & \cdots & 0   \\
 0      & 1  & -2       & 1     &        & \vdots    \\
        & 0  & 1        &-2     & \ddots & 0   \\
 \vdots &    & \ddots   &\ddots & \ddots & 1   \\
        &    &          &\ddots &\ddots  & -2  \\
 0      &    & \cdots   &       &   0    & 1   \\
\end{pmatrix}
\end{displaymath}
puis la matrice cherchée qui est une matrice bande de largeur 4
\begin{displaymath}
 \begin{pmatrix}
1      & -2     & 1      & 0      & 0      & \cdots & 0      \\
-2     & 5      & -4     & 1      & 0      &        & \vdots \\
 1     & -4     &  6     & -4     & 1      & \ddots &  0     \\
 0     &    1   & -4     &  6     &        & \ddots &  0     \\
 0     & \ddots & \ddots & \ddots & \ddots &        &  1     \\
\vdots & \ddots & \ddots &   1    &   -4   &    6   & -4     \\
 0     & \cdots &    0   &  0     & 1      &   -4   & 6      \\
 \end{pmatrix}
\end{displaymath}

\end{enumerate}
