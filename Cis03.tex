\begin{tiny}(Cis03)\end{tiny}
On va démontrer la limite demandée en utilisant la définition d'une limite.\newline
Soit $\varepsilon > 0$.\newline
En intégrant l'inégalité définissant $M$, on obtient
\[
 u_n \leq \left( \int_{\left[ a,b\right] } M^n\right)^{\frac{1}{n}}
 = (b-a)^{\frac{1}{n}}M.
\]
Comme $((b-a)^{\frac{1}{n}})_{n \in \N^*}\rightarrow 1$ et $\frac{M + \varepsilon}{M} > 1$, il existe $N_1$ tel que
\[
 n \geq N_1 \Rightarrow (b-a)^{\frac{1}{n}} \leq \frac{M + \varepsilon}{M}.
\]
La fonction $f$ est continue sur le segement $\left[ a,b \right]$. Elle atteint sa borne supérieure $M$. Il existe donc $u$ et $v$ tels que
$a\leq u < v \leq b$ et
\[
 \forall x \in \left[ u,v \right], \; M - \frac{\varepsilon}{2} \leq f(x). 
\]
De plus $f$ est à valeur positive donc 
\begin{multline*}
 \int_{\left[ a,b\right] }f^n \geq \int_{\left[ u,v\right] }f^n
 \geq \int_{\left[ u,v\right] }(M-\frac{\varepsilon}{2})^n \\
 \Rightarrow u_n \geq (v-u)^{\frac{1}{n}}(M-\frac{\varepsilon}{2})
\end{multline*}
Comme $((v-u)^{\frac{1}{n}})_{n \in \N^*}\rightarrow 1$ et $\frac{M - \varepsilon}{M-\frac{\varepsilon}{2}} < 1$, il existe $N_2$ tel que
\[
 n \geq N_2 \Rightarrow (b-a)^{\frac{1}{n}} \geq \frac{M - \varepsilon}{M-\frac{\varepsilon}{2}}.
\]
On obtient alors,
\begin{multline*}
 n \geq \max(N_1,N_2) 
 \Rightarrow \\
 M - \varepsilon = \frac{M - \varepsilon}{M-\frac{\varepsilon}{2}} M-\frac{\varepsilon}{2}
 \leq u_n \leq 
 \frac{M + \varepsilon}{M} M = M + \varepsilon.
\end{multline*}
Pour l'autre limite, on pose $g = f^{-1}$, alors
\[
 \min f = \frac{1}{\max g}.
\]
De plus
\[
 \left( \int_{\left[ a,b\right] }g^n\right)^{\frac{1}{n}} 
 = \left( \int_{\left[ a,b\right] }f^{-n}\right)^{\frac{1}{n}}
 = \frac{1}{v_n}.
\]
En utilisant le premier résultat:
\[
 (\frac{1}{v_n})\rightarrow \max g \Rightarrow (v_n)\rightarrow \frac{1}{\max g} = \min f.
\]


