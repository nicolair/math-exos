\begin{tiny}(Eaz02)\end{tiny} Dans tout l'exercice $a,$ $b,$ $c$, $d$, d{\'e}signent des naturels non nuls. On pourra utiliser la décomposition en facteurs premiers et des relations entre des $\max$, des $\min$, des sommes et des produits de valuations $p$-adiques.

  \begin{enumerate}
    \item  Montrer que $a\wedge (b\vee a)=a$ et $a\vee (b\wedge a)=a.$

    \item  Montrer que $a\wedge (bc)=a\wedge c$ et $a\vee (bc)=b(a\vee c)$
lorsque $a$ et $b$ sont premiers entre eux.

    \item  On suppose ici que $a$ divise $b$. Montrer que
\begin{align*}
b\wedge c &= (a\wedge c)\left[ \frac{c}{a\wedge c}\wedge \frac{b}{a}\right]\\
(a\vee c)\frac{b}{a} &= (b\vee c)\left[ \frac{c}{a\wedge c}\wedge \frac{b}{a}\right]
\end{align*}

    \item  Montrer que $c\vee (a\wedge b)$ divise $c\vee a$ et $c\vee b$ et que les quotients sont premiers entre eux.\newline
    En déduire que $\vee $ est distributive sur $\wedge$:
\[
 (c\vee a)\wedge (c \vee b)= c\vee(a\wedge b).
\]

    \item  Montrer que $\wedge $ est distributive sur $\vee $:
\[
 (c\wedge a)\vee (c \wedge b) = c\wedge(a\vee b).
\]
  \end{enumerate}
