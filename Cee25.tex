\begin{tiny}(Cee25)\end{tiny}
\begin{enumerate}
 \item L'énoncé n'impose rien à $u$ et $v$ sauf d'être orthogonaux à $w$. Choisissons
\begin{displaymath}
\begin{array}{ccc}
 u =& e_1 - 2e_3 =& (1,0,-2) \\
 v =& e_2 - 3e_3 =& (0,1,3)
\end{array} 
\end{displaymath}
La famille $(u,v)$ est libre et ses deux vecteurs sont orthogonaux à $w$ donc la famille $(u,v,w)$ est libre, c'est une base de $E$.
\item Par définition de $s$, $s(u)=u$, $s(v)=v$, $s(w)=-w$. On en déduit
\begin{displaymath}
 S'=
\begin{pmatrix}
 1 & 0 & 0 \\ 0 & 1 & 0 \\ 0 & 0 & -1
\end{pmatrix}
\end{displaymath}
\item On en déduit par changement de base
\begin{displaymath}
 S = P\,S' \,P^{-1} \text{ avec }
P=
\begin{pmatrix}
 1 & 0 & 2 \\0 & 1 & 3 \\ -2 & -3 & 1
\end{pmatrix}
\end{displaymath}
après calculs
\begin{multline*}
 P^{-1}= \frac{1}{4}
\begin{pmatrix}
 8 & -6 & 2 \\ 6 & -5 & 3 \\ -2 & 3 & -1
\end{pmatrix}, \\
S = \frac{1}{7}
\begin{pmatrix}
3 & 6 & -2 \\ -6 & -2 & -3 \\ -2 & -3 & 6 
\end{pmatrix}
\end{multline*}

\end{enumerate}
