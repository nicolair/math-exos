\begin{tiny}(Cre05)\end{tiny} En appliquant la relation avec $0$ et $1$, on obtient $|f(0)-f(1)|\geq 1$ ce qui, combiné avec $f(0)$ et $f(1)$ dans $[0,1]$, entraine que $f(0)=0$ et $f(1)=1$ ou $f(0)=1$ et $f(1)=0$. On passe d'un cas à l'autre en considérant $1-f$ qui vérifie la même propriété que $f$.\newline
On suppose donc $f(0)=0$ et $f(1)=1$. On applique la propriété entre $0$ et $x\in]0,1[$:
\begin{displaymath}
\left. 
\begin{aligned}
 &|f(x)|\geq |x| \\
 & f(x)\in [0,1]
\end{aligned}
\right\rbrace 
\Rightarrow
f(x)\geq x
\end{displaymath}
puis entre $1$ et $x$
\begin{multline*}
\left. 
\begin{aligned}
 &|f(x) -1|\geq |x-1| \\
 & f(x)\in [0,1]
\end{aligned}
\right\rbrace 
\Rightarrow
1-f(x)\geq 1-x \\
\Rightarrow
f(x)\leq x 
\end{multline*}