\begin{tiny}(Csl03)\end{tiny}
Système (1).
\begin{multline*}
  (1)\Leftrightarrow
\left\lbrace 
\begin{aligned}
  -x+2y+z &= b \\
  -3x -2z &= a-2b \\
  -3y+z &= c-2b
\end{aligned}
\right.\\
\Leftrightarrow
\left\lbrace 
\begin{aligned}
  -x+2y+z &= b \\
  -3y -2z &= a-2b \\
  -2y-z &= c-2b
\end{aligned}
\right.\\
\Leftrightarrow
\left\lbrace 
\begin{aligned}
  -x+z+2y &= b  \\
  -z-2y &= c-2b\\
  y &=a+2b-2c
\end{aligned}
\right.
\end{multline*}
Le système $(1)$ admet donc toujours une unique solution.

Système (2).
\begin{displaymath}
  (2)\Leftrightarrow 
\left\lbrace 
\begin{alignedat}{4}
  z & + & ax    & + & y      &= \alpha \\
    &   &(1-a)x & + & (a-1)y &= \beta -\alpha
\end{alignedat}
\right.
\end{displaymath}
\begin{itemize}
  \item Si $a\neq 1$, les solutions sont
\begin{displaymath}
  \left( \lambda + \frac{\beta - \alpha}{a-1}, \lambda, -a(\lambda + \frac{\beta - \alpha}{a-1})-\lambda + \alpha \right) \lambda \in \R.
\end{displaymath}
  \item Si $a=1$ et $\alpha \neq \beta$ pas de solution.
  \item Si $a=1$ et $\alpha = \beta$, le système est équivalent à l'unique équation
\[
  x+y+z=\alpha.
\]
\end{itemize}

Système (3). On va montrer que si $a$, $b$, $c$ sont deux à deux distincts, le système admet un unique triplet solution.\newline
Opérations $L_3 \leftarrow L_3 - L_2$, $L_2 \leftarrow L_2 - L_1$ puis on divise $L_2$ par $b-a$ et $L_3$ par $c-a$.
\begin{multline*}
  \left\lbrace 
  \begin{alignedat}{4}
    x &+& ay &+& a^2z   &= a^3 \\
      & & y  &+& (a+b)z &= b^2 + ab + a^2 \\
      & & y  &+& (c+b)z &= b^2 + cb + c^2
  \end{alignedat}
\right. \\
\Leftrightarrow
  \left\lbrace 
  \begin{alignedat}{4}
    x &+& ay &+& a^2z   &= a^3 \\
      & & y  &+& (a+b)z &= b^2 + ab + a^2 \\
      & &    & & (c-a)z &= (c-a)(a+b+c)
  \end{alignedat}
\right. .
\end{multline*}
On en déduit $z=a+b+c$, $x = abc$, $y=-(ab +bc+ca)$.

Système (4).\newline
Ne pas utiliser l'algorithme de Gauss mais ajouter toutes les lignes:
\[
  (3+\alpha)(x+y+z+t) = 1+\beta+\beta^2+\beta^3.
\]
Pour $\alpha + 3 \neq 0$, notons
\[
  S = \frac{1+\beta+\beta^2+\beta^3}{\alpha + 3}
\]
Le système admet une unique solution car il est équivalent à
\[
  \left\lbrace 
  \begin{aligned}
    (\alpha - 1)x + S = 1 \\
    (\alpha - 1)y + S = \beta \\
    (\alpha - 1)z + S = \beta^2 \\
    (\alpha - 1)t + S = \beta^3 
  \end{aligned}
\right.
\]


