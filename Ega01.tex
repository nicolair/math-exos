\begin{tiny}(Ega01)\end{tiny} Rappel sur les conditions d'alignement ou de concours.
\begin{enumerate}
 \item Dans un plan muni d'un repère, on considère trois points $A_1$, $A_2$, $A_3$ respectivement de coordonnées $(x_1,y_1)$, $(x_2,y_2)$, $(x_3,y_3)$. Montrer que ces trois points sont alignés si et seulement si
\begin{displaymath}
 \begin{vmatrix}
  x_1 & y_1 & 1 \\
  x_2 & y_2 & 1 \\
  x_3 & y_3 & 1 \\
 \end{vmatrix}
=0
\end{displaymath}
\item Dans un plan, trois points non alignés $U$, $V$, $W$ sont fixés. On considère trois points $A_1$, $A_2$, $A_3$ respectivement barycentres de $U$, $V$, $W$ avec les coefficients $(x_1,y_1,z_1)$, $(x_2,y_2,z_2)$, $(x_3,y_3,z_3)$.
 Montrer que ces trois points sont alignés si et seulement si
\begin{displaymath}
 \begin{vmatrix}
  x_1 & y_1 & z_1 \\
  x_2 & y_2 & z_2 \\
  x_3 & y_3 & z_3 \\
 \end{vmatrix}
=0
\end{displaymath}
\item Les fonctions coordonnées $x$ et $y$ sont relatives à un repère fixé d'un plan. On considère trois droites d'équations
\begin{displaymath}
 \left\lbrace
\begin{aligned}
 ax+by+c &= 0\\a'x+b'y+c' &= 0\\a''x+b''y+c'' &= 0
\end{aligned}\right. 
\end{displaymath}
Montrer que ces droites sont parallèles ou concourantes si et seulement si
 \begin{displaymath}
  \begin{vmatrix}
   a&b&c\\a'&b'&c'\\a''&b''&c''
  \end{vmatrix}
=0
 \end{displaymath}

\end{enumerate}