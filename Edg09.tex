\begin{tiny}(Edg09)\end{tiny}
\textbf{Fonctions holomorphes}.\newline On s'interesse ici aux fonctions à valeurs complexes d'une variable complexe. On commence par redéfinir des notations usuelles afin de les faire entrer dans le cadre des \emph{systèmes de coordonnées} 
\begin{align*}
 x=\Re &,& y=\Im &,& z= x+iy &,& \overline{z}=x-iy
\end{align*}
On notera bien que le dernière relation n'est pas une égalité entre nombres complexes mais entre \emph{fonctions complexes}. De même $\overline{z}$ n'est pas le conjugué d'un nombre complexe mais la \emph{conjugaison} elle même. Les fonctions $x$ et $y$ forment un système de fonctions coordonnées. On aura donc en particulier pour tous les $w$ complexes :
\begin{align*}
 z(w)=w &,& x(w)=\Re w &,& y(w)= \Im w &,& \overline{z}(w)=\overline{w}
\end{align*}
Normalement les opérateurs $\frac{\partial}{\partial x}$ et $\frac{\partial}{\partial y}$ agissent sur les fonctions à valeurs réelles, on les étend aux fonctions $f$ à valeurs complexes par linéarité (complexe). On pose donc :
\begin{align*}
 \dfrac{\partial f}{\partial x} = \dfrac{\partial \Re f}{\partial x} + i\dfrac{\partial \Im f}{\partial x}\\
\dfrac{\partial f}{\partial y} = \dfrac{\partial \Re f}{\partial y} + i\dfrac{\partial \Im f}{\partial y}
\end{align*}
On définit des opérateurs linéaires $\frac{\partial}{\partial z}$ et $\frac{\partial}{\partial \overline{z}}$ sur les fonctions à valeurs complexes par les formules:
\begin{align*}
 \dfrac{\partial}{\partial z} =& \dfrac{1}{2}\left( \dfrac{\partial}{\partial x} - i \dfrac{\partial}{\partial y}\right) \\
\dfrac{\partial}{\partial \overline{z}} =& \dfrac{1}{2}\left( \dfrac{\partial}{\partial x} + i\dfrac{\partial}{\partial y}\right)  
\end{align*}
Soit $f$ une fonction de classe $\mathcal C^1$ dans $\C$. 
\begin{enumerate}
 \item Montrer que, pour tous complexes $m$ et $h$ :
\begin{displaymath}
 f(m+h)-f(m) = \dfrac{\partial f}{\partial z}(m)h + \dfrac{\partial f}{\partial \overline{z}}(m)\overline{h} + r(h)
\end{displaymath}
avec 
\begin{displaymath}
 \dfrac{r(h)}{|h|} \xrightarrow{0} 0
\end{displaymath}
\item Pour tout complexe $m$, on note $\tau_m$ la fonction définie dans $\C^*$ par :
\begin{displaymath}
 \tau_m(h)=\dfrac{f(m+h)-f(m)}{h}
\end{displaymath}
Montrer l'équivalence entre les deux propriétés suivantes :
\begin{align*}
 &(1) & & \tau_m \text{ admet une limite en $0$ }\\
 &(2) & & \dfrac{\partial f}{\partial \overline{z}}(m) = 0
\end{align*}
On dit alors que $f$ est holomorphe en $m$.
\item Calculer
\begin{displaymath}
 \dfrac{\partial }{\partial \overline{z}}\circ \dfrac{\partial }{\partial z}
\text{ et }
\dfrac{\partial }{\partial z} \circ \dfrac{\partial }{\partial \overline{z}}
\end{displaymath}
Montrer que 
\begin{displaymath}
 \dfrac{\partial \overline{f}}{\partial z}
=
\overline{\dfrac{\partial f}{\partial \overline{z}}}
\end{displaymath}
En déduire que, lorsque $f$ est holomorphe, ses parties réelles et imaginaires sont harmoniques.
\end{enumerate}


