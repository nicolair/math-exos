\begin{tiny}(Ecp37)\end{tiny} On considère quatre nombres réels
\begin{displaymath}
  \alpha < \theta' < \beta < \theta \hspace{0.5cm}\text{ avec } \hspace{0.5cm}\theta - \alpha < 2\pi 
\end{displaymath}
et les nombres complexes
\begin{displaymath}
  a = e^{i\alpha},\hspace{0.5cm} b = e^{i\beta},\hspace{0.5cm} m = e^{i\theta},\hspace{0.5cm} m' = e^{i\theta'}
\end{displaymath}
Comparer les arguments de $\frac{b-m}{a-m}$ et de $\frac{b-m'}{a-m'}$.\newline
Interpréter géométriquement et généraliser à un cercle quelconque.
