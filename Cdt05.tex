\begin{tiny}(Edt05)\end{tiny}
\begin{enumerate}
 \item 
 \item
 \item
 \item Notons $\delta_n$ le déterminant cherché. En développant suivant la première colonne, on trouve, pour $n\geq 5$, $\delta_n=2\delta_{n-1}-3\delta_{n-1}$. On convient des petites valeurs pour que la formule soit toujours valable : $\delta_2=1$, $\delta_1=1$, $\delta_0=0$. Les racines du polynôme caractéristique de cette relation de récurrence sont $1+i\sqrt{2}$ et $1-i\sqrt{2}$. Après calculs, on trouve
\begin{displaymath}
 \delta_n = \frac{1}{\sqrt{2}}\Im(1+i\sqrt{2})^n
\end{displaymath}
\item En partant du bas et en remplaçant la ligne $L_k$ par $L_k - L_{k-1}$, on obtient le déterminant d'une matrice triangulaire supérieure avec des $1$ sur la diagonale. Le déterminant cherché est donc $1$.
\item Même méthode que pour le déterminant précédent avec des $\frac{1}{k}$ sur la diagonale à la fin. Le déterminant cherché est donc $\frac{1}{n!}$.
\item Toujours la même technique. Cette fois on obtient des $a_1 - a_2$ sur la diagonale sauf pour la première ligne. Le déterminant cherché est
\begin{displaymath}
 a_1(a_1 - a_2)^{n-1}
\end{displaymath}
 \item On développe en utilisant le caractère multilinéaire et antisymétrique des colonnes. On obtient
\begin{displaymath}
 b_1b_2\cdots b_n + a_1b_2\cdots b_n+ b_1a_2\cdots b_n + \cdots + b_1b_2\cdots a_n
\end{displaymath}

\end{enumerate}
