\begin{tiny}(Cre16)\end{tiny}
\begin{enumerate}
 \item Par définition de la partie entière supérieure puis de la partie entière usuelle 
\begin{multline*}
-x \leq \lceil -x \rceil \Rightarrow - \lceil -x \rceil \leq x
\Rightarrow - \lceil -x \rceil \leq \lfloor x \rfloor\\
\Rightarrow - \lfloor x \rfloor \leq \lceil -x \rceil
\end{multline*}

De manière analogue,
\begin{displaymath}
\lfloor x \rfloor \leq x \Rightarrow  -x\leq - \lfloor x \rfloor 
\Rightarrow  \lceil -x \rceil \leq  -\lfloor x \rfloor
\end{displaymath}
\item On décompose $x$ en partie entière et fractionnaire
\begin{multline*}
 x = \lfloor x \rfloor + \{x\}\Rightarrow nx = \underset{\in \Z}{n\lfloor x \rfloor} + \underset{\in [0,n[}{n\{x\}}\\
\Rightarrow \lfloor nx \rfloor = n\lfloor x \rfloor + \underset{\in \llbracket 0,n-1\rrbracket }{\lfloor n\{x\}\rfloor}\\
\Rightarrow \frac{1}{n} \lfloor nx \rfloor = \lfloor x \rfloor + \underset{\in [0,1[}{\frac{1}{n}\lfloor n\{x\}\rfloor}\\
\Rightarrow \lfloor \frac{1}{n} \lfloor nx \rfloor \rfloor = \lfloor x \rfloor
\end{multline*}

\item Notons que $0< x+\frac{k}{n}<2$, sa partie entière ne peut prendre que les valeurs $0$ ou $1$. La somme proposée est donc égale au nombre de $k$ pour lesquels la partie entière vaut $1$. Or
\begin{multline*}
 \lfloor x+\frac{k}{n} \rfloor =1
\Leftrightarrow 1\leq x+\frac{k}{n} \\
\Leftrightarrow n(1-x)\leq k \Leftrightarrow \lceil n(1-x) \rceil \leq k 
\end{multline*}

On en déduit que la somme cherchée vaut
\begin{multline*}
 \sharp \llbracket \lceil n(1-x) \rceil , n-1 \rrbracket
= n-1 - \lceil n(1-x) \rceil +1 \\
= n - \lceil n -nx \rceil
= - \lceil -nx \rceil = \lfloor nx \rfloor
\end{multline*}

en utilisant le a. et le fait que $\lceil y+m \rceil = m+\lceil y \rceil$ pour $y$ réel et $m$ entier. En utilisant une formle analogue pour la partie entière, on étend facilement la relation dans $\R$ à l'aide de la première formule de la question b.
\end{enumerate}
