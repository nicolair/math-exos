\begin{tiny}(Ega08)\end{tiny} Dans un $\R$-espace vectoriel $E$ de dimension $3$, on se donne un repère affine $(A,(i,j,k))$. Les fonctions coordonnées dans ce repère sont notées $x$, $y$, $z$.\newline
Soit $B$ le point de coordonnées $(1,1,1)$ dans ce repère. Soit $u$, $v$, $w$ trois vecteurs dont les coordonnées dans la base $(i,j,k)$ sont
\begin{displaymath}
u:(1,1,1),\hspace{0.3cm} v:(0,1,1),\hspace{0.3cm} w:(1,0,1)  
\end{displaymath}
Vérifier que $(B,(u,v,w))$ est un repère affine. On note $X$, $Y$, $Z$ les fonctions coordonnées dans ce repère. Exprimer $x$, $y$, $z$ en fonction de $X$, $Y$, $Z$ puis $X$, $Y$, $Z$ en fonction de $x$, $y$, $z$.