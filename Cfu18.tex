\begin{tiny}(Cfu18)\end{tiny} Après réduction au même dénominateur de la première équation, le système devient
\begin{multline*}
\left\lbrace 
\begin{aligned}
\sin(x+y) = \frac{1}{2} \\ \cos(x) \, \cos(y) = \frac{1}{2}  
\end{aligned}
\right. \\
\Leftrightarrow
\left\lbrace 
\begin{aligned}
\sin(x+y) = \frac{1}{2} \\ \cos(x+y) + \cos(x-y) = 1  
\end{aligned}
\right. \\
\Leftrightarrow
\end{multline*}
D'après la première équation, on doit avoir 
\begin{displaymath}
  \cos(x+y) = \pm \frac{\sqrt{3}}{2}
\end{displaymath}
Mais dans la deuxième équation, seul
\begin{displaymath}
  \cos(x+y) = \frac{\sqrt{3}}{2}
\end{displaymath}
est possible, sinon le $\cos$ serait plus grand que $1$. On en tire qu'il existe $u\in \Z$ tel que
\begin{displaymath}
  x+y = \frac{\pi}{6} + 2u\pi
\end{displaymath}
et la deuxième équation s'écrit:
\begin{displaymath}
\cos(2x-\frac{\pi}{6})=1-\frac{\sqrt{3}}{2}  
\end{displaymath}
Finalement, les solutions sont données par:
\begin{displaymath}
\left\lbrace 
\begin{aligned}
&x \equiv \frac{\pi}{12} \pm \alpha \mod \pi \\
&y \equiv -\frac{\pi}{6} + x \mod 2\pi
\end{aligned}
\right. 
\end{displaymath}
avec
\begin{displaymath}
  \alpha = \arccos(1-\frac{\sqrt{3}}{2})
\end{displaymath}
