\begin{tiny}(Car24)\end{tiny} 
\begin{enumerate}
  \item  Les polynômes $(X-1)^4$ et $(X+1)^4$ sont premiers entre eux. Voir dans le cours sur l'arithmétique euclidienne, la partie sur l'équation de Bézout.
  \item Traduit l'unicité du couple solution après substitution de $-X$ à $X$.
  \item  On utilise développe avec la formule du binôme la relation
\begin{displaymath}
 2^7 = \left( (X+1) + (X-1)\right)^7 
\end{displaymath}
On en déduit
\begin{multline*}
 U = (X+1)^3 + 7(X+1)^2(X-1)  \\ + 21(X+1)(X-1)^2 +35(X-1)^3
\end{multline*}
\end{enumerate}

 