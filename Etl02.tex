\begin{tiny}(Etl02)\end{tiny}
Soit $I$ un intervalle ouvert, $a\in I$ et $f\in \mathcal C^\infty(I)$. On définit $\tau$ dans $I$ par :
\begin{displaymath}
 \tau(x)=
\left\lbrace 
\begin{aligned}
 &\frac{f(x)-f(a)}{x-a} &\text{ si }& x\neq a \\
 &f'(a)  &\text{ si }& x= a 
\end{aligned}
\right. 
\end{displaymath}
La fonction $\tau$ est évidemment de classe $\mathcal C^\infty$ dans $I\setminus\{a\}$. Pour un naturel $n$ quelconque, exprimer $\tau^{(n)}(x)$ à l'aide de la formule de Leibniz puis à l'aide d'une formule de Taylor avec reste de Lagrange. Quelle est la limite en $a$? Que peut-on en déduire pour $\tau$ ?