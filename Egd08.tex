\begin{tiny}(Egd08)\end{tiny} Soit $I = [-1,1]$, $\alpha >0$. Un fonction $f$ dérivable dans $I$ est dite $\alpha - \mathcal{G}$ si et seulement si \footnote{$\mathcal{G}$ pour Gronwall: une notion utile pour certaines majorations liées aux équations différentielles}:
\begin{displaymath}
  \forall x\in I,\; \left| f'(x) \right| \leq \alpha \left| f(x) \right|
\end{displaymath}
\begin{enumerate}
  \item On suppose que $f$ est $\mathcal{C}^1$ et ne s'annule pas. Montrer qu'il existe $\alpha >0$ tel que  $f$ soit $\alpha-\mathcal{G}$.
  \item Soit $\lambda \in \R$, soit $f$ une fonction $\alpha - \mathcal{G}$, soit $g$ une fonction $\beta - \mathcal{G}$, soit $\psi\in \mathcal{C}^1(I)$ telle que $\psi(I)\subset I$. Préciser pour chaque fonction $\lambda f$ , $fg$, $f \circ \psi$ un $\gamma >0$ tel qu'elle soit $\gamma - \mathcal{G}$.
  \item Soit $f$ une fonction $\alpha - \mathcal{G}$ et à valeurs positives ou nulles. \'Etudier les variations de $\varphi = \left( x\mapsto f(x)e^{-\alpha x}\right)$.\newline 
Que peut-on en déduire si $f(0)=0$? En utilisant $\psi = \left(  x\mapsto -x\right)$ montrer que $f$ est la fonction nulle.
  \item Soit $f$ une fonction $\alpha - \mathcal{G}$ telle que $f(0)=0$. Montrer que $f$ est la fonction nulle.
  \item Montrer que toute fonction $\alpha - \mathcal{G}$ qui s'annule est identiquement nulle. On pourra considérer, pour $\lambda > 0$, des fonctions 
\begin{displaymath}
  \psi_\lambda:\; x\mapsto -1 + 2 \left( \frac{1 + x}{2}\right)^\lambda
\end{displaymath}
\end{enumerate}
