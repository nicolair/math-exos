\begin{tiny}(Esr02)\end{tiny}
\textbf{R{\'e}currence homographique}\newline
Les suites consid{\'e}r{\'e}es ici sont à valeurs complexes.\newline
Soit $a,b,c,d$ des nombres complexes tels que $c\neq 0$ et $\delta = ad-bc\neq 0$. On consid{\`e}re une fonction homographique $f$ définie dans $\C\setminus\{-\frac{d}{c}\}$ par :
\begin{displaymath}
 f(z) = \frac{az+b}{cz+d}
\end{displaymath}
ainsi que l'{\'e}quation du second degr{\'e} en $x$ :
\begin{equation}
x=\frac{ax+b}{cx+d}  \tag{E}
\end{equation}
caractérisant les points fixes de $f$.
Dans toute la suite, on supposera que une suite $(u_n)_{n\in \N}$ v{\'e}rifie une relation de r{\'e}currence homographique; c'est {\`a} dire que :
\[
\forall n\in \N, u_{n+1}=\frac{au_n+b}{cu_n+d}=f(u_n)
\]
\begin{enumerate}
\item Montrer que 
\begin{displaymath}
 f(x)=\frac{a}{c}-\frac{\delta}{c(cx+d)}
\end{displaymath}
En déduire des expressions factorisées pour $f(x)-f(y)$ et $f'(x)$ (avec $x$ réel dans le dernier cas).

\item  Dans cette question, on suppose que (E) admet deux racines distinctes $\alpha $ et $\beta $. On définit $g_n$ par :
\begin{displaymath}
 g_n = \frac{u_{n}-\alpha }{u_{n}-\beta}
\end{displaymath}

\begin{enumerate}
 \item Montrer que, si $u_{0}\neq \beta $, alors $u_{n}$ $\neq \beta $ pour tout $n$. Montrer que $(g)_{n\in \N}$ est une suite g{\'e}om{\'e}trique. Préciser la raison.
 \item Exprimer $u_n$ en fonction de $g_n$.\\ Quels sont les comportements possibles pour la suite $(u_n)_{n\in \N}$ ?
 \item Ici, $a$, $b$, $c$, $d$ sont réels et les deux points fixes sont réels. Calculer $f'(\alpha)f'(\beta)$.
\end{enumerate}
\item  Dans cette question, on suppose que (E) admet une racine double $\alpha $. On définit $a_n$ par :
\begin{displaymath}
 a_n = \frac{1}{u_n-\alpha}
\end{displaymath}
\begin{enumerate}
 \item Montrer que, si $u_{0}\neq \alpha $, alors $u_{n}$ $\neq \alpha $ pour tout $n$ et $(a_n)_{n\in \N}$ est une suite arithm{\'e}tique.
 \item Exprimer $u_n$ en fonction de $a_n$, en déduire sa limite.
\end{enumerate}
  \item  Exemples.
\[
u_{n+1}=\frac{4u_{n}+2}{u_{n}+5}, u_{n+1}=\frac{7u_{n}-12}{3u_{n}-5}, u_{n+1}=\frac{1}{1-u_{n}}
\]
\end{enumerate}
Dans le dernier exemple, comparer $u_{n+3}$ avec $u_n$.