\begin{tiny}(Cpo32)\end{tiny} Le polynôme $X^2+1$ divise le polynôme donné si et seulement si $i$ et $-i$ sont racines. Cela donne
\begin{displaymath}
 \left\lbrace 
\begin{aligned}
 \lambda - i \mu &=3-i \\  \lambda + i \mu &=3+i  
\end{aligned}\right. 
\Leftrightarrow 
\left\lbrace 
\begin{aligned}
 \lambda&=3 \\ \mu &=1
\end{aligned}\right. 
\end{displaymath}
Le polynôme se factorise en
\begin{displaymath}
 X^4+X^3+3X^2+X+2=(X^2+1)(X^2+X+1)
\end{displaymath}
