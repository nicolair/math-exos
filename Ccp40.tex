\begin{tiny}(Ccp40)\end{tiny}
\begin{enumerate}
 \item On résoud le système de 2 équations aux inconnues $u$ et $v$
\begin{displaymath}
\left\lbrace 
\begin{aligned}
 s(b) &= b \\ s(c) &= c
\end{aligned}
\right. 
\Leftrightarrow
\left\lbrace 
\begin{aligned}
 \overline{b}u + v &= b \\ \overline{c}u + v &= c
\end{aligned}
\right. 
\Leftrightarrow
\left\lbrace 
\begin{aligned}
 (\overline{b}-\overline{c})u &= b-c \\ (b-c)v &=b^2- c^2
\end{aligned}
\right. 
\end{displaymath}
Après simplification à cause du module $1$:
\begin{displaymath}
 u = -bc,\hspace{0.5cm} v =b+c
\end{displaymath}
\begin{displaymath}
 \forall z\in \C,\;s(z) = -bc\overline{z} + b+c.
\end{displaymath}

 \item Expression de $s\circ s$. Pour tout $z\in \C$,
\begin{multline*}
s\circ s(z) = s(s(z)) 
= -bc(\overline{-bc\overline{z}+b+c}) +b+c \\
= |bc|^2z -bc(\frac{1}{b}+\frac{1}{c}) + (b+c) = z
\end{multline*}
L'affixe $m$ d'un point de la droite $(BC)$ est de la forme
\begin{displaymath}
 c + \lambda(b-c) \text{ avec } \lambda\in \R
\end{displaymath}
Calculons son image par $s$
\begin{multline*}
 s(m) = -bc (\overline{c + \lambda(b-c)}) + b+c \\
 = s(c) -bc\overline{\lambda(b-c)}
 = c -\lambda bc(\frac{1}{b}-\frac{1}{c})\\
 = c -\lambda(c-b) = m
\end{multline*}
\end{enumerate}
