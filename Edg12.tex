\begin{tiny}(Edg12)\end{tiny} Dans cet exercice, $x$ et $y$ désignent les fonctions coordonnées dans un repère fixé et $\Omega$ est le demi-plan $y>0$. On souhaite déterminer des fonctions $f$ définies dans $\Omega$ et vérifiant
\begin{displaymath}
 \delta(f)=0 \text{ avec } \delta = x^2\frac{\partial^2 }{\partial x^2}-y^2\frac{\partial^2 }{\partial y^2}
\end{displaymath}
\begin{enumerate}
 \item Calculer
\begin{displaymath}
 \left(x\frac{\partial}{\partial x}-y\frac{\partial}{\partial y} \right)
\circ
\left(x\frac{\partial}{\partial x}+y\frac{\partial}{\partial y} \right) 
\end{displaymath}
\item On introduit le système de coordonnées $(u,v)$ défini par
\begin{displaymath}
 u=xy,\hspace{0.5cm} v=\frac{x}{y}
\end{displaymath}
Exprimer $\delta$ en fonction de $\frac{\partial}{\partial u}$ et $\frac{\partial}{\partial v}$. 
\item En déduire des solutions exprimées à l'aide de $u$ et $v$.
\end{enumerate}
