\begin{tiny}(Cmo21)\end{tiny}
\begin{enumerate}
 \item Dans ce cas, $C$ est de rang $3$ donc inversible. Après calcul de l'inverse, 
\begin{displaymath}
 B=AC^{-1}= A
\begin{pmatrix}
 2 & -2 & -3 \\1 & -1 & -1 \\ 3 & -4 & -5
\end{pmatrix}
=
\begin{pmatrix}
 1 & 1 & 1 \\1 & 1 & -1 \\ 1 & -1 & -1
\end{pmatrix}
\end{displaymath}
\item Cette fois $\rg C = 2$ donc $C$ n'est pas inversible.\newline
Analyse.\newline
Soit $a$, $b$, $c$ les endomorphismes de $\R^3$ dont les matrices dans la base canonique sont $A$, $B$, $C$.
\begin{displaymath}
 BC=A \Leftrightarrow b\circ c = a \Rightarrow \ker c \subset \ker a.
\end{displaymath}
Donc $k$ doit vérifier $\ker c \subset \ker a$. De plus
\begin{multline*}
 (x,y,z)\in \ker c \Leftrightarrow
\left\lbrace  
\begin{aligned}
 x+2y-z&=0\\2x-y-z &=0\\-5x + 3z &=0
\end{aligned}
\right. \\
\Leftrightarrow (x,y,z)=\frac{z}{5}(3,1,5).
\end{multline*}

Alors $\ker c \subset \ker a$ entraine
\begin{displaymath}
\begin{pmatrix}
-2 & 1 & 1 \\ 8 & 1 & -5 \\ 4 & 3 & k 
\end{pmatrix}
\begin{pmatrix}
 3\\1\\5
\end{pmatrix}
=
\begin{pmatrix}
 0\\0\\0
\end{pmatrix}
\Rightarrow k=-3 .
\end{displaymath}
Synthèse.\newline
Introduisons $\mathcal{C}=(e_1,e_2,e_3)$ la base canonique et $u=(3,1,5)$ un générateur du noyau.\newline
Alors $\mathcal{U}=(c(e_1),c(e_2),u)$ est une base. Définissons $b$ par prolongement linéaire 
\begin{displaymath}
 b(c(e_1))=a(e_1),\; b(c(e_2))=a(e_2),\; b(u)=0.
\end{displaymath}
Elle vérifie bien $b\circ c = a$. Ceci se traduit par
\begin{displaymath}
 \mathop{\mathrm{Mat}}_{\mathcal U \mathcal C}b =
\begin{pmatrix}
 -2 & 1 & 0 \\ 8 & 1 & 0\\ 4 & 3 &0
\end{pmatrix}.
\end{displaymath}
On en tire
\begin{displaymath}
 B = \mathop{\mathrm{Mat}}_{\mathcal C}b
= \mathop{\mathrm{Mat}}_{\mathcal U \mathcal C}b\, \mathop{\mathrm{Mat}}_{\mathcal C \mathcal U }\Id_E
=  \mathop{\mathrm{Mat}}_{\mathcal U \mathcal C}b \, P_{\mathcal{C} \mathcal U }^{-1}.
\end{displaymath}
Après calculs;
\[
 P_{\mathcal{C} \mathcal U }
=\frac{1}{10}
\begin{pmatrix}
 1 & 2 & 3 \\ 2 & -1 & 1 \\ -5 & 0 & 5
\end{pmatrix}
\]
\begin{displaymath}
 P_{\mathcal{C} \mathcal U }^{-1}
=\frac{1}{10}
\begin{pmatrix}
 1 & 2 & -1 \\ 3 & -4 & -1 \\ 1 & 2 & 1
\end{pmatrix}
\end{displaymath}
\begin{displaymath}
B =\frac{1}{10}
\begin{pmatrix}
 1 & -8 & 1 \\ 11 & 12 & -9 \\ 13 & -4 & -7
\end{pmatrix}
\end{displaymath}
Il y a plusieurs $B$ solutions. On peut multiplier le générateur $u$ du noyau par un scalaire non nul. Cela ne change pas $\MatBB{U}{C}b$ mais change les matrices de passage.\newline
Code pour la vérification en Python
\begin{verbatim}
import numpy as np
import numpy.linalg as la
C = np.array([[1,2,-1],[2,-1,-1]
                ,[-5,0,3]])
P = np.array([[1,2,3],[2,-1,1]
                ,[-5,0,5]])
Q = la.inv(P)
M = np.array([[-2,1,0],[8,1,0],[4,3,0]])
B = np.matmul(M,Q)

#Vérification
A = np.matmul(B,C)
\end{verbatim} 
\end{enumerate}
 
