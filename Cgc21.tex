\begin{tiny}(Cgc21)\end{tiny}
\begin{enumerate}
 \item De la relation additive on tire $f(0)=0$ et $f(-x) = -f(x)$ pour tous les $x$. De la relation multiplicative, on tire $f(1)=1$ car la fonction étantbijective, il existe un $x$ tel que $f(x)\neq 0$. De plus
\[
\forall p \in\N, \forall x\in \R,\; f(px) = p f(x)
\]
En prenant $x=1$, on en déduit $f(p)=p$. Pour $p\neq 0$, en prenant $x=\frac{1}{p}$, on tire $f(\frac{1}{p}) = \frac{1}{f(p)}$.\newline
Avec la propriété multiplicative, on déduit $f(x)=x$ pour tout $x$ rationnel.
 \item Soit $x < y$. Alors
\begin{multline*}
 f(y) = f(x) + f(y-x)
 = f(x) + f(\sqrt{y-x}^2)\\
 = f(x) + f(\sqrt{y-x})^2 > 0
\end{multline*}
L'inégalité est stricte car la fonction est bijective avec $f(0)=0$.
 \item La fonction est croissante et surjective donc $f(\R)=\R$ est un intervalle. Une fonction monotone sur intervalle et dont l'image est un intervalle est continue donc la fonction $f$ est continue. En approchant par densité un $x$ irrationnel par une suite de nombres rationnels, on montre en utilisant la continuité en $x$ que $f(x)=x$.
\end{enumerate}
