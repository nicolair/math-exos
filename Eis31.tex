\begin{tiny}(Eis31)\end{tiny} Soit $f$ une fonction $k$-lipschitzienne sur un intervalle $I$. (les questions b. et c. sont indépendantes)
\begin{enumerate}
 \item Montrer que $x\rightarrow f(x)-kx$ et $x\rightarrow f(x)+kx$ sont monotones.
 \item On suppose  $I=]a,b[$. Montrer que $f$ admet un prolongement $k$-lipschitzien à $[a,b]$.
 \item On suppose  $I=]a,b[$. Pour tout réel $y$, on définit $\varphi_y$ dans $I$ par :
\begin{displaymath}
 \forall x\in I, \hspace{0.5cm}
\varphi_y(x) = f(x) + k|x-y|
\end{displaymath}
Suivant la position de $y$ par rapport à $a$ et $b$, former les différents tableaux de variations de $\varphi_y$. En déduire que $\varphi_y$ est minorée et atteint sa borne inférieure. On note $g(y)=\min \varphi_y$.\newline
Montrer que $y\in I$ entraine $g(y)=f(y)$. Montrer que $g$ est $k$-lipschitzienne sur $\R$.
\end{enumerate}
