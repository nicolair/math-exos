\begin{tiny}(Cef03)\end{tiny} Pour montrer que c'est une base, on peut raisonner de plusieurs manières en combinant deux des propriétés suivantes de la famille $(L_0, \cdots, L_n)$.\newline
Elle contient $n+1 = \dim \C_n[X]$ vecteurs.\newline
Elle est libre. (prendre la valeur en $a_i$ de la combinaison nulle)\newline
Elle est génératrice avec $P = \sum_{i=0}^n \widetilde{P}(a_i)L_i$. (la différence admet $n+1$ racines alors que son degré est au plus $n$)
