\begin{tiny}(Cgs01)\end{tiny} Déterminons les orbites
\[
  \begin{matrix}
    1 & 3  & 6 &   &   \\
    2 & 10 & 9 & 8 & 5 \\
    4 &    &   &   &   \\
    7 &    &   &   &   
  \end{matrix}
\]
On en déduit la décomposition en cycles disjoints puis en transpositions
\begin{multline*}
  \sigma =
  \begin{pmatrix}
    2 & 10 & 9 & 8 & 5
  \end{pmatrix}
  \circ 
  \begin{pmatrix}
    1 & 3  & 6
  \end{pmatrix}\\
  =
  \begin{pmatrix}
    2 & 10
  \end{pmatrix}
  \circ
  \begin{pmatrix}
    10 & 9
  \end{pmatrix}
  \circ
  \begin{pmatrix}
    9 & 8
  \end{pmatrix}
  \circ
  \begin{pmatrix}
    8 & 5
  \end{pmatrix}
  \circ
  \begin{pmatrix}
    1 & 3
  \end{pmatrix}
  \circ
  \begin{pmatrix}
    3 & 6
  \end{pmatrix}
  .
\end{multline*}

Comme $\sigma$ se décompose en 6 transpositions, elle est paire: $\varepsilon(\sigma) = 1$.
