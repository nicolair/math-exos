\begin{tiny}(Ecr04)\end{tiny}
La \emph{podaire} d'un point fix{\'e} par rapport {\`a} une courbe param{\'e}tr{\'e}e est la courbe form{\'e}e par la projection orthogonale de ce point sur les tangentes {\`a} la courbe.
\begin{figure}[ht]
 \centering
 \input{Ecr04_1.pdf_t}
 \caption{Exercice \arabic{enumi} : podaire d'un cercle.}
 \label{fig:Ecr04_1}
\end{figure}

\begin{enumerate}
\item Montrer que la podaire de l'origine par rapport {\`a} une spirale logarithmique $r(\theta )=ae^{m\theta }$ est une spirale semblable {\`a} la premi{\`e}re (c'est {\`a} dire image par une similitude)
\item Les \emph{lima\c{c}ons de Pascal} sont les podaires d'un cercle.\\
On considère un cercle de rayon $1$ et de centre le point de coordonnées $(-a,0)$ (figure \ref{fig:Ecr04_1}). On note $h(\theta)$ le projeté de l'origine du repère sur la tangente en $f(\theta)=0+\overrightarrow e_\theta$. Discuter suivant $a>0$ des variations de $x\circ h$ et du signe de $y\circ h$. Dessiner les diverses formes (en particulier la formation de la boucle) de lima\c{c}on. Dans le cas o{\`u} le point est sur le cercle, la courbe podaire est la \emph{cardio{\"\i}de}.
\end{enumerate}