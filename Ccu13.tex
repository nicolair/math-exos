\begin{tiny}(Ecu13)\end{tiny} Les indices des sommes proposées recouvrent $\llbracket 0,n\rrbracket$, ils sont regroupés modulo $3$. Deux puissances de $j$ sont égales lorsque les exposants sont congrus modulo $3$. On peut combiner les sommes proposées pour faire apparaitre des formules du binôme.
\begin{displaymath}
\left\lbrace  
\begin{aligned}
  &A + B + C     &= (1+1)^n  &= 2^n\\
  &A + jB + j^2C &= (1+j)^n  &= (-j^2)^n\\
  &A + j^2B + jC &= (1+j^2)^n &= (-j)^n
\end{aligned}
\right. 
\end{displaymath}
En combinant linéairement, on déduit
\begin{align*}
&3A = 2^n +(-1)^n2\cos\frac{2n\pi}{3} \\
&3B = 2^n +(-1)^n2\cos\frac{2(n+1)\pi}{3} \\
&3C = 2^n +(-1)^n2\cos\frac{2(n-1)\pi}{3}
\end{align*}
