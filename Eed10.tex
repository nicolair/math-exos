\begin{tiny}(ed10)\end{tiny}
On considère les équations différentielles dont les inconnues $y$ sont des fonctions définies dans $\R$ et à valeurs réelles.
\begin{align*}
 (1+x^2)y''(x)+xy'(x)-4y(x) = -3x & &(1)\\
 (1+x^2)y''(x)+xy'(x)-4y(x) = 0 & &(2)\\
(1+x^2)y'(x)+xy(x) = 0 & &(3)
\end{align*}
\begin{enumerate}
\item Calculer, sous forme factorisée,  la dérivée de
\begin{displaymath}
 \frac{x\sqrt{1+x^2}}{2x^2+1}
\end{displaymath}

\item Déterminer une solution polynomiale $y_0$ de $(1)$.
\item Déterminer une solution polynomiale non nulle $y_1$ de $(2)$.
\item Pour toute fonction $z$, on définit $w$ par :
\begin{displaymath}
 w=\begin{vmatrix}
    z & y_1\\
z' & y_1'
   \end{vmatrix}
\end{displaymath}
Montrer que $z$ est solution de $(2)$ si et seulement si $w$ est solution de $(3)$.
\item Calculer les solutions de $(3)$. Résoudre l'équation
\begin{displaymath}
 \begin{vmatrix}
y & y_1\\
y' & y_1'
 \end{vmatrix} = w_0
\end{displaymath}
lorsque $w_0$ est une solution de $(3)$.
En déduire l'ensemble des solutions de $(1)$.
\end{enumerate}