\begin{tiny}(Een16)\end{tiny} Parties sans éléments consécutifs.
\begin{enumerate}
 \item Quel est le nombre de partie de $\llbracket 1,n \rrbracket$ à $p$ éléments sans éléments consécutifs? \newline
 On pourra considérer des fonctions \og très strictement croissantes\fg.
 \item Soit $t_n$ le nombre de parties de $\llbracket 1,n \rrbracket$ sans éléments consécutifs. Montrer que 
\begin{multline*}
 t_{n+2} = t_{n+1} +t_n, \; t_{2n+1} = t_{n}^2 + t_{n-1}^2, \\
 t_{2n} = t_{n}^2 - t_{n-2}^2.
\end{multline*}
 \item Calculer $t_{50}$.
\end{enumerate}
