\begin{tiny}(Egd13)\end{tiny} Soit $a$ et $b$ deux réels, $|a|\neq 1$. On définit $\varphi$:
\begin{displaymath}
  \forall x\in \R,\;\varphi(x) = ax +b
\end{displaymath}
On note $u$ l'unique point fixe de $\varphi$.
\begin{enumerate}
  \item Soit $f$ définie dans $\R$, continue en $u$ et telle que $f\circ \varphi =f$. Montrer que $f$ est constante.
  \item Soit $f$ définie dans $\R$ et telle que $f\circ \varphi = \varphi \circ f$. On cherche à montrer qu'il existe $\lambda\in \R$ tel que
\begin{displaymath}
  \forall x\in \R, \; f(x) = \lambda(x-u) +u
\end{displaymath}
Montrer le si $f\in \mathcal{C}^1(\R)$. Montrer le sous la seule hypothèse: $f$ dérivable en $u$. On pourra considérer
\begin{displaymath}
  \tau: x \mapsto \frac{f(x) -u}{x-u}
\end{displaymath}
\item \'Etudier l'équation fonctionnelle $f\circ f = \varphi$ où $f$ est définie dans $\R$ et dérivable en $u$.
\end{enumerate}
