\begin{tiny}(Ecu02)\end{tiny}\label{Ecu02}
\begin{enumerate}
  \item On considère un trinôme $ax^2+bx+c$ de discriminant $\Delta$. Montrer que ce trinôme admet deux racines réelles distinctes strictement inférieures à $1$ si et seulement si:
\begin{align*}
\left\lbrace  
\begin{aligned}
  &a>0 \\ &2a+b >0 \\ &\Delta >0 \\ &a+b+c >0
\end{aligned}
\right. 
& &\text{ ou }& &
\left\lbrace  
\begin{aligned}
  &a<0 \\ &2a+b <0 \\ &\Delta >0 \\ &a+b+c <0
\end{aligned}
\right. 
\end{align*}
\item Déterminer les réels $m$ tels que l'équation
\begin{displaymath}
(2m-1)x^{2}+2(m+1)x+m+3=0  
\end{displaymath}
ait deux racines r{\'e}elles inférieures ou égales à 1.
\end{enumerate}