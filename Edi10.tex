\begin{tiny}(Edi10)\end{tiny}
Soit $E=\R_{3}\left[ X\right]$ et $a$, $b$, $c$ trois r{\'e}els distincts. On consid{\`e}re les formes lin{\'e}aires
 $\phi _{1}$, $\phi _{2}$, $\phi _{3}$, $\psi $ d{\'e}finies sur $E$ par :
\begin{align*}
\phi_{1}(P) &= \widetilde{P}(a)&  & \phi _{2}(P)=\widetilde{P}(b)\\
\phi _{3}(P)&= \widetilde{P}(c)& & \psi(P)=\int_{a}^{b}\widetilde{P}(t)\,dt
\end{align*}
Montrer que $(\phi _{1}$, $\phi _{2}$, $\phi _{3})$ est libre.\newline
Montrer que $(\phi _{1}$, $\phi _{2}$, $\phi_{3},\psi )$ est li{\'e}e si et seulement si
\begin{displaymath}
 \psi ((X-a)(X-b)(X-c))=0
\end{displaymath}
En d{\'e}duire que $(\phi _{1}$, $\phi _{2}$, $\phi _{3},\psi )$ est li{\'e}e si et seulement si $c=\frac{a+b}{2}$. (on pourra poser $m=\frac{a+b}{2}$, $l=\frac{b-a}{2}$ et exprimer la polynôme à intégrer avec des puissances de $X-m$.
