\begin{tiny}(Ccu05)\end{tiny} Avec des identités remarquables:
\begin{multline*}
  a^3-b^3 = (a-b)(a^2+ab+b^2)\\ = (a-b)((a-b)^2+3ab) = d(d^2+3p)
\end{multline*}
Comme $a^3-b^3 = 6$ et $p = \frac{5}{3}$, le nombre $d$ est racine de l'équation
\begin{displaymath}
  z^3 + 5z = 6
\end{displaymath}
Comme $1$ est une racine évidente, on peut factoriser:
\begin{displaymath}
  z^3 + 5z -6 =(z-1)(z^2+z+6)
\end{displaymath}
L'équation $z^2+z+6$ est sans racine réelle, on en déduit que $d=1$.