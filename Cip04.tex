\begin{tiny}(Cip04)\end{tiny}
\begin{itemize}
 \item Primitive de $\frac{1}{4x^2+4x+5}$. 
\begin{multline*}
 \frac{1}{4x^2+4x+5}=\frac{1}{(2x+1)^2+4} \\
 = \frac{1}{4}\frac{1}{(x+\frac{1}{2})^2+1}.
\end{multline*}

Une primitive est donc
\begin{displaymath}
 F(t) = \frac{1}{4}\arctan(t+\frac{1}{2})
\end{displaymath}

 \item Primitive de $\frac{1}{x^2-4x+2} $
\begin{displaymath}
 \frac{1}{x^2-4x+2} = \frac{\frac{1}{2\sqrt{2}}}{x-2-\sqrt{2}}+ \frac{-\frac{1}{2\sqrt{2}}}{x-2+\sqrt{2}}
\end{displaymath}
Une primitive est donc
\begin{displaymath}
 F(t)= \frac{1}{2\sqrt{2}} \ln|x-2-\sqrt{2}| - \frac{1}{2\sqrt{2}} \ln|x-2+\sqrt{2}|
\end{displaymath}

 \item Primitive de $\frac{1}{5+3\cos x}$.
Pour $t$ dans $\R$, changement de variable $u=\tan \frac{x}{2}$ dans $F(t)$.
\begin{multline*}
 F(t) = \int^t\frac{1}{5+3\cos x}\,dx\\
 = \int^{\tan \frac{t}{2}}\frac{2\,du}{5(1+u^2)+3(1-u^2)}\\
= \frac{1}{4}\int^{\tan \frac{t}{2}}\frac{du}{1+(\frac{u}{2})^2}
= \frac{1}{2}\arctan\left(\frac{1}{2}\tan(\frac{t}{2}) \right) 
\end{multline*}

 \item Primitive de $\frac{1}{x}\sqrt{\frac{x-1}{x+1}}$. On se place dans $\left[ 1, +\infty\right[$.\newline
Première méthode.
\[
 F(x) = \int_1^{x}\sqrt{\frac{t-1}{t+1}}\frac{dt}{t}
 = \int_1^{x}\frac{\sqrt{t^2-1}}{t(t+1)}\,dt.
\]
Changement de variable $t = \ch u$.
\begin{multline*}
 F(x) = \int_{0}^{\argch(x)}\frac{\ch u -1}{\ch u}\, du \\
 = \argch(x) - \int_{0}^{\argch(x)}\frac{1}{\ch u}\, du \\
 = \argch(x) - \int_{0}^{\argch(x)}\frac{\ch u}{1 + \sh^2 u}\, du \\
 = \argch(x) - \arctan \circ \sh \circ \argch (x)\\
 = \argch(x) - \arctan(\sqrt{x^2 - 1}).
\end{multline*}

Deuxième méthode.\newline
Changement de variable $u = \sqrt{\frac{t-1}{t+1}}$.
\[
 F(x) = \int_{0}^{\sqrt{\frac{x-1}{x+1}}} \frac{4u^2}{(1+u^2)(1-u^2)}\,du = \cdots
\]
 
 
 \item Primitive de $\frac{x}{\sqrt{x^2+x+1}} $\\pas encore de correction
 \item Primitive de $\frac{1}{e^x+2e^{-x}} $\\pas encore de correction
 \item Primitive de $\frac{x^2}{\sqrt{1-x^2}} $\\pas encore de correction
 \item Primitive de $\frac{\sin x +2\cos x}{\sin x -\cos x} $\\pas encore de correction
\end{itemize}
