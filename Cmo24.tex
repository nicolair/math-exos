\begin{tiny}(Cmo24)\end{tiny} On transforme la matrice par l'algorithme du pivot partiel.
\begin{align*}
  &\begin{pmatrix}
    1  & -3 & -5 & 2  & y_1 \\
    2  & 5  & 12 & -7 & y_2 \\
    -1 & 1  & 1  & 0  & y_3 \\
    -2 & 4  & 6  & -2 & y_4
  \end{pmatrix}& \\
  &\begin{pmatrix}
    -1 & 1  & 1  & 0  & y_3 \\
    2  & 5  & 12 & -7 & y_2 \\
     1 & -3 & -5 & 2  & y_1 \\
    -2 & 4  & 6  & -2 & y_4
  \end{pmatrix}& L_1 \leftrightarrow L_3 \\
  &\begin{pmatrix}
    -1 & 1  & 1  & 0  & y_3 \\
    0  & 7  & 14 & -7 & y_2 + 2y_3\\
    0  & -2 & -4 & 2  & y_1 + y_3\\
    0  & 2  & 4  & -2 & y_4 - 2y_3
  \end{pmatrix}& 
  \begin{aligned}
    L_2 &\leftarrow L_2 + 2L_1 \\
    L_3 &\leftarrow L_3 + L_1 \\
    L_4 &\leftarrow L_4 - 2L_1
  \end{aligned}\\
  &\begin{pmatrix}
    -1 & 1  & 1  & 0  & y_3 \\
    0  & 1  & 2 & -1 & \frac{1}{7}(y_2 + 2y_3)\\
    0  & 1  & 2 & -1  & -\frac{1}{2}( y_1 + y_3)\\
    0  & 1  & 2  & -1 & \frac{1}{2}( y_4 - 2y_3)
  \end{pmatrix}& 
  \begin{aligned}
    L_2 &\leftarrow \frac{1}{7}L_2 \\
    L_3 &\leftarrow -\frac{1}{2}L_3 \\
    L_4 &\leftarrow \frac{1}{2}L_4
  \end{aligned}\\
\end{align*}
Avec $L_4 \leftarrow L_4 - L_3$ et $L_2 \leftarrow L_2 - L_3$ on aboutit (après une permutation des lignes) à la forme réduite
\[
\begin{pmatrix}
    -1 & 1  & 1 & 0  & y_3 \\
    0  & 1  & 2 & -1 & -\frac{1}{2}( y_1 + y_3)\\
    0  & 0  & 0 & 0  & A\\
    0  & 0  & 0 & 0  & B
\end{pmatrix}
\]
avec
\[
  \begin{aligned}
    A &= \frac{1}{7}(y_2 + 2y_3) + \frac{1}{2}( y_1 + y_3) \\
    B &= \frac{1}{2}( y_4 - 2y_3) + \frac{1}{2}( y_1 + y_3)
  \end{aligned}
\]
Sur la partie gauche de la forme réduite, on lit que le rang est $2$. Le noyau est invariant par les opérations sur les lignes (multiplication à gauche par des matrices inversibles.
Le système d'équations du noyau est donc
\[
\left\lbrace
\begin{aligned}
  -x_1 + x_2 + x_3 &= 0 \\
        x_2+ 2x_3 - x_4 &=0
\end{aligned}
\right.
\Leftrightarrow 
\left\lbrace
\begin{aligned}
  x_1 &= -x_3 + x_4 \\
  x_2 &= -2x_3 + x_4 
\end{aligned}
\right.
\]
Une base du noyau est
\[
  (
  \begin{pmatrix}
    -1 \\ -2 \\ 1 \\ 0
  \end{pmatrix}, \hspace{0.2cm}
  \begin{pmatrix}
    1 \\ 1 \\ 0 \\ 1
  \end{pmatrix}
  ).
\]
Le système d'équations de l'image est
\[
  \left\lbrace
  \begin{aligned}
    A &= 0 \\ B &= 0
  \end{aligned}
  \right.
  \Leftrightarrow
  \left\lbrace
  \begin{aligned}
    7y_1 +2 y_2 + 9y_3 &= 0 \\ y_1 - y_3 +y_4 &= 0
  \end{aligned}
  \right.  
\]
Toutes les colonnes de la matrice de départ sont dans l'image. Comme le rang est $2$, si on en prend $2$ formant une famille libre c'est une base de l'image. Par exemple les deux premières ne sont clairement pas colinéaires et forment une base de l'image.
\[
  (
  \begin{pmatrix}
    1 \\ 2 \\ -1 \\ -2
  \end{pmatrix}, \hspace{0.2cm}
  \begin{pmatrix}
    -3 \\ 5 \\ 1 \\ 4
  \end{pmatrix}
  ).
\]
