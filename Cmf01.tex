\begin{tiny}(mf01)\end{tiny} Par définition des matrices de passage:
\begin{displaymath}
 P_{\mathcal{B} \mathcal{U}}=
\begin{pmatrix}
 1 & 0 & 0 \\ 0 & 1 & 1\\ 0 & 0 & 1
\end{pmatrix}
\end{displaymath}
Cette matrice traduit un système que l'on peut inverser pour exprimer l'autre matrice de passage
\begin{multline*}
 \left\lbrace 
\begin{aligned}
 u_1 &= e_1\\ u_2 &= e_2 \\ u_3 &= e_2+e_3
\end{aligned}
\right. 
\Rightarrow
 \left\lbrace 
\begin{aligned}
 e_1 &= u_1\\ e_2 &= u_2 \\ e_3 &= -u_2+u_3
\end{aligned}
\right. 
\\
\Rightarrow
 P_{\mathcal{U} \mathcal{B} }=
\begin{pmatrix}
 1 & 0 & 0 \\ 0 & 1 & -1\\ 0 & 0 & 1
\end{pmatrix}.
\end{multline*}
