\begin{tiny}(Cis17)\end{tiny} Par linéarité, on se ramène au cas où $f(0)=0$. Notons 
\[
 M = \max_{[0,1]}|f|.
\]
Raisonnons à la Cesàro en coupant arbitrairement l'intégrale pour la majorer.
\[
 \forall \alpha \in \left] 0,1\right[, \;
 |I_n| \leq \alpha \max_{\left[ 0, \alpha^n\right] }|f| + (1-\alpha)M.
\]
Pour tout $\varepsilon >0$:
\begin{align*}
 &\exists \alpha \in \left] 0,1 \right[ \text{ (proche de $1$) tq } (1-\alpha)M \leq \frac{\varepsilon}{2}, \\
 &\exists \beta >0 \text{ tq } \max_{\left[ 0, \alpha^n\right] }|f| \leq \frac{\varepsilon}{2} \text{ (continuité en $0$)},\\
 &\exists N \text{ tq } n\geq N \Rightarrow \alpha^n < \beta.
\end{align*}
On écrit alors la définition de la convergence vers $0$.