\begin{tiny}(Cao02)\end{tiny} Pour chaque matrice, on indique les coordonnées du vecteur $u$ de l'axe obtenu par antisymétrisation, le coefficient $\pm 1$ de son image et la nature de l'automorphisme orthogonal.
\begin{align*}
&\frac{1}{9}\begin{pmatrix}
7 & -4 & 4 \\ 
-4 & 1 & 8 \\ 
4 & 8 & 1
\end{pmatrix}:
\end{align*}

\begin{align*}
\frac{1}{3}\begin{pmatrix}
-2 & 2 & 1 \\ 
2 & 1 & 2 \\ 
1 & 2 & -2
\end{pmatrix}:
\end{align*}

\begin{align*}
\frac{1}{9}\begin{pmatrix}
8 & 1 & -4 \\ 
-4 & 4 & -7 \\ 
1 & 8 & 4
\end{pmatrix}:
& & \begin{pmatrix}
3 \\ -1 \\ -1   
\end{pmatrix}
& &1
\end{align*}
rotation d'angle $\arccos \frac{7}{18}$ autour de $u$.

\begin{align*}
\frac{1}{3}\begin{pmatrix}
2 & 2 & 1 \\ 
-2 & 1 & 2 \\ 
-1 & 2 & -2
\end{pmatrix}:
& &\begin{pmatrix}
 0 \\ 1 \\ -2  
\end{pmatrix}
& &-1
\end{align*}
rotation miroir d'angle $\arccos \frac{2}{3}$ autour de $u$.

\begin{align*}
\frac{1}{4}\begin{pmatrix}
3 & 1 & \sqrt{6} \\ 
1 & 3 &  -\sqrt{6}\\ 
-\sqrt{6} & \sqrt{6} & 2
\end{pmatrix}:
& &\begin{pmatrix}
1 \\ 1 \\ 0  
\end{pmatrix}
& &1
\end{align*}
rotation d'angle $\frac{2\pi}{3}$ autour de $u$.

\begin{align*}
\frac{1}{3}
\begin{pmatrix}
2&1&2\\-2&2&1\\-1&-2&2 
\end{pmatrix}:
& &\begin{pmatrix}
-1 \\ 1 \\ -1  
\end{pmatrix}
& &1 
\end{align*}
rotation d'angle $\frac{2\pi}{3}$ autour de $u$. 