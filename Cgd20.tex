\begin{tiny}(Cgd20)\end{tiny}
\begin{enumerate}
  \item On fixe un $a >0$ puis on majore avec l'inégalité des accroissements finis.
\begin{multline*}
  \forall x \geq a, \;
  f(x) = f(a) + \left(f(x) - f(a)\right)\\
  \Rightarrow |f(x)| \leq |f(a)| + \left| f(x) - f(a)\right|\\
  \leq |f(a)| + (x-a)\max_{\left[a,X\right]}|f'| \\
  \Rightarrow \frac{|f(x)|}{x} \leq \frac{|f(a)|}{x} + \max_{\left[a,X\right]}|f'|.
\end{multline*}
On termine comme pour le théorème de Cesaro \index{théorème de Cesaro}.\newline
Pour tout $\varepsilon >0$, il existe $a>0$ tel que
\[
  \max_{\left[a,X\right]}|f'|\leq \frac{\varepsilon}{2}.
\]
Ce $a$ étant fixé, $\frac{|f(a)|}{x} \rightarrow 0$. Il existe $b >a$ tel que
\[
  x >b \Rightarrow
  \frac{|f(x)|}{x} \leq \frac{\varepsilon}{2} + \frac{\varepsilon}{2}.
\]

  \item On se ramène au a. appliqué à $g$ définie par
\[
  g(x) = f(x) - \lambda x.
\]

  \item Cette fois l'inégalité des accroissements finis permet d'écrire
\[
  \frac{f(x)}{x} \geq -\frac{|f(a)|}{x} + \frac{x-a}{x}\,\min_{\left[a,X\right]}|f'|. 
\]
Pour tout $E$, il existe $a$ assez grand pour que 
\[
 x\geq a \Rightarrow  \min_{\left[a,x\right]}|f'| \geq 4E.
\]
Ce $a$ étant fixé, il existe $b >a$ assez grand pour que
\[
  x \geq b \Rightarrow \frac{x-a}{x} \geq \frac{1}{2} \;\text{ et }\; \frac{|f(a)|}{x} \leq A
\]
On en déduit
\[
  x \geq b \Rightarrow \frac{f(x)}{x} \geq - A + \frac{1}{2} 4 A = A. 
\]

\end{enumerate}
