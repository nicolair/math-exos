\begin{tiny}(Eap02)\end{tiny} Formule de Stirling.\\
Pour $n$ dans $N^*$, on pose $S_n= \ln(n!)$ et 
\begin{displaymath}
 x_n=S_n-n\ln(n)+n,\hspace{0.5cm}
y_n = x_n-\frac{1}{2}\ln(n)
\end{displaymath}
\begin{enumerate}
 \item Montrer que $0\leq x_n$.
 \item Calculer un développement de $x_{n+1}-x_n$ dont le reste est $o(\frac{1}{n^2})$.
 \item Calculer un développement de $y_{n+1}-y_n$ dont le reste est $o(\frac{1}{n^2})$.
 \item Montrer que la suite est monotone à partir d'un certain rang, montrer ensuite qu'elle est convergente. En déduire l'existence d'un réel $K$ tel que
\begin{displaymath}
 n! \sim n^ne^{-n}\sqrt{n}K
\end{displaymath}
\end{enumerate}
(voir \href{\baseurl temptex/fexip.pdf}{exercice ip03} pour une expression de $K$.)