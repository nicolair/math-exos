\begin{tiny}(Eee02)\end{tiny} On prend $\arccos (\frac{(u / v)}{\left\| u\right\| \,\left\| v\right\| })$ comme d{\'e}finition de l'{\'e}cart angulaire entre deux vecteurs dans un espace pré-hilbertien.\newline
Existe-t-il trois vecteurs unitaires d'un espace euclidien dont les {\'e}carts angulaires deux {\`a} deux d{\'e}passent tous $\frac{2\pi }{3}$ ?\newline
On considère $p$ vecteurs unitaires $a_1,\cdots, a_p$, montrer qu'il existe un couple $(i,j)$ tels que $i\neq j$ et que l'écart angulaire entre $a_i$ et $a_j$ soit plus petit que 
\begin{displaymath}
  \arccos\left( -\frac{1}{p-1}\right) 
\end{displaymath}

