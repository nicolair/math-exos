\begin{tiny}(Cis01)\end{tiny} Comme l'intégrale de $f$ entre $a$ et $b$ est nulle, il existe $c_1\in]a,b[$ tel que $f$ s'annule en $c_1$ en changeant de signe.\newline
Considérons la fonction $x\rightarrow (x-c_1)f(x)$. Elle s'annule en $c_1$ mais sans changer de signe. Or, par linéarité son intégrale est nulle. Elle doit donc s'annuler en changeant de signe en un point $c_2\in ]a,b[$. Ce point est forcément distinct de $c_1$. On peut raisonner de même pour une troisième racine et d'autres autant de fois que les
\begin{displaymath}
 \int_{[a,b]}f(x)x^kdx
\end{displaymath}
sont nuls.