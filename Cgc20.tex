\begin{tiny}Cgc20\end{tiny} Notons $\varphi_k$ et $\psi_k$ les fonctions proposées
\begin{displaymath}
  \varphi_k(x) = f(x) -kx, \hspace{0.5cm} \psi_k(x) = f(x) +kx
\end{displaymath}
\begin{enumerate}
  \item Pour tous $x<y$ dans $I$:
\begin{multline*}
\varphi_k(y) - \varphi_k(x) = f(y)-f(x) - k(y-x)\\
\leq \left|f(y)-f(x)\right| - k(y-x)
\leq k\left( \left|y-x\right| -(y-x)\right)\\
= k\left( (y-x) -(y-x)\right) = 0
\end{multline*}

On en déduit $\varphi_k$ décroissante.\newline
On démontre de même que $\psi_k$ est croissante.

  \item Le caractère lipschitzien sur $]a,b[$ entraîne que la fonction $f$ est bornée. Il en est de même pour $\varphi_k$ et $\psi_k$ qui admettent donc des limites finies en $a$ et $b$ par monotonie bornée. Comme $f$ peut s'exprimer comme une combinaison de ces fonctions elle admet aussi des limites finies. Le caractère lipschitzien est conservé en $a$ et $b$ à cause du théorème de passage à la limite dans une inégalité.
  
  \item à compléter.
\end{enumerate}

