\begin{tiny}(Eli13)\end{tiny} Fonctions semi-continues.\newline
Soit $I$ un intervalle de $\R$ et $a\in I$. Une fonction $f$ définie dans $I$ est dite \emph{semi-continue supérieurement} en $a$ si et seulement si:
\begin{multline*}
\forall \varepsilon >0, \exists \alpha >0 \text{ tq }:\\
\forall x\in I, \left|x-a\right| \leq \alpha \Rightarrow f(x) - f(a) \leq \varepsilon
\end{multline*}

De même, une fonction $f$ définie dans $I$ est dite \emph{semi-continue inférieurement} en $a$ si et seulement si:
\begin{multline*}
\forall \varepsilon >0, \exists \alpha >0 \text{ tq }:\\
\forall x\in I, \left|x-a\right| \leq \alpha \Rightarrow \varepsilon \leq f(x) - f(a)
\end{multline*}

Pour tout $n\in \N$, $f_n$ est une fonction continue dans $I$. On suppose que de plus que pour tout $a\in I$, la suite de nombre réels $\left( f_n(a)\right)_{n\in \N}$ est bornée.\newline
Montrer que l'on peut définir dans $I$ des fonctions $g$ et $h$ par:
\begin{displaymath}
\forall a\in I,
\left\lbrace 
\begin{aligned}
  g(x) &= \inf\left\lbrace f_n(a), n\in \N\right\rbrace \\
  h(x) &= \sup\left\lbrace f_n(a), n\in \N\right\rbrace
\end{aligned}
\right. 
\end{displaymath}
et que l'une est semi-continue supérieurement et l'autre inférieurement.
