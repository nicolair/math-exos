\begin{tiny}(Ccp35)\end{tiny}
Faisons jouer un rôle particulier à la racine cubique $z_1$ de $A$. D'après le cours, les deux autres s'obtiennent en multipliant $z_1$ par les éléments de $\U_3$ (racines cubiques de l'unité). On a donc deux possibilités:
\begin{itemize}
  \item Cas 1. $z_2 = jz_1$ et $z_3=j^2z_1$.
  \item Cas 2. $z_2 = j^2z_1$ et $z_3=jz_1$.
\end{itemize}
Dans le cas 1, la relation se traduit par 
\begin{multline*}
  z_1^2 = \frac{1}{(1-j)j^2}=\frac{j}{1-j}\\
  \Rightarrow
  A^2 = \left(\frac{j}{(1-j)} \right)^3 = \frac{1}{1-3j+3j^2-1}=\frac{i}{3\sqrt{3}}
\end{multline*}

De manière analogue, dans le cas 2, la relation se traduit par:
\begin{multline*}
  z_1^2 = \frac{1}{(1-j^2)j}=\frac{j^2}{1-j^2} \\
  \Rightarrow
  A^2 = \left(\frac{j^2}{(1-j^2)} \right)^3 = \frac{1}{1-3j^2+3j-1}=-\frac{i}{3\sqrt{3}}
\end{multline*}

On obtient donc 4 valeurs possibles pour $A$ (qui sont les racines quatrièmes de $-\frac{1}{27}$).
