\begin{tiny}(Edt10)\end{tiny}
On considère des nombres réels 
\begin{displaymath}
 \alpha_1 < \alpha_2 < \cdots <\alpha_n \hspace{0.5cm}
\beta_1 < \beta_2 < \cdots <\beta_n 
\end{displaymath}
L'objet de cet exercice est de montrer que $\det M >0$ où $M\in \mathcal{M}_p(\R)$ avec
\begin{displaymath}
 m_{i,j} = e^{\alpha_i \beta_j}
\end{displaymath}
\begin{enumerate}
 \item Montrer qu'une fonction du type :
\begin{displaymath}
 x \rightarrow \sum_{i=1}^{p}c_ie^{a_ix}
\text{ avec }
(c_1,\cdots,c_p) \neq (0,\cdots,0)
\end{displaymath}
et $a_1 < a_2 < \cdots < a_p$ s'annule au plus $p-1$ fois.
\item Montrer le résultat annoncé. On pourra considérer le déterminant:
\begin{displaymath}
 \begin{vmatrix}
  e^{\alpha_1\beta_1}& \cdots & e^{\alpha_1\beta_{p-1}} & e^{\alpha_1x} \\
  e^{\alpha_2\beta_1}& \cdots & e^{\alpha_2\beta_{p-1}} & e^{\alpha_2x} \\
  \vdots &  &\vdots  & \vdots \\
 e^{\alpha_p\beta_1}& \cdots & e^{\alpha_p\beta_{p-1}} & e^{\alpha_px} 
 \end{vmatrix}
\end{displaymath}

\end{enumerate}
