\begin{tiny}(Cfu24)\end{tiny} Avec les indications de l'énoncé et $c^3+s^3=1$ :
\begin{multline*}
\left. 
\begin{aligned}
  c+s=(c^2+s^2)(c+s) &= 1 + c^2s + cs^2 \\
  (c+s)^3 &= 1 + 3(c^2s + cs^2)
\end{aligned}
\right\rbrace \Rightarrow \\
(c+s)^2-3(c+s) = -2
\end{multline*}
Or $1$ est racine évidente de
\begin{displaymath}
  z^3-3z+2 = 0
\end{displaymath}
En divisant par $z-1$, on factorise:
\begin{displaymath}
  z^3-3z+2=(z-1)^2(z-2)
\end{displaymath}
On doit donc avoir $c+s=1$ c'est à dire $x$ congru à $\frac{\pi}{2}$ ou $0$ modulo $2\pi$.