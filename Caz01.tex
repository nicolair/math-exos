\begin{tiny}(Caz01)\end{tiny}
\begin{enumerate}
 \item D'après l'expression des coefficients du binôme avec des produits:
 \begin{multline*}
  \binom{p}{n} = \frac{p}{n} \binom{p-1}{n-1}\\
  \Rightarrow
  p \text{ divise } n \binom{p}{n}
  \Rightarrow
  p \text{ divise } \binom{p}{n}
 \end{multline*}

 
lorsque $n\wedge p = 1$ d'après le théorème de Gauss.\newline
En développant $(a+1)^p$ avec la formule du binôme, tous les coefficients sauf les deux extrèmes disparaissement modulo $p$. On en déduit la formule demandée.
 \item On raisonne par  récurrence sur $a$. La formule est évidente pour $a= 0$ ou $1$. On passe de $a$ à $a+1$ avec la première question.\newline
 Lorsque $a\wedge p = 1$, il existe $b$ (théorème de Bezout) tel que $ab \equiv 1 \mod p$. Il suffit de multiplier la relation précédente par $b$ pour obtenir le petit théorème de Fermat. 
\end{enumerate}
