\begin{tiny}(Csn23)\end{tiny} On peut écrire 
\begin{displaymath}
 \sin \frac{\pi}{n} = \frac{\pi}{n} + r_n
 \text{ avec }
 r_n \in O(\frac{1}{n^3}).
\end{displaymath}
On interprète la somme des $\frac{\pi}{k}$ comme une somme de Riemann. La série de terme général $r_n$ est absolument convergente donc la somme associée converge vers $0$. La limite cherchée est donc
\begin{displaymath}
 \pi\int_{0}^{1}\frac{dt}{1+t} 
 = \pi \ln 2.
\end{displaymath}
