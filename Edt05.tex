\begin{tiny}(Edt05)\end{tiny}
Calculer les d{\'e}terminants des $A\in \mathcal{M}_n(\R)$ pour :
\begin{enumerate}
 \item $a_{ij}=\binom{i+j-2}{i-1}$
\item $a_{ij}=(a+i+j)^{2}$, $a$ r{\'e}el fix{\'e}
\item $a_{ij}=\left\{
\begin{array}{ccc}
a & \text{si} & j<i \\
a+b & \text{si} & j=i \\
b & \text{si} & j>i
\end{array}
\right. $\newline
Utiliser le d{\'e}terminant $D(x)$ obtenu en ajoutant des $x$ partout. Montrer que c'est une fonction de $x$ d'un type simple qui se calcule facilement en deux valeurs bien choisies.\newline \'Etendre en remplaçant $(a,a+b,b)$ par $(a,b,c)$. 
\item $a_{i,i}=2$, $a_{i i+1}=1$, $a_{i i-1}=3$ et $a_{i j}=0$ sinon.\newline
On mettra le résultat sous la forme de la partie imaginaire d'un nombre complexe. 
\item $a_{i j}=\min(i,j)$.
\item $a_{i j}=S_{\min(i,j)}$ avec
\begin{displaymath}
 S_k = 1 +\frac{1}{2}+\cdots + \frac{1}{k}
\end{displaymath}
\item $a_{i j}=
\left\lbrace 
\begin{aligned}
 1 &\text{ si }j\leq i \\ a_j &\text{ si }j> i
\end{aligned}\right. $
\item $a_{i j}=
\left\lbrace 
\begin{aligned}
 x_i &\text{ si } i\neq j \\ x_i + y_i &\text{ si } i=j
\end{aligned}\right. $
\end{enumerate}