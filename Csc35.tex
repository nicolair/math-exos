\begin{tiny}(Csc35)\end{tiny} 
\begin{enumerate}
 \item 
 \item
 \item Comme $(v_n)$ décroit vers $0$, tous les $v_n$ sont positifs ou nuls. Ceci entraine en particulier que parmi deux valeurs consécutives de $u_n$, au moins une est positive.\newline
 Si $u_n \geq 0$ alors $|u_n| = u_n \leq \max(u_n,u_{n-1}) = v_{n-1}$.\newline
 Si $u_n < 0$ alors $u_{n-1}$ et $u_{n+1}$ sont $\geq 0$.
\[
 \left. 
 \begin{aligned}
  u_{n-1} &\geq 0 \\ u_n &< 0
 \end{aligned}
 \right\rbrace \Rightarrow u_{n-1} = v_{n-1}.
\]
\[
 0 \leq u_{n+1} \leq \frac{u_{n-1} + u_n}{2}\Rightarrow u_{n-1} + u_n \geq 0.
\]
On en déduit
\[
 |u_n| = -u_n \leq u_{n-1} = v_{n-1}.
\]
On conclut par le théorème d'encadrement.

 \item On se ramène au cas précédent en considérant la suite $\left( u_n -v\right)_{n \in \N}$. 
\end{enumerate}
