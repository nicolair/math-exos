\begin{tiny}(Egc14)\end{tiny}\label{exo:Egc14}
Soit $I$ un intervalle de $\R$ et $f$ une application \emph{injective et continue} définie dans $I$. On se propose de montrer que $f$ est strictement monotone.\newline
Pour $u$, $v$, $w$ dans $I$ tels que $u<v<w$, on nomme $I_1(u,v,w)$, $I_2(u,v,w)$, $I_3(u,v,w)$ les trois implications suivantes.
\begin{align*}
 I_1(u,v,w):  &\left( f(v)<f(w) \Rightarrow f(u)<f(v) \right) \\
 I_2(u,v,w):  &\left( f(u)<f(w) \Rightarrow f(u)<f(v)<f(w)\right)  \\
 I_3(u,v,w):  &\left( f(u)<f(v) \Rightarrow f(v)<f(w) \right) 
\end{align*}

\begin{enumerate}
 \item On veut prouver ces trois implications.\newline
On considère $u<v<w$ fixés dans $I$. On va exploiter l'injectivité de $f$ et le théorème des valeurs intermédiaires pour tirer des conséquences de certaines inégalités entre des images.\newline
Dans le tableau suivant, ces inégalités figurent dans la colonne de gauche et leurs conséquences dans celle de droite.
\begin{center}
\renewcommand{\arraystretch}{1.3}
\begin{tabular}{c|c|c}
 inégalité & intervalle & conséquence\\ \hline
 $f(v)<f(w)$ &  & $f(u)\not\in [f(v),f(w)]$\\ \hline
 $f(v)<f(w)$ &  & $f(u)\geq f(w)$ faux\\ \hline
 $f(u)<f(w)$ &  & $f(v)\leq f(u)$ faux\\ \hline
 $f(u)<f(w)$ &  & $f(v)\geq f(w)$ faux\\ \hline
 $f(u)<f(v)$ &  & $f(w)\not\in[f(u),f(v)]$\\ \hline
 $f(u)<f(v)$ &  & $f(w)\leq f(u)$ faux
\end{tabular}
\end{center}
Compléter ce tableau en indiquant dans la colonne du milieu un intervalle dans lequel l'application du théorème des valeurs intermédiaires (associé à l'injectivité de $f$) prouve la conséquence à droite. 
\item On suppose qu'il existe $a<b$ dans $I$ tel que $f(a)<f(b)$.\newline
On veut montrer que $f$ est strictement croissante en considérant dans le tableau suivant les différents cas possibles pour $x<y$ quelconques dans $I$. 
\begin{center}
\renewcommand{\arraystretch}{1.2}
\begin{tabular}{c|c}
cas & arguments\\ \hline
$x<y<a<b$ & \phantom{$I_1(a,b,c)$} puis \phantom{$I_1(a,b,c)$}\\ \hline
$x<a<y<b$ & \phantom{$I_1(a,b,c)$} puis \phantom{$I_1(a,b,c)$}\\ \hline
$x<a<b<y$ & \phantom{$I_1(a,b,c)$} puis \phantom{$I_1(a,b,c)$}\\ \hline
$a<x<y<b$ & \phantom{$I_1(a,b,c)$} puis \phantom{$I_1(a,b,c)$}\\ \hline
$a<x<b<y$ & \phantom{$I_1(a,b,c)$} puis \phantom{$I_1(a,b,c)$}\\ \hline
$a<b<x<y$ & \phantom{$I_1(a,b,c)$} puis \phantom{$I_1(a,b,c)$}
\end{tabular}
\end{center}
Compléter ce tableau en justifiant que $f(x)<f(y)$. Cet argumentation sera constituée par deux implications successives de la forme  ($I$) indiquée au début pour un bon choix des lettres et de l'indice. (plusieurs réponses sont possibles)
\item Montrer le résultat annoncé.
\end{enumerate}
