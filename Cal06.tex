\begin{tiny}(Cal06)\end{tiny}
\begin{enumerate}
 \item Avec la définition de $g_a$, il s'agit d'une simple reformulation des propriétés. 
 \item Supposons $a$ inversible à gauche. Il existe alors $b$ tel que $ab=1_A$. On en déduit la surjectivité de $g_a$ car, pour tout $y\in A$,
\begin{displaymath}
 aby=y\Rightarrow g_a(by)=y
\end{displaymath}
Supposons $g_a$ surjectif. Il existe alors $b$ tel que $g_a(b)=1_A$ c'est à dire $ab=1_A$.
 \item Supposons $a$ inversible (des deux côtés). L'inversibilité à gauche entraine que $g_a$ est surjective. Si $x$ et $y$ dans $A$ sont tels que $g_a(x)=g_a(y)$ alors $ax=ay$ et en multipliant à gauche par l'inverse $a^{-1}$ on obtient $x=y$ c'est à dire l'injectivité de $g_a$.\newline
Supposons $g_a$ bijectif. De la surjectivité, on tire l'existence d'un $b$ tel que $ab=1_A$. On multiplie à droite par $a$ et on utilise l'injectivité
\begin{displaymath}
 aba=a\Rightarrow g_a(ba)=g_a(1_A) \Rightarrow ba=1_A
\end{displaymath}

 \item Il est immédiat que $g_x \circ g_y = g_{xy}$. On peut reformuler les hypothèses de manière ensembliste.
\begin{multline*}
 \left. 
\begin{aligned}
 g_a \circ g_b\text{ surjectif} &\Rightarrow g_a \text{ surjectif }\\
 g_b \circ g_a \text{ injectif} &\Rightarrow g_a \text{ injectif }
\end{aligned}
\right\rbrace \\ 
\Rightarrow
 g_a \text{ bijectif} \Rightarrow a \text{ inversible }
\end{multline*}
puis pour l'inversibilité de $b$ :
\begin{displaymath}
 \left. 
\begin{aligned}
 g_a \circ g_b\text{ surjectif}\\
 g_a \text{ bijectif}
\end{aligned}
\right\rbrace \Rightarrow g_b \text{ surjectif}
\end{displaymath}
\begin{displaymath}
 \left. 
\begin{aligned}
 g_b \circ g_a\text{ injectif}\\
 g_a \text{ bijectif}
\end{aligned}
\right\rbrace \Rightarrow g_b \text{ injectif}
\end{displaymath}

\end{enumerate}
