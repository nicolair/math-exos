\begin{tiny}(Eao05)\end{tiny}
Soit $E$ euclidien orienté de dimension 3,
\[\mathcal{B}=(
\overrightarrow{e_{1}},\overrightarrow{e_{2}},\overrightarrow{e_{3}})\]
une base orthonormée directe de $E$. Soit $\overrightarrow{v}$ le vecteur de coordonnées $(\alpha ,\beta,\gamma )$ dans $\mathcal{B}$ et 
\begin{displaymath}
A=
\begin{pmatrix}
0 & -\gamma  & \beta  \\
\gamma  & 0 & -\alpha  \\
-\beta  & \alpha  & 0
\end{pmatrix} 
\end{displaymath}
On note 
\begin{displaymath}
U=(I-A)(I+A)^{-1} 
\end{displaymath}
et $f$ l' endomorphisme dont la matrice dans $\mathcal{B}$ est $U$.\newline
On note $\varphi_{\overrightarrow{v}}$ l'application définie par
\begin{displaymath}
 \varphi_{\overrightarrow{v}}(\overrightarrow{x}) = \overrightarrow{v}\wedge \overrightarrow{x} 
\end{displaymath}
Quelle est la matrice de  $\varphi_{\overrightarrow{v}}$ dans $\mathcal B$ ? Exprimer $f$ comme une composition d'endomorphismes. Calculer la matrice de $f$ dans une base orthonormée directe 
\begin{displaymath}
\mathcal U=(\overrightarrow{a},\overrightarrow{b},\frac{1}{\Vert \overrightarrow{v}\Vert}\overrightarrow{v}) 
\end{displaymath}
En déduire que $f$ est une rotation d'axe $\Vect(\overrightarrow{v})$, déterminer l'angle autour de $\overrightarrow{v}$.\newline
(voir l'exercice ao01)