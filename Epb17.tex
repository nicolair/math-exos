\begin{tiny}(Epb17)\end{tiny} Le problème avec le Réparateur, c'est qu'il dit la même chose à tout le monde :
\begin{quotation}
  \og Oui oui. Demain je viens et c'est réglé!\fg
\end{quotation}
Effectivement, quand il vient, il répare toujours! Mais il ne vient pas forcément (seulement avec la probabilité $v$). Même s'il n'est pas venu, il vous dira la même chose quand vous le rappelerez.\newline
Le problème chez Antoine, c'est la Douche du premier étage. Il la vérifie tous les jours! Qu'elle fonctionne un jour n'assure pas qu'elle fonctionnera le lendemain (seulement avec la probabilité $f$). Quand ça ne marche pas, Antoine appelle paisiblement le Réparateur le jour même.\newline
Aujourd'hui tout va bien; mais dans un mois, son cousin descend de la montagne pour passer quelques jours chez lui. Quelle est la probabilité que la Douche fonctionne ? Ce serait bien si, à long terme, la probabilité de fonctionner était supérieure à $0.9$. Comment représenter graphiquement cette configuration d'un avenir satisfaisant.  
