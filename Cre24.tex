\begin{tiny}(Cre24)\end{tiny} La partie $A$ est non vide et minorée par $0$. Montrons qu'elle admet un plus petit éléments
\[
  \in A = \min A = n^2.
\]
En effet $n^2 \in A$ en prenant tous les $a_i=1$. D'autre part, on sait que
\[
  \forall t >0, \; t + \frac{1}{t} \geq 2.
\]
On en déduit,
\begin{multline*}
  \sum_{(i,j)\in \llbracket 1,n\rrbracket^2} \frac{x_i}{x_j} 
  = n +  2\sum_{i < j} \left(\underset{ \geq 2}{\underbrace{\frac{x_i}{x_j} + \frac{x_j}{x_i}}}\right)\\
  \geq n + \frac{n(n-1)}{2}\times 2 = n^2.
\end{multline*}
Donc $n^2$ est bien un minorant de $A$ qui appartient à $A$.
