\begin{tiny}(Efu26)\end{tiny} Soit $s>0$. Montrer que
\begin{displaymath}
  \left\lbrace \frac{1}{x}+\frac{1}{y}, (x,y)\in ]0,1[^2 \text{ tq } x+y = s\right\rbrace .
\end{displaymath}
admet un plus petit élément que l'on déterminera.\newline
Même question avec
\begin{displaymath}
  \left\lbrace \frac{1}{x}+\frac{1}{y}+\frac{1}{z}, (x,y,z)\in ]0,1[^3 \text{ tq } x+y+z = s\right\rbrace 
\end{displaymath}
puis généraliser avec une somme de $n$ (naturel quelconque) inverses.
