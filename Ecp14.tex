\begin{tiny}(Ecp14)\end{tiny} \label{cp14} Caractérisation des triangles équilatéraux.\newline
Dans cet exercice, on note $u=e^{i\frac{\pi}{3}}$. On désigne par $a$, $b$, $c$ les affixes de trois points $A$, $B$, $C$ deux à deux distincts.
\begin{enumerate}
 \item Exprimer $u$ en fonction de $j$. Déterminer l'ensemble des racines cubiques de $-1$. Former une équation dont les solutions sont $u$ et $\overline{u}$.
 \item Simplifier
\begin{displaymath}
 \frac{c-a}{b-a}\, \frac{a-b}{c-b}\, \frac{b-c}{a-c}
\end{displaymath}
Interpréter géométriquement ce résultat.

\item Montrer que le triangle $(A,B,C)$ est équilatéral si et seulement si
\begin{displaymath}
 \frac{c-a}{b-a} = \frac{a-b}{c-b} = \frac{b-c}{a-c} \in \left\lbrace u, \overline{u}\right\rbrace 
\end{displaymath}
Quelle est l'interprétation géométrique de ces cas ?

\item Montrer que
\begin{displaymath}
\frac{a-b}{c-b} = \frac{b-c}{a-c}
\Leftrightarrow
 a^2 + b^2 + c^2 =  ab+bc+ca
\end{displaymath}
En déduire que 
\begin{displaymath}
\frac{a-b}{c-b} = \frac{b-c}{a-c}
\Leftrightarrow
 \frac{c-a}{b-a} = \frac{a-b}{c-b} = \frac{b-c}{a-c}
\end{displaymath}

\item Montrer que les quatre propositions suivantes sont équivalentes :
\begin{itemize}
 \item $(A,B,C)$  est équilatéral
 \item $j$ ou $j^2$ est solution de l'équation $az^2+bz+c=0$ d'inconnue $z$.
 \item $a^2+b^2+c^2 = ab+bc+ca$.
 \item $(b-a)^2 + (c-b)^2 + (a-c)^2 =0$. 
\end{itemize}
\end{enumerate}
