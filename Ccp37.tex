\begin{tiny}(Ccp37)\end{tiny} On utilise la méthode usuelle pour transformer les différences d'exponentielles:
\begin{align*}
  &\frac{b-m}{a-m}   = e^{i\frac{\beta - \alpha}{2}}\,\frac{\sin\frac{\beta - \theta}{2}}{\sin\frac{\alpha - \theta}{2}} \\
  &\frac{b-m'}{a-m'} = e^{i\frac{\beta - \alpha}{2}}\,\frac{\sin\frac{\beta - \theta'}{2}}{\sin\frac{\alpha - \theta'}{2}}
\end{align*}
\`A cause des inégalités, on connait les signes des $\sin$, les demi-différences sont dans $]-\pi, \pi[$. On en déduit que la différence entre deux arguments est égale à $\pi$ modulo $2\pi$.\newline
On reconnait le théorème de l'arc capable. Il se transporte à un cercle quelconque avec une similitude $z\mapsto uz +v$.
