\begin{tiny}(Cdi16)\end{tiny} Supposons $\rg(f) = p$.\newline
Il existe une base $(a_1,\cdots,a_p)$ de $\Im(f)$. On la complète en une base $(a_1,\cdots, a_n)$ de $E$. Soit $(\alpha_1,\cdots,\alpha_n)$ la base duale des formes coordonnées. Pour tout $x\in E$, $f(x)$ se décompose mais seulement sur les premiers vecteurs
\begin{multline*}
 \forall x \in E, \; f(x) 
 = \alpha_1(x)a_1 + \cdots + \alpha_p(x)a_p + 0_E. 
\end{multline*}
La famille $(\alpha_1,\cdots,\alpha_p)$ extraite d'une base est libre.\newline
Réciproquement, supposons que
\begin{multline*}
  \forall x \in E, \; f(x) 
 = \alpha_1(x)a_1 + \cdots + \alpha_p(x)a_p + 0_E \\
 \text{ avec }
 \left\lbrace 
 \begin{aligned}
  (a_1,\cdots,a_p) &\text{ libre dans } E\\
  (\alpha_1,\cdots,\alpha_p) &\text{ libre dans } E^*
 \end{aligned}
\right. .
\end{multline*}
Alors $\ker \alpha_1 \cap \cdots \cap \ker \alpha_p \subset \ker(f)$.\newline
Comme $(a_1,\cdots,a_p)$ est libre; pour tout $x\in E$:
\begin{multline*}
 x\in \ker(f) 
 \Rightarrow \alpha_1(a)a_1 + \cdots + \alpha_p(x)a_p = 0_E\\
 \Rightarrow \alpha_1(x) = \cdots = \alpha_p(x) = 0_K\\
 \Rightarrow x \in \ker \alpha_1 \cap \cdots \cap \ker \alpha_p.
\end{multline*}
Donc $\ker \alpha_1 \cap \cdots \cap \ker \alpha_p = \ker(f)$.\newline
D'après le résultat hors programme (ex. \ref{exo: di06}) sur la dimension de l'intersection d'une famille d'hyperplans et le théorème du rang,
\begin{multline*}
(\alpha_1,\cdots,\alpha_p)\text{ libre }
\Rightarrow \dim \ker(f) = n-p \\
\Rightarrow \rg(f) = p .
\end{multline*}
