\begin{tiny}(Caz06)\end{tiny}
\begin{multline*}
  (S1): \;\left\lbrace
  \begin{aligned}
    x &\equiv 8 &\mod 15 \\
    x &\equiv 5 &\mod 6
  \end{aligned}
  \right.\\
\Leftrightarrow
\exists (\lambda, \mu) \in \Z^2 \text{ tq }
  \left\lbrace
  \begin{aligned}
    x &= 8 + 15 \lambda  \\
    x &= 5 + 6\mu
  \end{aligned}
\right.\\
\Leftrightarrow
\exists (\lambda, \mu) \in \Z^2 \text{ tq }
  \left\lbrace
  \begin{aligned}
    x &= 9 + 15 \lambda  \\
    3 &= -15 \lambda + 6\mu
  \end{aligned}
\right.
\end{multline*}
\[
  3 = -15 \lambda + 6 \mu \Leftrightarrow 1 = - 5\lambda + 2\mu.
\]
Une solution évidente est $\lambda = 1$, $\mu = 3$. Les autres solutions sont $\lambda = 1+2k$ et $\mu = 3 + 5k$ avec $k\in \Z$.
\[
\text{Ensemble des sols de }(S1) =  \left\lbrace 23 + 30k, k\in \Z \right\rbrace.
\]
\begin{multline*}
  (S2): \;\left\lbrace
  \begin{aligned}
    x &\equiv 5 &\mod 20 \\
    x &\equiv 7 &\mod 14
  \end{aligned}
  \right.\\
\Leftrightarrow
\exists (\lambda, \mu) \in \Z^2 \text{ tq }
  \left\lbrace
  \begin{aligned}
    x &= 5 + 20 \lambda  \\
    x &= 7 + 14\mu
  \end{aligned}
\right.\\
\Leftrightarrow
\exists (\lambda, \mu) \in \Z^2 \text{ tq }
  \left\lbrace
  \begin{aligned}
    x &= 5 + 20 \lambda  \\
    2 &= 20 \lambda - 14\mu
  \end{aligned}
\right.
\end{multline*}
\[
  2 = 20 \lambda - 14 \mu \Leftrightarrow 1 = 10\lambda - 7\mu.
\]
Une solution évidente est $\lambda = -2$, $\mu = -3$. Les autres solutions sont $\lambda = -2 + 7k$ et $\mu = -3 + 10k$ avec $k\in \Z$.
\[
\text{Ensemble des sols de }(S2) =  \left\lbrace -35 + 140k, k\in \Z \right\rbrace.
\]
\begin{multline*}
  (S3): \;\left\lbrace
  \begin{aligned}
    x &\equiv 7 &\mod 15 \\
    x &\equiv 5 &\mod 6
  \end{aligned}
  \right.\\
\Leftrightarrow
\exists (\lambda, \mu) \in \Z^2 \text{ tq }
  \left\lbrace
  \begin{aligned}
    x &= 7 + 15 \lambda  \\
    x &= 5 + 6\mu
  \end{aligned}
\right.\\
\Leftrightarrow
\exists (\lambda, \mu) \in \Z^2 \text{ tq }
  \left\lbrace
  \begin{aligned}
    x &= 7 + 15 \lambda  \\
    2 &= -15\lambda + 6\mu
  \end{aligned}
\right.
\end{multline*}
Il n'existe pas de couple $(\lambda, \mu)$ car $-15\lambda + 6\mu$ est toujours un multiple de $3$.\newline
L'ensemble des solutions de $(S3)$ est vide.
