\begin{tiny}(Cis31)\end{tiny}
\begin{enumerate}
 \item Notons $g$ et $h$ les fonctions considérées
\[
 g(x) = f(x) - kx, \; h(x) = f(x) + kx. 
\]
Montrons que $g$ est décroissante et $h$ croissante. Pour $x < y$:
\begin{multline*}
 g(y) - g(x) = f(y) - f(x) - k(y-x)\\
 = f(y) - f(x) - k|y-x| \\
 \leq |f(y) - f(x)| - k|y-x| 
 \leq 0 .
\end{multline*}

\begin{multline*}
 h(y) - h(x) = f(y) - f(x) + k(y-x)\\
 = f(y) - f(x) + k|y-x| \\
 \geq f(y) - f(x) + |f(y) - f(x)|  
 \geq 0 .
\end{multline*}

 \item Une fonction lipschitzienne est uniformément continue donc elle est bornée et converge au extrémités de l'intervalle voir l'exercice \ref{limunifcont} (Eis23). La fonction prolongée reste lipschitzienne de même rapport par passage à la limite dans les inégalités.
 
 \item La question a. permet de déterminer les variations
{%
\newcommand{\mc}[3]{\multicolumn{#1}{#2}{#3}}
\begin{center}
Cas $a < y < b$.
\begin{tabular}{|c|ccccc|}\hline
               & $a$ &            & $y$ &           & $b$\\ \hline
$\varphi_y(x)$ &     & $g(x)+ky$  &   $|$  & $h(x)-ky$ & \\ \hline
               &     & $\searrow$ &     & $\nearrow$   & \\ 
               &     & \mc{3}{c}{$f(y)$}           & \\ \hline
\end{tabular}
\vspace{0.2cm}

Cas $y < a$. $\varphi_y(x) = h(x) - ky$, $\varphi_y$ est croissante.
\vspace{0.2cm}

Cas $b < y$. $\varphi_y(x) = g(x) + ky$, $\varphi_y$ est décroissante.
\vspace{0.2cm}
\end{center}
}%

à compléter
\end{enumerate}
