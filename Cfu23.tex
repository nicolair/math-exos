\begin{tiny}(Cfu23)\end{tiny} Notons respectivement $\alpha$, $\beta$ et $\gamma$ les trois termes et calculons
\begin{multline*}
  \tan(\alpha + \beta) =
\frac{\frac{1}{2x^2} - \frac{x}{x+1}}{1+\frac{1}{2x^2}\frac{x}{x+1}}
= \frac{-2x^3 + x +1}{2x^3 + 2x^2 +x} \\
= \frac{(1-x)(2x^2+2x+1)}{x(2x^2+2x+1)}= \frac{1-x}{x}
\end{multline*}
en divisant $-2x^3 + x +1$ par $-x+1$ car $1$ est une racine évidente. Ceci permet de conclure:
\begin{displaymath}
\tan(\alpha + \beta) = - \tan \gamma \Rightarrow \alpha + \beta + \gamma \equiv 0 \mod \pi  
\end{displaymath}
Calcul de la valeur de la somme.
\begin{itemize}
  \item Dans $]-\infty, -1[$. Limite en $-\infty : \; 0 - \frac{\pi}{4} + \frac{\pi}{4} = 0$.
  \item Dans $]-1,0[$. Limite à gauche de $0 : \frac{\pi}{2} -0 + \frac{\pi}{2} = \pi$. 
  \item Dans $]1, +\infty[$. Limite en $+\infty : \; 0 - \frac{\pi}{4} + \frac{\pi}{4} = 0$.
\end{itemize}

