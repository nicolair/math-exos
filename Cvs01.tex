\begin{tiny}(Cvs01)\end{tiny}
Il est bien évident que
\begin{displaymath}
  B = C \Rightarrow 
  \left\lbrace 
  \begin{aligned}
    A \cap B &= A \cap C \\ A\cup B &= A\cup C
  \end{aligned}
\right. 
\end{displaymath}
Réciproquement, on veut montrer
\begin{displaymath}
  \left\lbrace 
  \begin{aligned}
    A \cap B &= A \cap C \\ A\cup B &= A\cup C
  \end{aligned}
\right. 
 \Rightarrow  B = C
\end{displaymath}
On veut donc montrer une égalité ensembliste $B=C$ en utilisant certaines hypothèses.\newline
On peut raisonner avec des éléments en prouvant deux inclusions.\newline
Pour montrer $B\subset C$, on commence le raisonnement par un \og Quelque soit $b\in B$\fg.\newline
Pour tout $b\in B$, on veut montrer que $b\in C$.
Comme $B\subset A\cup B = A\cup C$ par hypothèse, $b\in A\cup C$. Donc $b$ est soit dans $A$, soit dans $C$ soit dans les deux. S'il est dans $A$, il est dans $A\cap B = A\cap C$, il est donc forcément dans $C$. On en déduit $B\subset C$.\newline
Comme $B$ et $C$ jouent des rôles symétriques, l'autre inclusion se démontre de la même manière d'où $B=C$.
On peut aussi prouver ce résultat en \og calculant\fg~ avec les relations
\begin{align*}
  X\cap(Y\cup Z) &= (X\cap Y) \cup (X\cap Z)\\  X\cup(Y\cap Z) &= (X\cup Y) \cap (X\cup Z)
\end{align*}
En effet:
\begin{multline*}
B = B\cap(A\cup B)\hspace{0.3cm}\text{ car }B\subset A\cup B\\
 = B\cap(A\cup C) = (B\cap A) \cup (B\cap C) \\
 = (C\cap A) \cup (B\cap C) = C\cap(A\cup B) = C\cap(A\cup C) = C
\end{multline*}
