\begin{tiny}(Cmf12)\end{tiny} Le projeté de $x\in \R^3$ sur $\ker \varphi$ parallélement à $\Vect(u)$ est
\begin{multline*}
 x-\frac{\varphi(x)}{\varphi(u)}\,u \text{ avec } \varphi((x,y,z))= x - 2y + z,\\
 u =(1,0,1), \; \varphi(u) = 2.
\end{multline*}

Utilisons les matrices colonnes dans la base canonique pour former les images : $(1,0,0)$ se projete en
\[
  \begin{pmatrix}
    1 \\ 0 \\ 0
  \end{pmatrix}
- \frac{1}{2}\,\begin{pmatrix}
    1 \\ 0 \\ 1
  \end{pmatrix}
  = \begin{pmatrix}
    \frac{1}{2} \\ 0 \\ -\frac{1}{2}
  \end{pmatrix}.
\]
Avec des calculs analogues pour les autres vecteurs, on en déduit la matrice
\begin{displaymath}
 \begin{pmatrix}
  \frac{1}{2}& 1 & -\frac{1}{2} \\0 & 1 & 0\\ -\frac{1}{2} & 1 & \frac{1}{2}.
 \end{pmatrix}
\end{displaymath}
On peut aussi utiliser des coefficients indéterminés.\newline
Notons $A$ la matrice demandée. D'après la formule vectorielle :
\begin{multline*}
  \forall(x,y,z) \in \R^3,\; 
  A \begin{pmatrix} x \\ y \\ z \end{pmatrix}
  = \begin{pmatrix} x \\ y \\ z \end{pmatrix} - \frac{x - 2y +z}{2}\,\begin{pmatrix} 1 \\ 0 \\ 1 \end{pmatrix} \\
  = \frac{1}{2}\,
  \begin{pmatrix} x + 2y -z \\ 2y \\ -x + 2y +  z \end{pmatrix} \\
  \Rightarrow 
  A = \frac{1}{2}\,
  \begin{pmatrix}
    1 & 2 & -1 \\ 0 & 2 & 0 \\ -1 & 2 & 1
  \end{pmatrix}.
\end{multline*}

On peut vérifier que $A^2 = A$.
