\begin{tiny}(Ece02)\end{tiny} \emph{Cycloïde}\newline
\begin{figure}[ht]
 \centering
\input{Ece02_1.pdf_t}
\caption{Cycloïde}
\label{fig:Ece02_1}
\end{figure}

Dans $\C$ considéré comme un plan euclidien, un cercle $\mathcal C_2$ de rayon $r_2$ roule sans glisser sur un cercle fixe $\mathcal C_1$ de rayon $r_1$ et de centre $0$ (voir fig \ref{fig:Ece02_1}). Le point $z(\theta_1)$ est lié au cercle $\mathcal C_2$ comme indiqué sur la figure.
\begin{enumerate}
 \item Former la condition liant $r_1$, $r_2$, $\theta_1$, $\theta_2$ exprimant le roulement sans glissement.
\item Dans la suite, on notera $\theta$ au lieu de $\theta_1$. Montrer que :
\begin{displaymath}
 z(\theta)=
(r_1+r_2)e^{i\theta} - r_2 e^{i\frac{r_1+r_2}{r_2}\theta}
\end{displaymath}
\item Calculer $z'(\theta)$ sous forme trigonométrique. Il existe des points stationnaires, à quoi correspondent-ils ? En dehors de ces points, calculer la courbure en $z(\theta$).
\end{enumerate}
Le support de $z$ est une \emph{cycloïde}. On peut définir aussi des \emph{hypocycloïdes} et des \emph{épicycloïdes} en considérant le mouvement d'un point attaché à $\mathcal C_2$ mais à une distance du centre autre que $r_2$.
