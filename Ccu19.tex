\begin{tiny}(Ccu19)\end{tiny}
\begin{enumerate}
  \item Utilisons une décomposition idiote dans l'expression proposée
\begin{multline*}
  -n = \sum_{k=1}^nx^2_k -2\sum_{k=1}^nx_k 
  = \sum_{k=1}^n({x_k}^2 - 2x_k) \\
  = \sum_{k=1}^n\left( (x_k - 1)^2 - 1\right)
  = \left( \sum_{k=1}^n(x_k - 1)^2\right)  - n.
\end{multline*}

Ceci ne peut se produire que si $x_k = 1$ pour tous les $k$.
  \item L'inégalité de Cauchy-Schwarz appliquée aux $n$-uplets $(x_1,\cdots,x_n)$ et $(1,\cdots,1)$ conduit à
\begin{displaymath}
\left| x_1+\cdots+x_n\right|\leq 
\sqrt{x^2_1+\cdots+x^2_n} \sqrt{n} = n 
\end{displaymath}
On est donc dans le cas d'égalité et tous les $x_i$ sont égaux entre eux. La seule valeur possible est $1$. 
\end{enumerate}
