\begin{tiny}(Caz12)\end{tiny}
\begin{enumerate}
  \item On remarque que $n!+2$ est divisible par $2$, $n!+3$ est divisible par $3$, $\cdots$, $n!+n$ est divisible par $n$.\newline
De manière analogue, tout $x$ entre $2=p_1$ et $p_n$ admet un diviseur premier $p_i$ avec $i\leq n$. On en déduit que $q_n +x$ est divisible par $p_i$. 
  \item Un nombre premier n'est pas congru à $0$ mod $4$ car il serait divisible par $4$; ni congru à $2$ (sauf $2$) car il serait pair. On a donc 
\begin{displaymath}
  \mathcal{P}\setminus\left\lbrace 2 \right\rbrace = \mathcal{P}_4(1) \cup \mathcal{P}_4(3)
\end{displaymath}
Si $p$ est un diviseur premier de $m = 2\frac{q_n}{3} +3$ inférieur à $p_n$, il divise $3$. Le seul possible serait $3$ mais $3$ ne divise pas $m$. Tous les diviseurs premiers de $m$ sont donc strictement plus grands que $p_n$. Ils ne sont pas tous congrus à $1$ modulo $4$ car $2q_n +3 \equiv 3 \mod 4$ (car $p_1=2$). Il existe donc un nombre premier congru à $3$ modulo $4$ et plus grand que n'importe quel nombre premier.
  \item Un nombre premier autre que $2$ ou $3$ ne peut être congru modulo $6$ qu'à $1$ ou $5$ (sinon il serait divisible par $2$ ou $3$). Un diviseur premier de $m = \frac{q_n}{5}+5$ est strictement supérieur à $p_n$ sinon il diviserait $5$ et $5$ n'est pas un diviseur de $m$. De plus $m\equiv 5 \mod 6$ car $p_1=2$ et $p_2=3$ donc les diviseurs premiers de $m$ ne sont pas tous congrus à $1$ modulo $6$.   
\end{enumerate}
