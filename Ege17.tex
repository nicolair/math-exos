\begin{tiny}(Ege17)\end{tiny} Faisceau de plans.\\
Un repère étant fixé, on désigne par $x$, $y$, $z$ les fonctions coordonnées dans ce repère. On considère deux plans non parallèles $\mathcal{P}$ et $\mathcal{P}'$ d'intersection $\mathcal D$.
\begin{align*}
 &M\in \mathcal{P} \Leftrightarrow ax(M)+by(M)+cz(M)+d = 0 \\
 &M\in \mathcal{P}' \Leftrightarrow a'x(M)+b'y(M)+c'z(M)+d' = 0 
\end{align*}
On définit des \emph{fonctions} $\alpha$ et $\alpha'$ de l'espace et à valeurs réelles
\begin{align*}
 \alpha = ax+by+cz+d & & \alpha' = a'x+b'y+c'z+d' 
\end{align*}
Les plans $\mathcal{P}$ et $\mathcal{P}'$ sont donc les \emph{surfaces de niveau 0} des fonctions $\alpha$ et $\alpha'$.
\begin{enumerate}
 \item Soit $\lambda$ et $\mu$ deux réels (non tous nuls). Montrer que $\lambda \alpha(M)+\mu\alpha'(M)=0$ est l'équation d'un plan qui contient $\mathcal{D}$.
 \item Former l'équation du plan qui contient $\mathcal{D}$ et un point arbitraire $A\not\in \mathcal D$.
 \item Former un système d'équation de la droite $\mathcal D$ qui passe par le point de coordonnées $(2,3,-1)$ et dont un vecteur directeur a pour coordonnées $(-1,0,2)$. Former les équations des plans qui contiennent $\mathcal D$ et qui sont à la distance $1$ du point $B$ de coordonnées $(0,1,0)$.
 \item Former un système d'équations cartésiennes de la droite projection orthogonale de la droite
\begin{displaymath}
 (D) \left\lbrace 
\begin{aligned}
 &x+y+z-1 =0\\
 &x-y-2z = 0
\end{aligned}
\right. 
\end{displaymath}
sur le plan
\begin{displaymath}
 (P)\;x+2y+3z+6 = 0
\end{displaymath}

\end{enumerate}
