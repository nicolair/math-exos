\begin{tiny}(Cis28)\end{tiny}
\begin{enumerate}
  \item Avec des tableaux de variation, on montre que
\[
  \forall x\geq 0, \; x - x^2 \leq \sin x \leq x.
\]

  \item Introduisons des sommes de Riemann de fonctions dans $\mathcal{C}(\left[ 0,1 \right])$ avec la subdivision régulière $x_k = \frac{k}{n}$ pour $k$ entre $0$ et $n$.
\begin{multline*}
 a_n = \sum_{k=0}^{n} \frac{1}{k + n} = \sum_{k=0}^{n-1} \frac{1}{n}\frac{1}{\frac{k}{n} + 1} + \frac{1}{2n}\\
  = \sum_{k=0}^{n-1}(x_{k+1} - x_k)f(x_k) + \frac{1}{2n}\;
\text{ avec } f(x) = \frac{1}{x+1}. 
\end{multline*}

On en déduit $(a_n)_{n\in \N} \rightarrow \int_0^1 f(x)\,dx = \ln 2$.
\begin{multline*}
 b_n = \sum_{k=0}^{n} \frac{1}{(k + n)^2} = \sum_{k=0}^{n-1} \frac{1}{n^2}\frac{1}{(\frac{k}{n} + 1)^2} + \frac{1}{4n^2}\\
  = \frac{1}{n}\sum_{k=0}^{n-1}(x_{k+1} - x_k)g(x_k) + \frac{1}{4n^2}
\end{multline*}

avec $g(x) = \frac{1}{(x+1)^2}$. On en déduit 
\[
(b_n)_{n\in \N}\sim \frac{1}{n} \int_0^1 g(x)\,dx \rightarrow 0.  
\]

  \item Notons $s_n$ la dernière somme. En intégrant l'encadrement du a. puis en sommant:
\[
  a_n - b_n \leq s_n \leq a_n.
\]
Par le théorème d'encadrement, on conclut
\[
  (s_n)_{n \in \N} \rightarrow \ln 2.
\]

\end{enumerate}
