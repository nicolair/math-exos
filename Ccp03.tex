\begin{tiny}(Ccp03)\end{tiny} Ce qu'il est important de retenir dans cet exercice \emph{c'est que l'on a plus de prise sur un système de plusieurs équations polynomiales que sur une seule}. On peut transformer un système en un système équivalent avec une des deux équations de degré strictement plus petit.\newline
Si, pour $z$ réel, l'expression $z^{4}+6z^{2}-iz+3-i$ est nulle alors les parties réelles et imaginaires de cette expression sont nulles.
\begin{displaymath}
 \left\lbrace 
\begin{aligned}
z^{4}+6z^{2}+3 &= 0\\
-z -1 &= 0
\end{aligned}
\right. 
\end{displaymath}
La seule racine possible est donc $-1$ or $-1$ n'est pas une racine. L'équation n'admet donc pas de solution réelle.\newline
De même, si $z$ est une solution réelle de la deuxième équation,
\begin{displaymath}
 \left\lbrace 
\begin{aligned}
z^3+ 5z^2+7z+2 &= 0 \\
2z^2 +7z +6 &= 0
\end{aligned}
\right. 
\end{displaymath}
Pour pouvoir tirer parti de cette méthode dans un cas plus compliqué, il ne FAUT PAS résoudre la deuxième équation mais l'utiliser pour faire baisser le degré.
\begin{displaymath}
 \left\lbrace 
\begin{aligned}
(1)\\ (2)\end{aligned}
\right. 
\Leftrightarrow
 \left\lbrace 
\begin{aligned}
(1)-\frac{z}{2}\\ (2)\end{aligned}
\right. 
\Leftrightarrow
 \left\lbrace 
\begin{aligned}
\frac{3}{2}z^2+4z+2 &=0\\ 2z^2 +7z +6 &= 0\end{aligned}
\right. 
\end{displaymath}
 De même en remplaçant $(1)$ par $(1)-\frac{3}{4}(2)$, on obtient
\begin{displaymath}
 \left\lbrace 
\begin{aligned}
-\frac{5}{4}z-\frac{5}{2} &= 0 \\
2z^2 +7z +6 &= 0
\end{aligned}
\right. 
\end{displaymath}
Cette fois la seule racine possible $-2$ est racine des deux équations et l'équation admet bien une racine réelle $-2$.\newline
Pour la troisième équation, en formant les deux équations vérifiées par $t$ réel tel que $it$ soit racine, on trouve que les racines imaginaires pures sont $-2i$ et $2i$. 
