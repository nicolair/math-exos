\begin{tiny}(Ear22)\end{tiny} Racines primitives. Polynômes cyclotomiques.\newline
Soit $n\in \N^*$ et $w_n = e^{\frac{2i\pi}{n}}$. On appelle \emph{racine primitive} $n$-ième de l'unité tout élément du groupe multiplicatif $\U_n$ qui engendre $\U_n$. On note $\mathcal{D}(n)$ l'ensemble des diviseurs naturels de $n$. 
\begin{enumerate}
 \item Montrer que les racines primitives dans $\U_n$ sont les $\omega^k$ pour $k \in \llbracket 1,n \rrbracket$ et $k\wedge n = 1$. Combien existe-t-il de racines primitives ? Voir l'exercice sur \href{http://back.maquisdoc.net/data/exos_nicolair/_fex_az.pdf}{l'indicatrice d'Euler}.
 
 \item Soit $d \in \mathcal{D}(n)$ alors $\frac{n}{d}\in \mathcal{D}(n)$.\newline Quels sont les $k\in \llbracket 1,n \rrbracket$ tels que $k \wedge n = \frac{n}{d}$ ? Pour ces $k$, quel est l'ensemble des $w_n^k$?

 \item On note $\Phi_n$ (polynôme cyclotomique) le polynôme de $\C[X]$ dont les racines sont les racines primitives $n$-ièmes. Montrer que 
\[
 X^n - 1 = \prod_{d \in \mathcal{D}(n)} \Phi_d .
\]
 
 \item Calculer les $\Phi_n$ pour $n\in \llbracket 1,12\rrbracket$.
 
 \item Montrer que les polynômes cyclotomiques sont à coefficients dans $\Z$.
\end{enumerate}
