\begin{tiny}(Cpo19)\end{tiny} \begin{enumerate}
\item Comme le reste est de degré inférieur ou égal à 1, il est de la forme
\begin{displaymath}
R = uX + v
\end{displaymath}
avec $u$ et $v$ dans $K$. En substituant $\alpha$ et $\beta$ à $X$, on obtient un système de deux équations aux inconnues $\alpha$ et $\beta$ 
\begin{displaymath}
\left\lbrace 
\begin{aligned}
u\alpha + v &= \tilde{P}(\alpha) \\ u\beta + v &= \tilde{P}(\beta)
\end{aligned}
\right.
\end{displaymath}
que l'on résoud avec les formules de Cramer
\begin{displaymath}
 u= \frac{\tilde{P}(\alpha)- \tilde{P}(\beta)}{\alpha - \beta},\hspace{0.5cm}
 v= \frac{\alpha\tilde{P}(\beta)- \beta\tilde{P}(\alpha)}{\alpha - \beta}
\end{displaymath}
\item Comme $X^2+1=(X-i)(X+i)$, on peut appliquer la première question avec $\alpha=i$ et $\beta=-i$. Comme $P$ est à coefficients réels, on peut remarquer que $\tilde{P}(\beta)=\overline{\tilde{P}(\alpha)}$. On en déduit que $u$ est la partie imaginaire de $\tilde{P}(i)$ et $v$ sa partie réelle. En introduisant la fonction exponentielle, on obtient finalement que le reste demandé est
\begin{displaymath}
 X\sin S + \cos S \text{ avec } S=\sum_{k=1}^na_k
\end{displaymath}
\end{enumerate}