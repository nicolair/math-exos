Les sommets sont obtenus pour les valeurs $0$ et $\pi$ du paramètre. Ils ont pour abscisse (origine à un foyer)
\begin{displaymath}
 \frac{p}{1+e}\hspace{0.5cm} \frac{p}{e-1}
\end{displaymath}
Le centre est au milieu des deux. On en déduit
\begin{displaymath}
 c=\frac{p}{2}\left(\frac{1}{e+1}+\frac{1}{e-1} \right) = \frac{pe}{e^2-1}
\end{displaymath}
Pour une hyperbole, le sommet $S$ le plus proche de $F$ est entre $F$ et le centre $C$. On en déduit $FS + a =c$. On a $FS=\frac{p}{1+e}$ pour le sommet $S$ le plus proche de $F$. On en déduit
\begin{multline*}
 a=c-FS=\frac{p}{2}\left(-\frac{1}{e+1}+\frac{1}{e-1} \right) = \frac{p}{e^2-1}\\
\Rightarrow
b=\sqrt{c^2-a^2}=\frac{p}{\sqrt{e^2-1}}
\end{multline*}
