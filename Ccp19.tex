\begin{tiny}(Ccp19)\end{tiny}
Notons $x$ la partie réelle et $y$ la partie imaginaire de $z$.

Alignement de $1$, $z$, $z^3$.\newline
Les points sont deux à deux distincts si $z$ n'est pas dans $\left\lbrace 1, j, j^2, 0, -1 \right\rbrace$. Ils sont alors alignés si et seulement si
\begin{displaymath}
  \frac{z^3 - z}{z-1} = z(z+1)\in \R
\Leftrightarrow 2xy+y = 0
\end{displaymath}
Les points cherchés sont sur les droites d'équations $y=0$ et $x=\frac{1}{2}$. 

Cocyclicité de $1$, $z$, $z^{-1}$, $1-z$.\newline
D'après le \href{\baseurl C2002.pdf}{cours} la condition s'écrit:
\begin{displaymath}
  \frac{(z^{-1}-z)(1-z-1)}{(z^{-1}-1)(1-z-z)} \in \R
  \Leftrightarrow \frac{(z+1)z}{1-2z}\in \R
\end{displaymath}
