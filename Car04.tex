\begin{tiny}(Car04)\end{tiny} Soit $a_1, \cdots, a_p$ les racines distinctes de  $P$ et $\alpha_1, \cdots , \alpha_p$ leurs multiplicités. D'après la caractérisation de la multiplicté avec les dérivée, $\alpha_i$ est racine de $P'$ de multiplicité $\alpha_i - 1$. On en déduit que 
\[
 \deg(P \wedge P') = \sum_{i=1}^p(\alpha_i - 1) = \deg(P) - p.
\]
Soit $n$ le degré de $P$, si $\deg( P \wedge P') = 2$ alors $P$ admet $n-2$ racines distinctes. Les deux cas possibles sont
\begin{itemize}
 \item $n-3$ racines simples et une triple $a$,\\ $\deg( P \wedge P') = (X-a)^2$.
 \item $n-4$ racines simples et deux doubles $a \neq b$,\\ $\deg( P \wedge P') = (X-a)(X-b)$.
\end{itemize}
Par exemple, pour $P = X^4 + qX^2 + rX +s$, 
\[
 \deg(p \wedge P') = 2 \\
 \Leftrightarrow
 \left\lbrace 
 \begin{aligned}
  8r &= 4pq - p^3 \\ 64 s &= (4q - p^2)^2
 \end{aligned}
\right. 
\]

 
