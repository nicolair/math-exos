\begin{tiny}(Eee23)\end{tiny} Soit $E$ un espace euclidien, $(a,b)$ une famille libre de deux vecteurs, $V=\Vect(a,b)$. On note
\begin{displaymath}
\Phi:
\left\lbrace 
\begin{aligned}
&E\setminus\left\lbrace 0_E\right\rbrace  \rightarrow \R \\
&x \mapsto \frac{(x/a)(x/b)}{\left\|x\right\|^2}
\end{aligned}
\right. 
,\hspace{0.5cm}
  \mathcal{H} = \Phi(V \setminus\left\lbrace 0_E\right\rbrace) .
\end{displaymath}
Pour $\lambda \in \R$, on note aussi $x_\lambda = \lambda a  +b$.
\begin{enumerate}
  \item Préciser $K\in \R$ tel que
\[
 (x_\lambda /a)(x_\lambda /b) -(a/b)\left\|x_\lambda \right\|^2 = K \,\lambda .
\]

  \item Une fonction $\varphi$ est définie dans $\R$ par :
  \begin{displaymath}
    \lambda \mapsto \frac{\lambda}{\left\|\lambda a + b\right\|^2}\; \text{ avec } I = \varphi(\R).
  \end{displaymath}
Pourquoi $I$ est-il un intervalle? En étudiant $\varphi$, déterminer $I$.

  \item Déterminer $\mathcal{H}$ en utilisant $I$.
  
  \item \`A l'aide de la projection orthogonale sur $V$, montrer que $\Phi(x) \in \mathcal{H}$ pour tout $x$ non nul de $E$.\newline En déduire l'inégalité de Richard
\begin{multline*}
  \left( (a/b)-\Vert a\Vert \Vert b\Vert)\right) \Vert x \Vert^2
\leq 2(x/a)(x/b)  \\ \leq
  \left( (a/b)+\Vert a\Vert \Vert b\Vert)\right) \Vert x \Vert^2 .
\end{multline*}

\end{enumerate}

