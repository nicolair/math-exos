\begin{tiny}(Cgc10)\end{tiny} Comme $f$ est continue de $[0,1]$ dans $[0,1]$, elle admet un point fixe (exercice \ref{Egc17} (Egc17)) donc $E_f$ est non vide. Pour montrer que $E_f$ est un intervalle, nous allons montrer que c'est une partie convexe. C'est à dire que
\begin{displaymath}
\forall (a,b) \in E_f^2,\; a < b \Rightarrow [a,b] \subset E_f 
\end{displaymath}
Considérons donc un $c$ tel que $a<c<b$. Comme $a=f(a)$ et $b=f(b)$, le nombre $c$ est entre $f(a)$ et $f(b)$. D'après le théorème des valeurs intermédiaires, il existe $x\in [a,b]$ tel que $c = f(x)$. On compose par $f$:
\begin{displaymath}
 f(c) = f\circ f(x) = f(x) = c \Rightarrow c \in E_f
\end{displaymath}
On pouvait aussi démontrer que $E_f = f([0,1])$ par une double inclusion facile.