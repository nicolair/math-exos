\begin{tiny}(Eva05)\end{tiny} Loi géométrique tronquée.\newline
Soit $n$ un entier naturel. On procéde à $n$ (au plus) expériences de Bernoulli de paramètre $p$ indépendantes en s'arrêtant au premier succès. On note $X_n$ le rang du premier succès en convenant que $X_n$ prend la valeur $0$ si aucun succès n'est obtenu. Préciser la loi de $X_n$ et son espérance. \'Etudier la convergence des suites $\left( \p(X_n=k)\right) _{n\in \N}$ pour $k$ fixé dans $\N$ et $\left( E(X_n)\right) _{n\in \N}$.  