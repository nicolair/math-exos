\begin{tiny}(Csn13)\end{tiny} Calcul de sommes.\newline
Pour chaque cas, la convergence vient de ce que l'on peut exprimer la somme partielle comme une suite pour laquelle la limite résulte des propriétés usuelles sur les suites convergentes.

\begin{itemize}
  \item $\sum_{n=0}^{+\infty}\frac{n+1}{3^n}$
  \item $\sum_{n\geq 2} \frac{1}{n^3 - n}$
  \item Le terme général tend vers $0$ et la somme du terme d'indice $n$ (pair) et du suivant est nulle.
\[
 \sum_{n=2}^{+\infty}\ln\left( 1+\frac{(-1)^n}{n}\right) = 0.
\]

  \item Ici aussi, le terme général tend vers $0$. On peut factoriser les polynômes $(n+1)(n+2)$ et $n(n+3)$ et simplifier les logarithmes en dominos.
\[
\sum_{n\geq 1}\ln\frac{n^2+3n+2}{n^2+3n} = 2\ln 3 
\]

  \item Cette fois, il s'agit d'une combinaison de suites géométriques de raison $\frac{1}{16}$ (pairs) et $\frac{1}{4}$ (impairs):
\begin{multline*}
 1 + \frac{1}{2} + \frac{1}{4^2} + \frac{1}{2^3} + \frac{1}{4^4} + \cdots \\
 \Rightarrow 
\sum_{n\geq 0}\left( 3+(-1)^n\right)^{-n} = \frac{1}{1-\frac{1}{16}} + \frac{1}{2}\,\frac{1}{1-\frac{1}{4}}\\
= \frac{16}{15} + \frac{2}{3}.
\end{multline*}

  
  \item Cette fois, il s'agit plutôt de dominos multiplicatifs. Introduisons
\[
\left. 
\begin{aligned}
C_n &= \prod_{k=1}^{n}\cos \frac{x}{2^k} \\
\cos \frac{x}{2^k} & = \frac{\sin \frac{x}{2^{k-1}}}{2 \sin \frac{x}{2^{k}}}
\end{aligned}
\right\rbrace \Rightarrow
C_n = 2^{-n} \frac{\sin x}{\sin \frac{x}{2^{n}}}
\]
Pour $0< x < \frac{\pi}{2}$, tout est positif. Comme $\sin u \sim u$ en $0$. On obtient
\[
\sum_{n\geq 1}\ln(\cos\frac{x}{2^n}) = \ln \frac{\sin x}{x}. 
\]
 
\end{itemize}
