\begin{tiny}(Ccp24)\end{tiny} \label{Ccp24}
Calcul du discriminant
\begin{displaymath}
  4e^{2iu}-8ie^{iu}\sin u = 4e^{iu}\left( e^{iu}-2i\sin u\right)
  = 4e^{iu}e^{-iu} = 4
\end{displaymath}
On en déduit les racines:
\begin{displaymath}
  e^{iu} +1 = 2\cos \frac{u}{2}\,e^{i\frac{u}{2}},\hspace{0.5cm}
  e^{iu} -1 = 2i\sin \frac{u}{2}\,e^{i\frac{u}{2}}
\end{displaymath}
Pour la première racine, suivant le signe de $\cos \frac{u}{2}$, un argument est $\frac{u}{2}$ (cas positif) ou $\frac{u}{2} +\pi$ (cas négatif). Pour la seconde, suivant le signe de $\sin \frac{u}{2}$, un argument est $\frac{u}{2}+\frac{\pi}{2}$ (cas positif) ou $\frac{u}{2}-\frac{\pi}{2}$ (cas négatif).
