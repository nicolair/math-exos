\begin{tiny}(Ccp14)\end{tiny}
Un point clé de cet exercice est l'intervention d'expressions \emph{invariantes par permutation circulaire} des lettres $a$, $b$, $c$.
\begin{enumerate}
 \item Comme $u$ est une racine carrée de $j=j^4$, on doit avoir $u=j^2$ ou $-j^2$. \`A cause du signe des parties réelles et imaginaires, $u=-j^2$.\newline
Comme $-1$ est une racine cubique évidente de $-1$, d'après le cours, les racines cubiques de $-1$ sont $-1$, $-j=\overline{u}$ et $-j^2=u$. L'équation du second degré (inconnue $z$) dont les racines sont $u$ et $\overline{u}$ est
\begin{displaymath}
 z^2-z+1=0
\end{displaymath}
 
 \item L'expression est égale à $-1$ car les numérateurs et dénominateurs se simplifient au signe près. Cette relation traduit que la somme des angles du triangle est congrue à $\pi$ modulo $2\pi$.

 \item Lorsque le triangle est équilatéral direct, chacun des trois quotients est égal à $u$. Le fait que les modules des quotients soient $1$ traduit l'égalité des longueurs des côtés. La valeur $u$ traduit en plus que les trois angles sont $\frac{\pi}{3}$. Pour un triangle équilatéral indirect, la valeur commune est $\overline{u}$.

\item L'équivalence s'obtient simplement en développant
\begin{displaymath}
 (a-b)(a-c) = -(b-c)^2
\end{displaymath}
Comme l'expression obtenue est invariante par permutation circulaire des lettres $a$, $b$, $c$ :
\begin{displaymath}
 \frac{a-b}{c-b} = \frac{b-c}{a-c}
\Rightarrow
\frac{b-c}{a-c} = \frac{c-a}{b-a}
\end{displaymath}
ce qui entraine l'égalité des trois quotients.
\item En développant, on montre que
\begin{displaymath}
 \frac{c-a}{b-a}=u=-j^2
\Leftrightarrow
a+jb+j^2c=0
\end{displaymath}
Cette expression possède encore une propriété d'invariance par permutation circulaire. 
En multipliant par $j$ et $j^2$, on montre que 
\begin{displaymath}
 a+jb+j^2c=0 \Leftrightarrow b+jc+j^2a=0 \Leftrightarrow c+ja+j^2b=0
\end{displaymath}
La situation est analogue dans le cas indirect. On en tire l'équivalence entre les deux premières propriétés.\newline
L'égalité de la troisième condition traduit une égalité entre quotients. Mais d'après d., elle traduit l'égalité des \emph{trois} quotients entre eux. La valeur commune de ces quotients est alors une racine cubique de $-1$ d'après b. On a donc l'équivalence entre les propriétés 1 et 3.\newline
La propriété 4 est une simple reformulation de la propriété $3$.
\end{enumerate}
 