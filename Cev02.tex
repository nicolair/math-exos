\begin{tiny}(Cev02)\end{tiny} Soit $A$ et $B$ deux parties de $E$ (pas forcément ds sous-espaces).
\[
  A \cap B \subset A \subset \Vect(A) 
  \Rightarrow \Vect(A\cap B) \subset \Vect(A)
\]
car $\Vect(A)$ est un sous-espace. On raisonne de même pour $B$ donc 
\[
  \Vect(A\cap B) \subset \Vect(A) \cap \Vect(B).
\]
L'inclusion réciproque n'est pas vraie en général comme le montre la situation suivante.\newline
Soit $u\neq 0_E$, soit $a\notin \Vect(u)$, soit $b = \lambda a$ avec $\lambda \neq  1_K$. Notons $A=\left\lbrace u, a\right\rbrace$, $B=\left\lbrace u, b\right\rbrace$. Alors
\[
  u + b = u + \lambda a \in \Vect(A)\cap \Vect(B)
\]
mais $A\cap B= \left\lbrace u\right\rbrace$ donc $u + b \notin \Vect(A\cap B)$.\newline
De même pour l'union
\begin{multline*}
  \left.
  \begin{aligned}
    &A \subset \Vect(A) \subset \Vect(A) + \Vect(B)\\
    &B \subset \Vect(B) \subset \Vect(A) + \Vect(B)
  \end{aligned}
\right\rbrace \\
\Rightarrow \Vect(A \cup B) \subset \Vect(A) + \Vect(B).
\end{multline*}
Tout $x \in \Vect(A) + \Vect(B)$ est somme de deux vecteurs, combinaisons respectivement de vecteurs de $A$ et de $B$. Donc $x$ est combinaison de vecteurs de $A\cup B$ d'où
\[
  \Vect(A\cap B) = \Vect(A) + \Vect(B).
\]
