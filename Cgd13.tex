\begin{tiny}(Cgd13)\end{tiny}
\begin{enumerate}
  \item Pour un $x$ réel quelconque, on définit une suite par 
\begin{displaymath}
x_0 = x \;\text{ et }\; x_{n+1} = \varphi(x_n) 
\end{displaymath}
Alors
\begin{displaymath}
\forall n\in \N,\;  x_n = a^n(x-u) +u\; \text{ et }\; f(x_n)=f(x)
\end{displaymath}
Si $|a|<1$ la suite converge vers $u$ et $f(x)=f(u)$ car $f$ est continue en $u$.\newline
Si $|a|>1$, on remarque que l'invariance de $f$ est aussi valable pour la bijection réciproque ce qui permet de se ramener au premier cas.
  \item Invariance de la dérivée ou du taux.
  \item On se ramène au b. en considérant $f\circ f \circ f$. Le coefficient $a$ doit être strictement positif pour qu'il existe des solutions.
\end{enumerate}
