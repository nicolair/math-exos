 \documentclass[a4paper,twocolumn]{article}
\usepackage[hmargin={1.1cm,1.1cm},vmargin={2.2cm,2cm}]{geometry}
%       includehead,     scale=0.85,centering,hoffset=-0.1cm,voffset=-0.5cm]{geometry} headheight=13.1pt ,portrait

%\usepackage[a4paper,portrait,twocolumn,includeheadfoot,
%            scale=0.85,centering,hoffset=-1cm]{geometry}
\usepackage[pdftex]{graphicx,color}
\usepackage{amsmath}
\usepackage{amssymb}
\usepackage{stmaryrd}
\usepackage[french]{babel}
\selectlanguage{french}
\usepackage{fancyhdr}
\usepackage{floatflt}
\usepackage{ucs}
\usepackage[utf8]{inputenc}
\usepackage[T1]{fontenc}
\usepackage[pdftex,colorlinks={true},urlcolor={blue},pdfauthor={remy Nicolai}]{hyperref}
\usepackage{makeidx}


%Options de hyperref pour les fichiers pdf g{\'e}n{\'e}r{\'e}s
%\hypersetup{pdfpagemode=None,colorlinks=true,pdffitwindow=true}
%\hypersetup{pdfpagemode=None,colorlinks=true}


%                 Chargement des symboles de l'AMS
%\input amssym
%pour que la compilation aille au bout
%\nofiles\scrollmode

%pr{\'e}sentation du compteur de niveau 2 dans les listes
\makeatletter
\renewcommand{\labelenumii}{\theenumii.}
\makeatother

%dimension des pages, en-t{\^e}te et bas de page
  %utilisation avec vmargin
   %\setpapersize{custom}{21cm}{29.7cm}
   %\setmarginsrb{1.5cm}{0cm}{3.5cm}{1cm}{15mm}{10mm}{0mm}{0mm}
%\setlength{\voffset}{-2cm}
%\setlength{\oddsidemargin}{-1cm}
%\setlength{\textheight}{25cm}
%\setlength{\textwidth}{17.3cm}
%\columnsep=5pt
% \columnseprule=0.5pt
%\columnseprule=0.5pt
%En tete et pied de page
\pagestyle{fancy}
\lhead{Lycée Hoche MPSI B}
%\rhead{}
%\rhead{25/11/05}
\lfoot{\tiny{Cette création est mise à disposition selon le Contrat\\ Paternité-Partage des Conditions Initiales à l'Identique 2.0 France\\ disponible en ligne http://creativecommons.org/licenses/by-sa/2.0/fr/
} }
\rfoot{\tiny{Rémy Nicolai \jobname pdf du \today}}

%\pagestyle{fancy}
%\lhead{MPSI B}
%\rhead{\today}
%\rfoot{\small{\jobname}}
\newcommand{\baseurl}{http://back.maquisdoc.net/data/}
\newcommand{\textesurl}{http://back.maquisdoc.net/data/devoirs_nicolair/}
\newcommand{\exosurl}{http://back.maquisdoc.net/data/exos_nicolair/}
\newcommand{\coursurl}{http://back.maquisdoc.net/data/cours_nicolair/}

\newcommand{\N}{\mathbb{N}}
\newcommand{\Z}{\mathbb{Z}}
\newcommand{\C}{\mathbb{C}}
\newcommand{\R}{\mathbb{R}}
\newcommand{\K}{\mathbf{K}}
\newcommand{\Q}{\mathbb{Q}}
\newcommand{\F}{\mathbf{F}}
\newcommand{\U}{\mathbb{U}}
\newcommand{\p}{\mathbb{P}}


\newcommand{\card}{\mathop{\mathrm{Card}}}
\newcommand{\Id}{\mathop{\mathrm{Id}}}
\newcommand{\Ker}{\mathop{\mathrm{Ker}}}
\newcommand{\Vect}{\mathop{\mathrm{Vect}}}
\newcommand{\cotg}{\mathop{\mathrm{cotan}}}
\newcommand{\cotan}{\mathop{\mathrm{cotan}}}
\newcommand{\sh}{\mathop{\mathrm{sh}}}
\newcommand{\ch}{\mathop{\mathrm{ch}}}
\newcommand{\argch}{\mathop{\mathrm{argch}}}
\newcommand{\argsh}{\mathop{\mathrm{argsh}}}
\newcommand{\tr}{\mathop{\mathrm{tr}}}
\newcommand{\rg}{\mathop{\mathrm{rg}}}
\newcommand{\rang}{\mathop{\mathrm{rg}}}
\newcommand{\val}{\mathop{\mathrm{val}}}

\newcommand{\Mat}{\mathop{\mathrm{Mat}}}
\newcommand{\MatB}[2]{\mathop{\mathrm{Mat}}_{\mathcal{#1}}\left( #2\right) }
\newcommand{\MatBB}[3]{\mathop{\mathrm{Mat}}_{\mathcal{#1} \mathcal{#2}}\left( #3\right) }

\renewcommand{\Re}{\mathop{\mathrm{Re}}}
\newcommand{\Ima}{\mathop{\mathrm{Im}}}
\renewcommand{\Im}{\mathop{\mathrm{Im}}}
\renewcommand{\th}{\mathop{\mathrm{th}}}
\newcommand{\repere}{$(O,\overrightarrow{i},\overrightarrow{j},\overrightarrow{k})$ }
\newcommand{\trans}{\mathstrut^t\!}
\newcommand{\cov}{\mathop{\mathrm{Cov}}}
\newcommand{\orth}[1]{#1^{\perp}}

\newcommand{\absolue}[1]{\left| #1 \right|}
\newcommand{\fonc}[5]{#1 : \begin{cases}#2 &\rightarrow #3 \\ #4 &\mapsto #5 \end{cases}}
\newcommand{\depar}[2]{\dfrac{\partial #1}{\partial #2}}
\newcommand{\norme}[1]{\left\| #1 \right\|}
\newcommand{\se}{\geq}
\newcommand{\ie}{\leq}
\newcommand{\serie}[1]{\left( \sum {#1}_n \right)_{n\in\N}}

\batchmode

\begin{document}
\section*{Notations}
Soit $p$ et $q$ des naturels non nuls et $n = p + q$.\newline
On note $\mathfrak{S}_m$ le groupe des permutations de $\llbracket 1, m \rrbracket$ pour un $m\in \N^*$ quelconque. Comme $\mathfrak{S}_n$ revient très souvent, on le note seulement $\mathfrak{S}$.\newline
Soit $E$ et $F$ des $\R$-espaces vectoriels et $\mathcal{M}$ l'espace des applications multilinéaires de $E^n$ dans $F$.\newline
Pour tout $\sigma \in \mathfrak{S}_n$ et tout $\omega\in \mathcal{M}$, on définit $\sigma^* \omega$ par
\begin{multline*}
 \forall (x_1,\cdots,x_n)\in E^n, \\
 \sigma^*\omega(x_1,\cdots,x_n) = \omega(x_{\sigma(1)},\cdots,x_{\sigma(n)}).
\end{multline*}
On vérifie facilement:

\begin{multline} \label{sigma*}
 \forall (\sigma, \theta) \in \mathfrak{S}_{n}^2, \forall(\omega, \omega')\in \mathcal{M}^2, \forall \lambda \in \R,\\
 \sigma^*(\omega + \omega') = \sigma^*\omega + \sigma^*\omega', \;
 \sigma^*(\lambda \omega) = \lambda \sigma^* \omega, \\
 \theta^* \sigma^* \omega = (\theta \circ \sigma)^* \omega.
\end{multline} 

Une application multilinéaire $\omega \in \mathcal{M}$ est dite \emph{antisymétrique} si et seulement si
\[
 \forall \sigma \in \mathfrak{S}_n, \;
 \sigma^* \omega = \varepsilon(\sigma) \,\omega
\]
où $\varepsilon(\sigma)$ est la signature de $\sigma$.

\section*{Antisymétrisation}
Soit $\omega \in \mathcal{M}$, on définit $\omega^a$ par:
\[
 \omega^a = \sum_{\sigma \in \mathfrak{S}} \varepsilon(\sigma)\,\sigma^* \omega.
\]
On vérifie facilement que $\omega^a$ est antisymétrique avec les relations \ref{sigma*}.

\section*{Sous-groupe $G_p$}
On note $G_p$ le sous-groupe de $\mathfrak{S}$ stabilisant $\llbracket 1,p \rrbracket$.
\[
 \sigma \in G_p \Leftrightarrow \forall k \in \llbracket 1,p \rrbracket, \sigma(k) \in \llbracket 1,p \rrbracket.
\]
Comme il s'agit de bijections, les élements de $G_p$ stabilisent aussi $\llbracket p+1,n\rrbracket$ et $G_p$ est isomorphe à $\mathfrak{S}_p\times \mathfrak{S}_q$.  Le cardinal de $G_p$ est $p!\, q!$.\newline
Pour tout $\theta \in G_p$, on peut écrire la décomposition
\begin{equation}\label{theta1-2}
 \theta = \theta_1 \circ \theta_2 = \theta_2 \circ \theta_1 
\end{equation}
où $\theta_1$ fixe les éléments de$\llbracket p+1, n\rrbracket$ et permute ceux de $\llbracket 1,p\rrbracket$ et $\theta_2$ fixe les éléments de$\llbracket 1, p\rrbracket$ et permute ceux de $\llbracket p+1,n\rrbracket$.  
Considérons \emph{les classes à droite} modulo $G_p$:
\begin{multline*}
 \sigma' \sim \sigma \Leftrightarrow \sigma^{-1} \circ \sigma' \in G_p \\
 \Leftrightarrow \exists \theta \in G_p \text{ tq } \sigma' = \sigma \circ \theta \\
 \Leftrightarrow \sigma'(\llbracket 1,p \rrbracket) = \sigma(\llbracket 1,p \rrbracket).
\end{multline*}
Comme chaque classe contient $p!\,q!$ éléments, il existe $\binom{n}{p}$ classes distinctes. Elles sont clairement en relation avec les parties à $p$ éléments de $\llbracket 1,n \rrbracket$.\newline
Précisons cette relation. Pour chaque partie $A$ à $p$ éléments, on peut écrire de manière unique
\[
 \left\lbrace 
 \begin{aligned}
  A &= \left\lbrace i_1, \cdots, i_p\right\rbrace &\text{ avec } i_1 < \cdots < i_p \\
  \llbracket 1,n \rrbracket \setminus A &= \left\lbrace j_1, \cdots, j_q\right\rbrace &\text{ avec } j_1 < \cdots < j_q 
 \end{aligned}
 \right. .
\]
On définit une unique permutation $\sigma_A$ attachée à $A$ par:
\[
 \left\lbrace 
 \begin{aligned}
  \sigma_A(k) &= i_k \text{ pour } k\in \llbracket 1,p \rrbracket \\
  \sigma_A(p+k) &= j_k \text{ pour } k\in \llbracket 1,q \rrbracket \\
 \end{aligned}
\right. 
\]
on peut alors écrire une unique décomposition.\newline
Pour tout $\sigma \in \mathfrak{S}$, il existe une unique partie $A$ à $p$ éléments et une unique permutation $\theta \in G_p$ tels que
\[
 \sigma = \sigma_A \circ \theta.
\]

\section*{Antisymétrisations particulières.}
\subsection*{Applications $p,q$ antisymétriques}
On considère ici une application multilinéaire $\omega$ qui est antisymétrique partiellement par rapport aux $p$ premières places et aux $q$ dernières. Cela se traduit par:
\[
 \forall \theta \in G_p,\; \theta^* \omega = \varepsilon(\theta_1) \varepsilon(\theta_2)\,\omega = \varepsilon(\theta)\, \omega
\]
avec la décomposition \ref{theta1-2}.\newline
Que donne l'antisymétrisation d'une telle application ?\newline
Notons $\mathcal{P}_p$ l'ensemble des parties à $p$ éléments de $\llbracket 1,n\rrbracket$ et regroupons par classes:
\begin{multline*}
 \omega^a = \sum_{A \in \mathcal{P}_p} \varepsilon(\sigma_A)\,{\sigma_A}^*\left( \sum_{\theta \in G_p}\varepsilon(\theta)\,\underset{=\varepsilon(\theta)\,\omega}{\underbrace{\theta^* \omega}}\right) \\
 = p!\,q!\,\sum_{A \in \mathcal{P}_p} \varepsilon(\sigma_A)\,{\sigma_A}^*\omega.
\end{multline*}
On en déduit que 
\[
 \sum_{A \in \mathcal{P}_p} \varepsilon(\sigma_A)\,{\sigma_A}^*\omega
\]
est antisymétrique.

\subsection*{Application $p$-symétrique et $q$-antisymétrique}
On considère ici une application multilinéaire $\omega$ qui est symétrique partiellement par rapport aux $p$ premières places et antisymétrique par rapport aux $q$ dernières. Cela se traduit par:
\[
 \forall \theta \in G_p,\; \theta^* \omega = \varepsilon(\theta_2)\,\omega
\]
avec la décomposition \ref{theta1-2}.\newline
Que donne l'antisymétrisation d'une telle application ?
\begin{multline*}
 \omega^a = \sum_{A \in \mathcal{P}_p} \varepsilon(\sigma_A)\,{\sigma_A}^*\left( \sum_{\theta \in G_p}\varepsilon(\theta)\,\underset{=\varepsilon(\theta_2)\,\omega}{\underbrace{\theta^* \omega}}\right) \\
 = \underset{ = 0}{\underbrace{\left( \sum_{\theta_1 \in \mathfrak{S}_p} \varepsilon(\theta_1)\right)}}q!\,\sum_{A \in \mathcal{P}_p} \varepsilon(\sigma_A)\,{\sigma_A}^* \omega
 = 0.
\end{multline*}
La somme étant nulle car il existe autant de permutations indirectes que directes.
\end{document}
