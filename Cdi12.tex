\begin{tiny}(Cdi12)\end{tiny} Par définition de la composition:
\[
  g\circ f = 0_{\mathcal{L}(E,G)}
  \Leftrightarrow
  \Im f \subset \ker g .
\]
Donc $\ker g = \Im f \Rightarrow g\circ f = 0_{\mathcal{L}(E,G)}$. De plus, d'après le théorème du rang appliqué à $g$
\[
  \rg f + \rg g = \rg f + \dim F - \dim(\ker g)
  = \dim F
\]
car $\rg f = \dim(\ker g)$.\newline
Réciproquement, 
\[
  g\circ f = 0_{\mathcal{L}(E,G)}
  \rightarrow
  \Im f \subset \ker g .
\]
L'égalité des sous espaces vient de l'égalité des dimensions et du théorème du rang appliqué à $g$:
\begin{multline*}
  \rg f + \rg g = \dim F \\
  \Rightarrow \dim(\ker g) = \dim F - \rg g = \rg f.
\end{multline*}
