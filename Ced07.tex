\begin{tiny}(Ced07)\end{tiny} Analyse. Soit $f$ une fonction vérifiant la relation 
\[
  f'(x) + f(-x) = e^x
\]
de l'énoncé. \'Ecrivons la relation en $-x$ et la dérivée de la relation puis sommons:
\[
  \begin{aligned}
    f'(-x) + f(x) &= e^{-x} \\
    f''(x) - f'(-x) &= e^x \\ \hline
    f''(x) + f(x) &= e^{-x} + e^x
  \end{aligned}
\]
On en déduit que $f$ est solution de l'équation
\[
  (1)\; y'' + ' = 2\ch.
\]
Une fonction $z$ est solution de $(1)$ si et seulement si
\[
  \exists(\lambda, \mu)\in \R^2 \text{ tq }
  z = \ch + \lambda \cos + \mu \sin.
\]
Il existe donc $\lambda$ et $\mu$ réels tels que
\[
\begin{aligned}
f(x) &= \ch(x) + \lambda \cos(x) + \mu\sin(x) \\
f(-x) &= \ch(x) + \lambda \cos(x) - \mu\sin(x) \\
f'(x) &= \sh(x) + \mu\cos(x) - \lambda \sin(x)  \\ \hline 
e^x &= e^x + (\lambda + \mu)(\cos(x) - \sin(x))
\end{aligned}
\]
Comme ceci est vrai pour tous les $x$, on en déduit $\lambda + \mu = 0$.\newline
Synthèse.\newline
On peut vérifier que les fonctions
\[
  \ch + \lambda(\cos - \sin)
\]
sont solutions.
