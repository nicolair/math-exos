\begin{tiny}(Eis10)\end{tiny}
Pour $x\neq \pm 1$ réel, justifier la définition de $I(x)$ par
\begin{displaymath}
I(x) = \int_{0}^{\pi }\ln (x^{2}-2x\cos t+1)dt 
\end{displaymath}
\begin{enumerate}
 \item On pose $t_k=k\frac{2\pi}{2n}=k\frac{\pi}{n}$ pour $k$ entre $0$ et $n$. En remarquant que $e^{it_k}$ et $e^{-it_k}$ sont des racines $2n$-ièmes de l'unité donner une expression simple de
\begin{displaymath}
 \prod_{k=0}^{n-1}(x-e^{it_k})(x-e^{-it_k})
\end{displaymath}

 \item Calculer $I(x)$ à l'aide de somme de Riemann.
\end{enumerate}
