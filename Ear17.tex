\begin{tiny}(Ear17)\end{tiny} Transformation de Tschirnhaus.
\begin{enumerate}
  \item Soit $P\in \C[X]$ et $y \in \C$. On exécute formellement l'algorithme d'Euclide avec les entrées $P$ et $X^2-y$. Il renvoie un nombre complexe qui est une expression de $y$ désignée par $A(y)$.
Montrer que $A(y)=0$ si et seulement si $y$ est le carré d'une racine de $P$.\newline
Application. Soit $P = X^3+2X^2-X+3$.\newline
Former un polynôme $Q$ dont les racines sont les carrés de celles de $P$.
  \item Soit $A$, $B$ dans $\C[X]$, $y\in \C$ et $B_y = \widehat{B}(X-y)$.\newline
  Montrer que $A$ et $B_y$ ont une racine en commun si et seulement si $y$ est la somme d'une racine de $A$ et d'une racine de $B$.\newline
Application. Former le polynôme unitaire de degré $4$ dont les racines sont $\pm \sqrt{2} \pm \sqrt{3}$.
\end{enumerate}

