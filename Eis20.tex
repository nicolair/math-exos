\begin{tiny}(Eis20)\end{tiny} Pour $n\in \N$, on considère une fonction $k_n$ continue dans $[0,1]$, croissante, à valeurs positives telle que:
\begin{displaymath}
 \left( \int_{[0,1]}k_n\right)_{n\in \N} \rightarrow 0 
\end{displaymath}
et que, pour tout $a \in [0,1[$, la suite $\left( k_n(a)\right) _{n\in \N}$  soit négligeable devant
\begin{displaymath}
 \left( \int_{[0,1]}k_n\right)_{n\in \N}
\end{displaymath}
Pour $f\in\mathcal{C}([0,1])$, on pose $K_n(f) = \int_{[0,1]}fk_n$.
\begin{enumerate}
 \item Montrer que $k_n$ défini dans $[0,1]$ par $k_n(x)=x^n$ satisfait aux conditions.
 \item Montrer que $\left( K_n(f)\right) _{n\in \N}\rightarrow 0$.
 \item Montrer que si $f(1)=0$ alors $\left( K_n(f)\right) _{n\in \N}$ est négligeable devant $\left( \int_{[0,1]}k_n\right)_{n\in \N}$.
 \item Que dire de $\left( K_n(f)\right) _{n\in \N}$ dans le cas général où $f(1)\neq 0$ ?
\end{enumerate}
