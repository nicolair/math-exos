\begin{tiny}(Een01)\end{tiny} 
Soit $p$ et $n$ deux entiers ($1\leq p\leq n$). On définit les ensembles suivants :
\begin{itemize}
 \item $E = \llbracket 1,p \rrbracket$,  $F = \llbracket 1,n \rrbracket$,
 \item $\mathcal{C}_{p,n}\subset \mathcal{F}(E,F)$ : fonctions croissantes,
 \item $\mathcal{S}_{p,n}\subset \mathcal{F}(E,F)$ : fonctions strictement croissantes.
\end{itemize}
\begin{enumerate}
\item  Montrer que 
\begin{displaymath}
\card(\mathcal{S}_{p,n})= \binom{n}{p}.
\end{displaymath}
Montrer que le nombre d'{\'e}l{\'e}ments de $\mathcal{C}_{p,n}$ est le nombre de $n$-uplets 
\begin{displaymath}
 \left( x_{1},x_{2},\cdots ,x_{n}\right)\in\llbracket 0,p \rrbracket^n \text{ tq } x_{1}+x_{2}+\cdots +x_{n}=p.
\end{displaymath}
\item  \`A chaque {\'e}l{\'e}ment $f$ de $\mathcal{C}_{p,n}$, on associe une fonction $g$ d{\'e}finie par
\[
\forall x\in \llbracket 1, p\rrbracket ,\;g(x) = f(x) + x - 1 .
\]
Montrer que $g\in \mathcal{S}_{p,n+p-1}$. Montrer que $\mathcal{C}_{p,n}$ et $\mathcal{S}_{p,n+p-1}$ ont le m{\^e}me nombre d'{\'e}l{\'e}ments.
\end{enumerate}