\begin{tiny}(Emm20)\end{tiny} Pour une matrice $A\in \mathcal{M}_2(\K)$ fixée, on définit l'application
\begin{displaymath}
  \Phi:
\left\lbrace 
\begin{aligned}
\mathcal{M}_2(\K) &\rightarrow \mathcal{M}_2(\K) \\
M &\mapsto AM -MA
\end{aligned}
\right. 
\end{displaymath}
On note
\begin{multline*}
  A =
\begin{pmatrix}
a & b \\ c & d  
\end{pmatrix},\;
\Delta =
\begin{pmatrix}
0 & b \\ -c & 0  
\end{pmatrix},\\
T_s =
\begin{pmatrix}
-c & a-d \\ 0 & c  
\end{pmatrix},\;
T_i =
\begin{pmatrix}
b & 0 \\ d-a & -b  
\end{pmatrix} 
\end{multline*}
\begin{enumerate}
  \item Montrer que $\Vect(I_2,A)\subset \ker \Phi$.
  \item Montrer que $\Im \Phi = \Vect(\Delta, T_s, T_i)$.
  \item Montrer que $A\notin \Vect(I)$ entraîne $(\Delta, T_s, T_i)$ liée et préciser une relation linéaire.
  \item Montrer que
\begin{displaymath}
\rg(\Phi) = 
\left\lbrace 
\begin{aligned}
0 &\text{ si } A\in \Vect(I_2) \\ 2 &\text{ si } A\notin \Vect(I_2)  
\end{aligned}
\right. 
\end{displaymath}
En déduire que $B$ commute avec $A$ si et seulement si $(I_2,A,B)$ liée.
\end{enumerate}
