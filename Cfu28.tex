\begin{tiny}(Cfu28)\end{tiny} En utilisant la définition de $\sh$ et la propriété fondamentale de l'exponentielle puis en multipliant par $2$, on écrit
\begin{multline*}
 \sum_{k=0}^{n-1}\sh(x+k) = 0
\Leftrightarrow
\sum_{k=0}^{n-1}e^x\,e^k = \sum_{k=0}^{n-1}e^{-x}\,e^{-k} \\
\Leftrightarrow
e^{2x} = \frac{\sum_{k=0}^{n-1}e^{-k}}{\sum_{k=0}^{n-1}e^{k}}
= \frac{1+e^{-1}+\cdots+e^{-(n-1)}}{1+e^{1}+\cdots+e^{n-1}}\\
= \frac{e^{-(n-1)}\left( e^{n-1} + e^{n-2} + \cdots + 1\right) }{1+e^{1}+\cdots+e^{n-1}}
= e^{1-n}
\end{multline*}
Comme la fonction exponentielle réelle est bijective, il existe une unique solution:
\begin{displaymath}
  \frac{1-n}{2}
\end{displaymath}
