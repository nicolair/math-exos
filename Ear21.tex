\begin{tiny}(Ear21)\end{tiny} Soit $p$ un nombre premier, on note $v(x)$ la valuation $p$-adique d'un entier $x$. Soit $m\in\N^*$ et $q=p^m$. 
On définit\footnote{D'après \emph{THE BOOK} p 87} le polynôme $P$ par:
\begin{displaymath}
 P = \frac{1}{(q-2)!}(X-2)(X-3)\cdots (X-q+1)
\end{displaymath}
\begin{enumerate}
 \item Soit $a$ et $b$ dans $\Z$. Montrer que 
\begin{displaymath}
\left. 
\begin{aligned}
&a \equiv b \mod q \\ &a\not \equiv 0 \mod q  
\end{aligned}
\right\rbrace \Rightarrow v(a) = v(b)
\end{displaymath}
 \item Montrer que $v(x)v(x-1)=0$ pour $x\in \Z$.
 \item Quel est le degré de $P$ ? Montrer que $P(z)\in\Z$ pour tout $z\in\Z$.
 \item Montrer que, pour tout $z\in\Z\setminus \{2,3,\cdots,q-1\}$,
\begin{displaymath}
 P(z)\equiv 0 \mod p
\Leftrightarrow
\left\lbrace  
\begin{aligned}
z\not\equiv 0 \mod q \\ z\not\equiv 1 \mod q 
\end{aligned}
 \right. 
\end{displaymath}
\end{enumerate} 