\begin{tiny}(Ccu18)\end{tiny} Avec les quadruplets de réels $(a,b,c,d)$ et $(b,c,d,a)$, on utilise l'inégalité de Cauchy-Schwarz. On en déduit
\begin{displaymath}
 ab+bc+cd+da \leq a^2 + b^2 + c^2 +d^2
\end{displaymath}
car les deux sommes de carrés sont égales. Le cas d'égalité dans l'inégalité entraine l'existence d'un $\lambda$ tel que $b=\lambda a$, $c=\lambda b$, $d=\lambda c$, $a=\lambda d$. On en tire $\lambda^4 = 1$ donc $\lambda$ vaut $1$ ou $-1$.\newline
Le cas $-1$ conduisant à
\begin{displaymath}
 ab+bc+cd+da \leq - (a^2 + b^2 + c^2 +d^2)
\end{displaymath}
On en déduit le résultat demandé.