\begin{tiny}(Cev16)\end{tiny} On suppose $A\cap B = C\cap D$.\newline
Soit $x \in \left(A+(B\cap C) \right) \cap \left(A+(B\cap D) \right)$.
\begin{multline*}
  x \in A+(B\cap C) \\ \Rightarrow \exists a\in A \text{ et } u\in B\cap C \text{ tq } x = a+u.
\end{multline*}
\begin{multline*}
x \in A+(B\cap D) \\ \Rightarrow \exists a'\in A \text{ et } v\in B\cap D\text{ tq } x = a'+v . 
\end{multline*}
Alors:
\begin{multline*}
  \underset{\in A}{\underbrace{a-a'}} = \underset{\in B}{\underbrace{v-u}}
\Rightarrow v-u \in A\cap B = C \cap D \\
\Rightarrow v = \underset{\in C}{\underbrace{(v-u)}} + \underset{\in C}{\underbrace{u}} \in D \cap C = A\cap B \\
\Rightarrow v \in A \Rightarrow x = a' + v \in A
\end{multline*}
L'autre inclusion est immédiate car on peut écrire $a=a+0_E$ avec $0_E$ dans tous les sous-espaces vectoriels.
