Les formules sont rassemblées en plusieurs groupes selon l'idée principale de la démonstration.\\
\begin{figure}[h!t]
 \centering
 \input{./Cgp10_1.pdf_t}
 \caption{exercice \arabic{enumi} : expressions de $R$}
 \label{fig:Cgp10_1}
\end{figure}

Le premier groupe de formules vient du théorème de l'angle au centre dans le cercle circonscrit. La figure \ref{fig:Cgp10_1} permet d'écrire $R\sin \widehat C = \frac{c}{2}$. On obtient les autres relations par des considérations analogues. La dernière s'obtient en ajoutant les expressions obtenues pour $a$, $b$, $c$.

Pour la deuxième formule on utilise le déterminant pour exprimer l'aire et la première relation du premier groupe: 
\begin{displaymath}
 2S = \det(\overrightarrow{AB},\overrightarrow{AC}) = cb\sin \widehat A = \frac{abc}{2R}
\end{displaymath}

Les quatre formules suivantes viennent de décomposition d'aires à l'aide de triangles.
\begin{figure}[h!t]
 \centering
 \input{./Cgp10_2.pdf_t}
 \caption{exercice \arabic{enumi} : rayon du cercle inscrit}
 \label{fig:Cgp10_2}
\end{figure}
Pour le calcul du rayon $r$ du cercle inscrit, on découpe le triangle $(ABC)$ en trois triangles (figure \ref{fig:Cgp10_2}) dont on connait les bases ($a$, $b$, $c$) et hauteur ($r$). La formule se déduit de
\begin{displaymath}
 A = \frac{ar}{2}+\frac{br}{2}+\frac{cr}{2} 
\end{displaymath}
