\begin{tiny}(Cvs09)\end{tiny} Considérons l'ensemble indiqué par l'énoncé 
\begin{displaymath}
  A = \left\lbrace x \in E \text{ tq } x\notin f(x)\right\rbrace 
\end{displaymath}
On va montrer que $A$ n'est pas l'image par $f$ d'un élément de $E$.\newline
En effet, pour tout $a\in A$, $a\notin f(a)$ par définition de $A$. On en déduit que $f(a)\neq A$ car $a\in A$ et $a \notin f(A)$.\newline
D'autre part, pour tout $x\notin A$, on a $x\in f(x)$ par définition de $A$. Cette fois encore, $f(x)\neq A$ car $x\in f(x)$ et $x\notin A$. \newline
Il est donc impossible que $A$ soit une image par $f$, la fonction $f$ ne peut pas être surjective.

