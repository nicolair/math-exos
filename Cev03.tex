\begin{tiny}(Cev03)\end{tiny} Le principe de démonstration est le même dans les 3 cas. On considère une combinaison 
\[
  \lambda_1 f_1 + \cdots + \lambda_p f_p = \theta
\]
où les $\lambda_i$ sont des réels et $\theta$ désigne la fonction nulle définie sur la partie de $\R$ spécifique à chaque cas. Pour chaque $i$, on indique une propriété analytique locale de $\theta$ qui ne peut se produire que si $\lambda_i = 0$. Ceci assure que la famille est libre.
\begin{enumerate}
  \item Limite nulle en $t_i$.
  \item Dérivable en $t_i$.
  \item Continue en $t_i$.
\end{enumerate}
