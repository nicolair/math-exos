\begin{tiny}(Cap12)\end{tiny}
\begin{enumerate}
 \item Il s'agit d'un résultat du cours de deuxième année qui peut se traiter à l'aide du concept de \emph{suite de Cauchy} ou de la manière suivante.\newline
Notons 
\begin{align*}
a_n &=w_n^+=\max(w_n,0) \\ b_n &=w_n^-=\max(-w_n,0) 
\end{align*}
Il est important de remarquer que $w_n^+$ et $w_n^-$ sont à valeurs positives. On vérifie de plus, en considérant tous les cas possibles, que 
\begin{displaymath}
 w_n = a_n - b_n,\hspace{0.5cm} |w_n| = a_n + b_n
\end{displaymath}
Formons les sommes de termes consécutifs
\begin{multline*}
 W_n = \sum_{k=0}^nw_k, \; N_n = \sum_{k=0}^n|w_k|, \\ A_n = \sum_{k=0}^na_k, \;B_n = \sum_{k=0}^nb_k
\end{multline*}

Les trois dernières suites sont croissantes car elles sont formées avec des sommes de termes positifs. De plus, par linéarité,
\begin{displaymath}
 A_n + B_n = N_n \Rightarrow \left\lbrace 
\begin{aligned}
 A_n \leq N_n \\ B_n\leq N_n
\end{aligned}
\right. 
\end{displaymath}
Si la suite $\left( N_n\right) _{n\in \N}$ converge, elle est majorée donc $\left( A_n\right) _{n\in \N}$ et $\left( B_n\right) _{n\in \N}$ le sont aussi. Elles convergent car elles sont croissantes ce qui assure la convergence de $\left( W_n\right) _{n\in \N}$ car $W_n =A_n - B_n$.
 \item Par hypothèse, il existe $K>0$ tel que 
\begin{displaymath}
 0\leq |w_n| \leq \frac{K}{n^{1+\beta}}
\end{displaymath}
On somme les inégalités et on compare avec une intégrale
\begin{multline*}
 \sum_{k=2}^n|w_k|\leq K \sum_{k=2}^n\frac{1}{k^{1+\beta}}\leq K\int_{1}^{n}\frac{dx}{x^{1+\beta}} \\
= \left[-\frac{K}{\beta}x^{-\beta} \right]_{1}^{n} \leq  \frac{K}{\beta}
\end{multline*}

Ceci montre la convergence de $\left( \sum_{k=0}^n|w_k|\right) _{n\in \N}$. On en déduit, d'après la question a., la convergence demandée.
 \item En faisant le produit des deux développements limité
\begin{align*}
 \frac{u_{n+1}}{u_n} &= 1 -\frac{\alpha}{n} +O(\frac{1}{n^{1+\beta}}) \\
 \left(\frac{n+1}{n} \right)^\alpha &= 1 +\frac{\alpha}{n} +O(\frac{1}{n^{2}})
\end{align*}
on obtient
\begin{displaymath}
 \frac{v_{n+1}}{v_n} = 1 + O(\frac{1}{n^{1+\beta'}})
\end{displaymath}
avec $\beta'=\min(\beta,1)$ puis
\begin{displaymath}
 \ln\left( \frac{v_{n+1}}{v_n}\right)  = O(\frac{1}{n^{1+\beta'}})
\end{displaymath}
La question b. montre alors que $\ln(v_n)$ converge. Notons $u$ sa limite et $A=e^u$, on a alors
\begin{displaymath}
 v_n \rightarrow A \Rightarrow u_n \sim \frac{A}{n^\alpha}
\end{displaymath}

\end{enumerate}
