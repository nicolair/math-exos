\begin{tiny}(Cfu02)\end{tiny} Solutions des équations trigonométriques.
\begin{itemize}
  \item (1) (transformation d'une somme en produit)
\begin{displaymath}
 (1) \Leftrightarrow \sqrt{3}\cos(5x) = 2\cos(5x)\cos(7x) 
\end{displaymath}
\begin{multline*}
\cos(5x)=0\Leftrightarrow 5x\equiv \frac{\pi}{2}\mod \pi\\
\Leftrightarrow x\equiv \frac{\pi}{10}\mod \frac{\pi}{5}
\end{multline*}

\begin{multline*}
\cos(7x)=\frac{\sqrt{3}}{2}\Leftrightarrow
\left\lbrace 
\begin{aligned}
&7x\equiv \frac{\pi}{6}\mod 2\pi \\ &\text{ ou } \\ &7x\equiv -\frac{\pi}{6}\mod 2\pi   
\end{aligned}
\right. \\
\Leftrightarrow x\equiv \pm\frac{\pi}{42}\mod \frac{2\pi}{7}
\end{multline*}

  \item (2) (arc double)
\begin{multline*}
(2)\Leftrightarrow \frac{2\sin x \cos x}{\cos^2x - \sin^2 x}= 3\frac{\sin x}{\cos x} \\
\Leftrightarrow
\left\lbrace 
\begin{aligned}
&\sin x = 0 \\ &\text{ ou }\\
 &\frac{2\cos x}{\cos^2x - \sin^2 x}= 3\frac{1}{\cos x} \;(E)
\end{aligned}
\right. 
\end{multline*}

Avec
\begin{displaymath}
\sin x = 0 \Leftrightarrow x\equiv 0 \mod \pi  
\end{displaymath}
et
\begin{multline*}
(E) \Leftrightarrow 2\cos^2 x = 3(\cos^2x -\sin^2 x) \\
\Leftrightarrow \cos^2 x = 3\sin^2 x 
\Leftrightarrow x \equiv \pm \frac{\pi}{6} \mod \pi
\end{multline*} 

  \item (3) Désignons $\sin x$ par $s$ et $\cos x$ par $c$ et faisons baisser le degré en remarquant que
\begin{multline*}
1 =(c^2+s^2)^3 = c^6+3c^4s^2+3c^2s^4+s^6 \\
=s^6+c^6+3c^2s^2
\Rightarrow 
s^6+c^6 = 1-3s^2c^2
\end{multline*}

Alors:
\begin{multline*}
(3) \Leftrightarrow 4s^2c^2 =1 
\Leftrightarrow (\sin(2x))^2 = 1 \\
\Leftrightarrow 2x \equiv \frac{\pi}{2}\mod \pi 
\Leftrightarrow x \equiv \frac{\pi}{4}\mod \frac{\pi}{2}
\end{multline*}

  \item (4) On se ramène à des $\frac{x}{6}$ en utilisant
\begin{align*}
  &\sin 3y = 3\sin y \cos^2y - \sin^3 y \\
  &\cos 2y = \cos^2 y - \sin^2 y
\end{align*}
On note $s=\sin \frac{x}{6}$ et $c=\cos \frac{x}{6}$, puis on exprime en fonction de $s$ seulement
\begin{multline*}
(4)\Leftrightarrow  
2(c^2-s^2)-3sc^2 + s^3 = 2 \\
\Leftrightarrow 2(1-2s^2)-3s(1-s^2) + s^3 = 2 \\
\Leftrightarrow s(4s^2-4s-3)=0
\Leftrightarrow s(2s-3)(2s+1)=0
\end{multline*}

Comme $s$ est un sinus, $s\neq\frac{3}{2}$. Les solutions sont donc seulement obtenues pour $s=0$ ou $-\frac{1}{2}$ soit
les nombres congrus à $0$ modulo $\pi$ ou à $-\frac{\pi}{6}$ modulo $2\pi$ ou à $\frac{7\pi}{6}$ modulo $2\pi$.
\end{itemize}
