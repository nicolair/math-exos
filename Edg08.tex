\begin{tiny}(Edg08)\end{tiny}
Soit $f$ une fonction homogène de degré $k>$ et $\mathcal{C}^1$ dans le plan. Cela se traduit par :
\begin{displaymath}
 f\circ h_{O,\lambda}=\lambda^kf
\end{displaymath}
où $h_{O,\lambda}$ désigne l'homothétie de centre $O$ et de rapport $\lambda$.
\begin{enumerate}
 \item Montrer que $f(O)=0$.
 \item En appliquant l'opérateur $\dfrac{\partial}{\partial x}$ sur cette relation, montrer que $\dfrac{\partial f}{\partial x}$ et $\dfrac{\partial f}{\partial y}$ sont homogènes de degré $k-1$.
\item Pour un point $m$ fixé, en dérivant l'application
\[\lambda \rightarrow f\circ h_{o,\lambda}(m)\]
démontrer la \emph{formule d'Euler} 
\begin{displaymath}
 \dfrac{\partial f}{\partial x}(m)x(m) + \dfrac{\partial f}{\partial y}(m)y(m) = kf(m)
\end{displaymath}

\end{enumerate}