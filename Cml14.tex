\begin{tiny}(Cml14)\end{tiny} Dans le cas où $(x,y)$ est libre, considérons $x+y$ et exploitons la linéarité de $f$
\begin{multline*}
  f(x+y)=f(x) + f(y)\\
\Rightarrow \lambda(x+y)(x+y) = \lambda(x) x + \lambda(y) y\\
\Rightarrow \left( \lambda(x+y) -\lambda(x)\right) x + \left( \lambda(x+y) -\lambda(y)\right) y = 0_E \\
\Rightarrow \lambda(x) = \lambda(x+y) = \lambda(y)
\end{multline*}

car la famille $(x,y)$ est libre.\newline
Dans le cas où la famille est liée; comme $x \neq 0_E$, il existe $\mu \in \K$ tel que $y=\mu x$ avec $\mu \neq 0_\K$ car $y \neq 0_E$:
\begin{multline*}
  f(y) = \mu f(x)
\Rightarrow \lambda(y) y = \mu \lambda(x)x \\
\Rightarrow \lambda(y) \mu x = \mu \lambda(x) x
\Rightarrow \lambda(y) = \lambda(x)
\end{multline*}

car $\mu \neq 0_K$ et $x\neq 0_E$.
