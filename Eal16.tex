\begin{tiny}(Eal16)\end{tiny} Dérivations dans un anneau.\newline
Soit $A$ un anneau (pas forcément commutatif). On dira qu'une application $D$ de $A$ dans $A$ est une \emph{dérivation} lorsqu'elle vérifie :
\begin{displaymath}
\forall (a,b)\in A^2:
\left\lbrace
\begin{aligned}
 D(a+b) &= D(a)+D(b)\\
 D(ab) &= D(a)b+aD(b)
\end{aligned}
\right. 
\end{displaymath}
\begin{enumerate}
 \item Montrer que $D(1_A)=0_A$ et $D(0_A)=0_A$.
 \item Soit $D_1$ et $D_2$ deux dérivations, montrer que $$D_1\circ D_2 - D_2\circ D_1$$ est une dérivation.
 \item On suppose $A$ commutatif. Montrer que pour tout $a\in A$ et $n\in \N^*$
\begin{displaymath}
 D(a^n)=nD(a)a^{n-1}
\end{displaymath}
Si $a$ est inversible, montrer que la formule est valable pour $n\in \Z$.
\end{enumerate}

