\begin{tiny}(Cpo08)\end{tiny} Notons $a_1$, $a_2$, $a_3$ les racines:
\[
  P = X^3 + 2 X^2 + 3X + 4 = (X-a_1)(X-a_2)(X-a_3).
\]
Avec les notations usuelles des polynômes symétriques élémentaires:
\[
  S_2 = \sum_i a_i^2 = \sigma_1^2 - 2 \sigma_2.
\]
D'après les relations entre coefficients et racines:
\[
  \sigma_1 = -2, \; \sigma_2 = 3 \Rightarrow  S_2 = -2.
\]
Pour calculer $S_7 = \sum_i a_i^7$, on divise $X^7$ par $P$. Après calculs,
\[
  X^7 = Q\,P + R \text{ avec }
\left\lbrace
\begin{aligned}
  Q &= X^4 - 2X^3 + X^2 + 5\\
  R &= -14X^2 + 15X +20
\end{aligned}
\right.
\]
Comme $\widetilde{P}(a_i) = 0$,
\begin{multline*}
  a_i^7 = -14 a_i^2 + 15 a_1 + 20 \\
  \Rightarrow
  S_7 = -14 S_2 + 15 \sigma_1 + 30 = -2.
\end{multline*}
