\begin{tiny}(Cis16)\end{tiny} Avec l'indication de l'énoncé et par linéarité,
\begin{multline*}
 I_n = 
\int_{0}^{\pi}\frac{ \sin x}{ 1 + \frac{x}{n}}\, dx
= \int_{0}^{\pi}\sin x\, dx -\frac{1}{n}\int_{0}^{\pi}x\sin x\, dx\\
+ \frac{1}{n^2} \int_{0}^{\pi} \frac{x^2 \sin x}{1 + \frac{x}{n}}\, dx.
\end{multline*}
Comme $\int_0^\pi x\sin x\,dx$ est un nombre fixé, la suite $\left(\frac{1}{n} \int_0^\pi x\sin x\,dx\right)_{n\in \N}$ converge vers $0$. De plus,
\[
\forall x\in [0,\pi], \; 0 \leq \int_{0}^{\pi} \frac{x^2 \sin x}{1 + \frac{x}{n}}\, dx 
\leq \int_{0}^{\pi} x^2 \sin x\, dx 
\]
qui est un nombre fixé. On en déduit que
\[
 \left( I_n \right)_{n \in \N^*} \rightarrow \int_{0}^{\pi}\sin x\, dx = 2.
\]
