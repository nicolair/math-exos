\begin{tiny}(Eap09)\end{tiny} Soit $f$ une fonction $\mathcal C^5[-a,a]$. On définit $\varphi$ sur $[0,a]$ par :
\begin{displaymath}
 \varphi(x)=
f(x)-f(-x)-\frac{x}{3}\left(f'(-x)+4f'(0)+f'(x) \right) 
\end{displaymath}
\begin{enumerate}
 \item Calculer $\varphi^{(3)}(x)$. Le majorer par l'inégalité des accroissements finis.
 \item Montrer qu'il existe $c\in [-a,a]$ tel que
\begin{multline*}
f(a)-f(-a)
=\frac{a}{3}\left(f'(-a)+4f'(0)+f'(a)\right)\\ -\frac{1}{90}a^5f^{(5)}(c) 
\end{multline*}
On pourra s'inspirer de la méthode de majoration de l'erreur pour la méthode des rectangles présentée dans \href{http://back.maquisdoc.net/data/cours_nicolair/C2196.pdf}{Approximations d'intégrales}.
\end{enumerate}
