\begin{tiny}(Egd30)\end{tiny} Autour des dérivées successives de $\arctan$.
\begin{enumerate}
  \item On note $\theta(x) = \arctan x$.\newline
  Exprimer, $\cos \theta(x)$, $\sin \theta(x)$, $\theta'(x)$ en fonction de $x$ (sans $\theta$) et $\theta'(x)$ en fonction de $\cos \theta(x)$. Former des expressions trigonométriques des $\theta^{(k)}$ pour les premières valeurs de $k$. En déduire, pour $n\geq 1$,  
\begin{displaymath}
\arctan^{(n)}(x) = \frac{(n-1)!}{(1+x^2)^{\frac{n}{2}}}\sin\left(n( \frac{\pi}{2} + \arctan x ) \right) 
\end{displaymath}
\item Montrer que $\arctan^{(n)}(x)$ est de la forme
\begin{displaymath}
  \frac{P_n(x)}{(1+x^2)^n}
\end{displaymath}
où $P_n$ est une fonction polynomiale de degré $n-1$. Montrer, sans utiliser la première question, que $P_n$ admet $n-1$ racines réelles. 
\item Exprimer les racines de $P_n$ en utilisant a..
\end{enumerate}
