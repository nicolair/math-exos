\begin{tiny}(Cef12)\end{tiny}
\begin{enumerate}
 \item Supposons $E = U \cup V$ et $U \neq E$. Montrons que $U \subset V$.\newline
 Il existe un élement de $E$ qui n'est pas dans $U$, il est donc forcément dans $V$, notons le $v$ : $v\in V$ et $v\notin U$.\newline
 Pour tout $u\in \U$, considérons $u+v$. Il appartient à $U \cup V$ mais il ne peut pas appartenir à $U$ sinon $v$ appartiendrait à $U$. On conclut
\[
 u + v \in V \Rightarrow u \in V.
\]

 \item Comme $A$ et $B$ sont différents de $E$ (degré strictement plus petit), $A \cup B \neq E$. Il existe donc un vecteur $x$ qui n'est ni dans $A$ ni dans $B$. Il est donc non nul. La famille $(x)$ à un seul élément est libre et vérifie la condition imposée.
 
 \item Parmi les familles vérifiant les conditions de b, considérons en une maximale : $(x_1,\cdots,x_q)$. Une telle famille existe car les familles libres ont moins de $\dim E$ éléments. Notons $C = \Vect(x_1,\cdots,x_q)$.\newline
 Pour tout $x\notin C$, la famille $(x_1,\cdots,x_q,x)$ est libre. \`A cause de la maximalité,
 \begin{multline*}
\Vect(x_1,\cdots,x_q,x) \cap A \neq \left\lbrace 0_E \right\rbrace  \\
\text{ ou }
  \Vect(x_1,\cdots,x_q,x) \cap B \neq \left\lbrace 0_E \right\rbrace.  
 \end{multline*}
Remarquons que
\[
\left. 
\begin{aligned}
 \Vect(x_1,\cdots,x_q,x) \cap A &\neq \left\lbrace 0_E \right\rbrace\\ \Vect(x_1,\cdots,x_q) \cap A &= \left\lbrace 0_E \right\rbrace
\end{aligned}
\right\rbrace 
\Rightarrow
x \in C + A.
\]
En effet, si 
\[ 
\lambda_1 x_1 + \cdots + \lambda_q x_q + \lambda x = a\neq 0_E
\]
alors $\lambda \neq 0$ d'où $x \in A + C$.
On en déduit
\[
 E = (C+A) \cup (C+B).
\]
car lorsque $x\notin C$ l'une ou l'autre alternative se réalise (les deux si $x\in C$.\newline
On en déduit d'après a. que l'un des deux espaces est $E$ : par exemple $E = C + A$. La somme est alors directe car $A\cap C = \left\lbrace 0_E\right\rbrace$. Utilisons la dimension:
\begin{multline*}
 E = C \oplus A \Rightarrow \dim E = \dim C + \dim A \\
 = \dim C + \dim B \text{ car } \dim A = \dim B \\
 \Rightarrow E = C \oplus B \text{ car } C \cap B = \left\lbrace 0_E\right\rbrace.
\end{multline*}


\end{enumerate}
