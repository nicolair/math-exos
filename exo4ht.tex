% make4ht
%\documentclass[a4paper,twocolumn]{article}
\documentclass{article}
% make4ht
%\usepackage[hmargin={1.1cm,1.1cm},vmargin={2.2cm,2cm}]{geometry}
%       includehead,     scale=0.85,centering,hoffset=-0.1cm,voffset=-0.5cm]{geometry} headheight=13.1pt ,portrait

%\usepackage[a4paper,portrait,twocolumn,includeheadfoot,
%            scale=0.85,centering,hoffset=-1cm]{geometry}
% make4ht
%\usepackage[pdftex]{graphicx,color}
\usepackage{graphicx}
\usepackage{amsmath}
\usepackage{amssymb}
\usepackage{stmaryrd}
\usepackage[french]{babel}
\selectlanguage{french}
\usepackage{fancyhdr}
\usepackage{floatflt}
\usepackage{ucs}
\usepackage[utf8]{inputenc}
\usepackage[T1]{fontenc}
% make4ht
%\usepackage[pdftex,colorlinks={true},urlcolor={blue},pdfauthor={remy Nicolai}]{hyperref}
\usepackage{makeidx}


%Options de hyperref pour les fichiers pdf g{\'e}n{\'e}r{\'e}s
%\hypersetup{pdfpagemode=None,colorlinks=true,pdffitwindow=true}
%\hypersetup{pdfpagemode=None,colorlinks=true}


%                 Chargement des symboles de l'AMS
%\input amssym
%pour que la compilation aille au bout
%\nofiles\scrollmode

%pr{\'e}sentation du compteur de niveau 2 dans les listes
\makeatletter
\renewcommand{\labelenumii}{\theenumii.}
\makeatother

% indication de licence CC
%\usepackage{ hyperref, hyperxmp} % hyperref est déjà présent plus haut
%\usepackage[type={CC}, modifier={by-sa}, version={4.0}]{doclicense}


%dimension des pages, en-t{\^e}te et bas de page
  %utilisation avec vmargin
   %\setpapersize{custom}{21cm}{29.7cm}
   %\setmarginsrb{1.5cm}{0cm}{3.5cm}{1cm}{15mm}{10mm}{0mm}{0mm}
%\setlength{\voffset}{-2cm}
%\setlength{\oddsidemargin}{-1cm}
%\setlength{\textheight}{25cm}
%\setlength{\textwidth}{17.3cm}
%\columnsep=5pt
% \columnseprule=0.5pt
%\columnseprule=0.5pt
%En tete et pied de page
%\pagestyle{fancy}
%\lhead{MPSI maths exercices}
%\rhead{\tiny{\jobname pdf du \today}}
%\rhead{}
%\rhead{25/11/05}
%\lfoot{\tiny{Cette création est mise à disposition selon le Contrat\\ Paternité-Partage des Conditions Initiales à l'Identique 2.0 France\\ disponible en ligne http://creativecommons.org/licenses/by-sa/2.0/fr/
%} }

%\rfoot{\tiny{\doclicenseText}}

%\lfoot{\doclicenseImage[imagewidth=1.5cm]\tiny{ Rémy Nicolai pour maquisdoc}}

%\pagestyle{fancy}
%\lhead{MPSI B}
%\rhead{\today}
%\rfoot{\small{\jobname}}
\newcommand{\baseurl}{http://81.28.96.244/data/}
\newcommand{\textesurl}{http://81.28.96.244/data/devoirs_nicolair/}
\newcommand{\exosurl}{http://81.28.96.244/data/exos_nicolair/}
\newcommand{\coursurl}{http://81.28.96.244/data/cours_nicolair/}

\newcommand{\N}{\mathbb{N}}
\newcommand{\Z}{\mathbb{Z}}
\newcommand{\C}{\mathbb{C}}
\newcommand{\R}{\mathbb{R}}
\newcommand{\K}{\mathbf{K}}
\newcommand{\Q}{\mathbb{Q}}
\newcommand{\F}{\mathbf{F}}
\newcommand{\U}{\mathbb{U}}
\newcommand{\p}{\mathbb{P}}


\newcommand{\card}{\mathop{\mathrm{Card}}}
\newcommand{\Id}{\mathop{\mathrm{Id}}}
\newcommand{\Ker}{\mathop{\mathrm{Ker}}}
\newcommand{\Vect}{\mathop{\mathrm{Vect}}}
\newcommand{\cotg}{\mathop{\mathrm{cotan}}}
\newcommand{\cotan}{\mathop{\mathrm{cotan}}}
\newcommand{\sh}{\mathop{\mathrm{sh}}}
\newcommand{\ch}{\mathop{\mathrm{ch}}}
\newcommand{\argch}{\mathop{\mathrm{argch}}}
\newcommand{\argsh}{\mathop{\mathrm{argsh}}}
\newcommand{\tr}{\mathop{\mathrm{tr}}}
\newcommand{\rg}{\mathop{\mathrm{rg}}}
\newcommand{\rang}{\mathop{\mathrm{rg}}}
\newcommand{\val}{\mathop{\mathrm{val}}}

\newcommand{\Mat}{\mathop{\mathrm{Mat}}}
\newcommand{\MatB}[2]{\mathop{\mathrm{Mat}}_{\mathcal{#1}}\left( #2\right) }
\newcommand{\MatBB}[3]{\mathop{\mathrm{Mat}}_{\mathcal{#1} \mathcal{#2}}\left( #3\right) }

\renewcommand{\Re}{\mathop{\mathrm{Re}}}
\newcommand{\Ima}{\mathop{\mathrm{Im}}}
\renewcommand{\Im}{\mathop{\mathrm{Im}}}
\renewcommand{\th}{\mathop{\mathrm{th}}}
\newcommand{\repere}{$(O,\overrightarrow{i},\overrightarrow{j},\overrightarrow{k})$ }
\newcommand{\trans}{\mathstrut^t\!}
\newcommand{\cov}{\mathop{\mathrm{Cov}}}
\newcommand{\orth}[1]{#1^{\perp}}

\newcommand{\absolue}[1]{\left| #1 \right|}
\newcommand{\fonc}[5]{#1 : \begin{cases}#2 &\rightarrow #3 \\ #4 &\mapsto #5 \end{cases}}
\newcommand{\depar}[2]{\dfrac{\partial #1}{\partial #2}}
\newcommand{\norme}[1]{\left\| #1 \right\|}
\newcommand{\se}{\geq}
\newcommand{\ie}{\leq}
\newcommand{\serie}[1]{\left( \sum {#1}_n \right)_{n\in\N}}

\newcommand{\href}[2]{#2 (#1)}

%\batchmode
