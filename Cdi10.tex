\begin{tiny}(Cdi10)\end{tiny} On considère, dans $\R_3[X]$, les trois polynômes de Lagrange 
\begin{multline*}
 L_a = \frac{(X-b)(X-c)}{(a-b)(a-c)},\;
 L_b = \frac{(X-a)(X-c)}{(b-a)(b-c)}, \\
 L_c = \frac{(X-a)(X-b)}{(c-a)(c-B)} 
\end{multline*}

et une combinaison nulle des trois formes linéaires.
\begin{displaymath}
 \lambda_1\phi_1 + \lambda_2 \phi_2 + \lambda_3 \phi_3
\end{displaymath}
En prenant la valeur de cette fonction en $L_a$, $L_b$, $L_c$, on obtient
$\lambda_1 = \lambda_2 = \lambda_3$.\newline
Notons $M=(X-a)(X-b)(X-c)$. La famille $(L_a,L_b,L_c)$ est libre (prendre les valeurs en $a$, $b$, $c$). Le polynôme $M$ n'est pas combinaison des $L_a, L_b, L_c$ (considérer le degré). La famille $(L_a,L_b,L_c,M)$ est donc libre dans un espace de dimension $4$, c'est une base.\newline
Comme $(\phi_1,\phi_2,\phi_3)$ est libre:
\begin{displaymath}
  (\phi_1,\phi_2,\phi_3,\phi)\text{ liée } \Leftrightarrow \phi\in \Vect(\phi_1,\phi_2,\phi_3)
\end{displaymath}
Définissons une forme $\varphi$:
\begin{displaymath}
 \varphi = \phi(L_a)\phi_1 + \phi(L_b)\phi_2 + \phi(L_c)\phi_3 
\end{displaymath}
Par construction: $L_a$, $L_b$, $L_c$ sont dans $\ker(\phi - \varphi)$. On en déduit (prolongement linéaire) :
\begin{displaymath}
 \varphi = \phi \Leftrightarrow \varphi(M)=\phi(M) \Leftrightarrow \phi(M) = 0
\end{displaymath}
Avec les notations préconisées par l'énoncé,
\begin{displaymath}
 a=m-l,\hspace{0.5cm} b=m+l,\hspace{0.5cm} c = m + (c-m)
\end{displaymath}
d'où, après développement,
\begin{displaymath}
 M = (X-m)^3 -l^2(X-m) + l^2(c-m)
\end{displaymath}
Les puissances impaires disparaissent par symétrie
\begin{displaymath}
 \psi(M) = l^2(c-m)(b-a)
\end{displaymath}

