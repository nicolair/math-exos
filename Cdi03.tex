\begin{tiny}(Cdi03)\end{tiny} Notons $r_A$ l'application \og restriction à $A$\fg.
\begin{enumerate}
  \item La linéarité de $r_A$ vient des propriétés des opérations fonctionnelles. Montrons qu'elle est surjective.\newline
  Soit $g \in \mathcal{L}(A,E)$ et $(a_1,\cdots,a_\alpha)$ une base de $A$. On complète en une base $(a_1,\cdots,a_n)$ de $E$ et on définit $f$ par prolongement linéaire avec
\[
  f(a_i) = 
  \left\lbrace
  \begin{aligned}
    g(a_i) &\text{ si } i\in \llbracket 1, \alpha \rrbracket \\
    0  &\text{ si } i\in \llbracket \alpha +1, n\rrbracket
  \end{aligned}
  \right. .
\]
Alors $r_A(f) = g$.
  \item Avec  $L = \left\lbrace f\in \mathcal{L}(E) \text{ tq } A \subset \ker f\right\rbrace = \ker r_A$ et les dimensions des espaces d'applications linéaires, le théorème du rang appliqué à $r_A$ conduit à
\[
  \dim L = (\dim E - \dim A) \dim E.
\]

\end{enumerate}
