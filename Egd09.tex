\begin{tiny}(Egd09)\end{tiny}
On définit une fonction $f$ dans $[0,+\infty[$ par :
\begin{displaymath}
 f(x) = \left\lbrace \begin{aligned} 
0 &\text{ si } x=0 \\
e^{-\frac{1}{x}} &\text{ si } x>0
\end{aligned}
\right. 
\end{displaymath}
\begin{enumerate}
 \item Montrer que $f$ est continue dans $[0,+\infty[$ et de classe $\mathcal C^\infty$ dans $]0,+\infty[$.
\item Calculer $f'(x)$, $f''(x)$, $f^{(3)}(x)$ pour tout $x>0$. Montrer que, pour tout entier $n\geq1$, il existe une fonction $P_n$ polynomiale tel que 
\begin{displaymath}
 f^{(n)}(x) = \dfrac{1}{x^{n+1}}P_n(\dfrac{1}{x})e^{-\frac{1}{x}}
\end{displaymath}
Préciser $P_1$, $P_2$, $P_3$. Former une relation liant $P_n$, $P'_n$ et $P_{n+1}$. En déduire que $P_n$ est une fonction polynomiale. Préciser son degré, la valeur $P_n(0)$ et le coefficient du terme de plus haut degré.
\item Former un équivalent de $f^{(n)}$ strictement à droite de $0$ et montrer que $f\in \mathcal C^\infty([0,+\infty[)$ et que $f^{(n)}(0)=0$ pour tous les entiers $n$.
\item Montrer que $P_n$ admet $n-1$ racines strictement positives distinctes.
\end{enumerate}
