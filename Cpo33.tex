\begin{tiny}(Cpo33)\end{tiny}
\begin{enumerate}
 \item Formules faciles en considérant le coefficient du terme de plus bas degré.
 \item Pour l'unicité, on raisonne comme pour la division euclidienne en faisant jouer à la valuation le rôle du degré. Pour l'existence, on raisonne par récurrence sur $n$. Notons
\begin{displaymath}
 a= c_0(A)\neq 0,\hspace{0.5cm} b = c_0(B)\neq 0
\end{displaymath}
Pour $n=0$, $Q_0=\frac{b}{a}$ convient.\newline
S'il existe $Q_n$ et $R_n$ vérifiant les conditions, on peut écrire 
\begin{displaymath}
 R_n=X^{n+1}R\text{ avec } r=c_0(R)
\end{displaymath}
 On vérifie alors que l'on peut prendre
\begin{displaymath}
 Q_{n+1}= Q_n+\frac{r}{a}X^{n+1}
\end{displaymath}
\item On trouve
\begin{align*}
 Q_3 &= 1+X+X^2-X^3\\
 R_3 &= X^4(1+X-X^2)
\end{align*}
\end{enumerate}
