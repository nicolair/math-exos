\begin{tiny}(Esc18)\end{tiny} Une suite $\left( x_n\right) _{n\in \N}$ de réels est dite \emph{de Cauchy}\index{suite de Cauchy} si et seulement si, pour tout $\varepsilon> 0$, il existe un entier $N_\varepsilon$ tel que:
\begin{displaymath}
\forall(p,q)\in \N^2:
\left. 
\begin{aligned}
p>N_\varepsilon \\ q>N_\varepsilon 
\end{aligned}
 \right\rbrace \Rightarrow |x_p -x_q|<\varepsilon
\end{displaymath}
\begin{enumerate}
 \item Montrer que toute suite convergente est de Cauchy. 
 \item Montrer que toute suite de Cauchy est bornée.
 \item En utilisant le théorème de Bolzano-Weirstrass, montrer que toute suite de Cauchy est convergente.
\end{enumerate}
