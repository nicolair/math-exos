\begin{tiny}(Cli04)\end{tiny}
\begin{enumerate}
  \item à compléter
  \item à compléter
  \item En fait $u$ est le point fixe de $\varphi$:
\begin{displaymath}
  \varphi(u) = u \Leftrightarrow au+b = u \Leftrightarrow u = \frac{b}{1-a}
\end{displaymath}
Il conduit à une expression commode
\begin{displaymath}
\forall x\in \R,\; \varphi(x) = a(x-u) + u  
\end{displaymath}
Pour tout $n\in \N$ et $x\in \R$ on en déduit
\begin{displaymath}
  f(x) = f(a^n(x-u) + u)
\end{displaymath}
Si $|a|<1$, la suite $\left( a^n(x-u) + u\right)_{n\in \N}$ converge vers $u$. Par continuité, la limite de cette suite constante est $f(x) = f(u)$.
La fonction $f$ est donc constante. \newline
Si $|a|>1$, on raisonne de même avec $\varphi^{-1}$. En effet, $\varphi$ est bijective avec
\begin{displaymath}
\forall x\in \R,\; \varphi^{-1}(x) = \frac{1}{a}(x-u) + u 
\end{displaymath}
et $f\circ \varphi^{-1} =f$ avec $\left| \frac{1}{a}\right| <1$.
\item à compléter
\end{enumerate}
