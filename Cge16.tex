Supposons que $M$ vérifie les trois relations demandée:
\begin{displaymath}
 \left\lbrace 
\begin{aligned}
 (\overrightarrow{AM}/\overrightarrow{BM})=0 \\
 (\overrightarrow{BM}/\overrightarrow{CM})=0 \\
 (\overrightarrow{CM}/\overrightarrow{AM})=0 
\end{aligned}
\right. 
\end{displaymath}
En formant les différences, on obtient
\begin{displaymath}
 \left\lbrace 
\begin{aligned}
 (\overrightarrow{AC}/\overrightarrow{BM})=0 \\
 (\overrightarrow{BA}/\overrightarrow{CM})=0 \\
 (\overrightarrow{CB}/\overrightarrow{AM})=0 
\end{aligned}
\right. 
\end{displaymath}
La première égalité traduit que $M$ est dans le plan orthogonal à $\overrightarrow{AC}$ qui passe par $B$. Ce plan est orthogonal au plan du triangle et contient la hauteur. On peut interpréter de la même manière les autres équations. On en déduit que $M$ est sur la droite perpendiculaire au plan du triangle et qui perce ce plan en l'orthocentre. Il reste à vérifier que sur cette droite deux points conviennent.