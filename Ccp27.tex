\begin{tiny}(Ccp27)\end{tiny} On écrit le numérateur à l'aide d'une exponentielle. Pour le dénominateur, on utilise d'abord
\begin{displaymath}
 \sin(2a) = \cos\left(\frac{\pi}{2}-2a \right)
= \cos 2 \left(\frac{\pi}{4}-a \right)
\end{displaymath}
d'où on tire
\begin{displaymath}
 \begin{aligned}
  1+\sin(2a) &= 2\cos^2\left( \frac{\pi}{4}-a\right) \\
  1-\sin(2a) &= 2\sin^2\left( \frac{\pi}{4}-a\right)
 \end{aligned}
\end{displaymath}
La condition sur $a$ entraine que 
\begin{displaymath}
 0<\frac{\pi}{4}-a<\frac{\pi}{2}
\end{displaymath}
On peut donc supprimer les racines carrées et l'expression exponentielle demandée est :
\begin{displaymath}
 \sqrt{2}\cos(a)e^{-i\left( \frac{a}{2}+\frac{\pi}{4}\right) }
\end{displaymath}
