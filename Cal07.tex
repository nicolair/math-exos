\begin{tiny}(Cal07)\end{tiny} Théorème de Lagrange.
\begin{enumerate}
  \item Si $Hg = Hg'$ alors $g=eg\in Hg=Hg'$ donc il existe $h\in H$ tel que
\begin{displaymath}
  g = h g' \Rightarrow gg'^{-1} \in H
\end{displaymath}
Réciproquement, supposons $gg'^{-1} \in H$.
\begin{displaymath}
  \forall x \in Hg,\; \exists h\in H\text{ tq } x = hg = \underset{\in H}{\underbrace{h(gg'^{-1})}}\,g' \in Hg'
\end{displaymath}
On en déduit $Hg \subset Hg'$.\newline
On obtient l'autre inclusion en échangeant les rôles de $g$ et $g'$. On peut le faire car 
\begin{displaymath}
  g'g^{-1} = (gg'^{-1})^{-1} \in H
\end{displaymath}

  \item Réflexivité: un sous-groupe contient le neutre
\begin{displaymath}
gg^{-1}=e\in H  \Rightarrow g \mathcal{R}_H g
\end{displaymath}
Symétrie: par stabilité de $H$ pour l'inversion:
\begin{multline*}
g \mathcal{R}_H g' \Rightarrow gg'^{-1} \in H \Rightarrow (gg'^{-1})^{-1} \in H\\
\Rightarrow g'g^{-1} \in H \Rightarrow g' \mathcal{R}_H g
\end{multline*}

Transitivité: stabilité de $H$ pour l'opération.
\begin{multline*}
\left. 
\begin{aligned}
g &\mathcal{R}_H g' \\ g' &\mathcal{R}_H g''  
\end{aligned}
\right\rbrace \Rightarrow
\left. 
\begin{aligned}
gg'^{-1} &\in H \\ g'g''^{-1} &\in H  
\end{aligned}
\right\rbrace \\
\Rightarrow
gg''^{-1} = gg'^{-1}  g'g''^{-1} \in H 
\Rightarrow g \mathcal{R}_H g''
\end{multline*}

  \item Comme $\mathcal{R}_H$ est une relation d'équivalence, ses classes forment une partition de $G$. D'après la quetion a., chacune de ses classes est de la forme $Hg$ et contient donc $\card H$ éléments. Toutes les classes ont le même nombre d'éléments $\card H$.
\begin{center}
$\card G$ = Nb classes $\times$ Nb elts dans une classe  
\end{center}
Si on note $p$ le nombre de classes, on obtient
\begin{displaymath}
  \card G = p\times \card H
\end{displaymath}
qui traduit que $\card H$ divise $\card G$.
  \item L'ensemble
\begin{displaymath}
  N = \left\lbrace k\in \Z \text{ tq } g^k = e\right\rbrace 
\end{displaymath}
est un sous groupe de $(\Z, +)$. Il existe donc (cours) un unique $m_g \in \N$ tel que 
\begin{displaymath}
  N = m_g \Z
\end{displaymath}

  \item On vérifie que le sous-groupe de $G$ engendré par $g$ est 
\begin{displaymath}
  \left\lbrace e, g, \cdots , g^{m_g -1}\right\rbrace 
\end{displaymath}
Il est donc de cardinal $m_g$. D'après la question c., il existe un entier $p$ tel que 
\begin{displaymath}
\card G = p\,m_g \Rightarrow g^{\card G} = (g^{m_g})^{p} = e^p = e  
\end{displaymath}

\end{enumerate}

