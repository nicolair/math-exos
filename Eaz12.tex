\begin{tiny}(Eaz12)\end{tiny} Pour $n$ naturel non nul, soit $p_1, \cdots, p_n$ les $n$ premiers nombres premiers. On note aussi
\begin{displaymath}
  q_n = p_1p_2 \cdots p_n=2\times 3 \times 5 \times \cdots \times p_n
\end{displaymath}
On note $\mathcal{P}$ l'ensemble des nombres premiers et, pour $0<a<b$, on note $\mathcal{P}_a(b)$ l'ensemble des nombres premiers congrus à $b$ modulo $a$.
\begin{enumerate}
  \item Montrer que $\llbracket n! + 2, n! + n \rrbracket$ et $\llbracket q_n + 2 , q_n + p_n \rrbracket$ ne contiennent aucun nombre premier.
  \item Montrer que $\mathcal{P}\setminus\left\lbrace 2 \right\rbrace = \mathcal{P}_4(1) \cup \mathcal{P}_4(3)$. En considérant $2\frac{q_n}{3}+3$, montrer que $\mathcal{P}_4(3)$ est infini.
  \item Montrer que $\mathcal{P}\setminus\left\lbrace  2, 3\right\rbrace = \mathcal{P}_6(1) \cup \mathcal{P}_6(5)$. En considérant $\frac{q_n}{5}+5$, montrer que $\mathcal{P}_6(5)$ est infini.
\end{enumerate}
