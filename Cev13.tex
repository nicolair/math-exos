\begin{tiny}(Cev13)\end{tiny}
Soit $n\in \N$ et $\lambda_1, \cdots, \lambda_n$ tels que
\begin{displaymath}
  \lambda_1 f_1 + \cdots + \lambda_n f_n = 0
\end{displaymath}
Première méthode. Il s'agit ici de la \emph{fonction} nulle. Comme les fonctions sont définies par des $\sin$, en dérivant $2m$ fois, on obtient, pour tout $t\in \R$,
\begin{displaymath}
\lambda_1 \sin(t) + \lambda_2 2^{2m}\sin(2t) + \cdots + \lambda_n n^{2m}\sin(nt) = 0
\end{displaymath}
Fixons un $t$ tel que $\sin(nt)\neq 0$, la suite
\begin{multline*}
  \left((\frac{1}{n})^{2m}\lambda_1 \sin(t) +  \cdots  \right.\\
  \left. +\lambda_{n-1} (\frac{n-1}{n})^{2m}\sin(nt) + \lambda_n\sin(nt) \right)_{m\in \N} 
\end{multline*}
est constante nulle et converge vers $\lambda_n \sin(nt)$. On en déduit $\lambda_n = 0$. On raisonne de la même manière pour les autres coefficients.\newline
Deuxième méthode. Introduisons une fonction $\varphi$ :
\begin{displaymath}
\forall t \in \R,\hspace{0.5cm}  \varphi(t) = \sum_{k=1}^n\lambda_k\,e^{ikt}
\end{displaymath}
et réinterprétons trigonométriquement la relation
\begin{multline*}
\varphi(t)\in \R \Leftrightarrow \varphi(t) = \overline{\varphi(t)}
\Leftrightarrow
\sum_{k=1}^n\lambda_k\,e^{ikt} = \sum_{k=1}^n\lambda_k\,e^{-ikt}\\
\Leftrightarrow
\left( \sum_{k=1}^n\lambda_k\,e^{ikt}\right)e^{int}  = \left( \sum_{k=1}^n\lambda_k\,e^{-ikt}\right) e^{int}\\
\Leftrightarrow
\widetilde{P}(e^{it}) = 0 \text{ avec } 
 P = \lambda_n + \lambda_{n-1}X + \cdots + \lambda_1 X^{n-1}\\
 - \lambda_1 X^{n+1} - \cdots - \lambda_n X^{2n} .
\end{multline*}
Le polynôme $P$, de degré au plus $2n$, admet une infinité de racines (tous les nombres complexes de module 1), il est donc nul: tous ses coefficients sont nuls.
