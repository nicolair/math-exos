\begin{tiny}(Cde06)\end{tiny} D'après la formule de Taylor-Young, si le coefficient de $x^6$ dans un dl en $0$ d'une fonction $f \in \mathcal{C}^6$ est $c$,
$f^{(6)}(0) = c \, 6!$.\newline
Calculons ces développements pour $(1+x\sin x) ^{x\, \tan x}$.
\[
  1 + x \sin x = 1 + x^2 - \frac{1}{6}x^4 + o(x^5).
\]
Utilisons $\ln(1+u) = u - \frac{1}{2}u^2 + O(u^3)$ avec
\begin{multline*}
  u = x^2 - \frac{1}{6}x^4 + o(x^5) \sim x^2 \\
  \Rightarrow O(u^3) = O(x^6) \subset o(x^5).
\end{multline*}

On en tire
\begin{multline*}
\left.
\begin{aligned}
  \ln(1 + x\sin x) &= x^2 - \frac{2}{3}x^4 + o(x^5) \\
  x\,\tan x &= x^2 + \frac{1}{3}x^4 + o(x^5)
\end{aligned} \right\rbrace 
\Rightarrow \\
x\,\tan x\, \ln(1 + x\sin x)
= x^4 - \frac{1}{3}x^6 + o(x^7).
\end{multline*}

On termine en utilisant $e^u = 1 + u + O(u^2)$ avec 
\[
  u = x^4 - \frac{1}{3}x^6 + o(x^7) \sim x^4 
  \Rightarrow O(u^2) = O(x^8) \subset o(x^6).
\]
Donc $f^{(6)}(0) = -\frac{6!}{3} = -240$.\newline
De même pour $(\cos x) ^{x\, \tan x}$.
\begin{multline*}
  \cos x = 1 - \frac{1}{2}x^2 + \frac{1}{24}x^4 + o(x^4) \Rightarrow \\
  \ln(\cos x) = - \frac{1}{2}x^2 - \frac{1}{12}x^4 + o(x^4).
\end{multline*}

Donc $f^{(6)}(0) = -\frac{6!}{12} = -80$.
