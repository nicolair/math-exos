\begin{tiny}(Eef12)\end{tiny} Supplementaire commun à deux sous-espaces.\newline
Soit $E$ un $\K$-espace vectoriel.
\begin{enumerate}
 \item Soit $U$ et $V$ deux sous-espaces vectoriels de $E$. Montrer que
 \[
  U \cup V = E \Rightarrow U = E \text{ ou } V = E.
 \]
 Dans la suite de l'exercice, $E$ est de dimension finie, $A$ et $B$ sont deux sous-espaces de même dimension strictement plus petite que $\dim E$.
 \item Montrer qu'il existe des familles libres $\mathcal{F}$ de vecteurs de $E$ telles que 
\[
 A \cap \Vect(\mathcal{F}) = \left\lbrace 0_E\right\rbrace \text{ et } B \cap \Vect(\mathcal{F}) = \left\lbrace 0_E\right\rbrace.
\]
 \item Montrer qu'il existe un sous-espace $C$ de $E$ tel que
\[
 A \oplus C = B \oplus C = E.
\]

\end{enumerate}
