\begin{tiny}(Cdi25)\end{tiny} Un sens est facile :
\begin{displaymath}
 h \circ g = f \circ k
\end{displaymath}
avec $h$ bijectif entraine
\begin{displaymath}
 \rg(g) = \rg(h \circ g) = \rg(f \circ k) \leq \rg(f).
\end{displaymath}
Réciproquement, supposons $\rg(g) \leq \rg(f)$.\newline
Soit $(g_1,\cdots, g_p)$ une base de $\Im(g)$ que l'on complète en une base $(g_1,\cdots, g_n)$ de $F$ et $(f_1, \cdots , f_q)$ une base de $\Im(f)$ que l'on complète en une base $(f_1, \cdots , f_n)$ de $F$. On définit $h\in GL(F)$ par:
\begin{displaymath}
 \forall i \in \llbracket 1,n \rrbracket, \; h(g_i) = f_i.
\end{displaymath}
C'est une bijection car l'image d'une base particulière est une base. Il existe aussi des familles libres de $E$ 
\begin{displaymath}
 (e_1,\cdots,e_p) \text{ et } (e'_1,\cdots,e'_q)
\end{displaymath}
telles que 
\begin{align*}
 \forall i \in \llbracket 1,p \rrbracket,& &g(e_i) = g_i\\
 \forall i \in \llbracket 1,q \rrbracket,& &f(e'_i) = f_i
\end{align*}
Considérons une base $(e_{p+1},\cdots,e_m)$ de $\ker g$. D'après le lemme noyau-image, $(e_1,\cdots,e_m)$ est une base de $E$. On définit $k$ en posant
\begin{multline*}
 k(e_1) = e'_1, \cdots, k(e_p) = e'_p, \\ k(e_{p+1})= \cdots = k(e_m) = 0
\end{multline*}
Alors, par $h\circ g$:
\begin{align*}
 e_1 &\rightarrow g_1 \rightarrow f_1 \\
 \vdots &\rightarrow \vdots \rightarrow \vdots \\
 e_p &\rightarrow g_p \rightarrow f_p \\
 e_{p+1} &\rightarrow 0_F \rightarrow 0_F \\
 \vdots &\rightarrow \vdots \rightarrow \vdots \\
 e_{m} &\rightarrow 0_F \rightarrow 0_F
\end{align*}
et, par $f\circ k$:
\begin{align*}
 e_1 &\rightarrow e'_1 \rightarrow f_1 \\
 \vdots &\rightarrow \vdots \rightarrow \vdots \\
 e_p &\rightarrow e'_p \rightarrow f_p \\
 e_{p+1} &\rightarrow 0_F \rightarrow 0_F \\
 \vdots &\rightarrow \vdots \rightarrow \vdots \\
 e_{m} &\rightarrow 0_F \rightarrow 0_F
\end{align*}

