\begin{tiny}(Efr14)\end{tiny} Démonstration arithmétique de l'existence et de l'unicité d'une partie polaire.\newline
Soit $a\in \C$ un pôle de multiplicité $\alpha$ de $F\in \C(X)$. Montrer, avec des outils arithmétiques qu'il existe un unique polynôme $\Lambda$ tel que $a$ ne soit pas un pôle de $F-\frac{\Lambda}{(X-a)^\alpha}$ et que $\deg(\Lambda) < \alpha$.
