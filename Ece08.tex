\begin{tiny}(Ece08)\end{tiny} Dans un plan affine euclidien muni d'un repère orthonormé $(O,(\overrightarrow i, \overrightarrow j))$, on considère la courbe paramétrée
\begin{displaymath}
\forall \theta\in \R:\; f(\theta)= O+\sin^4(\frac{\theta}{2})\overrightarrow{e}_\theta
\end{displaymath}
On note $\Gamma$ le support de $f$.
\begin{enumerate}
 \item Déterminer des symétries de $\Gamma$. Faire tracer la courbe par un logiciel de calcul formel ou une calculatrice.
 \item Calculer la longueur totale de la courbe.
 \item On oriente la courbe $\Gamma$ dans le sens des $\theta$ croissants. Déterminer l'angle $(\widehat{\overrightarrow{i},\overrightarrow{\tau}(\theta)})$ puis le rayon de courbure en $f(\theta)$.
\end{enumerate}
 