\begin{tiny}(ed15)\end{tiny} Méthode de variation d'une constante pour une équation différentielle linéaire d'ordre $2$.\newline
Soit $I$ un intervalle de $\R$ et $a$, $b$, $f$ des fonctions continues à valeurs réelles. On considère les équations suivantes d'inconnue $y$
\begin{align*}
 &(E)& &y''+ay'+by = f\\&(H)& &y''+ay'+by = 0
\end{align*}
Soit $z_0$ une solution de $(H)$ qui ne prend pas la valeur $0$ dans $I$. Former une équation $(L)$ d'inconue $y$ telle que 
\begin{displaymath}
 qz_0\text{ solution de } (E) \Leftrightarrow q \text{ solution de } (L)
\end{displaymath}
Application. Résoudre les équations suivantes par la méthode précédente en utilisant la fonction $z_0$ indiquée
\begin{align*}
 &(1)  &y''(t)-4y'(t)+3y(t)=e^{t}(1+\cos t + t\cos t) &z_0(t)=e^t\\
 &(2)  &(1+t)y''(t)-2y'(t)+(1-t)y(t)=te^{-t}&z_0(t)=e^t\\
 &(3) &(1-t^2)y''(t)-ty'(t)+y(t)=\sqrt{1-t^2}&z_0(t)=e^t
\end{align*}
Pour $(2)$, se limiter à $]-1,+\infty[$. Pour $(3)$, se limiter à $]-1,+1[$.