\begin{tiny}(Eee17)\end{tiny} On veut montrer que, pour $k$ entre $1$ et un entier $p>1$ fixé, les fonctions
\begin{displaymath}
 s_k:
\left\lbrace 
\begin{aligned}
 [0,\frac{\pi}{2}] \R &\rightarrow \R \\
t &\mapsto \sin(kt)
\end{aligned}
\right. 
\end{displaymath}
forment une famille libre dans l'espace $\mathcal{C}([0,\frac{\pi}{2}])$.
\begin{enumerate}
 \item Calculer
\begin{displaymath}
 \int_0 ^\pi \sin(it)\sin(jt)\,dt
\end{displaymath}
En déduire que la famille des fonctions définies dans $[0,\pi]$ est libre.
\item Soit $\lambda_1,\cdots,\lambda_p$ des nombres complexes. Montrer que si la fonction
\begin{displaymath}
 t\rightarrow \lambda_1 \sin t + \lambda_2\sin(2t)+\cdots + \lambda_p \sin(pt)
\end{displaymath}
s'annule strictement plus de $2p$ fois alors les $\lambda_i$ sont tous nuls. Que peut-on en déduire relativement à des restrictions des fonctions considérées?
\end{enumerate}
