\begin{tiny}(Cap09)\end{tiny} 
\begin{enumerate}
 \item On peut remarquer que $\varphi$ est impaire. En dérivant, on trouve
\begin{multline*}
 \varphi^{(4)}(x)
= -\frac{1}{3}\left(-f^{(4)}(-x)+f^{(4)}(x) \right) \\
-\frac{x}{3} \left( f^{(5)}(-x)+f^{(5)}(x) \right)
\end{multline*}
\item Avec les notations usuelles $m_5$ et $M_5$, on obtient pour $x>0$:
\begin{displaymath}
 \frac{4m_5}{3}x\leq -\varphi^{(4)}(x) \leq \frac{4M_5}{3}x
\end{displaymath}
En intégrant ces inégalités plusieurs fois, on obtient
\begin{displaymath}
 \frac{m_5}{90}x^5\leq -\varphi(x) \leq \frac{M_5}{90}x^5
\end{displaymath}
qui conduit à l'existence du $c$ par le théorème des valeurs intermédiaires appliqué à la fonction continue $f^{(5)}$.
\end{enumerate}
