\begin{tiny}(Csc26)\end{tiny} Notons $u_n$, $v_n$, $w_n$ les termes d'indices $n$ pour les trois suites proposées. 
Pour chacune, on forme des inégalités qui permettent de conclure par le théorème d'encadrement ou ses variantes.\newline
Pour tout $k \in \llbracket 1,n \rrbracket$:
\[
 \sqrt{n^2 + k} > n \Rightarrow u_n \leq n\, \frac{1}{n} = 1
\]
\[
 \sqrt{n^2 + k} < \sqrt{n^2 + 2n +1} \Rightarrow u_n > n\, \frac{1}{n+1}
\]
\[
 \frac{n}{n+1} \leq u_n \leq 1 \Rightarrow \left( u_n \right)_{n \in \N^*} \rightarrow 1.
\]
Pour $v_n$, on somme la minoration jusqu'à $n^2$:
\[
 \frac{n^2}{n+1} \leq v_n \Rightarrow \left( v_n \right)_{n \in \N^*} \rightarrow +\infty.
\]
Avec la définition de la partie entière
\begin{multline*}
 \frac{1}{n^3}\sum_{k=1}^nk(kx-1) < w_n \leq \frac{1}{n^3}\sum_{k=1}^nk^2x \\
 \Rightarrow
 \frac{n(n+1)(2n+1)}{6n^3} x  - \frac{n(n+1)}{2n^3} \\
 \leq w_n \leq \frac{n(n+1)(2n+1)}{6n^3} x.
\end{multline*}

On en déduit 
\[
 \left( x_n \right)_{n \in \N^*} \rightarrow \frac{x}{3}.
\]

