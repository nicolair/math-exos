\begin{tiny}(Egp16)\end{tiny} Puissance d'un point par rapport à un cercle. Axe radical de deux cercles.\newline
On note $x$ et $y$ les fonctions coordonnées dans un repère orthonormal fixé. Soit $\mathcal{C}$ un cercle, il existe des réels $a$, $b$, $c$ tels que, pour tout point $M$,
\begin{displaymath}
 M \in \mathcal{C}
\Leftrightarrow p_{\mathcal{C}}(M)=0
\text{ avec }
p_{\mathcal{C}} = x^2 + y^2 +ax+by+c
\end{displaymath}
\begin{enumerate}
 \item Soit $M$ un point et $\mathcal{D}$ une droite passant par $M$ et coupant $\mathcal{C}$ en deux points $P$ et $Q$ (évenuellement confondus). Montrer que
\begin{displaymath}
 p_{\mathcal{C}}(M) = (\overrightarrow{MP}/\overrightarrow{MQ})
\end{displaymath}
\item Soit $\mathcal{C}'$ un autre cercle avec
\begin{displaymath}
p_{\mathcal{C}'} = x^2 + y^2 +a'x+b'y+c'
\end{displaymath}
Montrer que l'ensemble des points $M$ tels que 
\begin{displaymath}
 p_{\mathcal{C}}(M) = p_{\mathcal{C}'}(M)
\end{displaymath}
est une droite (axe radical des deux cercles). Montrer que l'axe radical est orthogonal à la droite des centres.
\item On se donne un cercle $\mathcal{C}$ et deux points $A$ et $B$ qui ne sont pas sur $\mathcal{C}$. \`A toute droite $\mathcal{D}$ passant par $A$ et coupant $\mathcal{C}$ en $M$ et $N$, on associe le cercle passant par $M$, $N$, $B$. Il est noté $\mathcal{C}(\mathcal{D})$. Montrer que tous les $\mathcal{C}(\mathcal{D})$ passent par un même point (autre que $B$).
\item On se donne trois cercles et les trois axes radicaux de ces cercles pris deux à deux. Montrer que ces trois droites sont parallèles ou concourantes.
\end{enumerate}

 