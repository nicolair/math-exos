\begin{tiny}(Cgs10)\end{tiny} L'application $\sigma_g$ est une permutation car elle est bijective de bijection réciproque $\sigma_{g^{-1}}$. L'orbite d'un élément $h\in G$ selon $\sigma_g$ est 
\[
 \left\lbrace h, gh, \cdots, g^{k-1}h \right\rbrace .
\]
Chaque orbite contient $p$ éléments et elles constituent une partition de $G$ donc $n = \text{Nb orbites}\times p$. La permutation $\sigma_g$ se décompose donc en $\frac{n}{p}$ cycles disjoints de longueur $p$:
\[
 \varepsilon(\sigma_g = \left((-1)^{p-1} \right)^{\frac{n}{p}} = (-1)^{\frac{n(p-1)}{p}}. 
\]
En particulier, $\sigma_g$ est paire si et seulement si
\[
 \frac{n(p-1)}{p} \equiv 0 \mod 2 \Leftrightarrow n \equiv \frac{n}{p}\mod 2.
\]
C'est toujours réalisé si $n$ est impair car alors $\frac{n}{p}$ est aussi impair.