\begin{tiny}(Cal18)\end{tiny}
\begin{enumerate}
 \item D'après les définitions de $\sh$ et $\ch$,
\begin{multline*}
 \sh(t)*\sh(u)=\sh(t)\ch(u)+\sh(u)\ch(t)\\
=\frac{1}{4}\left(e^{t+u}+e^{-t+u}-e^{-t+u}-e^{-t-u} \right)\\
=\sh(t+u) 
\end{multline*}

\item L'application $\sh$ est une bijection de $\R$ dans $\R$ qui transporte l'opération "$+$" sur l'opération "$*$". Celle ci définit donc une structure de groupe sur $\R$. Pour cette opération, $\sh$ est un isomorphisme de groupe de $(\R,+)$ sur $(\R,*)$.
\end{enumerate}
 
