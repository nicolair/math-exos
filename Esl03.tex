\begin{tiny}(Esl03)\end{tiny} Discuter suivant les paramètres de l'existence et du nombre de solutions des systèmes aux inconnues $(x,y,z)$.
\begin{multline*}
(1) \;\left\lbrace 
\begin{aligned}
 -2x + y + 2z &= a \\
 -x + 2y + 2z &= b\\
 -2x +2y +3z &= c
\end{aligned}
\right. ,\hspace{0.3cm}
(2)\;\left\lbrace 
\begin{aligned}
  ax+y+z &= \alpha \\
  x+ay+z &= \beta
\end{aligned}
\right. \\
(3)\left\lbrace 
\begin{aligned}
 x + ay + a^2z &= a^3 \\
 x + by + b^2z &= b^3\\
 x + cy + c^2z &= c^3
\end{aligned}
\right.  ,\;
(4)\left\lbrace 
\begin{aligned}
 \alpha x + y + z + t &= 1 \\
 x + \alpha y + z + t &= \beta\\
 x + y + \alpha z + t &= \beta^2 \\
 x + y + z + \alpha t &= \beta^3
\end{aligned}
\right.
\end{multline*}
Pour le système $(4)$, on pourra additionner les lignes.
