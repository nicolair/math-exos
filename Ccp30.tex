\begin{tiny}(Ccp30)\end{tiny} Nommons les équations d'inconnue $z$ qui interviennent dans la première équation.
\begin{align*}
 (1)& &z^4+6z^2+25=0\\
 (2)& &z^2+6z+25=0
\end{align*}
Pour tout nombre complexe $w$, il est clair que $w$ est solution de $(1)$ si et seulement si $w^2$ est solution de $(1)$. On se ramène donc aux techniques de cours : résolution d'une équation du second degré puis recherche de racines carrées. On trouve que les quatre racines sont :
\begin{align*}
 &1+2i,& &-1-2i,& &1-2i, &-1+2i 
\end{align*}
Calcul analogue pour la deuxième équation. On résoud une équation du second degré puis on extrait les racines carrées des solutions. On trouve
\begin{align*}
 1-i, & & 3-2i, & & -1+i, & & -3+2i
\end{align*}
Calcul analogue pour la troisième. Les solutions de $z^3+z^2+z+1=0$ sont les racines quatrièmes de l'unité autres que $1$ soit $-1$, $i$ et $-i$. On doit donc résoudre
\begin{displaymath}
 \frac{z+i}{z-i} = -1 \hspace{0.5cm}(= i) \hspace{0.5cm}(= -i)
\end{displaymath}
On trouve $0$, $1$, $-1$.