\begin{tiny}(Cen02)\end{tiny} Une relation binaire sur un ensemble $E$ s'identifie à une partie $\Omega$ de $E\times E$. Un élément $a$ est en relation avec $b$ si et seulement si $(a,b)$ est dans la partie du produit cartésien associé.\newline
La relation est réflexive si et seulement si $\Omega$ contient la diagonale (la partie de $E\times E$ formée par les couples $(a,a)$). Le nombre de relations réflexives est donc le nombre de parties du complémentaire de cette diagonale soit
\begin{displaymath}
  2^{n^2 - n}
\end{displaymath}
En numérotant arbitrairement $a_1,\cdots,a_n$ les éléments de $E$, on peut caractériser les relations reflexives et symétriques par les parties du \og triangle \fg~ de $E\times E$ formé par les $(a_i,a_j)$ avec $i<j$. L'autre partie de $\Omega$ étant obtenue par symétrie. Le nombre de relations réflexives et symétriques est donc
\begin{displaymath}
  2^{\frac{n(n-1)}{2}}.
\end{displaymath}

