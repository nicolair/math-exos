\begin{tiny}(Cef10)\end{tiny} Pour savoir si $(v_1,v_2,v_3)$ est libre, on forme le système traduisant $x_1 v_1 + x_2v_2 + x_3 v_3 = 0_{\R^4}$.
\begin{multline*}
 \left\lbrace
 \begin{aligned}
  x_1 + x_2 + 2x_3 &= 0 \\
  2x_1 + x_2 + x_3 &= 0 \\
  3x_1 + x_2 + x_3 &= 0 \\
  4x_1 + 3x_2 + x_3 &= 0
 \end{aligned}
\right. 
\Rightarrow
 \left\lbrace
 \begin{aligned}
  x_1 + x_2 + 2x_3 &= 0 \\
  x_1  - x_3 &= 0 \\
  2x_1 - x_3 &= 0 \\
  x_1 + 2x_2  &= 0
 \end{aligned}
\right. \\
\Rightarrow x_1 = x_2 = x_3 = 0
\Rightarrow \dim(A) = 3.
\end{multline*}

$(v_4,v_5)$ est libre (position des $0$) donc $\dim(B) = 2$.\newline
Pour savoir si $(v_1,v_2,v_3,v_4)$ est libre, on forme le système traduisant $x_1 v_1 + x_2v_2 + x_3 v_3 + x_4v_4= 0_{\R^4}$.
\begin{multline*}
 \left\lbrace
 \begin{aligned}
  x_1 + x_2 + 2x_3 - x_4 &= 0 \\
  2x_1 + x_2 + x_3 &= 0 \\
  3x_1 + x_2 + x_3 -x_4 &= 0 \\
  4x_1 + 3x_2 + x_3 + 2x_4 &= 0
 \end{aligned}
\right. \\
\Rightarrow
 \left\lbrace
 \begin{aligned}
  x_1 + x_2 + 2x_3 - x_4 &= 0 \\
  x_1  - x_3 + x_4 &= 0 \\
  2x_1 - x_3 &= 0 \\
  x_1 + 2x_2 + 3x_4 &= 0
 \end{aligned}
\right. \\
\Rightarrow
 \left\lbrace
 \begin{aligned}
  5x_1 + x_2  - x_4 &= 0 \\
  -x_1   + x_4 &= 0 \\
  2x_1 - x_3 &= 0 \\
  x_1 + 2x_2 + 3x_4 &= 0
 \end{aligned}
\right. 
\Rightarrow
 \left\lbrace
 \begin{aligned}
  4x_1 + x_2   &= 0 \\
  -x_1   + x_4 &= 0 \\
  2x_1 - x_3 &= 0 \\
  4x_1 + 2x_2 &= 0
 \end{aligned}
\right. \\
\Rightarrow x_1 = x_2 = x_3 = x_4 = 0
\end{multline*}

La famille de 4 vecteurs est libre dans $\R^4$, c'est donc une base. On en déduit
\[
 \dim(A+b) = 4 \text{ et } \dim(A \cap B) = 1
\]
d'après la formule de Grassmann.
