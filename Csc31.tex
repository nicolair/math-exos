\begin{tiny}(Csc31)\end{tiny} 
\begin{enumerate}
  \item La fonction $f_n$
\begin{displaymath}
  \left\lbrace 
\begin{aligned}
  \left[ 0 , +\infty \right[ &\rightarrow \R \\
  x &\mapsto x + x^2 + \cdots + x^n
\end{aligned}
\right. 
\end{displaymath}
est strictement croissante de $0$ à $+\infty$. Le réel $a>0$ admet donc un unique antécédent $x_n$. On peut noter que $x_1=a$.
  \item On remarque que
\begin{displaymath}
  f_{n+1}(x_n) = a + x_{n}^{n+1} > a
\end{displaymath}
L'antécédent $x_{n+1}$ de $a$ par $f_{n+1}$ est donc plus petit que $x_n$. La suite $\left( x_n\right)_{n\in \N^*}$ est décroissante. Elle est minorée par $0$ donc elle converge. On note $l$ sa limite.\newline
Comme $f_n(1)=n$, pour $n$ assez grand, on aura $a<f(n)$. Il existe donc un $N$ tel que 
\begin{displaymath}
\forall n\geq N, \; x_n \leq x_N < 1  
\end{displaymath}
On en déduit que $\left( x_n^{n+1}\right)_{n\in \N^*}\rightarrow 0$ (majoration par une suite géométrique convergente). En utilisant la formule pour une somme de termes en progression géométrique:
\begin{displaymath}
  a = f_n(x_n) = \frac{x_n - x_n^{n+1}}{1-x_n}
\end{displaymath}
Par opérations sur les suites convergentes, on déduit
\begin{displaymath}
a = \frac{l}{1-l} \Rightarrow l = \frac{a}{1+a}  
\end{displaymath}

\end{enumerate}
