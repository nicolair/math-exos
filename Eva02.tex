\begin{tiny}(Eva02)\end{tiny} Loi hypergéométrique.\newline
Une urne contient $N$ boules dont une proportion $p$ de boules blanches et $q = 1-p$ de noires.\newline 
On tire sans remise $n \leq \min(pN,qN)$ boules et on note $X$ le nombre de boules blanches tirées.
\begin{enumerate}
  \item Déterminer la loi de $X$. On modélisera de deux manières: avec des parties à $n$ éléments dans $\llbracket 1, N\rrbracket$, ou avec des fonctions injectives de $\llbracket 1,n\rrbracket$ dans $\llbracket 1,N \rrbracket$ et on vérifiera que la loi est la même.
  \item Calculer $E(X)$ par sa définition avec la loi de $X$.
  \item Pour le modèle avec des injections, préciser des variables de Bernoulli $X_1, \cdots, X_n$ dont la somme est $X$. Retrouver $E(X)$.
  \item Pour le modèle avec des parties, préciser des variables de Bernoulli $B_1, \cdots, B_{pN}$ dont la somme est $X$. Retrouver $E(X)$. Montrer que
\[
  V(X) = npq\, \frac{N - n}{N - 1}.
\]

\end{enumerate}
