\begin{tiny}(Egd20)\end{tiny}Une forme continue du théorème de Césaro\footnote{Voir l'exercice sc8 de la \href{http://back.maquisdoc.net/data/temptex/fexsc.pdf}{feuille d'exercices les suites de réels}}.\\
Soit $f$ une fonction $\mathcal C^1(]0,+\infty[)$.
\begin{enumerate}
 \item Montrer que
\begin{displaymath}
 f'\xrightarrow{+\infty} 0 \Rightarrow \frac{f(x)}{x} \xrightarrow{+\infty} 0
\end{displaymath}
 \item Montrer que
\begin{displaymath}
 f'\xrightarrow{+\infty} \lambda\in \R \Rightarrow \frac{f(x)}{x} \xrightarrow{+\infty} \lambda
\end{displaymath}
 \item Montrer que
\begin{displaymath}
 f'\xrightarrow{+\infty} +\infty \Rightarrow \frac{f(x)}{x} \xrightarrow{+\infty} +\infty
\end{displaymath}
\end{enumerate}
On peut considérer un $a>0$ et écrire, pour $x\geq a$:
\begin{displaymath}
 \frac{f(x)}{x}=\frac{f(x)-f(a)}{x} + \frac{f(a)}{x}
\end{displaymath}
